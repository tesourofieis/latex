\documentclass[paper=a5,10pt,openany]{scrbook}
\usepackage[footnotesep=1.5\baselineskip,twoside, top=0.06\paperheight, bottom=0.06\paperheight, left=0.125\paperwidth,
right=0.125\paperwidth, headsep=0.02\paperheight, foot=0.01\paperheight, bindingoffset=0\paperheight,heightrounded,includeheadfoot]{geometry}

\usepackage{background}

\parskip=0pt
\raggedbottom
\usepackage{opcdevocionario}%%%DEFINICOES%%%
\usepackage{tipografia}

\setkomafont{subject}{\scshape}
\setkomafont{title}{\Huge}
\setkomafont{subtitle}{\scshape\color{red}}
\setkomafont{author}{\scshape}
\setkomafont{date}{\itshape\color{red}}

\setkomafont{chapter}{\normalfont\Large\scshape}
\renewcommand*{\raggedchapter}{\centering}

\begin{document}%%%DOC%%%
\pagenumbering{gobble}% Remove page numbers (and reset to 1)
\backgroundsetup{scale=1,opacity=1,angle=0,position=current page.center,contents={\includegraphics[width=1.1\textwidth, height=1.1\textheight]{media/frame6}}}

\begin{titlepage}
\begin{center}
\vspace*{2.5cm}
\Huge{Tabela de Oração e Penitência}\\

\vspace*{1cm}

\includegraphics[width=0.85\textwidth]{media/sanctemichael}
\vspace*{1cm}

\Large{\redx Militia Sancti Michæli}\\
\large{2018}\\

\end{center}
\end{titlepage}

\sloppy%Justifica texto
\pretolerance=-1
\tolerance=3000
\adjdemerits=6400
\doublehyphendemerits=8000000
\finalhyphendemerits=14400
\hbadness=99999

\backgroundsetup{scale=1,opacity=1,angle=0,position=current page.center,contents={\includegraphics[width=1.1\textwidth, height=1.1\textheight]{media/frame6}}}

\NewDocumentEnvironment{hangparacol}{mo}% LADO A LADO
  {\IfNoValueTF{#2} {\begin{paracol}{#1}}{\begin{paracol}{#1}[#2]}%
   \raggedright
   \parindent=3em \leftskip=3em}
  {\end{paracol}}
\columnratio{0.47}
\setlength{\columnseprule}{0pt}
\setlength{\columnsep}{3mm}
\colseprulecolor{red}

\pagenumbering{arabic}

Caríssimos amigos,

A \textit{Militia Sancti Michæli}, como sabemos, é um pequeno grupo de homens, unido informalmente em torno da comum devoção à Tradição da Santa Igreja Católica, em especial à Santa Missa rezada segundo os livros do rito romano até 1962, dita \textit{Tradicional, Tridentina, Antiga, de Todas as Eras, de Sempre, Eterna.}

Na nossa informalidade – de facto, não fomos instituídos, não temos estatutos, nem dirigentes, nem estarmos integrados em paróquia ou movimento –, fomos abençoados por Sua Excelência Reverendíssima o Senhor Dom Athanasius Schneider, Bispo Auxiliar de Maria Santíssima de Astana, Cónego Regular da Ordem de Santa Cruz de Coimbra, no lugar dos Valinhos, em Fátima, na tarde de 14 de Julho de 2017, em pleno Centenário das Aparições de Nossa Senhora do Rosário, depois da recitação do terço. 

Temos sido assistidos por um bom sacerdote, cujo nome contamos na lista de milicianos; e temos sido apoiados por outros padres e pessoas devotas.  
Há mais de um ano que conversamos diariamente, sobre vários temas da Tradição; e temos vindo também a organizar ou a convergir para alguns encontros, assistindo então à Santa Missa, fazendo adoração eucarística, peregrinando, recintando o terço; recebendo formação. 

Recentemente, iniciámos a recitação do terço e de orações de reparação no antigo coro baixo da Basílica do Sagrado Coração de Jesus da Estrela, em Lisboa, todas as quintas-feiras, de manhã. Para este propósito, saiu à luz a nossa primeira publicação, o livrinho \textit{Sanctum Rosarium}, editado por \textit{Michæl, Miles Sancti Ægidii}, com um esquema de oração em latim e em vernáculo. 

Temos também uma bandeira do nosso glorioso patrono, São Miguel Arcanjo, benzida e hasteada pela primeira vez no Santuário do Senhor Jesus da Pedra, em Óbidos, no passado dia 31 de Maio, onde assistimos à Santa Missa. 
Têm entrado neste grupo novos milicianos e, com felicidade, crescemos juntos nas virtudes que Nosso Senhor nos concede com abundância, as quais vão sendo cultivadas nas nossas almas pela Santíssima Virgem, Nossa Mãe e Rainha; por São Miguel Arcanjo, pelo glorioso São José e pelos santos que veneramos pessoalmente. 

Louvado seja Deus por tudo o que nos tem dado.

Mas é preciso rezar mais e fazer mais penitência. 

Convoco-vos, pois, à oração e à penitência, nas mesmas intenções anunciadas em cada quinta-feira, naquela Basílica:

\begin{compactitem}
\item Em desagravo do Santíssimo Coração de Jesus e do Coração Imaculado de Nossa Senhora;

\item Segundo as Intenções do Santo Padre, a saber: 
\begin{compactitem}
\item Exaltação da Santa Igreja Católica; 

\item Propagação da fé católica; 

\item Extirpação das heresias; 

\item Conversão dos pecadores; 

\item Paz e a concórdia entre os príncipes e reis católicos.
\end{compactitem}
\item Pelos homens da \textit{Militia Sancti Michæli} e suas Famílias;

\item Por outras intenções.
\end{compactitem}

A presente tabela atribui a cada miliciano um dia de oração e de penitência por mês, durante o qual se obrigará a cumprir o seguinte:

\begin{compactitem}
\item Estar em estado de graça santificante, recorrendo ao sacramento da Confissão numa data anterior ou no próprio dia;

\item Rezar o terço ou o rosário de Nossa Senhora, seguindo o esquema proposto no livrinho \textit{Sanctum Rosarium}, se possível com a sua Família;

\item Oferecer, nas mãos do glorioso São Miguel Arcanjo, uma penitência, segundo entender e estiver autorizado a praticar pelo seu director espiritual ou confessor; e, se casado, tendo procurado a opinião da sua mulher;

\item Assistir à Santa Missa tradicional e receber a Sagrada Comunhão, se lhe for possível.
\end{compactitem}

Exorto-vos a perseverar nesta nova missão da \textit{Militia Sancti Michæli}.

Do Céu nos virá todo o auxílio para não esmorecermos. 
\textit{Laus Deo semper}.

\textit{Ioseph, Miles Sanctæ Mariæ}

Lisboa, Junho de 2018

 \subsection{Tabela}

\begin{paracol}{2}
\begin{nscenter}Milicianos\end{nscenter}
\switchcolumn
\begin{nscenter}Dia de cada mês\end{nscenter}
\switchcolumn*
Aloisius, Miles Sancti Aloisii Mariæ Grignion
\switchcolumn
\begin{nscenter}1\end{nscenter}
\switchcolumn*
Aloisius, Miles Sancti Ioannis Fisher
\switchcolumn
\begin{nscenter}2\end{nscenter}
\switchcolumn*
Alphonsus, Miles Sancti Pii X
\switchcolumn
\begin{nscenter}3\end{nscenter}
\switchcolumn*
Bernardus, Miles Sanctæ Hyacinthæ
\switchcolumn
\begin{nscenter}4\end{nscenter}
\switchcolumn*
Emanuel, Miles Sancti Nonii de Sancta Maria
\switchcolumn
\begin{nscenter}5\end{nscenter}
\switchcolumn*
Emanuel, Miles Sancti Pauli
\switchcolumn
\begin{nscenter}6\end{nscenter}
\switchcolumn*
Didacus, Miles Sancti Francisci Salesii
\switchcolumn
\begin{nscenter}7\end{nscenter}
\switchcolumn*
Ferdinandus, Miles Sanctæ Crucis
\switchcolumn
\begin{nscenter}8\end{nscenter}
\switchcolumn*
Georgius, Miles Sancti Ioseph
\switchcolumn
\begin{nscenter}9\end{nscenter}
\switchcolumn*
Helder, Miles Sancti Patris Pii
\switchcolumn
\begin{nscenter}10\end{nscenter}
\switchcolumn*
Hubertus, Miles Sancti Pii V
\switchcolumn
\begin{nscenter}11\end{nscenter}
\switchcolumn*
Ioannes, Miles Sancti Petri
\switchcolumn
\begin{nscenter}12\end{nscenter}
\switchcolumn*
Ioannes, Miles Sancti Ioannis Baptista
\switchcolumn
\begin{nscenter}13\end{nscenter}
\switchcolumn*
Ioseph, Miles Sancti Antonius Olisiponensis
\switchcolumn
\begin{nscenter}14\end{nscenter}
\switchcolumn*
Ioseph, Miles Sanctæ Catharinæ Senensis
\switchcolumn
\begin{nscenter}15\end{nscenter}
\switchcolumn*
Ioseph, Miles Sanctæ Mariæ
\switchcolumn
\begin{nscenter}16\end{nscenter}
\switchcolumn*
Michæl, Miles Sancti Fratris Ægidii
\switchcolumn
\begin{nscenter}17\end{nscenter}
\switchcolumn*
Michæl, Miles Sanctæ Mariæ Magdalenæ
\switchcolumn
\begin{nscenter}18\end{nscenter}
\switchcolumn*
Michæl, Miles Sancti Nicolai
\switchcolumn
\begin{nscenter}19\end{nscenter}
\switchcolumn*
Marcus, Miles Sancti Gregorii Pauli
\switchcolumn
\begin{nscenter}20\end{nscenter}
\switchcolumn*
Nonius, Miles Sancti Nonii de Sancta Maria
\switchcolumn
\begin{nscenter}21\end{nscenter}
\switchcolumn*
Petrus, Miles Sancti Gregorii
\switchcolumn
\begin{nscenter}22\end{nscenter}
\switchcolumn*
Simon, Miles Sancti Athanasii
\switchcolumn
\begin{nscenter}23\end{nscenter}
\switchcolumn*
Alphonsus, Miles Sanctæ Mariæ
\switchcolumn
\begin{nscenter}24\end{nscenter}
\switchcolumn*
Petrus, Miles Sanctæ Hyacinthæ
\switchcolumn
\begin{nscenter}25\end{nscenter}
\end{paracol}

\mbox{}
\vfill
\begin{nscenter}
  Edição por Michæl, Miles Sancti Fratris Ægidii
\end{nscenter}

\end{document}