\begin{paracol}{2}\latim{
\rlettrine{T}{e} decet hymnus, Deus, in Sion: * et tibi reddétur votum in Jerúsalem.
}\switchcolumn\portugues{
\rlettrine{A}{} Vós, ó Deus, são devidos os hinos em Sião: * e a Vós serão prestados votos em Jerusalém.
}\switchcolumn*\latim{
Exáudi oratiónem meam: * ad Te omnis caro véniet.
}\switchcolumn\portugues{
Ouvi a minha oração: * a Vós toda a carne virá.
}\switchcolumn*\latim{
Verba iniquórum prævaluérunt super nos: * et impietátibus nostris Tu propitiáberis.
}\switchcolumn\portugues{
As palavras dos iníquos prevaleceram sobre nós: * mas Vós perdoareis as nossas impiedades.
}\switchcolumn*\latim{
Beátus, quem elegísti, et assumpsísti: * inhabitábit in átriis tuis.
}\switchcolumn\portugues{
Bem-aventurado o que elegestes e adoptastes: * ele habitará nos vossos átrios.
}\switchcolumn*\latim{
Replébimur in bonis domus tuæ: * sanctum est templum tuum, mirábile in æquitáte.
}\switchcolumn\portugues{
Seremos cheios dos bens da vossa casa: * santo é o vosso templo, maravilhoso em equidade.
}\switchcolumn*\latim{
Exáudi nos, Deus, salutáris noster, * spes ómnium fínium terræ, et in mari longe.
}\switchcolumn\portugues{
Ouvi-nos, ó Deus, nosso Salvador, * esperança de todos os confins da terra e no mar longínquo.
}\switchcolumn*\latim{
Prǽparans montes in virtúte tua, accínctus poténtia: * qui contúrbas profúndum maris sonum flúctuum ejus.
}\switchcolumn\portugues{
Dais firmeza aos montes com vossa força, cingido de poder: * conturbais o fundo do mar, o estrondo das suas ondas.
}\switchcolumn*\latim{
Turbabúntur gentes, et timébunt qui hábitant términos a signis tuis: * éxitus matutíni, et véspere delectábis.
}\switchcolumn\portugues{
Perturbar-se-ão as gentes e os que habitam os confins da terra temerão aos vossos prodígios: * dareis alegria às saídas matutinas e vespertinas.
}\switchcolumn*\latim{
Visitásti terram, et inebriásti eam: * multiplicásti locupletáre eam.
}\switchcolumn\portugues{
Visitastes a terra e a inebriastes: * multiplicastes suas riquezas.
}\switchcolumn*\latim{
Flumen Dei replétum est aquis, parásti cibum illórum: * quóniam ita est præparátio ejus.
}\switchcolumn\portugues{
O rio de Deus encheu-se de águas, preparastes o seu sustento: * porque tal é a sua disposição.
}\switchcolumn*\latim{
Rivos ejus inébria, multíplica genímina ejus: * in stillicídiis ejus lætábitur gérminans.
}\switchcolumn\portugues{
Inebriai os seus ribeiros, multiplicai as suas produções: * com o destilar do orvalho alegrar-se-á nos frutos.
}\switchcolumn*\latim{
Benedíces corónæ anni benignitátis tuæ: * et campi tui replebúntur ubertáte.
}\switchcolumn\portugues{
Bendireis a coroa do ano da vossa bondade: * e os vossos campos se encherão de abundância.
}\switchcolumn*\latim{
Pinguéscent speciósa desérti: * et exsultatióne colles accingéntur.
}\switchcolumn\portugues{
O deserto ficará viçoso: * e as colinas vestir-se-ão de alegria.
}\switchcolumn*\latim{
Indúti sunt aríetes óvium, et valles abundábunt fruménto: * clamábunt, étenim hymnum dicent.
}\switchcolumn\portugues{
Os carneiros dos rebanhos se agasalharão e os vales estarão cheios de trigo: * clamarão, deveras cantarão hinos.
}\end{paracol}
