\begin{paracol}{2}\latim{
\rlettrine{D}{ómine,} quis habitábit in tabernáculo tuo? * Aut quis requiéscet in monte sancto tuo?
}\switchcolumn\portugues{
\rlettrine{S}{enhor,} quem habitará no vosso tabernáculo? * Ou quem descansará no vosso santo monte?
}\switchcolumn*\latim{
Qui ingréditur sine mácula, * et operátur justítiam:
}\switchcolumn\portugues{
O que vive sem mácula, * e pratica a justiça:
}\switchcolumn*\latim{
Qui lóquitur veritátem in corde suo, * qui non egit dolum in lingua sua:
}\switchcolumn\portugues{
O que fala verdade no seu coração, * o que não forjou dolos com sua língua:
}\switchcolumn*\latim{
Nec fecit próximo suo malum, * et oppróbrium non accépit advérsus próximos suos.
}\switchcolumn\portugues{
Nem mal fez ao seu próximo, * nem consentiu que seus próximos fossem desonrados.
}\switchcolumn*\latim{
Ad níhilum dedúctus est in conspéctu ejus malígnus: * timéntes autem Dóminum gloríficat:
}\switchcolumn\portugues{
Na sua apreciação considera o malvado como um nada, * mas honra os que temem o Senhor:
}\switchcolumn*\latim{
Qui jurat próximo suo, et non décipit, * qui pecúniam suam non dedit ad usúram, et múnera super innocéntem non accépit.
}\switchcolumn\portugues{
Faz juramento ao seu próximo e o não engana, * não empresta o seu dinheiro com usura, nem aceita subornos contra o inocente.
}\switchcolumn*\latim{
Qui facit hæc: * non movébitur in ætérnum.
}\switchcolumn\portugues{
Quem procede assim: * jamais será abalado.
}\end{paracol}
