\begin{paracol}{2}\latim{
\rlettrine{C}{antáte} Dómino cánticum novum: * laus ejus in ecclésia sanctórum.
}\switchcolumn\portugues{
\rlettrine{C}{antai} ao Senhor um cântico novo: * o seu louvor na igreja dos santos.
}\switchcolumn*\latim{
Lætétur Israël in eo, qui fecit eum: * et fílii Sion exsúltent in rege suo.
}\switchcolumn\portugues{
Alegre-se Israel n’Aquele que o criou: * e os filhos de Sião exultem-se em seu rei.
}\switchcolumn*\latim{
Laudent nomen ejus in choro: * in týmpano, et psaltério psallant ei:
}\switchcolumn\portugues{
Louvem em coro o seu nome: * cantem ao som do tambor e do saltério:
}\switchcolumn*\latim{
Quia beneplácitum est Dómino in pópulo suo: * et exaltábit mansuétos in salútem.
}\switchcolumn\portugues{
Pois o Senhor tem-se comprazido no seu povo: * e há-de exaltar os mansos até salvá-los.
}\switchcolumn*\latim{
Exsultábunt sancti in glória: * lætabúntur in cubílibus suis.
}\switchcolumn\portugues{
Exultar-se-ão os santos na glória: * eles alegrar-se-ão nas suas mansões.
}\switchcolumn*\latim{
Exaltatiónes Dei in gútture eórum: * et gládii ancípites in mánibus eórum.
}\switchcolumn\portugues{
As exaltações de Deus estarão na sua boca: * e espadas de dous gumes nas suas mãos.
}\switchcolumn*\latim{
Ad faciéndam vindíctam in natiónibus: * increpatiónes in pópulis.
}\switchcolumn\portugues{
Para exercer a vingança entre as nações: * e o castigo entre os povos.
}\switchcolumn*\latim{
Ad alligándos reges eórum in compédibus: * et nóbiles eórum in mánicis férreis.
}\switchcolumn\portugues{
Para prender os seus reis com grilhões: * e os seus nobres com algemas de ferro.
}\switchcolumn*\latim{
Ut fáciant in eis judícium conscríptum: * glória hæc est ómnibus sanctis ejus.
}\switchcolumn\portugues{
Para executar contra eles a sentença escrita: * tal é a glória reservada a todos seus santos.
}\end{paracol}
