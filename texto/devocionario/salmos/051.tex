\begin{paracol}{2}\latim{
\qlettrine{Q}{uid} gloriáris in malítia, * qui potens es in iniquitáte?
}\switchcolumn\portugues{
\rlettrine{P}{orque} te glorias de tua malícia, * tu que és poderoso em iniquidade?
}\switchcolumn*\latim{
Tota die injustítiam cogitávit lingua tua: * sicut novácula acúta fecísti dolum.
}\switchcolumn\portugues{
Todo o dia a tua língua meditou injustiça: * como navalha afiada dolos fizeste.
}\switchcolumn*\latim{
Dilexísti malítiam super benignitátem: * iniquitátem magis quam loqui æquitátem.
}\switchcolumn\portugues{
Amaste o mal sobre o bem: * a linguagem da iniquidade mais que a da justiça.
}\switchcolumn*\latim{
Dilexísti ómnia verba præcipitatiónis, * lingua dolósa.
}\switchcolumn\portugues{
Amaste todas as palavras de ruína, * ó língua enganadora.
}\switchcolumn*\latim{
Proptérea Deus déstruet te in finem, * evéllet te, et emigrábit te de tabernáculo tuo: et radícem tuam de terra vivéntium.
}\switchcolumn\portugues{
Por isso Deus destruir-te-á para sempre: * arrancar-te-á, expulsar-te-á de tua morada e a tua estirpe da terra dos vivos.
}\switchcolumn*\latim{
Vidébunt justi, et timébunt, et super eum ridébunt, et dicent: * Ecce homo, qui non pósuit Deum adjutórem suum:
}\switchcolumn\portugues{
Vê-lo-ão os justos, temerão e dele se rirão, dizendo: * eis o homem que não tomou a Deus por seu protector:
}\switchcolumn*\latim{
Sed sperávit in multitúdine divitiárum suárum: * et præváluit in vanitáte sua.
}\switchcolumn\portugues{
Contudo, esperou na multidão das suas riquezas: * e prevaleceu na sua vaidade.
}\switchcolumn*\latim{
Ego autem, sicut olíva fructífera in domo Dei, * sperávi in misericórdia Dei in ætérnum: et in sǽculum sǽculi.
}\switchcolumn\portugues{
Eu, porém, sou como oliveira frutífera na casa de Deus, * espero na misericórdia de Deus para sempre e pelos séculos dos séculos.
}\switchcolumn*\latim{
Confitébor tibi in sǽculum, quia fecísti: * et exspectábo nomen tuum, quóniam bonum est in conspéctu sanctórum tuórum.
}\switchcolumn\portugues{
Louvar-Vos-ei eternamente, devido ao que fizestes: * e esperarei no vosso nome, porque é bom ante vossos santos.
}\end{paracol}
