\section{Pequeno Ofício de Nossa Senhora}

\paragraphinfo{Glória}{Depois de cada Salmo}
\begin{paracol}{2}\latim{
℣. Glória Patri, et Fílio, et Spíritui Sancto.
}\switchcolumn\portugues{
℣. Glória ao Pai, e ao Filho e ao Espírito Santo.
}\switchcolumn*\latim{
℟. Sicut erat in pricípio, et nunc, et semper, et in sǽcula sæculórum. Amen.
}\switchcolumn\portugues{
℟. Assim como era no princípio, agora e sempre, e por todos os séculos dos séculos. Amen.
}\end{paracol}


\subsection{Matinas 1}

\begin{paracol}{2}\latim{
℣. Domine, \cruz labia mea aperies.
}\switchcolumn\portugues{
℣. Abri, Senhor, \cruz os meus lábios.
}\switchcolumn*\latim{
℟. Et os meum annuntiabit laudem tuam.
}\switchcolumn\portugues{
℟. E a minha boca anunciará o vosso louvor.
}\switchcolumn*\latim{
℣. Deus \cruz in adjutórium meum inténde.
}\switchcolumn\portugues{
℣. Deus, \cruz vinde em meu auxílio.
}\switchcolumn*\latim{
℟. Dómine, ad adjuvándum me festína.
}\switchcolumn\portugues{
℟. Senhor, apressai-Vos em socorrer-me.
}\switchcolumn*\latim{
Glória Patri, \&c.
}\switchcolumn\portugues{
Glória ao Pai, \&c.
}\end{paracol}

\emph{Desde o Sábado antes do Domingo da Septuagésima até ás vésperas do Sábado Santo, em vez de Allelúja, é dito:}

\begin{paracol}{2}\latim{
Laus tibi, Domine, Rex æternæ gloriæ.
}\switchcolumn\portugues{
Louvado sejais, ó Senhor, Rei da glória eterna.
}\end{paracol}

\paragraph{Invitatório}
\begin{paracol}{2}\latim{
Ave Maria, gratia plena, Dominus tecum.
}\switchcolumn\portugues{
Ave, Maria, cheia de graça, o Senhor é convosco.
}\end{paracol}

\paragraph{Salmo 94}
\begin{paracol}{2}\latim{
\rlettrine{V}{eníte,} exsultémus Dómino: * jubilémus Deo salutári nostro:
}\switchcolumn\portugues{
\rlettrine{V}{inde,} exultemos no Senhor: * cantemos alegres a de Deus nosso salvador:
}\switchcolumn*\latim{
Præoccupémus fáciem ejus in confessióne: * et in psalmis jubilémus ei.
}\switchcolumn\portugues{
Apresentemo-nos diante d’Ele em acção de graças: * e celebremo-l’O com salmos.
}\switchcolumn*\latim{
Ave Maria, gratia plena, Dominus tecum.
}\switchcolumn\portugues{
Ave, Maria, cheia de graça, o Senhor é convosco.
}\switchcolumn*\latim{
Quóniam Deus magnus Dóminus: * et Rex magnus super omnes deos.
}\switchcolumn\portugues{
Porque o Senhor é o Deus grande: * e o Rei grande sobre todos os deuses.
}\switchcolumn*\latim{
Quia in manu ejus sunt omnes fines terræ: * et altitúdines móntium ipsíus sunt.
}\switchcolumn\portugues{
Pois na sua mão estão todos os confins da terra: * e as alturas dos montes são suas.
}\switchcolumn*\latim{
Dominus tecum.
}\switchcolumn\portugues{
O Senhor é convosco.
}\switchcolumn*\latim{
Quóniam ipsíus est mare, et ipse fecit illud: * et siccam manus ejus formavérunt.
}\switchcolumn\portugues{
Seu é o mar e Ele o fez: * e as suas mãos formaram a terra árida.
}\switchcolumn*\latim{
Veníte, adorémus, et procidámus, * et plorémus ante Dóminum qui fecit nos.
}\switchcolumn\portugues{
Vinde, adoremos e prostremo-nos, * e choremos diante do Senhor que nos criou.
}\switchcolumn*\latim{
Quia ipse est Dóminus Deus noster, * et nos pópulus páscuæ ejus, et oves manus ejus.
}\switchcolumn\portugues{
Pois Ele é o Senhor nosso Deus, * e nós somos o povo do seu pasto e as ovelhas da sua manada.
}\switchcolumn*\latim{
Ave Maria, gratia plena, Dominus tecum.
}\switchcolumn\portugues{
Ave, Maria, cheia de graça, o Senhor é convosco.
}\switchcolumn*\latim{
Hódie si vocem ejus audiéritis, * nolíte obduráre corda vestra:
}\switchcolumn\portugues{
Se hoje ouvirdes a sua voz, * não endureceis os vossos corações:
}\switchcolumn*\latim{
Sicut in irritatióne secúndum diem tentatiónis in desérto: * ubi tentavérunt me patres vestri, probavérunt me, et vidérunt ópera mea.
}\switchcolumn\portugues{
Como quando me provocaram à ira, no dia da tentação no deserto: * onde vossos pais me tentaram, me testaram e viram as minhas obras.
}\switchcolumn*\latim{
Dominus tecum.
}\switchcolumn\portugues{
O Senhor é convosco.
}\switchcolumn*\latim{
Quadragínta annis offénsus fui generatióni illi, * et dixi: semper hi errant corde.
}\switchcolumn\portugues{
Quarenta anos estive irritado contra esta geração, * e disse: é um povo de coração errante.
}\switchcolumn*\latim{
Et isti non cognovérunt vias meas, ut jurávi in ira mea: * Si introíbunt in réquiem meam.
}\switchcolumn\portugues{
Eles não conheceram os meus caminhos, pelo que jurei na minha ira: * no meu repouso não entrarão.
}\switchcolumn*\latim{
Ave Maria, gratia plena, Dominus tecum.
}\switchcolumn\portugues{
Ave, Maria, cheia de graça, o Senhor é convosco.
}\switchcolumn*\latim{
Gloria Patri, et Filio, et Spiritui sancto: Sicut erat in principio, et nunc, et semper, et in sæcula sæculorum. Amen.
}\switchcolumn\portugues{
Glória ao Pai, e ao Filho e ao Espírito Santo. Assim como era no princípio, agora e sempre, e por todos os séculos dos séculos. Amen.
}\switchcolumn*\latim{
Dominus tecum.
}\switchcolumn\portugues{
O Senhor é convosco.
}\switchcolumn*\latim{
Ave Maria, gratia plena, Dominus tecum.
}\switchcolumn\portugues{
Ave, Maria, cheia de graça, o Senhor é convosco.
}\end{paracol}

\paragraphinfo{Hino Quem terra}{Página \pageref{quemterra}}

\emph{Os três Salmos seguintes, com suas Antífonas, dizem-se no Domingo, Segunda-feira e Quinta-feira}

\subsubsection{Primeiro Nocturno}
\begin{paracol}{2}\latim{
\emph{Ant.} Benedicta tu in mulieribus, et benedictus fructus ventris tui.
}\switchcolumn\portugues{
\emph{Ant.} Bendita sois vós entre as mulheres, e bendito é o fruto do vosso ventre.
}\end{paracol}

\paragraphinfo{Salmo 8}{Página \pageref{salmo8}}

\begin{paracol}{2}\latim{
\emph{Ant.} Benedicta tu in mulieribus, et benedictus fructus ventris tui.
}\switchcolumn\portugues{
\emph{Ant.} Bendita sois vós entre as mulheres, e bendito é o fruto do vosso ventre.
}\end{paracol}

\begin{paracol}{2}\latim{
\emph{Ant.} Sicut myrrha electa, odorem dedisti suavitatis, sancta Dei Genitrix.
}\switchcolumn\portugues{
\emph{Ant.} Como a preciosa mirra, exalastes suavíssima fragrância, ó santa Mãe de Deus.
}\end{paracol}

\paragraphinfo{Salmo 18}{Página \pageref{salmo18}}

\begin{paracol}{2}\latim{
\emph{Ant.} Sicut myrrha electa, odorem dedisti suavitatis, sancta Dei Genitrix.
}\switchcolumn\portugues{
\emph{Ant.} Como a preciosa mirra, exalastes suavíssima fragrância, ó santa Mãe de Deus.
}\end{paracol}

\begin{paracol}{2}\latim{
\emph{Ant.} Ante torum hujus Virginis frequentate nobis dulcia cantica dramatis.
}\switchcolumn\portugues{
\emph{Ant.} Multiplicai-nos doces cânticos ante o precioso leito desta Virgem.
}\end{paracol}

\paragraphinfo{Salmo 23}{Página \pageref{salmo23}}

\begin{paracol}{2}\latim{
\emph{Ant.} Ante torum hujus Virginis frequentate nobis dulcia cantica dramatis.
}\switchcolumn\portugues{
\emph{Ant.} Multiplicai-nos doces cânticos ante o precioso leito desta Virgem.
}\end{paracol}

\emph{Absolvições, Lições e Responsórios, como no fim do terceiro Nocturno.}

\emph{Os três Salmos seguintes, com suas Antífonas, dizem-se na Terça-Feira e Sexta-feira}

\subsubsection{Segundo Nocturno}
\begin{paracol}{2}\latim{
\emph{Ant.} Specie tua et pulchritudine tua intende, prospere procede, et regna.
}\switchcolumn\portugues{
\emph{Ant.} Ornada de glória e de formosura, caminhai prosperamente e reinai.
}\end{paracol}

\paragraphinfo{Salmo 44}{Página \pageref{salmo44}}

\begin{paracol}{2}\latim{
\emph{Ant.} Specie tua et pulchritudine tua intende, prospere procede, et regna.
}\switchcolumn\portugues{
\emph{Ant.} Ornada de glória e de formosura, caminhai prosperamente e reinai.
}\end{paracol}

\begin{paracol}{2}\latim{
\emph{Ant.} Adjuvabit eam Deus vultu suo: Deus in medio ejus, non commovebitur.
}\switchcolumn\portugues{
\emph{Ant.} Ajudou-a Deus com seu favorável aspecto; e como Deus assiste no meio dela, não se verá perturbada.
}\end{paracol}

\paragraphinfo{Salmo 45}{Página \pageref{salmo45}}

\begin{paracol}{2}\latim{
\emph{Ant.} Adjuvabit eam Deus vultu suo: Deus in medio ejus, non commovebitur.
}\switchcolumn\portugues{
\emph{Ant.} Ajudou-a Deus com seu favorável aspecto; e como Deus assiste no meio dela, não se verá perturbada.
}\end{paracol}

\begin{paracol}{2}\latim{
\emph{Ant.} Sicut lætantium omnium nostrum habitatio est in te, sancta Dei Genitrix.
}\switchcolumn\portugues{
\emph{Ant.} Santa Mãe de Deus, todos nossos que por amor habitam convosco estão cheios de alegria.
}\end{paracol}

\paragraphinfo{Salmo 86}{Página \pageref{salmo86}}

\begin{paracol}{2}\latim{
\emph{Ant.} Sicut lætantium omnium nostrum habitatio est in te, sancta Dei Genitrix.
}\switchcolumn\portugues{
\emph{Ant.} Santa Mãe de Deus, todos nossos que por amor habitam convosco estão cheios de alegria.
}\end{paracol}

\emph{Absolvições, Lições e Responsórios, como no fim do terceiro Nocturno.}

\emph{Os três Salmos seguintes, com suas Antífonas, dizem-se na Quarta-Feira e Sábado}

\subsubsection{Terceiro Nocturno}
\begin{paracol}{2}\latim{
\emph{Ant.} Gaude, Maria Virgo: cunctas hæreses sola intermenisti in universo mundo.
}\switchcolumn\portugues{
\emph{Ant.} Alegrai-vos, Virgem Maria: porque só vós haveis destruído todas as heresias em todo o mundo.
}\end{paracol}

\paragraphinfo{Salmo 95}{Página \pageref{salmo95}}

\begin{paracol}{2}\latim{
\emph{Ant.} Gaude, Maria Virgo: cunctas hæreses sola intermenisti in universo mundo.
}\switchcolumn\portugues{
\emph{Ant.} Alegrai-vos, Virgem Maria: porque só vós haveis destruído todas as heresias em todo o mundo.
}\end{paracol}

\begin{paracol}{2}\latim{
\emph{Ant.} Dignare me laudare te, Virgo sacrata: da mihi virtutem contra hostes tuos.
}\switchcolumn\portugues{
\emph{Ant.} Dignai-vos, sagrada Virgem, de que eu vos louve; dai-me esforço contra vossos inimigos.
}\end{paracol}

\paragraphinfo{Salmo 96}{Página \pageref{salmo96}}

\begin{paracol}{2}\latim{
\emph{Ant.} Dignare me laudare te, Virgo sacrata: da mihi virtutem contra hostes tuos.
}\switchcolumn\portugues{
\emph{Ant.} Dignai-vos, sagrada Virgem, de que eu vos louve; dai-me esforço contra vossos inimigos.
}\end{paracol}

\begin{paracol}{2}\latim{
\emph{Ant.} Post partum virgo inviolata permansisti: Dei Genitrix, intercede pro nobis.
}\switchcolumn\portugues{
\emph{Ant.} Depois do parto permanecestes virgem imaculada; Mãe de Deus, intercedei por nós.
}\end{paracol}

\paragraphinfo{Salmo 97}{Página \pageref{salmo97}}

\begin{paracol}{2}\latim{
\emph{Ant.} Post partum virgo inviolata permansisti: Dei Genitrix, intercede pro nobis.
}\switchcolumn\portugues{
\emph{Ant.} Depois do parto permanecestes virgem imaculada; Mãe de Deus, intercedei por nós.
}\end{paracol}

\paragraph{Versículo}
\begin{paracol}{2}\latim{
℣. Diffusa est gratia in labiis tuis.
}\switchcolumn\portugues{
℣. Estão cheios de graça vossos lábios.
}\switchcolumn*\latim{
℟. Propterea benedixit te Deum in æternum.
}\switchcolumn\portugues{
℟. Por isso Deus vos abençoou para sempre.
}\switchcolumn*\latim{
Pater Noster (secreto usque ad).
}\switchcolumn\portugues{
Pai Nosso (em silêncio).
}\switchcolumn*\latim{
℣. Et ne nos inducas in tentationem.
}\switchcolumn\portugues{
℣. E nos não deixeis cair em tentação.
}\switchcolumn*\latim{
℟. Sed libera nos a malo.
}\switchcolumn\portugues{
℟. Mas livrai-nos do mal.
}\end{paracol}

\paragraph{Absolvição}
\begin{paracol}{2}\latim{
\rlettrine{P}{recibus} et meritis beatæ Mariæ semper Virginis, et omnium Sanctorum, perducat nos Dominus ad regna cælorum.
}\switchcolumn\portugues{
\rlettrine{P}{elos} rogos e merecimentos da bem-aventurada Virgem Maria, e de todos os Santos, nos conduza o Senhor ao reino dos céus.
}\switchcolumn*\latim{
℟. Amen.
}\switchcolumn\portugues{
℟. Amen.
}\switchcolumn*\latim{
℣. Jube, Domine, benedicere.
}\switchcolumn\portugues{
℣. Dai-me, Senhor, a vossa bênção.
}\end{paracol}

\paragraph{Benção}
\begin{paracol}{2}\latim{
Nos cum prole pia benedicat Virgo Maria.
}\switchcolumn\portugues{
Nos abençoe a Virgem Maria com seu piíssimo Filho.
}\switchcolumn*\latim{
℟. Amen.
}\switchcolumn\portugues{
℟. Amen.
}\end{paracol}

\paragraphinfo{Lição 1}{Ecl. 24, 11-13}
\begin{paracol}{2}\latim{
\rlettrine{I}{n} omnibus requiem quæsivi, et in hereditate Domini morabor. Tunc præcepit, et dixit mihi Creator omnium: et qui creavit me, requievit in tabernaculo meo. Et dixit mihi: In Jacob inhabita, et in Israël hereditare, et in electis meis mitte radices.
}\switchcolumn\portugues{
\rlettrine{E}{m} todas as cousas procurei descanso, e na herança do Senhor farei morada. Então ordenou, e me disse o Criador de tudo; e O que me criou descansou no meu Tabernáculo, e disse-me: Tem a tua morada em Jacob, e a tua herança em Israel, e nos meus escolhidos lança raízes.
}\switchcolumn*\latim{
℣. Tu autem, Dómine, miserére nobis.
}\switchcolumn\portugues{
℣. E Vós, Senhor, tende misericórdia de nós.
}\switchcolumn*\latim{
℟. Deo grátias.
}\switchcolumn\portugues{
℟. Graças a Deus.
}\switchcolumn*\latim{
℟. Sancta et immaculáta virginitas, quibus te laudibus efferam nescio: Quia quem cæli cápere non póterant, tuo gremio contulísti.
}\switchcolumn\portugues{
℟. Santa e imaculada Virgindade, não sei com que louvores possa exaltar-vos. Porque encerrastes no vosso seio Aquele a quem os céus não podiam abranger.
}\switchcolumn*\latim{
℣. Benedicta tu in muliéribus, et benedíctus fructus ventris tui.
}\switchcolumn\portugues{
℣. Bendita sois vós entre as mulheres e bendito é o fruto do vosso ventre.
}\switchcolumn*\latim{
℟. Quia quem cæli cápere non póterant, tuo gremio contulisti.
}\switchcolumn\portugues{
℟. Porque encerrastes no vosso seio Aquele a quem os céus não podiam abranger.
}\switchcolumn*\latim{
℣. Iube domne benedicere.
}\switchcolumn\portugues{
℣. Dai-me, Senhor, a vossa bênção.
}\switchcolumn*\latim{
Ipsa Virgo Vírginum intercédat pro nobis ad Dóminum.
}\switchcolumn\portugues{
A mesma Virgem das virgens interceda por nós ao Senhor.
}\switchcolumn*\latim{
℟. Amen.
}\switchcolumn\portugues{
℟. Amen.
}\end{paracol}

\paragraphinfo{Lição 2}{Ecl. 24, 15-16}
\begin{paracol}{2}\latim{
\rlettrine{E}{t} sic in Sion firmata sum, et in civitate sanctificata similiter requievi, et in Ierúsalem potestas mea. Et radicavi in populo honorificato, et in parte Dei mei hereditas illius, et in plenitudine sanctorum detentio mea.
}\switchcolumn\portugues{
\rlettrine{E}{} desta maneira estou fundada em Sião, e semelhantemente repousei na cidade santificada; e em Jerusalém é o meu poder. E lancei raízes no povo honorificado, e na parte do meu Deus, herança sua, e na congregação dos santos fiz a minha morada.
}\switchcolumn*\latim{
℣. Tu autem, Dómine, miserére nobis.
}\switchcolumn\portugues{
℣. E Vós, Senhor, tende misericórdia de nós.
}\switchcolumn*\latim{
℟. Deo grátias.
}\switchcolumn\portugues{
℟. Demos graças a Deus.
}\switchcolumn*\latim{
℟. Beata es, Virgo Maria, quæ Dominum portasti, Creatorem mundi: Genuisti qui te fecit, et in æternum permanes Virgo.
}\switchcolumn\portugues{
℟. Bem-aventurada sois, ó Virgem Maria, que trouxestes no vosso ventre o Criador do mundo. Gerastes o que vos deu o ser, e ficastes para sempre Virgem.
}\switchcolumn*\latim{
℣. Ave Maria, gratia plena, Dominus tecum.
}\switchcolumn\portugues{
℣. Ave Maria, cheia de graça, o Senhor é convosco.
}\switchcolumn*\latim{
℟. Genuisti qui te fecit, et in æternum permanes Virgo.
}\switchcolumn\portugues{
℟. Gerastes O que vos deu o ser, e ficastes para sempre Virgem.
}\end{paracol}

\emph{Quando o Te Deum é dito depois da Terceira Lição, adiciona-se o seguinte no fim do Responsório:}

\begin{paracol}{2}\latim{
℣. Glória Patri, et Fílio, et Spirítui Sancto.
}\switchcolumn\portugues{
℣. Glória ao Pai e ao Filho e ao Espírito Santo.
}\switchcolumn*\latim{
℟. Genuisti qui te fecit, et in æternum permanes Virgo.
}\switchcolumn\portugues{
℟. Gerastes O que vos deu o ser, e ficastes para sempre Virgem.
}\switchcolumn*\latim{
℣. Iube domne benedicere.
}\switchcolumn\portugues{
℣. Dai-me, Senhor, a vossa bênção.
}\end{paracol}

\paragraph{Benção}
\begin{paracol}{2}\latim{
Per Vírginem matrem concédat nobis Dóminus salútem et pacem.
}\switchcolumn\portugues{
Pela Virgem Maria, nos conceda o Senhor a paz e a salvação.
}\switchcolumn*\latim{
℟. Amen.
}\switchcolumn\portugues{
℟. Amen.
}\end{paracol}

\paragraphinfo{Lição 3}{Ecl. 24, 17-20}
\begin{paracol}{2}\latim{
\qlettrine{Q}{uasi} cedrus exaltata sum in Libano, et quasi cypressus in monte Sion: Quasi palma exaltata sum in Cades, et quasi plantatio rosæ in Iericho: Quasi oliva speciosa in campis, et quasi platanus exaltata sum iuxta aquam in plateis. Sicut cinnamomum et balsamum aromatizans odorem dedi; quasi myrrha electa dedi suavitatem odoris:
}\switchcolumn\portugues{
\rlettrine{E}{xaltada} sou, qual cedro no Líbano, e qual cipreste no monte Sião. Exaltada sou, qual palma em Cades e como as rosas em Jericó. Qual especial oliveira nos campos, e qual plátano, sou exaltada junto da água nas praças. Assim como o cinamomo e o bálsamo, que difundem cheiro, dei eu fragrância; como a mirra, dei cheiro de suavidade.
}\switchcolumn*\latim{
℣. Tu autem, Dómine, miserére nobis.
}\switchcolumn\portugues{
℣. E Vós, Senhor, tende misericórdia de nós.
}\switchcolumn*\latim{
℟. Deo grátias.
}\switchcolumn\portugues{
℟. Demos graças a Deus.
}\end{paracol}

\emph{O Te Deum não é dito no Advento, ou da Septuagésima até à Páscoa, excluindo as Festas de Nossa Senhora. O seguinte Responsório é dito quando o Te Deum é omitido:}

\begin{paracol}{2}\latim{
℟. Felix namque es, sacra Virgo Maria, et omni laude dignissima: Quia ex te ortus est sol justitiæ, Christus Deus noster.
}\switchcolumn\portugues{
℟. Ditosa sois, ó sagrada Virgem Maria, e digníssima de todo o louvor. Porque de vós nasceu o sol de justiça, Jesus Cristo nosso Deus.
}\switchcolumn*\latim{
℣. Ora pro populo, interveni pro clero, intercede pro devoto femineo sexu: sentiant omnes tuum juvamen, quicumque celebrant tuam sanctam commemorationem.
}\switchcolumn\portugues{
℣. Rogai pelo povo, intercedei pelo clero, advogai pelo devoto sexo feminino; experimentem o vosso patrocínio os que celebram a vossa santa memória. Porque de vós nasceu o Sol de justiça, Jesus Cristo, nosso Deus.
}\switchcolumn*\latim{
℟. Quia ex te ortus est sol justitiæ.
}\switchcolumn\portugues{
℟. Porque de ti nasceu o Sol de justiça.
}\switchcolumn*\latim{
℣. Glória Patri, et Fílio, et Spirítui Sancto.
}\switchcolumn\portugues{
℣. Glória ao Pai e ao Filho e ao Espírito Santo.
}\switchcolumn*\latim{
℟. Christus Deus noster.
}\switchcolumn\portugues{
℟. Jesus Cristo, nosso Deus.
}\end{paracol}

\paragraphinfo{Te Deum}{Página \pageref{tedeum}}

\emph{As Matinas acabam depois do Terceiro Responsório ou do Te Deum, porque é usual depois das Matinas passar-se directamente para as Laudes. No entanto, se não continuar para as Laudes diz:}

\begin{paracol}{2}\latim{
℣. Domine, exaudi orationem meam.
}\switchcolumn\portugues{
℣. Ouvi, Senhor, a minha oração.
}\switchcolumn*\latim{
℟. Et clamor meus ad te veniat.
}\switchcolumn\portugues{
℟. E o meu clamor chegue até Vós.
}\switchcolumn*\latim{
\begin{nscenter} Orémus. \end{nscenter}
}\switchcolumn\portugues{
\begin{nscenter} Oremos. \end{nscenter}
}\switchcolumn*\latim{
\rlettrine{C}{oncede} nos famulos tuos, quǽsumus, Domine Deus, perpetua mentis et corporis sanitate gaudere: et gloriosa beatæ Mariæ semper Virginis intercessione, a præsenti liberari tristitia, et æterna perfrui lætitia. Per Dominum nostrum Jesum Christum.
}\switchcolumn\portugues{
\rlettrine{S}{enhor} Deus, nós Vos suplicamos que concedais a vossos servos lograr uma perpétua saúde de corpo e alma, e que pela intercessão gloriosa da bem-aventurada sempre Virgem Maria sejamos livres da presente tristeza, e gozemos da eterna alegria. Por Jesus Cristo nosso Senhor.
}\switchcolumn*\latim{
℟. Amen.
}\switchcolumn\portugues{
℟. Amen.
}\switchcolumn*\latim{
℣. Domine, exaudi orationem meam.
}\switchcolumn\portugues{
℣. Ouvi, Senhor, a minha oração.
}\switchcolumn*\latim{
℟. Et clamor meus ad te veniat.
}\switchcolumn\portugues{
℟. E o meu clamor chegue até Vós.
}\switchcolumn*\latim{
℣. Benedicamus Domino.
}\switchcolumn\portugues{
℣. Bendigamos o Senhor.
}\switchcolumn*\latim{
℟. Deo gratias.
}\switchcolumn\portugues{
℟. Graças a Deus.
}\switchcolumn*\latim{
℣. Fidelium animæ per misericordiam Dei, requiescant in pace.
}\switchcolumn\portugues{
℣. E que as almas dos fiéis, pela misericórdia de Deus, descansem em paz.
}\switchcolumn*\latim{
℟. Amen.
}\switchcolumn\portugues{
℟. Amen.
}\end{paracol}


\subsection{Laudes 1}

\begin{paracol}{2}\latim{
℣. Deus \cruz in adjutórium meum inténde.
}\switchcolumn\portugues{
℣. Deus, \cruz vinde em meu auxílio.
}\switchcolumn*\latim{
℟. Dómine, ad adjuvándum me festína.
}\switchcolumn\portugues{
℟. Senhor, apressai-Vos em socorrer-me.
}\switchcolumn*\latim{
Glória Patri, \&c.
}\switchcolumn\portugues{
Glória ao Pai, \&c.
}\switchcolumn*\latim{
\emph{Ant.} Assumpta est Maria in cælum: gaudete angeli, laudantes benedicunt Dominum.
}\switchcolumn\portugues{
\emph{Ant.} Maria foi exaltada ao céu; os anjos se alegram, louvam, e glorificam o Senhor.
}\end{paracol}

\paragraphinfo{Salmo 92}{Página \pageref{salmo92}}

\begin{paracol}{2}\latim{
\emph{Ant.} Assumpta est Maria in cælum: gaudete angeli, laudantes benedicunt Dominum.
}\switchcolumn\portugues{
\emph{Ant.} Maria foi exaltada ao céu; os anjos se alegram, louvam, e glorificam o Senhor.
}\end{paracol}

\begin{paracol}{2}\latim{
\emph{Ant.} Maria Virgo assumpta est ad ætherum thalamum, in quo Rex regum stellato sedet solio.
}\switchcolumn\portugues{
\emph{Ant.} Maria Virgem foi sublimada ao tálamo celeste, onde o Rei dos reis está sentado num trono de estrelas.
}\end{paracol}

\paragraphinfo{Salmo 99}{Página \pageref{salmo99}}

\begin{paracol}{2}\latim{
\emph{Ant.} Maria Virgo assumpta est ad ætherum thalamum, in quo Rex regum stellato sedet solio.
}\switchcolumn\portugues{
\emph{Ant.} Maria Virgem foi sublimada ao tálamo celeste, onde o Rei dos reis está sentado num trono de estrelas.
}\end{paracol}

\begin{paracol}{2}\latim{
\emph{Ant.} In odorem unguentorum tuorum currimus: adolescentulæ dilexerunt te nimis.
}\switchcolumn\portugues{
\emph{Ant.} Todos corremos à fragrância dos vossos aromas, as donzelas vos amarão grandemente.
}\end{paracol}

\paragraphinfo{Salmo 62}{Página \pageref{salmo62}}

\begin{paracol}{2}\latim{
\emph{Ant.} In odorem unguentorum tuorum currimus: adolescentulæ dilexerunt te nimis.
}\switchcolumn\portugues{
\emph{Ant.} Todos corremos à fragrância dos vossos aromas, as donzelas vos amarão grandemente.
}\end{paracol}

\begin{paracol}{2}\latim{
\emph{Ant.} Benedicta filia tu a Domino: quia per te fructum vitaæ communicavimus.
}\switchcolumn\portugues{
\emph{Ant.} Sois filha bendita do Senhor, porque por vós recebemos o fruto da vida.
}\end{paracol}

\paragraphinfo{Benedicite}{Página \pageref{benedicite}}
\emph{Não se diz a Glória no fim.}

\begin{paracol}{2}\latim{
\emph{Ant.} Benedicta filia tu a Domino: quia per te fructum vitaæ communicavimus.
}\switchcolumn\portugues{
\emph{Ant.} Sois filha bendita do Senhor, porque por vós recebemos o fruto da vida.
}\end{paracol}

\begin{paracol}{2}\latim{
\emph{Ant.} Pulchra es et decora, filia Jerusalem: terribus ut castrorum acies ordinara.
}\switchcolumn\portugues{
\emph{Ant.} Filha de Jerusalém, sois bela e decorosa, terrível como um exército formado em linha.
}\end{paracol}

\paragraphinfo{Salmo 148}{Página \pageref{salmo148}}

\begin{paracol}{2}\latim{
\emph{Ant.} Pulchra es et decora, filia Jerusalem: terribus ut castrorum acies ordinara.
}\switchcolumn\portugues{
\emph{Ant.} Filha de Jerusalém, sois bela e decorosa, terrível como um exército formado em linha.
}\end{paracol}

\paragraphinfo{Pequeno Capítulo}{Ct. 6, 8}
\begin{paracol}{2}\latim{
\rlettrine{V}{iderunt} eam filiæ Sion, et beatissimam prædicaverunt, et reginæ laudaverunt eam.
}\switchcolumn\portugues{
\rlettrine{V}{iram-na} as Filhas de Sião, e a declararam beatíssima; e as Rainhas a louvaram.
}\switchcolumn*\latim{
℟. Deo grátias.
}\switchcolumn\portugues{
℟. Graças a Deus.
}\end{paracol}

\paragraphinfo{Hino O Gloriosa Virginum}{Página \pageref{ogloriosavirginum}}

\paragraph{Ofício 1}
\begin{paracol}{2}\latim{
\emph{Ant.} Beata dei genitrix, Maria, Virgo perpetua, templum Domini, sacrarium Spiritus Sancti, sola sine exemplo placuisti Domino nostro Jesu Christo: ora pro populo, interveni pro clero, intercede pro devoto femineo sexu.
}\switchcolumn\portugues{
\emph{Ant.} Ó Santa Mãe de Deus, Maria sempre Virgem, templo de Deus, sacrário do Espírito Santo; vós apenas, sem exemplo, agradastes Nosso Senhor Jesus Cristo: rezei por nós, intervinde pelo clero, intercedei pelo devoto sexo feminino.
}\end{paracol}

\paragraph{No Tempo Pascal}
\begin{paracol}{2}\latim{
\emph{Ant.} Regína Cæli, lætáre, allelúja; Quia quem meruísti portáre, allelúja; Resurréxit, sicut dixit, allelúja; Ora pro nóbis Deum, allelúja.
}\switchcolumn\portugues{
\emph{Ant.} Rainha do Céu, alegrai-Vos, Aleluia! Porque Aquele que merecestes trazer em vosso ventre, Aleluia! Ressuscitou como disse, Aleluia! Rogai por nós a Deus, Aleluia!
}\end{paracol}

\paragraphinfo{Benedictus}{Página \pageref{benedictus}}

\paragraph{Ofício 1}
\begin{paracol}{2}\latim{
\emph{Ant.} Beata dei genitrix, Maria, Virgo perpetua, templum Domini, sacrarium Spiritus Sancti, sola sine exemplo placuisti Domino nostro Jesu Christo: ora pro populo, interveni pro clero, intercede pro devoto femineo sexu.
}\switchcolumn\portugues{
\emph{Ant.} Ó Santa Mãe de Deus, Maria sempre Virgem, templo de Deus, sacrário do Espírito Santo; vós apenas, sem exemplo, agradastes Nosso Senhor Jesus Cristo: reza por nós, intervém pelo clero, intercede pelo devoto sexo feminino.
}\end{paracol}

\paragraph{No Tempo Pascal}
\begin{paracol}{2}\latim{
\emph{Ant.} Regína Cæli, lætáre, allelúja; Quia quem meruísti portáre, allelúja; Resurréxit, sicut dixit, allelúja; Ora pro nóbis Deum, allelúja.
}\switchcolumn\portugues{
\emph{Ant.} Rainha do Céu, alegrai-Vos, Aleluia! Porque Aquele que merecestes trazer em vosso ventre, Aleluia! Ressuscitou como disse, Aleluia! Rogai por nós a Deus, Aleluia!
}\switchcolumn*\latim{
\begin{nscenter} Orémus. \end{nscenter}
}\switchcolumn\portugues{
\begin{nscenter} Oremos. \end{nscenter}
}\switchcolumn*\latim{
\rlettrine{D}{eus,} qui de beatæ Mariæ Virginis utero Verbum tuum, Angelo nuntiante, carnem suscipere voluisti: præsta supplicibus tuis; ut qui vere eam Genetricem Dei credimus, ejus apud te intercessionibus adjuvemur. Per eundem Dominum nostrum Jesum Christum.
}\switchcolumn\portugues{
\slettrine{Ó}{} Deus, que pela anunciação do Anjo quisestes que o vosso Verbo se vestisse da nossa carne nas entranhas da bem-aventurada Virgem Maria: nós, vossos humildes servos, cremos ser ela verdadeira a Mãe de Deus, concedei-nos que nos ajudem as suas intercessões para convosco. Pelo mesmo Jesus Cristo Senhor Nosso.
}\switchcolumn*\latim{
℟. Amen.
}\switchcolumn\portugues{
℟. Amen.
}\switchcolumn*\latim{
℣. Domine, exaudi orationem meam.
}\switchcolumn\portugues{
℣. Ouvi, Senhor, a minha oração.
}\switchcolumn*\latim{
℟. Et clamor meus ad te veniat.
}\switchcolumn\portugues{
℟. E o meu clamor chegue até Vós.
}\switchcolumn*\latim{
℣. Benedicamus Domino.
}\switchcolumn\portugues{
℣. Bendigamos o Senhor.
}\switchcolumn*\latim{
℟. Deo gratias.
}\switchcolumn\portugues{
℟. Graças a Deus.
}\switchcolumn*\latim{
℣. Fidelium animæ per misericordiam Dei, requiescant in pace.
}\switchcolumn\portugues{
℣. E que as almas dos fiéis, pela misericórdia de Deus, descansem em paz.
}\switchcolumn*\latim{
℟. Amen.
}\switchcolumn\portugues{
℟. Amen.
}\end{paracol}

\emph{Acabar com uma Antífona de Nossa Senhora na página \pageref{antifonasnossasenhora}.}


\subsection{Prima 1}

\begin{paracol}{2}\latim{
℣. Deus \cruz in adjutórium meum inténde.
}\switchcolumn\portugues{
℣. Deus, \cruz vinde em meu auxílio.
}\switchcolumn*\latim{
℟. Dómine, ad adjuvándum me festína.
}\switchcolumn\portugues{
℟. Senhor, apressai-Vos em socorrer-me.
}\switchcolumn*\latim{
Glória Patri, \&c.
}\switchcolumn\portugues{
Glória ao Pai, \&c.
}\end{paracol}

\paragraphinfo{Hino Memento rerum conditor}{Página \pageref{mementorerumconditor}}

\begin{paracol}{2}\latim{
\emph{Ant.} Assumpta est Maria in cælum: gaudete angeli, laudantes benedicunt Dominum.
}\switchcolumn\portugues{
\emph{Ant.} Maria foi exaltada ao céu; os anjos se alegram, louvam, e glorificam o Senhor.
}\end{paracol}

\paragraphinfo{Salmo 53}{Página \pageref{salmo53}}

\paragraphinfo{Salmo 84}{Página \pageref{salmo84}}

\paragraphinfo{Salmo 116}{Página \pageref{salmo116}}

\begin{paracol}{2}\latim{
\emph{Ant.} Assumpta est Maria in cælum: gaudete angeli, laudantes benedicunt Dominum.
}\switchcolumn\portugues{
\emph{Ant.} Maria foi exaltada ao céu; os anjos se alegram, louvam, e glorificam o Senhor.
}\end{paracol}

\paragraphinfo{Pequeno Capítulo}{Ct. 6, 9}
\begin{paracol}{2}\latim{
\qlettrine{Q}{uæ} est ista, quæ progréditur quasi auróra consúrgens, pulchra ut luna, elécta ut sol, terribilis ut castrorum acies ordinata?
}\switchcolumn\portugues{
\qlettrine{Q}{uem} é esta que aparece como a aurora quando desponta, formosa como a lua, eleita, como o sol, terrível como um exército formado em linha?
}\switchcolumn*\latim{
℟. Deo grátias.
}\switchcolumn\portugues{
℟. Graças a Deus.
}\switchcolumn*\latim{
℣. Dignare me laudare te, Virgo sacrata.
}\switchcolumn\portugues{
℣. Dignai-vos, sagrada Virgem, de que eu vos louve.
}\switchcolumn*\latim{
℟. Da mihi virtutem contra hostes tuos.
}\switchcolumn\portugues{
℟. Dai-me esforço contra vossos inimigos.
}\end{paracol}


\begin{paracol}{2}\latim{
\emph{(Hic genuflectitur)} Kyrie eleison
}\switchcolumn\portugues{
\emph{(Genuflectir)} Senhor, tende piedade de nós.
}\switchcolumn*\latim{
Christe, eléison.
}\switchcolumn\portugues{
Cristo, tende piedade de nós.
}\switchcolumn*\latim{
Kyrie, eléison.
}\switchcolumn\portugues{
Senhor, tende piedade de nós.
}\switchcolumn*\latim{
℣. Domine, exaudi orationem meam.
}\switchcolumn\portugues{
℣. Ouvi, Senhor, a minha oração.
}\switchcolumn*\latim{
℟. Et clamor meus ad te veniat.
}\switchcolumn\portugues{
℟. E o meu clamor chegue até Vós.
}\end{paracol}

\begin{paracol}{2}\latim{
\begin{nscenter} Orémus. \end{nscenter}
}\switchcolumn\portugues{
\begin{nscenter} Oremos. \end{nscenter}
}\switchcolumn*\latim{
\rlettrine{D}{eus,} qui virginalem aulam beatae Mariae in qua habitares, eligere dignatus es: da, quaesumus, ut sua nos defensione munitos; jucundos facias suae interesse commemorationi. Qui vivis et regnas \emph{\&c.}
}\switchcolumn\portugues{
\slettrine{Ó}{} Deus, que Vos dignastes eleger puríssimas entranhas da bem-aventurada Virgem Maria para vossa morada: concedei-nos que com o presente culto, que alegres lhe tributamos, nos façamos beneméritos do seu patrocínio. Vós que viveis e reinais \emph{\&c.}
}\switchcolumn*\latim{
℟. Amen.
}\switchcolumn\portugues{
℟. Amen.
}\switchcolumn*\latim{
℣. Domine, exaudi orationem meam.
}\switchcolumn\portugues{
℣. Ouvi, Senhor, a minha oração.
}\switchcolumn*\latim{
℟. Et clamor meus ad te veniat.
}\switchcolumn\portugues{
℟. E o meu clamor chegue até Vós.
}\switchcolumn*\latim{
℣. Benedicamus Domino.
}\switchcolumn\portugues{
℣. Bendigamos o Senhor.
}\switchcolumn*\latim{
℟. Deo gratias.
}\switchcolumn\portugues{
℟. Graças a Deus.
}\switchcolumn*\latim{
℣. Fidelium animæ per misericordiam Dei, requiescant in pace.
}\switchcolumn\portugues{
℣. E que as almas dos fiéis, pela misericórdia de Deus, descansem em paz.
}\switchcolumn*\latim{
℟. Amen.
}\switchcolumn\portugues{
℟. Amen.
}\end{paracol}


\subsection{Terça 1}

\begin{paracol}{2}\latim{
℣. Deus \cruz in adjutórium meum inténde.
}\switchcolumn\portugues{
℣. Deus, \cruz vinde em meu auxílio.
}\switchcolumn*\latim{
℟. Dómine, ad adjuvándum me festína.
}\switchcolumn\portugues{
℟. Senhor, apressai-Vos em socorrer-me.
}\switchcolumn*\latim{
Glória Patri, \&c.
}\switchcolumn\portugues{
Glória ao Pai, \&c.
}\end{paracol}

\paragraphinfo{Hino Memento rerum conditor}{Página \pageref{mementorerumconditor}}

\begin{paracol}{2}\latim{
\emph{Ant.} Maria Virgo assumpta est ad ætherum thalamum, in quo Rex regum stellato sedet solio.
}\switchcolumn\portugues{
\emph{Ant.} Maria Virgem foi sublimada ao tálamo celeste, onde o Rei dos reis está sentado num trono de estrelas.
}\end{paracol}

\paragraphinfo{Salmo 119}{Página \pageref{salmo119}}

\paragraphinfo{Salmo 120}{Página \pageref{salmo120}}

\paragraphinfo{Salmo 121}{Página \pageref{salmo121}}

\begin{paracol}{2}\latim{
\emph{Ant.} Maria Virgo assumpta est ad ætherum thalamum, in quo Rex regum stellato sedet solio.
}\switchcolumn\portugues{
\emph{Ant.} Maria Virgem foi sublimada ao tálamo celeste, onde o Rei dos reis está sentado num trono de estrelas.
}\end{paracol}

\paragraphinfo{Pequeno Capítulo}{Ecl. 24, 15}
\begin{paracol}{2}\latim{
\rlettrine{E}{t} sic in Sion firmata sum, et in civitate sanctificata similiter requievi, et in Jerúsalem potestas mea.
}\switchcolumn\portugues{
\rlettrine{E}{} desta maneira estou fundada em Sião, e semelhantemente repousei na cidade santificada, e em Jerúsalem é o meu poder.
}\switchcolumn*\latim{
℟. Deo grátias.
}\switchcolumn\portugues{
℟. Graças a Deus.
}\switchcolumn*\latim{
℣. Diffusa est gratia in labiis tuis.
}\switchcolumn\portugues{
℣. A graça derramou-se nos vossos lábios.
}\switchcolumn*\latim{
℟. Propterea benedixit te Deus in æternum.
}\switchcolumn\portugues{
℟. Por isso vos abençoou Deus para sempre.
}\end{paracol}


\begin{paracol}{2}\latim{
\emph{(Hic genuflectitur)} Kyrie eleison
}\switchcolumn\portugues{
\emph{(Genuflectir)} Senhor, tende piedade de nós.
}\switchcolumn*\latim{
Christe, eléison.
}\switchcolumn\portugues{
Cristo, tende piedade de nós.
}\switchcolumn*\latim{
Kyrie, eléison.
}\switchcolumn\portugues{
Senhor, tende piedade de nós.
}\switchcolumn*\latim{
℣. Domine, exaudi orationem meam.
}\switchcolumn\portugues{
℣. Ouvi, Senhor, a minha oração.
}\switchcolumn*\latim{
℟. Et clamor meus ad te veniat.
}\switchcolumn\portugues{
℟. E o meu clamor chegue até Vós.
}\end{paracol}

\begin{paracol}{2}\latim{
\begin{nscenter} Orémus. \end{nscenter}
}\switchcolumn\portugues{
\begin{nscenter} Oremos. \end{nscenter}
}\switchcolumn*\latim{
\rlettrine{D}{eus,} qui salutis aeternae, beatae Mariae virginitate fecunda, humano generi praemia praestitisti: tribue, quaesumus; ut ipsam pro nobis intercedere sentiamus, per quam meruimus auctorem vitae suscipere, Dominum nostrum Jesum Christum Filium tuum: Qui tecum vivit et regnat \emph{\&c.}
}\switchcolumn\portugues{
\slettrine{Ó}{} Deus, que pela virgindade fecunda da B. Maria, participastes ao género humano os prémios da salvação eterna: concedei-nos, Vos rogamos, que experimentemos quanto é poderosa a nosso favor a intercessão daquela Virgem, pela qual merecemos receber o autor da vida nosso Senhor Jesus Cristo, Filho vosso: que convosco Vive e reina \emph{\&c.}
}\switchcolumn*\latim{
℟. Amen.
}\switchcolumn\portugues{
℟. Amen.
}\switchcolumn*\latim{
℣. Domine, exaudi orationem meam.
}\switchcolumn\portugues{
℣. Ouvi, Senhor, a minha oração.
}\switchcolumn*\latim{
℟. Et clamor meus ad te veniat.
}\switchcolumn\portugues{
℟. E o meu clamor chegue até Vós.
}\switchcolumn*\latim{
℣. Benedicamus Domino.
}\switchcolumn\portugues{
℣. Bendigamos o Senhor.
}\switchcolumn*\latim{
℟. Deo gratias.
}\switchcolumn\portugues{
℟. Graças a Deus.
}\switchcolumn*\latim{
℣. Fidelium animæ per misericordiam Dei, requiescant in pace.
}\switchcolumn\portugues{
℣. E que as almas dos fiéis, pela misericórdia de Deus, descansem em paz.
}\switchcolumn*\latim{
℟. Amen.
}\switchcolumn\portugues{
℟. Amen.
}\end{paracol}


\subsection{Sexta 1}

\begin{paracol}{2}\latim{
℣. Deus \cruz in adjutórium meum inténde.
}\switchcolumn\portugues{
℣. Deus, \cruz vinde em meu auxílio.
}\switchcolumn*\latim{
℟. Dómine, ad adjuvándum me festína.
}\switchcolumn\portugues{
℟. Senhor, apressai-Vos em socorrer-me.
}\switchcolumn*\latim{
Glória Patri, \&c.
}\switchcolumn\portugues{
Glória ao Pai, \&c.
}\end{paracol}

\paragraphinfo{Hino Memento rerum conditor}{Página \pageref{mementorerumconditor}}

\begin{paracol}{2}\latim{
\emph{Ant.} In odorem unguentorum tuorum currimus: adolescentulæ dilexerunt te nimis.
}\switchcolumn\portugues{
\emph{Ant.} Todos corremos à fragrância dos vossos aromas, as donzelas vos amarão grandemente.
}\end{paracol}

\paragraphinfo{Salmo 122}{Página \pageref{salmo122}}

\paragraphinfo{Salmo 123}{Página \pageref{salmo123}}

\paragraphinfo{Salmo 124}{Página \pageref{salmo124}}

\begin{paracol}{2}\latim{
\emph{Ant.} In odorem unguentorum tuorum currimus: adolescentulæ dilexerunt te nimis.
}\switchcolumn\portugues{
\emph{Ant.} Todos corremos à fragrância dos vossos aromas, as donzelas vos amarão grandemente.
}\end{paracol}

\paragraphinfo{Pequeno Capítulo}{Ecl. 24, 16}
\begin{paracol}{2}\latim{
\rlettrine{E}{t} radicavi in populo honorificato, et in parte Dei mei hereditas illius et in plenitudine sanctorum detentio mea.
}\switchcolumn\portugues{
\rlettrine{E}{} lancei raízes no povo honorificado, e na parte de meu Deus, herança sua; e na congregação dos santos fiz a minha morada.
}\switchcolumn*\latim{
℟. Deo grátias.
}\switchcolumn\portugues{
℟. Graças a Deus.
}\switchcolumn*\latim{
℣. Benedicta tu in mulieribus.
}\switchcolumn\portugues{
℣. Bendita sois v´so entre as mulheres.
}\switchcolumn*\latim{
℟. Et benedictus fructus ventris tui.
}\switchcolumn\portugues{
℟. E bendito é o fruto do vosso ventre.
}\end{paracol}


\begin{paracol}{2}\latim{
\emph{(Hic genuflectitur)} Kyrie eleison
}\switchcolumn\portugues{
\emph{(Genuflectir)} Senhor, tende piedade de nós.
}\switchcolumn*\latim{
Christe, eléison.
}\switchcolumn\portugues{
Cristo, tende piedade de nós.
}\switchcolumn*\latim{
Kyrie, eléison.
}\switchcolumn\portugues{
Senhor, tende piedade de nós.
}\switchcolumn*\latim{
℣. Domine, exaudi orationem meam.
}\switchcolumn\portugues{
℣. Ouvi, Senhor, a minha oração.
}\switchcolumn*\latim{
℟. Et clamor meus ad te veniat.
}\switchcolumn\portugues{
℟. E o meu clamor chegue até Vós.
}\end{paracol}

\begin{paracol}{2}\latim{
\begin{nscenter} Orémus. \end{nscenter}
}\switchcolumn\portugues{
\begin{nscenter} Oremos. \end{nscenter}
}\switchcolumn*\latim{
\rlettrine{C}{oncede,} misericors Deus, fragilitati nostrae praesidium: ut qui sanctae Dei Genitricis memoriam agimus, intercessionis ejus auxilio, a nostris iniquitatibus resurgamus. Per eúmdem Dóminum \emph{\&c.}
}\switchcolumn\portugues{
\rlettrine{C}{oncedei,} misericordioso Deus, um esforço grande à nossa fragilidade, para que os que celebramos a memória da santa Mãe de Deus, com o auxílio da sua intercessão, ressuscitemos das nossas iniquidades. Pelo mesmo Senhor \emph{\&c.}
}\switchcolumn*\latim{
℟. Amen.
}\switchcolumn\portugues{
℟. Amen.
}\switchcolumn*\latim{
℣. Domine, exaudi orationem meam.
}\switchcolumn\portugues{
℣. Ouvi, Senhor, a minha oração.
}\switchcolumn*\latim{
℟. Et clamor meus ad te veniat.
}\switchcolumn\portugues{
℟. E o meu clamor chegue até Vós.
}\switchcolumn*\latim{
℣. Benedicamus Domino.
}\switchcolumn\portugues{
℣. Bendigamos o Senhor.
}\switchcolumn*\latim{
℟. Deo gratias.
}\switchcolumn\portugues{
℟. Graças a Deus.
}\switchcolumn*\latim{
℣. Fidelium animæ per misericordiam Dei, requiescant in pace.
}\switchcolumn\portugues{
℣. E que as almas dos fiéis, pela misericórdia de Deus, descansem em paz.
}\switchcolumn*\latim{
℟. Amen.
}\switchcolumn\portugues{
℟. Amen.
}\end{paracol}


\subsection{Noa 1}

\begin{paracol}{2}\latim{
℣. Deus \cruz in adjutórium meum inténde.
}\switchcolumn\portugues{
℣. Deus, \cruz vinde em meu auxílio.
}\switchcolumn*\latim{
℟. Dómine, ad adjuvándum me festína.
}\switchcolumn\portugues{
℟. Senhor, apressai-Vos em socorrer-me.
}\switchcolumn*\latim{
Glória Patri, \&c.
}\switchcolumn\portugues{
Glória ao Pai, \&c.
}\end{paracol}

\paragraphinfo{Hino Memento rerum conditor}{Página \pageref{mementorerumconditor}}

\begin{paracol}{2}\latim{
\emph{Ant.} Pulchra es et decora, filia Jerusalem: terribus ut castrorum acies ordinara.
}\switchcolumn\portugues{
\emph{Ant.} Filha de Jerusalém, sois bela e decorosa, terrível como um exército formado em linha.
}\end{paracol}

\paragraphinfo{Salmo 125}{Página \pageref{salmo125}}

\paragraphinfo{Salmo 126}{Página \pageref{salmo126}}

\paragraphinfo{Salmo 127}{Página \pageref{salmo127}}

\begin{paracol}{2}\latim{
\emph{Ant.} Pulchra es et decora, filia Jerusalem: terribus ut castrorum acies ordinara.
}\switchcolumn\portugues{
\emph{Ant.} Filha de Jerusalém, sois bela e decorosa, terrível como um exército formado em linha.
}\end{paracol}

\paragraphinfo{Pequeno Capítulo}{Ecl. 24, 19-20}
\begin{paracol}{2}\latim{
\rlettrine{S}{icut} cinnamomum et balsamum aromatizans odorem dedi: quasi myrrha electa dedi suavitatem odoris.
}\switchcolumn\portugues{
\rlettrine{A}{ssim} como o cinamomo e o bálsamo, que difundem cheiro, dei eu fragrância: como a mirra escolhida, dei cheiro de suavidade.
}\switchcolumn*\latim{
℟. Deo grátias.
}\switchcolumn\portugues{
℟. Graças a Deus.
}\switchcolumn*\latim{
℣. Post partum, Virgo, invioláta permansísti.
}\switchcolumn\portugues{
℣. Despois do parto permanecestes imaculada.
}\switchcolumn*\latim{
℟. Dei Génetrix, intercéde pro nobis.
}\switchcolumn\portugues{
℟. Intercedei por nós, ó Mãe de Deus.
}\end{paracol}


\begin{paracol}{2}\latim{
\emph{(Hic genuflectitur)} Kyrie eleison
}\switchcolumn\portugues{
\emph{(Genuflectir)} Senhor, tende piedade de nós.
}\switchcolumn*\latim{
Christe, eléison.
}\switchcolumn\portugues{
Cristo, tende piedade de nós.
}\switchcolumn*\latim{
Kyrie, eléison.
}\switchcolumn\portugues{
Senhor, tende piedade de nós.
}\switchcolumn*\latim{
℣. Domine, exaudi orationem meam.
}\switchcolumn\portugues{
℣. Ouvi, Senhor, a minha oração.
}\switchcolumn*\latim{
℟. Et clamor meus ad te veniat.
}\switchcolumn\portugues{
℟. E o meu clamor chegue até Vós.
}\end{paracol}

\begin{paracol}{2}\latim{
\begin{nscenter} Orémus. \end{nscenter}
}\switchcolumn\portugues{
\begin{nscenter} Oremos. \end{nscenter}
}\switchcolumn*\latim{
\rlettrine{F}{amulorum} tuorum, quaesumus, Domine, delictis ignosce: ut qui tibi placere de actibus nostris non valemus, Genitricis Filii tui Domini nostri intercessione salvemur:
Qui tecum vivit et regnat \emph{\&c.}
}\switchcolumn\portugues{
\rlettrine{P}{erdoai,} Senhor, como Vos pedimos, os delictos dos vossos servos; para que não podendo agradar-Vos com as nossas obras, sejamos salvos, pela intercessão da Virgem Mãe de vosso Filhos e Senhor nosso: Que convosco vive e reina \emph{\&c.}
}\switchcolumn*\latim{
℟. Amen.
}\switchcolumn\portugues{
℟. Amen.
}\switchcolumn*\latim{
℣. Domine, exaudi orationem meam.
}\switchcolumn\portugues{
℣. Ouvi, Senhor, a minha oração.
}\switchcolumn*\latim{
℟. Et clamor meus ad te veniat.
}\switchcolumn\portugues{
℟. E o meu clamor chegue até Vós.
}\switchcolumn*\latim{
℣. Benedicamus Domino.
}\switchcolumn\portugues{
℣. Bendigamos o Senhor.
}\switchcolumn*\latim{
℟. Deo gratias.
}\switchcolumn\portugues{
℟. Graças a Deus.
}\switchcolumn*\latim{
℣. Fidelium animæ per misericordiam Dei, requiescant in pace.
}\switchcolumn\portugues{
℣. E que as almas dos fiéis, pela misericórdia de Deus, descansem em paz.
}\switchcolumn*\latim{
℟. Amen.
}\switchcolumn\portugues{
℟. Amen.
}\end{paracol}


\subsection{Vésperas 1}
\begin{paracol}{2}\latim{
℣. Deus \cruz in adjutórium meum inténde.
}\switchcolumn\portugues{
℣. Deus, \cruz vinde em meu auxílio.
}\switchcolumn*\latim{
℟. Dómine, ad adjuvándum me festína.
}\switchcolumn\portugues{
℟. Senhor, apressai-Vos em socorrer-me.
}\switchcolumn*\latim{
Glória Patri, \&c.
}\switchcolumn\portugues{
Glória ao Pai, \&c.
}\switchcolumn*\latim{
\emph{Ant.} Dum esset Rex in acubitu suo, nardus mea dedit odorem suavitatis.
}\switchcolumn\portugues{
\emph{Ant.} Estando o Rei no seu repouso, exalou o meu frasco um suavíssimo cheiro.
}\end{paracol}

\paragraphinfo{Salmo 109}{Página \pageref{salmo109}}

\begin{paracol}{2}\latim{
\emph{Ant.} Dum esset Rex in acubitu suo, nardus mea dedit odorem suavitatis.
}\switchcolumn\portugues{
\emph{Ant.} Estando o Rei no seu repouso, exalou o meu frasco um suavíssimo cheiro.
}\end{paracol}

\begin{paracol}{2}\latim{
\emph{Ant.} Læva ejus sub capite meo, et dextera ilius amplexabitur me.
}\switchcolumn\portugues{
\emph{Ant.} Sua mão esquerda estará debaixo de minha cabeça, e a sua direita me dará um abraço.
}\end{paracol}

\paragraphinfo{Salmo 112}{Página \pageref{salmo112}}

\begin{paracol}{2}\latim{
\emph{Ant.} Læva ejus sub capite meo, et dextera ilius amplexabitur me.
}\switchcolumn\portugues{
\emph{Ant.} Sua mão esquerda estará debaixo de minha cabeça, e a sua direita me dará um abraço.
}\end{paracol}

\begin{paracol}{2}\latim{
\emph{Ant.} Nigra sum, sed formosa, filiæ Jerusalem; ideo dilexit me rex, et introduxit me in cubiculom suum.
}\switchcolumn\portugues{
\emph{Ant.} Sou negra, mas sou formosa, ó filhas de Jerusalém; por isso o Rei me amou, e me levou a seu aposento.
}\end{paracol}

\paragraphinfo{Salmo 121}{Página \pageref{salmo121}}

\begin{paracol}{2}\latim{
\emph{Ant.} Nigra sum, sed formosa, filiæ Jerusalem; ideo dilexit me rex, et introduxit me in cubiculom suum.
}\switchcolumn\portugues{
\emph{Ant.} Sou negra, mas sou formosa, ó filhas de Jerusalém; por isso o Rei me amou, e me levou a seu aposento.
}\end{paracol}

\begin{paracol}{2}\latim{
\emph{Ant.} Jam hiems transiit, imber abiit et recessit: surge, amica mea, et veni.
}\switchcolumn\portugues{
\emph{Ant.} Já se foi o Inverno, e passou o chuveiro; levanta-te e vem, ó minha amada.
}\end{paracol}

\paragraphinfo{Salmo 126}{Página \pageref{salmo126}}

\begin{paracol}{2}\latim{
\emph{Ant.} Jam hiems transiit, imber abiit et recessit: surge, amica mea, et veni.
}\switchcolumn\portugues{
\emph{Ant.} Já se foi o Inverno, e passou o chuveiro; levanta-te e vem, ó minha amada.
}\end{paracol}

\begin{paracol}{2}\latim{
\emph{Ant.} Speciosa facta es et suavis in deliciis tuis, sancta Dei Genitrix.
}\switchcolumn\portugues{
\emph{Ant.} Especiosa sois, e suave nas vossas delicias, ó santa Mãe de Deus.
}\end{paracol}

\paragraphinfo{Salmo 147}{Página \pageref{salmo147}}

\begin{paracol}{2}\latim{
\emph{Ant.} Speciosa facta es et suavis in deliciis tuis, sancta Dei Genitrix.
}\switchcolumn\portugues{
\emph{Ant.} Especiosa sois, e suave nas vossas delicias, ó santa Mãe de Deus.
}\end{paracol}

\paragraphinfo{Pequeno Capítulo}{Ecl. 24, 14}
\begin{paracol}{2}\latim{
\rlettrine{A}{b} initio et ante sæcula creata sum, et usque ad futurum sæculum non desinam, et in habitatione sancta coram ipso ministravi.
}\switchcolumn\portugues{
\rlettrine{E}{u} fui criada desde o princípio, antes dos séculos, e não deixarei de existir até ao fim dos séculos, e exerci diante dele o meu ministério na morada santa.
}\switchcolumn*\latim{
℟. Deo grátias.
}\switchcolumn\portugues{
℟. Graças a Deus.
}\end{paracol}

\paragraphinfo{Ave Maris Stella}{Página \pageref{avemariastella}}

\paragraph{Ofício 1}
\begin{paracol}{2}\latim{
\emph{Ant.} Beata Mater et intacta Virgo, gloriosa Regina mundi, intercede pro nobis ad Dominum.
}\switchcolumn\portugues{
\emph{Ant.} Santa Mãe e Virgem intacta, gloriosa Rainha do mundo, intercedei a Deus por nós.
}\switchcolumn*\latim{
℟. Amen.
}\switchcolumn\portugues{
℟. Amen.
}\end{paracol}

\paragraph{Tempo Pascal}
\begin{paracol}{2}\latim{
\emph{Ant.} Regína Cæli, lætáre, allelúja; Quia quem meruísti portáre, allelúja; Resurréxit, sicut dixit, allelúja; Ora pro nóbis Deum, allelúja. Gaude et lætáre, Virgo Maria, allelúja. Quia surréxit Dóminus vere, allelúja.
}\switchcolumn\portugues{
\emph{Ant.} Rainha do Céu, alegrai-Vos, Aleluia!
Porque Aquele que merecestes trazer em vosso ventre, Aleluia! Ressuscitou como disse, Aleluia! Rogai por nós a Deus, Aleluia! Alegrai-Vos e exultai, ó Virgem Maria, Aleluia! Porque o Senhor ressuscitou verdadeiramente, Aleluia!
}\end{paracol}

\paragraphinfo{Magnificat}{Página \pageref{magnificat}}

\paragraph{Ofício 1}
\begin{paracol}{2}\latim{
\emph{Ant.} Beata Mater et intacta Virgo, gloriosa Regina mundi, intercede pro nobis ad Dominum.
}\switchcolumn\portugues{
\emph{Ant.} Santa Mãe e Virgem intacta, gloriosa Rainha do mundo, intercedei a Deus por nós.
}\switchcolumn*\latim{
℟. Amen.
}\switchcolumn\portugues{
℟. Amen.
}\end{paracol}

\paragraph{Tempo Pascal}
\begin{paracol}{2}\latim{
\emph{Ant.} Regína Cæli, lætáre, allelúja; Quia quem meruísti portáre, allelúja; Resurréxit, sicut dixit, allelúja; Ora pro nóbis Deum, allelúja. Gaude et lætáre, Virgo Maria, allelúja. Quia surréxit Dóminus vere, allelúja.
}\switchcolumn\portugues{
\emph{Ant.} Rainha do Céu, alegrai-Vos, Aleluia! Porque Aquele que merecestes trazer em vosso ventre, Aleluia! Ressuscitou como disse, Aleluia! Rogai por nós a Deus, Aleluia! Alegrai-Vos e exultai, ó Virgem Maria, Aleluia! Porque o Senhor ressuscitou verdadeiramente, Aleluia!
}\switchcolumn*\latim{
℣. Domine, exaudi orationem meam.
}\switchcolumn\portugues{
℣. Ouvi, Senhor, a minha oração.
}\switchcolumn*\latim{
℟. Et clamor meus ad te veniat.
}\switchcolumn\portugues{
℟. E o meu clamor chegue até Vós.
}\switchcolumn*\latim{
\begin{nscenter} Orémus. \end{nscenter}
}\switchcolumn\portugues{
\begin{nscenter} Oremos. \end{nscenter}
}\switchcolumn*\latim{
\rlettrine{C}{oncede} nos famulos tuos, quǽsumus, Domine Deus, perpetua mentis et corporis sanitate gaudere: et gloriosa beatæ Mariæ semper Virginis intercessione, a præsenti liberari tristitia, et æterna perfrui lætitia. Per Dominum nostrum Jesum Christum.
}\switchcolumn\portugues{
\rlettrine{S}{enhor} Deus, nós Vos suplicamos que concedais a vossos servos lograr uma perpétua saúde de corpo e alma, e que pela intercessão gloriosa da bem-aventurada sempre Virgem Maria sejamos livres da presente tristeza, e gozemos da eterna alegria. Por Jesus Cristo nosso Senhor.
}\switchcolumn*\latim{
℟. Amen.
}\switchcolumn\portugues{
℟. Amen.
}\switchcolumn*\latim{
℣. Domine, exaudi orationem meam.
}\switchcolumn\portugues{
℣. Ouvi, Senhor, a minha oração.
}\switchcolumn*\latim{
℟. Et clamor meus ad te veniat.
}\switchcolumn\portugues{
℟. E o meu clamor chegue até Vós.
}\switchcolumn*\latim{
℣. Benedicamus Domino.
}\switchcolumn\portugues{
℣. Bendigamos o Senhor.
}\switchcolumn*\latim{
℟. Deo gratias.
}\switchcolumn\portugues{
℟. Graças a Deus.
}\switchcolumn*\latim{
℣. Fidelium animæ per misericordiam Dei, requiescant in pace.
}\switchcolumn\portugues{
℣. E que as almas dos fiéis, pela misericórdia de Deus, descansem em paz.
}\switchcolumn*\latim{
℟. Amen.
}\switchcolumn\portugues{
℟. Amen.
}\end{paracol}

\emph{Acabar com uma Antífona de Nossa Senhora na página \pageref{antifonasnossasenhora}.}


\subsection{Completas 1}

\begin{paracol}{2}\latim{
℣. Convérte nos \cruz Deus, salutáris noster.
}\switchcolumn\portugues{
℣. Converte-nos, \cruz Deus nosso Salvador.
}\switchcolumn*\latim{
℟. Et avérte iram tuam a nobis.
}\switchcolumn\portugues{
℟. E afasta de nós a vossa ira.
}\switchcolumn*\latim{
℣. Deus \cruz in adjutórium meum inténde.
}\switchcolumn\portugues{
℣. Deus, \cruz vinde em meu auxílio.
}\switchcolumn*\latim{
℟. Dómine, ad adjuvándum me festína.
}\switchcolumn\portugues{
℟. Senhor, apressai-Vos em socorrer-me.
}\switchcolumn*\latim{
Glória Patri, \&c.
}\switchcolumn\portugues{
Glória ao Pai, \&c.
}\end{paracol}

\paragraphinfo{Salmo 128}{Página \pageref{salmo128}}

\paragraphinfo{Salmo 129}{Página \pageref{salmo129}}

\paragraphinfo{Salmo 130}{Página \pageref{salmo130}}

\paragraphinfo{Hino Memento rerum conditor}{Página \pageref{mementorerumconditor}}

\paragraphinfo{Pequeno Capítulo}{Ecl. 24}
\begin{paracol}{2}\latim{
\rlettrine{E}{go} mater pulchræ dilectionis, et timoris, et agnitionis, et sanctæ spei.
}\switchcolumn\portugues{
\rlettrine{E}{u} sou a Mãe do amor belo e do temor, e do conhecimento antigo, e da santa esperança.
}\switchcolumn*\latim{
℟. Deo grátias.
}\switchcolumn\portugues{
℟. Graças a Deus.
}\switchcolumn*\latim{
℣. Ora pro nobis sancta Dei Génetrix.
}\switchcolumn\portugues{
℣. Rogai por nós, Santa Mãe de Deus.
}\switchcolumn*\latim{
℟. Ut digni efficiamur promissionibus Christi.
}\switchcolumn\portugues{
℟. Para que sejamos dignos das promessas de Cristo.
}\end{paracol}

\paragraph{Ofício 1}
\begin{paracol}{2}\latim{
\emph{Ant.} Sub tuum præsídium confúgimus, sancta Dei Génetrix; nostras deprecatiónes ne despícias in necessitátibus; sed a perículis cunctis líbera nos semper,
Virgo gloriósa et benedícta.
}\switchcolumn\portugues{
\emph{Ant.} À vossa protecção recorremos, Santa Mãe de Deus; não desprezeis as nossas súplicas em nossas necessidades; mas livrai-nos sempre de todos os perigos, ó Virgem gloriosa e bendita.
}\switchcolumn*\latim{
℟. Amen.
}\switchcolumn\portugues{
℟. Amen.
}\end{paracol}

\paragraph{Tempo Pascal}
\begin{paracol}{2}\latim{
\emph{Ant.} Regína Cæli, lætáre, allelúja; Quia quem meruísti portáre, allelúja; Resurréxit, sicut dixit, allelúja; Ora pro nóbis Deum, allelúja. Gaude et lætáre, Virgo Maria, allelúja. Quia surréxit Dóminus vere, allelúja.
}\switchcolumn\portugues{
\emph{Ant.} Rainha do Céu, alegrai-Vos, Aleluia! Porque Aquele que merecestes trazer em vosso ventre, Aleluia! Ressuscitou como disse, Aleluia! Rogai por nós a Deus, Aleluia!
Alegrai-Vos e exultai, ó Virgem Maria, Aleluia! Porque o Senhor ressuscitou verdadeiramente, Aleluia!
}\end{paracol}

\paragraphinfo{Cântico Nunc Dimittis}{Página \pageref{nuncdimittis}}

\paragraph{Ofício 1}
\begin{paracol}{2}\latim{
\emph{Ant.} Sub tuum præsídium confúgimus, sancta Dei Génetrix; nostras deprecatiónes ne despícias in necessitátibus; sed a perículis cunctis líbera nos semper, Virgo gloriósa et benedícta.
}\switchcolumn\portugues{
\emph{Ant.} À vossa protecção recorremos, Santa Mãe de Deus; não desprezeis as nossas súplicas em nossas necessidades; mas livrai-nos sempre de todos os perigos, ó Virgem gloriosa e bendita.
}\switchcolumn*\latim{
℟. Amen.
}\switchcolumn\portugues{
℟. Amen.
}\end{paracol}

\paragraph{Tempo Pascal}
\begin{paracol}{2}\latim{
\emph{Ant.} Regína Cæli, lætáre, allelúja; Quia quem meruísti portáre, allelúja; Resurréxit, sicut dixit, allelúja; Ora pro nóbis Deum, allelúja. Gaude et lætáre, Virgo Maria, allelúja. Quia surréxit Dóminus vere, allelúja.
}\switchcolumn\portugues{
\emph{Ant.} Rainha do Céu, alegrai-Vos, Aleluia! Porque Aquele que merecestes trazer em vosso ventre, Aleluia! Ressuscitou como disse, Aleluia! Rogai por nós a Deus, Aleluia!
Alegrai-Vos e exultai, ó Virgem Maria, Aleluia! Porque o Senhor ressuscitou verdadeiramente, Aleluia!
}\switchcolumn*\latim{
℣. Domine, exaudi orationem meam.
}\switchcolumn\portugues{
℣. Ouvi, Senhor, a minha oração.
}\switchcolumn*\latim{
℟. Et clamor meus ad te veniat.
}\switchcolumn\portugues{
℟. E o meu clamor chegue até Vós.
}\switchcolumn*\latim{
\begin{nscenter} Orémus. \end{nscenter}
}\switchcolumn\portugues{
\begin{nscenter} Oremos. \end{nscenter}
}\switchcolumn*\latim{
\rlettrine{B}{eatæ} et gloriosæ semper Virginis Mariæ, quǽsumus, Domine, intercessio gloriosa nos protegat: et ad vitam perducat æternam. Per Dominum, \&c.
}\switchcolumn\portugues{
\qlettrine{Q}{ue} a gloriosa intercessão da abençoada e gloriosa Maria sempre Virgem, nos proteja, nós Vos pedimos Senhor, e que nos traga a vida eterna. Por nosso Senhor, \&c.
}\switchcolumn*\latim{
℟. Amen.
}\switchcolumn\portugues{
℟. Amen.
}\switchcolumn*\latim{
℣. Domine, exaudi orationem meam.
}\switchcolumn\portugues{
℣. Ouvi, Senhor, a minha oração.
}\switchcolumn*\latim{
℟. Et clamor meus ad te veniat.
}\switchcolumn\portugues{
℟. E o meu clamor chegue até Vós.
}\switchcolumn*\latim{
℣. Benedicamus Domino.
}\switchcolumn\portugues{
℣. Bendigamos o Senhor.
}\switchcolumn*\latim{
℟. Deo gratias.
}\switchcolumn\portugues{
℟. Graças a Deus.
}\end{paracol}

\emph{Acabar com uma Antífona de Nossa Senhora na página \pageref{antifonasnossasenhora}.}


\emph{O qual se deve dizer desde as Vésperas no Sábado antes do primeiro Domingo do Advento até a Nona na Véspera do Nascimento de N. Senhor, e assim mesmo no dia da Anunciação da Santíssima Virgem, a 25 de Março.}\par

\subsection{Matinas 2}

\emph{Tudo como no primeiro oficio, excepto o seguinte:
No Terceiro Nocturno, na Quarta-feira, no Sábado e na Festa da Anunciação, a última Antífona é a seguinte; são próprias as Lições e Responsórios que seguem.}

\begin{paracol}{2}\latim{
\emph{Ant.} Angelus Domini nuntiavit Mariæ, et concepit de Spiritu Sancto, (Allelúja)
}\switchcolumn\portugues{
\emph{Ant.} O Anjo do Senhor anunciou a Maria, e ela concebeu do Espírito Santo.
}\end{paracol}

\paragraphinfo{Salmo 97}{Página \pageref{salmo97}}

\begin{paracol}{2}\latim{
\emph{Ant.} Angelus Domini nuntiavit Mariæ, et concepit de Spiritu Sancto, (Allelúja)
}\switchcolumn\portugues{
\emph{Ant.} O Anjo do Senhor anunciou a Maria, e ela concebeu do Espírito Santo.
}\end{paracol}

\paragraph{Absolvição}
\begin{paracol}{2}\latim{
\rlettrine{P}{recibus} et meritis beatæ Mariæ semper Virginis, et omnium Sanctorum, perducat nos Dominus ad regna cælorum.
}\switchcolumn\portugues{
\rlettrine{P}{elos} rogos e merecimentos da bem-aventurada Virgem Maria, e de todos os Santos, nos conduza o Senhor ao reino dos céus.
}\switchcolumn*\latim{
℟. Amen.
}\switchcolumn\portugues{
℟. Amen.
}\switchcolumn*\latim{
℣. Jube, Domine, benedicere.
}\switchcolumn\portugues{
℣. Dai-me, ó Senhor, a vossa bênção.
}\end{paracol}

\paragraph{Benção}
\begin{paracol}{2}\latim{
Nos cum prole pia benedicat Virgo Maria.
}\switchcolumn\portugues{
Benza-nos a Virgem Maria com seu piíssimo Filho.
}\switchcolumn*\latim{
℟. Amen.
}\switchcolumn\portugues{
℟. Amen.
}\end{paracol}

\paragraphinfo{Lição 1}{Lc. 1, 26-28}
\begin{paracol}{2}\latim{
\rlettrine{M}{issus} est Angelus Gabriel a Deo in civitatem Galilææ, cui nomen Nazareth, ad virginem desponsatam viro, cui nomen erat Joseph, de domo David: et nomen virginis Maria. Et ingressus Angelus ad eam dixit: Ave gratia plena: Dominus tecum: benedicta tu in mulieribus.
}\switchcolumn\portugues{
\rlettrine{O}{} Anjo Gabriel foi mandado por Deus à cidade de Galileia chamada Nazaré, a uma virgem denominada Maria, desposada com um varão cujo nome era José, da casa de David. E entrando o Anjo onde ela estava, lhe disse: Deus vos salve, ó cheia de graça, o Senhor é convosco; bendita sois vós entre as mulheres.
}\switchcolumn*\latim{
℣. Tu autem, Dómine, miserére nobis.
}\switchcolumn\portugues{
℣. E vós, Senhor, tende misericórdia de nós.
}\switchcolumn*\latim{
℟. Deo grátias.
}\switchcolumn\portugues{
℟. Demos graças a Deus.
}\switchcolumn*\latim{
℟. Missus est Gabriel Angelus ad Maríam Vírginem desponsatam Joseph, nuntians ei verbum; et expavescit Virgo de lúmine: ne timeas, María, invenísti grátiam apud Dóminum: Ecce concipies et paries, et vocábitur Altíssimi Fílius.
}\switchcolumn\portugues{
℟. O Anjo Gabriel foi enviado a Maria Virgem, desposada com José, para lhe anunciar o verbo; e a Virgem assustou-se com o esplendor da sua luz. Não temas, Maria, que achaste graça para com o Senhor. Conceberás, e darás à luz um filho que será chamado o filho do Altíssimo.
}\switchcolumn*\latim{
℣. Dabit ei Dóminus Deus sedem David, patris ejus, et regnábit in domo Jacob in ætérnum.
}\switchcolumn\portugues{
℣. O Senhor Deus lhe dará o trono de David seu Pai, e reinará eternamente na casa de Jacob.
}\switchcolumn*\latim{
℟. Ecce concipies et paries, et vocábitur Altíssimi Fílius.
}\switchcolumn\portugues{
℟. Conceberás, e darás á luz um filho que será chamado o Filho do Altíssimo.
}\switchcolumn*\latim{
℣. Iube domne benedicere.
}\switchcolumn\portugues{
℣. Dai-me, ó Senhor, a vossa bênção.
}\switchcolumn*\latim{
Ipsa Virgo Vírginum intercédat pro nobis ad Dóminum.
}\switchcolumn\portugues{
A mesma Virgem das virgens interceda por nós ao Senhor.
}\switchcolumn*\latim{
℟. Amen.
}\switchcolumn\portugues{
℟. Amen.
}\end{paracol}

\paragraphinfo{Lição 2}{Lc. 1, 29-33}
\begin{paracol}{2}\latim{
\qlettrine{Q}{uæ} cum audisset, turbata est in sermone ejus, et cogitabat qualis esset ista salutatio. Et ait Angelus ei: Ne timeas, Maria: invenisti enim gratiam apud Deum: ecce concipies in utero, et paries filium, et vocabis nomen ejus Jesum: hic erit magnus, et Filius Altissimi vocabitur, et dabit illi Dominus Deus sedem David patris ejus: et regnabit in domo Jacob in æternum, et regni ejus non erit finis.
}\switchcolumn\portugues{
\rlettrine{O}{uvindo} ela estas palavras, perturbou-se pelo que se lhe dizia; e considerava que saudação seria. Então o Anjo disse-lhe: Não temas, Maria, porque achaste graça para com Deus. Conceberás no teu ventre, e darás à luz um filho a quem darás o nome de Jesus. Este será grande e se chamará Filho do Altíssimo, e o Senhor Deus lhe dará o trono de David seu Pai, e reinará eternamente na casa de Jacob, e o seu Reino não terá fim.
}\switchcolumn*\latim{
℣. Tu autem, Dómine, miserére nobis.
}\switchcolumn\portugues{
℣. E vós, Senhor, tende misericórdia de nós.
}\switchcolumn*\latim{
℟. Deo grátias.
}\switchcolumn\portugues{
℟. Demos graças a Deus.
}\switchcolumn*\latim{
℟. Ave, María, grátia plena; Dóminus tecum: Spíritus Sanctus supervéniet in te, et virtus Altíssimi obumbrábit tibi: quod enim ex te nascétur Sanctum, vocábitur Fílius Dei.
}\switchcolumn\portugues{
℟. Ave, Maria, cheia de graça; o Senhor é convosco. Virá sobre vós o Espírito Santo e a virtude do Altíssimo vos fará sombra: por isso o santo que nascerá de vós será chamado Filho de Deus.
}\switchcolumn*\latim{
℣. Quómodo fiet istud, quóniam virum non cognósco? Et respóndens Angelus, dixit ei.
}\switchcolumn\portugues{
℣. Como se fará isto, pois não conheço varão? E respondendo o Anjo, lhe disse:
}\switchcolumn*\latim{
℟. Spíritus Sanctus supervéniet in te, et virtus Altíssimi obumbrábit tibi: quod enim ex te nascétur Sanctum, vocábitur Fílius Dei.
}\switchcolumn\portugues{
℟. Virá sobre vós o Espírito Santo, e a virtude do Altíssimo vos fará sombra; por isso o santo que nascerá de vós será chamado Filho de Deus.
}\end{paracol}

\emph{Quando o Te Deum (página \pageref{tedeum}) é dito depois da Terceira Lição, adiciona-se o seguinte no fim do Responsório:}

\begin{paracol}{2}\latim{
℣. Glória Patri, et Fílio, et Spirítui Sancto.
}\switchcolumn\portugues{
℣. Glória ao Pai e ao Filho e ao Espírito Santo.
}\switchcolumn*\latim{
℟. Spíritus Sanctus supervéniet in te, et virtus Altíssimi obumbrábit tibi: quod enim ex te nascétur Sanctum, vocábitur Fílius Dei.
}\switchcolumn\portugues{
℟. Virá sobre vós o Espírito Santo, e a virtude do Altíssimo vos fará sombra; por isso o santo que nascerá de vós será chamado Filho de Deus.
}\switchcolumn*\latim{
℣. Iube domne benedicere.
}\switchcolumn\portugues{
℣. Dai-me, Senhor, a vossa bênção.
}\end{paracol}

\paragraph{Benção}
\begin{paracol}{2}\latim{
Per Vírginem Matrem concédat nobis Dóminus salútem et pacem.
}\switchcolumn\portugues{
Pela Virgem Maria, nos conceda o Senhor a paz e a salvação.
}\switchcolumn*\latim{
℟. Amen.
}\switchcolumn\portugues{
℟. Amen.
}\end{paracol}

\paragraphinfo{Lição 3}{Lc. 1, 34-38}
\begin{paracol}{2}\latim{
\rlettrine{D}{ixit} autem Maria ad Angelum: Quomodo fiet istud, quoniam virum non cognosco? Et respondens Angelus dixit ei: Spiritus Sanctus superveniet in te, et virtus Altissimi obumbrabit tibi. Ideoque et quod nascetur ex te Sanctum, vocabitur Filius Dei. Et ecce Elisabeth cognata tua, et ipsa concepit filium in senectute sua: et hic mensis sextus est illi, quæ vocatur sterilis: quia non erit impossibile apud Deum omne verbum. Dixit autem Maria: Ecce ancilla Domini: fiat mihi secundum verbum tuum.
}\switchcolumn\portugues{
\rlettrine{D}{isse} então Maria ao Anjo: Como se fará isto, por quando não conheço varão? E respondendo o Anjo, lhe disse: Virá sobre vós o Espírito Santo, e a virtude do Altíssimo vos fará sombra; e por isso o santo que nascerá de vós se chamará Filho de Deus. E também Isabel, vossa parenta, que é chamada estéril, concebeu um filho na sua velhice, está já no sexto mês; porque a Deus nada é impossível. Disse então Maria: Eis aqui a escrava do Senhor, faça-se em mim segundo a vossa palavra.
}\switchcolumn*\latim{
℣. Tu autem, Dómine, miserére nobis.
}\switchcolumn\portugues{
℣. E vós, Senhor, tende misericórdia de nós.
}\switchcolumn*\latim{
℟. Deo grátias.
}\switchcolumn\portugues{
℟. Demos graças a Deus.
}\end{paracol}

\emph{O Te Deum não é dito no Advento, excluindo as Festas de Nossa Senhora. O seguinte Responsório é dito quando o Te Deum é omitido:}

\begin{paracol}{2}\latim{
℟. Súscipe verbum, Virgo María, quod tibi a Dómino per Angelum transmíssum est: concípies et páries Deum páriter et hóminem, ut benedícta dicáris inter omnes mulíeres.
}\switchcolumn\portugues{
℟. Recebei, Maria Virgem, a palavra que Senhor vos transmite pelo seu Anjo. Concebereis, e dareis à luz a Deus e Homem juntamente: pelo que sereis chamada Bendita entre todas as mulheres.
}\switchcolumn*\latim{
℣. Paries quidem fílium, et virginitátis non patiéris detriméntum: efficiéris grávida, et eris mater semper intácta.
}\switchcolumn\portugues{
℣. Dareis à luz um filho, e ficareis sempre Virgem. Concebereis e ficareis mãe, continuareis sempre pura e imaculada.
}\switchcolumn*\latim{
℟. Ut benedícta dicáris inter omnes mulíeres.
}\switchcolumn\portugues{
℟. Pelo que sereis chamada Bendita entre todas as mulheres.
}\switchcolumn*\latim{
℣. Glória Patri, et Fílio, et Spirítui Sancto.
}\switchcolumn\portugues{
℣. Glória ao Pai e ao Filho e ao Espírito Santo.
}\switchcolumn*\latim{
℟. Ut benedícta dicáris inter omnes mulíeres.
}\switchcolumn\portugues{
℟. Pelo que sereis chamada Bendita entre todas as mulheres.
}\end{paracol}


\subsection{Laudes 2}

\begin{paracol}{2}\latim{
℣. Deus \cruz in adjutórium meum inténde.
}\switchcolumn\portugues{
℣. Deus, \cruz vinde em meu auxílio.
}\switchcolumn*\latim{
℟. Dómine, ad adjuvándum me festína.
}\switchcolumn\portugues{
℟. Senhor, apressai-Vos em socorrer-me.
}\switchcolumn*\latim{
Glória Patri, \&c.
}\switchcolumn\portugues{
Glória ao Pai, \&c.
}\switchcolumn*\latim{
\emph{Ant.} Missus est Gábriel Angelus ad Maríam Vírginem desponsátam Joseph.
}\switchcolumn\portugues{
\emph{Ant.} O Anjo Gabriel foi mandado à Virgem Maria, desposada com José.
}\end{paracol}

\paragraphinfo{Salmo 92}{Página \pageref{salmo92}}

\begin{paracol}{2}\latim{
 \emph{Ant.} Missus est Gábriel Angelus ad Maríam Vírginem desponsátam Joseph.
 }\switchcolumn\portugues{
 \emph{Ant.} O Anjo Gabriel foi mandado à Virgem Maria, desposada com José.
}\end{paracol}

\begin{paracol}{2}\latim{
\emph{Ant.} Ave, María, grátia plena; Dóminus tecum: benedícta tu in muliéribus.
}\switchcolumn\portugues{
\emph{Ant.} Ave, Maria, cheia de graça, o Senhor é convosco; bendita sois vós entre as mulheres.
}\end{paracol}

\paragraphinfo{Salmo 99}{Página \pageref{salmo99}}

\begin{paracol}{2}\latim{
\emph{Ant.} Ave, María, grátia plena; Dóminus tecum: benedícta tu in muliéribus.
}\switchcolumn\portugues{
\emph{Ant.} Ave, Maria, cheia de graça, o Senhor é convosco; bendita sois vós entre as mulheres.
}\end{paracol}

\begin{paracol}{2}\latim{
\emph{Ant.} Ne timeas, María, invenísti grátiam apud Dóminum: ecce concípies et páries fílium.
}\switchcolumn\portugues{
\emph{Ant.} Não temais, ó Maria, achastes graça para com o Senhor: concebereis, e dareis à luz um filho.
}\end{paracol}

\paragraphinfo{Salmo 62}{Página \pageref{salmo62}}

\begin{paracol}{2}\latim{
\emph{Ant.} Ne timeas, María, invenísti grátiam apud Dóminum: ecce concípies et páries fílium.
}\switchcolumn\portugues{
\emph{Ant.} Não temais, ó Maria, achastes graça para com o Senhor: concebereis, e dareis à luz um filho.
}\end{paracol}

\begin{paracol}{2}\latim{
\emph{Ant.} Dabit ei Dóminus sedem David, patris ejus, et regnábit in ætérnum.
}\switchcolumn\portugues{
\emph{Ant.} O Senhor lhe dará o trono de David seu Pai, e reinará eternamente.
}\end{paracol}

\paragraphinfo{Benedicite}{Página \pageref{benedicite}}
\emph{Não se diz a Glória no fim.}

\begin{paracol}{2}\latim{
\emph{Ant.} Dabit ei Dóminus sedem David, patris ejus, et regnábit in ætérnum.
}\switchcolumn\portugues{
\emph{Ant.} O Senhor lhe dará o trono de David seu Pai, e reinará eternamente.
}\end{paracol}

\begin{paracol}{2}\latim{
\emph{Ant.} Ecce ancílla Dómini: fiat mihi secúndum verbum tuum.
}\switchcolumn\portugues{
\emph{Ant.} Eis aqui a escrava do Senhor, faça-se em mim segundo a vossa palavra.
}\end{paracol}

\paragraphinfo{Salmo 148}{Página \pageref{salmo148}}

\begin{paracol}{2}\latim{
\emph{Ant.} Ecce ancílla Dómini: fiat mihi secúndum verbum tuum.
}\switchcolumn\portugues{
\emph{Ant.} Eis aqui a escrava do Senhor, faça-se em mim segundo a vossa palavra.
}\end{paracol}

\paragraphinfo{Pequeno Capítulo}{Is. 11, 1-2}
\begin{paracol}{2}\latim{
\rlettrine{E}{gredietur} virga de radice Jesse, et flos de radice ejus ascendet. Et requiescet super eum Spiritus Domini.
}\switchcolumn\portugues{
\rlettrine{S}{airá} uma vara da raiz de Jessé, e subirá uma flor da sua raiz, e descansará sobre ele o Espírito do Senhor.
}\switchcolumn*\latim{
℟. Deo grátias.
}\switchcolumn\portugues{
℟. Graças a Deus.
}\end{paracol}

\paragraphinfo{Hino O Gloriosa Virginum}{Página \pageref{ogloriosavirginum}}

\begin{paracol}{2}\latim{
℣. Benedicta tu in mulieribus.
}\switchcolumn\portugues{
℣. Bendita sois vóo entre as mulheres.
}\switchcolumn*\latim{
℟. Et benedictus fructus ventris tui.
}\switchcolumn\portugues{
℟. E bendito é o fruto do vosso ventre.
}\end{paracol}

\begin{paracol}{2}\latim{
\emph{Ant.} Spiritus Sanctus in te descendet, Maria: ne timeas, habebis in utero filium Dei, (allelúja).
}\switchcolumn\portugues{
\emph{Ant.} O Espírito Santo descerá sobre vós, ó Maria, não temais: concebereis, e tereis no ventre o Filho de Deus(aleluia).
}\end{paracol}

\paragraphinfo{Benedictus}{Página \pageref{benedictus}}

\begin{paracol}{2}\latim{
\emph{Ant.} Spiritus Sanctus in te descendet, Maria: ne timeas, habebis in utero filium Dei, (allelúja).
}\switchcolumn\portugues{
\emph{Ant.} O Espírito Santo descerá sobre vós, ó Maria, não temais: concebereis, e tereis no ventre o Filho de Deus(aleluia).
}\switchcolumn*\latim{
℣. Domine, exaudi orationem meam.
}\switchcolumn\portugues{
℣. Ouvi, Senhor, a minha oração.
}\switchcolumn*\latim{
℟. Et clamor meus ad te veniat.
}\switchcolumn\portugues{
℟. E o meu clamor chegue até Vós.
}\switchcolumn*\latim{
\begin{nscenter} Orémus. \end{nscenter}
}\switchcolumn\portugues{
\begin{nscenter} Oremos. \end{nscenter}
}\switchcolumn*\latim{
\rlettrine{D}{eus,} qui de beatæ Mariæ Virginis utero Verbum tuum, Angelo nuntiante, carnem suscipere voluisti: præsta supplicibus tuis; ut qui vere eam Genetricem Dei credimus, ejus apud te intercessionibus adjuvemur. Per eundem Dominum nostrum Jesum Christum.
}\switchcolumn\portugues{
\slettrine{Ó}{} Deus, que pela anunciação do Anjo quisestes que o vosso Verbo se vestisse da nossa carne nas entranhas da bem-aventurada Virgem Maria: nós, vossos humildes servos, cremos ser ela a verdadeira Mãe de Deus, concedei-nos que nos ajudem as suas intercessões para convosco. Pelo mesmo Jesus Cristo Senhor Nosso.
}\switchcolumn*\latim{
℟. Amen.
}\switchcolumn\portugues{
℟. Amen.
}\switchcolumn*\latim{
℣. Domine, exaudi orationem meam.
}\switchcolumn\portugues{
℣. Ouvi, Senhor, a minha oração.
}\switchcolumn*\latim{
℟. Et clamor meus ad te veniat.
}\switchcolumn\portugues{
℟. E o meu clamor chegue até Vós.
}\switchcolumn*\latim{
℣. Benedicamus Domino.
}\switchcolumn\portugues{
℣. Bendigamos o Senhor.
}\switchcolumn*\latim{
℟. Deo gratias.
}\switchcolumn\portugues{
℟. Graças a Deus.
}\switchcolumn*\latim{
℣. Fidelium animæ per misericordiam Dei, requiescant in pace.
}\switchcolumn\portugues{
℣. E que as almas dos fiéis, pela misericórdia de Deus, descansem em paz.
}\switchcolumn*\latim{
℟. Amen.
}\switchcolumn\portugues{
℟. Amen.
}\end{paracol}

\emph{Acabar com uma Antífona de Nossa Senhora na página \pageref{antifonasnossasenhora}.}


\subsection{Prima 2}

\textit{Tudo como no primeiro oficio, excepto o seguinte:}

\begin{paracol}{2}\latim{
\emph{Ant.} Missus est Gabriel Angelus ad Maríam, Vírginem, desponsatam Joseph.
}\switchcolumn\portugues{
\emph{Ant.} O Anjo Gabriel foi enviado a Maria Virgem, desposada com José.
}\end{paracol}

\paragraphinfo{Pequeno Capítulo}{Is. 7, 14-15}
\begin{paracol}{2}\latim{
\rlettrine{E}{cce} Virgo concipiet, et pariet filium, et vocabitur nomen ejus Emmanuel. Butyrum et mel comedet, ut sciat reprobare malum, et eligere bonum.
}\switchcolumn\portugues{
\rlettrine{P}{ois} por isso o mesmo Senhor vos dará este sinal: Uma virgem conceberá e dará à luz um filho, e o seu nome será Emanuel. Ele comerá manteiga e mel, até que saiba rejeitar o mal e escolher o bem.
}\switchcolumn*\latim{
℟. Deo grátias.
}\switchcolumn\portugues{
℟. Graças a Deus.
}\switchcolumn*\latim{
℣. Dignare me laudare te, Virgo sacrata.
}\switchcolumn\portugues{
℣. Dignai-vos, sagrada Virgem, de que eu vos louve.
}\switchcolumn*\latim{
℟. Da mihi virtutem contra hostes tuos.
}\switchcolumn\portugues{
℟. Dai-me esforço contra vossos inimigos.
}\end{paracol}


\begin{paracol}{2}\latim{
\emph{(Hic genuflectitur)} Kyrie eleison
}\switchcolumn\portugues{
\emph{(Genuflectir)} Senhor, tende piedade de nós.
}\switchcolumn*\latim{
Christe, eléison.
}\switchcolumn\portugues{
Cristo, tende piedade de nós.
}\switchcolumn*\latim{
Kyrie, eléison.
}\switchcolumn\portugues{
Senhor, tende piedade de nós.
}\switchcolumn*\latim{
℣. Domine, exaudi orationem meam.
}\switchcolumn\portugues{
℣. Ouvi, Senhor, a minha oração.
}\switchcolumn*\latim{
℟. Et clamor meus ad te veniat.
}\switchcolumn\portugues{
℟. E o meu clamor chegue até Vós.
}\end{paracol}

\begin{paracol}{2}\latim{
\begin{nscenter} Orémus. \end{nscenter}
}\switchcolumn\portugues{
\begin{nscenter} Oremos. \end{nscenter}
}\switchcolumn*\latim{
\rlettrine{D}{eus,} qui de beatæ Mariæ Virginis utero Verbum tuum, Angelo nuntiante, carnem suscipere voluisti: præsta supplicibus tuis; ut qui vere eam Genetricem Dei credimus, ejus apud te intercessionibus adjuvemur. Per eundem Dominum nostrum Jesum Christum.
}\switchcolumn\portugues{
\slettrine{Ó}{} Deus, que pela anunciação do Anjo quisestes que o vosso Verbo se vestisse da nossa carne nas entranhas da bem-aventurada Virgem Maria: nós, vossos humildes servos, cremos ser ela a verdadeira Mãe de Deus, concedei-nos que nos ajudem as suas intercessões para convosco. Pelo mesmo Jesus Cristo Senhor Nosso.
}\switchcolumn*\latim{
℟. Amen.
}\switchcolumn\portugues{
℟. Amen.
}\end{paracol}

\subsection{Terça 2}

\textit{Tudo como no primeiro oficio, excepto o seguinte:}

\begin{paracol}{2}\latim{
\emph{Ant.} Ave, María, grátia plena; Dóminus tecum: benedícta tu in muliéribus.
}\switchcolumn\portugues{
\emph{Ant.} Ave, Maria, cheia de graça, o Senhor é convosco; bendita sois vós entre as mulheres.
}\end{paracol}

\paragraphinfo{Pequeno Capítulo}{Is. 6, 1-2}
\begin{paracol}{2}\latim{
\rlettrine{E}{gredietur} virga de radice Jesse, et flos de radice ejus ascendet. Et requiescet super eum Spiritus Domini.
}\switchcolumn\portugues{
\rlettrine{S}{airá} uma vara da raiz de Jessé, e subirá uma flor da sua raiz, e descansará sobre ele o Espírito do Senhor.
}\switchcolumn*\latim{
℟. Deo grátias.
}\switchcolumn\portugues{
℟. Graças a Deus.
}\switchcolumn*\latim{
℣. Diffusa est gratia in labiis tuis.
}\switchcolumn\portugues{
℣. Estão cheios de graça vossos lábios.
}\switchcolumn*\latim{
℟. Propterea benedixit te Deum in æternum.
}\switchcolumn\portugues{
℟. Por isso Deus vos abençoou para sempre.
}\end{paracol}

\begin{paracol}{2}\latim{
\emph{(Hic genuflectitur)} Kyrie eleison
}\switchcolumn\portugues{
\emph{(Genuflectir)} Senhor, tende piedade de nós.
}\switchcolumn*\latim{
Christe, eléison.
}\switchcolumn\portugues{
Cristo, tende piedade de nós.
}\switchcolumn*\latim{
Kyrie, eléison.
}\switchcolumn\portugues{
Senhor, tende piedade de nós.
}\switchcolumn*\latim{
℣. Domine, exaudi orationem meam.
}\switchcolumn\portugues{
℣. Ouvi, Senhor, a minha oração.
}\switchcolumn*\latim{
℟. Et clamor meus ad te veniat.
}\switchcolumn\portugues{
℟. E o meu clamor chegue até Vós.
}\end{paracol}

\begin{paracol}{2}\latim{
\begin{nscenter} Orémus. \end{nscenter}
}\switchcolumn\portugues{
\begin{nscenter} Oremos. \end{nscenter}
}\switchcolumn*\latim{
\rlettrine{D}{eus,} qui de beatæ Mariæ Virginis utero Verbum tuum, Angelo nuntiante, carnem suscipere voluisti: præsta supplicibus tuis; ut qui vere eam Genetricem Dei credimus, ejus apud te intercessionibus adjuvemur. Per eundem Dominum nostrum Jesum Christum.
}\switchcolumn\portugues{
\slettrine{Ó}{} Deus, que pela anunciação do Anjo quisestes que o vosso Verbo se vestisse da nossa carne nas entranhas da bem-aventurada Virgem Maria: nós, vossos humildes servos, cremos ser ela a verdadeira Mãe de Deus, concedei-nos que nos ajudem as suas intercessões para convosco. Pelo mesmo Jesus Cristo Senhor Nosso.
}\switchcolumn*\latim{
℟. Amen.
}\switchcolumn\portugues{
℟. Amen.
}\end{paracol}


\subsection{Sexta 2}

\textit{Tudo como no primeiro oficio, excepto o seguinte:}

\begin{paracol}{2}\latim{
\emph{Ant.} Ne timeas, María, invenísti grátiam apud Dóminum: ecce concípies et páries fílium, (alleluia).
}\switchcolumn\portugues{
\emph{Ant.} Não temais, ó Maria, achastes graça para com o Senhor: concebereis, e dareis à luz um filho, (aleluia).
}\end{paracol}

\paragraphinfo{Pequeno Capítulo}{Lc. 1, 32}
\begin{paracol}{2}\latim{
\rlettrine{D}{abit} illi Dominus Deus sedem David patris ejus: et regnabit in domo Jacob in æternum, et regni ejus non erit finis.
}\switchcolumn\portugues{
\rlettrine{O}{} Senhor Deus lhe dará o trono de David seu Pai, e reinará eternamente na casa de Jacob, e o seu Reino não terá fim.
}\switchcolumn*\latim{
℟. Deo grátias.
}\switchcolumn\portugues{
℟. Graças a Deus.
}\switchcolumn*\latim{
℣. Benedicta tu in mulieribus.
}\switchcolumn\portugues{
℣. Bendita sois vóo entre as mulheres.
}\switchcolumn*\latim{
℟. Et benedictus fructus ventris tui.
}\switchcolumn\portugues{
℟. E bendito é o fruto do vosso ventre.
}\end{paracol}


\begin{paracol}{2}\latim{
\emph{(Hic genuflectitur)} Kyrie eleison
}\switchcolumn\portugues{
\emph{(Genuflectir)} Senhor, tende piedade de nós.
}\switchcolumn*\latim{
Christe, eléison.
}\switchcolumn\portugues{
Cristo, tende piedade de nós.
}\switchcolumn*\latim{
Kyrie, eléison.
}\switchcolumn\portugues{
Senhor, tende piedade de nós.
}\switchcolumn*\latim{
℣. Domine, exaudi orationem meam.
}\switchcolumn\portugues{
℣. Ouvi, Senhor, a minha oração.
}\switchcolumn*\latim{
℟. Et clamor meus ad te veniat.
}\switchcolumn\portugues{
℟. E o meu clamor chegue até Vós.
}\end{paracol}

\begin{paracol}{2}\latim{
\begin{nscenter} Orémus. \end{nscenter}
}\switchcolumn\portugues{
\begin{nscenter} Oremos. \end{nscenter}
}\switchcolumn*\latim{
\rlettrine{D}{eus,} qui de beatæ Mariæ Virginis utero Verbum tuum, Angelo nuntiante, carnem suscipere voluisti: præsta supplicibus tuis; ut qui vere eam Genetricem Dei credimus, ejus apud te intercessionibus adjuvemur. Per eundem Dominum nostrum Jesum Christum.
}\switchcolumn\portugues{
\slettrine{Ó}{} Deus, que pela anunciação do Anjo quisestes que o vosso Verbo se vestisse da nossa carne nas entranhas da bem-aventurada Virgem Maria: nós, vossos humildes servos, cremos ser ela a verdadeira Mãe de Deus, concedei-nos que nos ajudem as suas intercessões para convosco. Pelo mesmo Jesus Cristo Senhor Nosso.
}\switchcolumn*\latim{
℟. Amen.
}\switchcolumn\portugues{
℟. Amen.
}\end{paracol}

\subsection{Noa 2}

\textit{Tudo como no primeiro oficio, excepto o seguinte:}

\begin{paracol}{2}\latim{
\emph{Ant.} Ecce ancilla Domini: fiat mihi secundum verbum tuum.
}\switchcolumn\portugues{
\emph{Ant.} Eis aqui a escrava do Senhor, faça-se em mim segundo a vossa palavra.
}\end{paracol}

\paragraphinfo{Pequeno Capítulo}{Is. 7, 14-15}
\begin{paracol}{2}\latim{
\rlettrine{E}{cce} Virgo concipiet, et pariet filium, et vocabitur nomen ejus Emmanuel. Butyrum et mel comedet, ut sciat reprobare malum, et eligere bonum.
}\switchcolumn\portugues{
\rlettrine{P}{ois} por isso o mesmo Senhor vos dará este sinal: Uma virgem conceberá e dará à luz um filho, e o seu nome será Emanuel. Ele comerá manteiga e mel, até que saiba rejeitar o mal e escolher o bem.
}\switchcolumn*\latim{
℟. Deo grátias.
}\switchcolumn\portugues{
℟. Graças a Deus.
}\switchcolumn*\latim{
℣. Benedicta tu in mulieribus.
}\switchcolumn\portugues{
℣. Bendita sois vóo entre as mulheres.
}\switchcolumn*\latim{
℟. Et benedictus fructus ventris tui.
}\switchcolumn\portugues{
℟. E bendito é o fruto do vosso ventre.
}\end{paracol}


\begin{paracol}{2}\latim{
\emph{(Hic genuflectitur)} Kyrie eleison
}\switchcolumn\portugues{
\emph{(Genuflectir)} Senhor, tende piedade de nós.
}\switchcolumn*\latim{
Christe, eléison.
}\switchcolumn\portugues{
Cristo, tende piedade de nós.
}\switchcolumn*\latim{
Kyrie, eléison.
}\switchcolumn\portugues{
Senhor, tende piedade de nós.
}\switchcolumn*\latim{
℣. Domine, exaudi orationem meam.
}\switchcolumn\portugues{
℣. Ouvi, Senhor, a minha oração.
}\switchcolumn*\latim{
℟. Et clamor meus ad te veniat.
}\switchcolumn\portugues{
℟. E o meu clamor chegue até Vós.
}\end{paracol}

\begin{paracol}{2}\latim{
\begin{nscenter} Orémus. \end{nscenter}
}\switchcolumn\portugues{
\begin{nscenter} Oremos. \end{nscenter}
}\switchcolumn*\latim{
\rlettrine{D}{eus,} qui de beatæ Mariæ Virginis utero Verbum tuum, Angelo nuntiante, carnem suscipere voluisti: præsta supplicibus tuis; ut qui vere eam Genetricem Dei credimus, ejus apud te intercessionibus adjuvemur. Per eundem Dominum nostrum Jesum Christum.
}\switchcolumn\portugues{
\slettrine{Ó}{} Deus, que pela anunciação do Anjo quisestes que o vosso Verbo se vestisse da nossa carne nas entranhas da bem-aventurada Virgem Maria: nós, vossos humildes servos, cremos ser ela a verdadeira Mãe de Deus, concedei-nos que nos ajudem as suas intercessões para convosco. Pelo mesmo Jesus Cristo Senhor Nosso.
}\switchcolumn*\latim{
℟. Amen.
}\switchcolumn\portugues{
℟. Amen.
}\end{paracol}

\input{texto/oficio/vespers2}

\subsection{Completas 2}

\emph{Tudo como no primeiro oficio, excepto o seguinte:}

\paragraphinfo{Pequeno Capítulo}{Is. 7, 14-15}
\begin{paracol}{2}\latim{
\rlettrine{E}{cce} Virgo concipiet, et pariet filium, et vocabitur nomen ejus Emmanuel. Butyrum et mel comedet, ut sciat reprobare malum, et eligere bonum.
}\switchcolumn\portugues{
\rlettrine{P}{ois} por isso o mesmo Senhor vos dará este sinal: Uma virgem conceberá e dará à luz um filho, e o seu nome será Emanuel. Ele comerá manteiga e mel, até que saiba rejeitar o mal e escolher o bem.
}\switchcolumn*\latim{
℟. Deo gratias.
}\switchcolumn\portugues{
℟. Graças a Deus.
}\switchcolumn*\latim{
℣. Angelus Dómini nuntiávit Maríæ.
}\switchcolumn\portugues{
℣. O Anjo do Senhor anunciou a Maria.
}\switchcolumn*\latim{
℟. Et concépit de Spíritu Sancto.
}\switchcolumn\portugues{
℟. E Ela concebeu do Espírito Santo.
}\switchcolumn*\latim{
\emph{Nunc. Ant.} Spiritus Sanctus in te descendet, Maria: ne timeas, habebis in utero Filium Dei,(Allelúja).
}\switchcolumn\portugues{
\emph{Nunc. Ant.} O Espírito Santo descerá sobre vós, ó Maria; não temais: concebereis, e tereis no ventre o Filho de Deus, (Aleluia).
}\switchcolumn*\latim{
\begin{nscenter} Orémus. \end{nscenter}
}\switchcolumn\portugues{
\begin{nscenter} Oremos. \end{nscenter}
}\switchcolumn*\latim{
\rlettrine{D}{eus,} qui de beatæ Mariæ Virginis utero Verbum tuum, Angelo nuntiante, carnem suscipere voluisti: præsta supplicibus tuis; ut qui vere eam Genetricem Dei credimus, ejus apud te intercessionibus adjuvemur. Per eundem Dominum nostrum Jesum Christum.
}\switchcolumn\portugues{
\slettrine{Ó}{} Deus, que pela anunciação do Anjo quisestes que o vosso Verbo se vestisse da nossa carne nas entranhas da bem-aventurada Virgem Maria: nós, vossos humildes servos, cremos ser ela a verdadeira Mãe de Deus, concedei-nos que nos ajudem as suas intercessões para convosco. Pelo mesmo Jesus Cristo Senhor Nosso.
}\switchcolumn*\latim{
℟. Amen.
}\switchcolumn\portugues{
℟. Amen.
}\end{paracol}

\emph{Acabar com uma Antífona de Nossa Senhora na página \pageref{antifonasnossasenhora}.}


\emph{O qual se deve dizer desde as primeiras Vésperas do nascimento de N. Senhor (a 24 do mês de Dezembro) até ao fim do dia da Purificação da SS. Virgem, a 2 de Fevereiro.}\par

\input{texto/oficio/lauds3}

\subsection{Vésperas 3}

\begin{paracol}{2}\latim{
℣. Deus \cruz in adjutórium meum inténde.
}\switchcolumn\portugues{
℣. Deus, \cruz vinde em meu auxílio.
}\switchcolumn*\latim{
℟. Dómine, ad adjuvándum me festína.
}\switchcolumn\portugues{
℟. Senhor, apressai-Vos em socorrer-me.
}\switchcolumn*\latim{
Glória Patri, \&c.
}\switchcolumn\portugues{
Glória ao Pai, \&c.
}\switchcolumn*\latim{
\emph{Ant.} O admirabile commercium: Creator generis humani, animatum corpus sumens, de Virgine nasci dignatus est: et procedens homo sine semine, largitus est nobis suam Deitatem.
}\switchcolumn\portugues{
\emph{Ant.} Ó admirável permuta! O Criador do género humano, tomando corpo e alma, dignou-Se nascer duma Virgem; e, feito homem sem progenitor, tornou-nos participantes da sua divindade.
}\end{paracol}

\paragraphinfo{Salmo 109}{Página \pageref{salmo109}}

\begin{paracol}{2}\latim{
\emph{Ant.} O admirabile commercium: Creator generis humani, animatum corpus sumens, de Virgine nasci dignatus est: et procedens homo sine semine, largitus est nobis suam Deitatem.
}\switchcolumn\portugues{
\emph{Ant.} Ó admirável permuta! O Criador do género humano, tomando corpo e alma, dignou-Se nascer duma Virgem; e, feito homem sem progenitor, tornou-nos participantes da sua divindade.
}\end{paracol}

\begin{paracol}{2}\latim{
\emph{Ant.} Quando natus es inefabilitre ex Virgnine, tunc impletæ sunt Scripturæ: sicut pluvia in vellus descendisti, ut salvum faceres genus humanum: te laudamus, Deus noster.
}\switchcolumn\portugues{
\emph{Ant.} Quando nascestes misteriosamente da Virgem, então se cumpriram as Escrituras: descestes como a chuva sobre a lã, para salvar a humanidade. Nós Vos louvamos, ó Nosso Deus.
}\end{paracol}

\paragraphinfo{Salmo 112}{Página \pageref{salmo112}.}

\begin{paracol}{2}\latim{
\emph{Ant.} Quando natus es inefabilitre ex Virgnine, tunc impletæ sunt Scripturæ: sicut pluvia in vellus descendisti, ut salvum faceres genus humanum: te laudamus, Deus noster.
}\switchcolumn\portugues{
\emph{Ant.} Quando nascestes misteriosamente da Virgem, então se cumpriram as Escrituras: descestes como a chuva sobre a lã, para salvar a humanidade. Nós Vos louvamos, ó Nosso Deus.
}\end{paracol}

\begin{paracol}{2}\latim{
\emph{Ant.} Rubum, quem viderat Moyses incombustum, conservatam agnovimus tuam laudabilem virginitatem: Dei Genitrix, intercede pro nobis.
}\switchcolumn\portugues{
\emph{Ant.} Na sarça que Moisés via sem se consumirr, reconhecemos a vossa admirável virgindade conservada: rogai por nós, Santa Mãe de Deus.
}\end{paracol}

\paragraphinfo{Salmo 121}{Página \pageref{salmo121}}

\begin{paracol}{2}\latim{
\emph{Ant.} Rubum, quem viderat Moyses incombustum, conservatam agnovimus tuam laudabilem virginitatem: Dei Genitrix, intercede pro nobis.
}\switchcolumn\portugues{
\emph{Ant.} Na sarça que Moisés via sem se consumirr, reconhecemos a vossa admirável virgindade conservada: rogai por nós, Santa Mãe de Deus.
}\end{paracol}

\begin{paracol}{2}\latim{
\emph{Ant.} Germinavit radix Jesse, orta est stella ex Jacob; virgo peperit Salvatorem: te laudamus, Deus noster.
}\switchcolumn\portugues{
\emph{Ant.} Floresceu a raiz de Jessé, surgiu a estrela de Jacob. A Virgem deu à luz o Salvador: Nós Vos louvamos, ó Nosso Deus.
}\end{paracol}

\paragraphinfo{Salmo 126}{Página \pageref{salmo126}}

\begin{paracol}{2}\latim{
\emph{Ant.} Germinavit radix Jesse, orta est stella ex Jacob; virgo peperit Salvatorem: te laudamus, Deus noster.
}\switchcolumn\portugues{
\emph{Ant.} Floresceu a raiz de Jessé, surgiu a estrela de Jacob. A Virgem deu à luz o Salvador: Nós Vos louvamos, Senhor nosso Deus.
}\end{paracol}

\begin{paracol}{2}\latim{
\emph{Ant.} Ecce, Maria genuit nobis Salvatorem, quem Joannes videns exclamavit, dicens: Ecce Agnus Dei, ecce qui tollit peccata mundi, (allelúja).
}\switchcolumn\portugues{
\emph{Ant.} Maria deu à luz o nosso Salvador, que João reconheceu e exclamou: eis o Cordeiro de Deus, Aquele que tira o pecado do mundo, (aleluia).
}\end{paracol}

\paragraphinfo{Salmo 147}{Página \pageref{salmo147}}

\begin{paracol}{2}\latim{
\emph{Ant.} Ecce, Maria genuit nobis Salvatorem, quem Joannes videns exclamavit, dicens: Ecce Agnus Dei, ecce qui tollit peccata mundi, (allelúja).
}\switchcolumn\portugues{
\emph{Ant.} Maria deu à luz o nosso Salvador, que João reconheceu e exclamou: eis o Cordeiro de Deus, Aquele que tira o pecado do mundo, (aleluia).
}\end{paracol}

\paragraphinfo{Pequeno Capítulo}{Ecl. 24, 14}
\begin{paracol}{2}\latim{
\rlettrine{A}{b} initio et ante sæcula creata sum, et usque ad futurum sæculum non desinam, et in habitatione sancta coram ipso ministravi.
}\switchcolumn\portugues{
\rlettrine{E}{u} fui criada desde o princípio, antes dos séculos, e não deixarei de existir até ao fim dos séculos, e exerci diante dele o meu ministério na morada santa.
}\switchcolumn*\latim{
℟. Deo grátias.
}\switchcolumn\portugues{
℟. Graças a Deus.
}\end{paracol}

\paragraphinfo{Ave Maris Stella}{Página \pageref{avemariastella}}

\begin{paracol}{2}\latim{
℣. Diffusa est gratia in labiis tuis.
}\switchcolumn\portugues{
℣. A graça derramou-se nos vossos lábios.
}\switchcolumn*\latim{
℟. Propterea benedixit te Deus in æternum.
}\switchcolumn\portugues{
℟. Por isso vos abençoou Deus para sempre.
}\switchcolumn*\latim{
\emph{Ant.} Magnum hæreditatis mysterium: templum Dei factus est uterus nescientis virum: non est pollutus ex ea carnem assumens; omnes gentes venient, dicentes: Gloria tibi, Domine.
}\switchcolumn\portugues{
\emph{Ant.} Grande mystério de herança: o ventre daquela que não conheceu varão, é feito templo de Deus; o qual se não manchou, tomando dela carne humana. Virão todas as gentes, dizendo: Glória a Vós, ó Senhor.
}\end{paracol}

\paragraphinfo{Magnificat}{Página \pageref{magnificat}}

\begin{paracol}{2}\latim{
\emph{Ant.} Magnum hæreditatis mysterium: templum Dei factus est uterus nescientis virum: non est pollutus ex ea carnem assumens; omnes gentes venient, dicentes: Gloria tibi, Domine.
}\switchcolumn\portugues{
\emph{Ant.} Grande mystério de herança: o ventre daquela que não conheceu varão, é feito templo de Deus; o qual se não manchou, tomando dela carne humana. Virão todas as gentes, dizendo: Glória a Vós, ó Senhor.
}\switchcolumn*\latim{
℣. Domine, exaudi orationem meam.
}\switchcolumn\portugues{
℣. Ouvi, Senhor, a minha oração.
}\switchcolumn*\latim{
℟. Et clamor meus ad te veniat.
}\switchcolumn\portugues{
℟. E o meu clamor chegue até Vós.
}\switchcolumn*\latim{
\begin{nscenter} Orémus. \end{nscenter}
}\switchcolumn\portugues{
\begin{nscenter} Oremos. \end{nscenter}
}\switchcolumn*\latim{
\rlettrine{D}{eus,} qui salutis æternæ, beatæ Mariæ virginitate fœcunda, humano generi præmia præstitisti: tribue, quǽsumus; ut ipsam pro nobis intercedere sentiamus, per quam meruimus auctorem vitæ suscipere, Dominum nostrum Jesum Christum Filium tuum. Qui tecum vivit et regnat in unitate Spiritus Sancti, Deus, per omnia sæcula sæculorum.
}\switchcolumn\portugues{
\slettrine{Ó}{} Deus, que pela virgindade fecunda da bem-aventurada Maria, destes ao género humano as gratificações da salvação eterna: concedei-nos, Vos rogamos, que experienciemos sua intercessão por nós, dela pela qual recebemos o autor da vida, Nosso Senhor Jesus Cristo, vosso Filho. Que convosco, e com o Espírito Santo, vive e reina por todos os séculos.
}\switchcolumn*\latim{
℟. Amen.
}\switchcolumn\portugues{
℟. Amen.
}\switchcolumn*\latim{
℣. Domine, exaudi orationem meam.
}\switchcolumn\portugues{
℣. Ouvi, Senhor, a minha oração.
}\switchcolumn*\latim{
℟. Et clamor meus ad te veniat.
}\switchcolumn\portugues{
℟. E o meu clamor chegue até Vós.
}\switchcolumn*\latim{
℣. Benedicamus Domino.
}\switchcolumn\portugues{
℣. Bendigamos o Senhor.
}\switchcolumn*\latim{
℟. Deo gratias.
}\switchcolumn\portugues{
℟. Graças a Deus.
}\switchcolumn*\latim{
℣. Fidelium animæ per misericordiam Dei, requiescant in pace.
}\switchcolumn\portugues{
℣. E que as almas dos fiéis, pela misericórdia de Deus, descansem em paz.
}\switchcolumn*\latim{
℟. Amen.
}\switchcolumn\portugues{
℟. Amen.
}\end{paracol}

\emph{Acabar com uma Antífona de Nossa Senhora na página \pageref{antifonasnossasenhora}.}


\subsection{Completas 3}

\paragraphinfo{Pequeno Capítulo}{Ecl. 24}
\begin{paracol}{2}\latim{
\rlettrine{E}{go} mater pulchræ dilectionis, et timoris, et agnitionis, et sanctæ spei.
}\switchcolumn\portugues{
\rlettrine{E}{u} sou a Mãe do amor belo e do temor, e do conhecimento antigo, e da santa esperança.
}\switchcolumn*\latim{
℟. Deo grátias.
}\switchcolumn\portugues{
℟. Graças a Deus.
}\switchcolumn*\latim{
℣. Ora pro nobis sancta Dei Génetrix.
}\switchcolumn\portugues{
℣. Rogai por nós, Santa Mãe de Deus.
}\switchcolumn*\latim{
℟. Ut digni efficiamur promissionibus Christi.
}\switchcolumn\portugues{
℟. Para que sejamos dignos das promessas de Cristo.
}\switchcolumn*\latim{
\emph{Nunc. Ant.} Magnum hæreditatis mysterium: templum Dei factus est uterus nescientis virum: non est pollutus ex ea carnem assumens; omnes gentes venient, dicentes: Gloria tibi, Domine.
}\switchcolumn\portugues{
\emph{Nunc. Ant.} Grande mystério de herança: o ventre daquela que não conheceu varão, é feito templo de Deus; o qual se não manchou, tomando dela carne humana. Virão todas as gentes, dizendo: Glória a Vós, ó Senhor.
}\switchcolumn*\latim{
\begin{nscenter} Orémus. \end{nscenter}
}\switchcolumn\portugues{
\begin{nscenter} Oremos. \end{nscenter}
}\switchcolumn*\latim{
\rlettrine{D}{eus,} qui salutis æternæ, beatæ Mariæ virginitate fœcunda, humano generi præmia præstitisti: tribue, quǽsumus; ut ipsam pro nobis intercedere sentiamus, per quam meruimus auctorem vitæ suscipere, Dominum nostrum Jesum Christum Filium tuum. Qui tecum vivit et regnat in unitate Spiritus Sancti, Deus, per omnia sæcula sæculorum.
}\switchcolumn\portugues{
\slettrine{Ó}{} Deus, que pela virgindade fecunda da bem-aventurada Maria, destes ao género humano as gratificações da salvação eterna: concedei-nos, Vos rogamos, que experienciemos sua intercessão por nós, dela pela qual recebemos o autor da vida, Nosso Senhor Jesus Cristo, vosso Filho. Que convosco, e com o Espírito Santo, vive e reina por todos os séculos.
}\switchcolumn*\latim{
℟. Amen.
}\switchcolumn\portugues{
℟. Amen.
}\end{paracol}

\emph{Acabar com uma Antífona de Nossa Senhora na página \pageref{antifonasnossasenhora}.}

