\subsectioninfo{Missa dos Anjos}{Para a Terça-feira}

\paragraphinfo{Intróito}{Página \pageref{gabrielarcanjo}}

\paragraphinfo{Oração}{Página \pageref{aparicaoarcanjomiguel}}

\paragraphinfo{Epístola}{Ap. 5, 11-14}
\begin{paracol}{2}\latim{
Léctio libri Apocalýpsis beáti Joánnis Apóstoli.
}\switchcolumn\portugues{
Lição do Apocalipse do B. Ap.º João.
}\switchcolumn*\latim{
\rlettrine{I}{n} diébus illis: Audívi vocem Angelórum multórum in circúitu throni, et animálium, et seniórum: et erat númerus eórum mília mílium, dicéntium voce magna: Dignus est Agnus, qui occísus est, accípere virtútem, et divinitátem, et sapiéntiam, et fortitúdinem, et honórem, et glóriam, et benedictiónem. Et omnem creatúram, quæ in cœlo est, et super terram, et sub terra, et quæ sunt in mari, et quæ in eo: omnes audívi dicéntes: Sedénti in throno, et Agno: benedíctio, et honor, et glória, et potéstas in sǽcula sæculórum. Et quátuor animália dicébant: Amen. Et vigínti quátuor senióres cecidérunt in fácies suas: et adoravérunt vivéntem in sǽcula sæculórum.
}\switchcolumn\portugues{
\rlettrine{N}{aqueles} dias, ouvi em tomo do trono a voz de muitos Anjos e de anciãos. Havia milhares e milhares deles, os quais diziam com voz forte: «O Cordeiro, que foi morto, é digno de receber o poder, a divindade, a sabedoria, a força, a honra, a glória e as bênçãos!» E ouvi todas as criaturas, que estão no céu, na terra, debaixo da terra, no mar e em toda sua extensão que diziam: «Àquele que está assentado no trono e ao Cordeiro bênçãos, honra, glória em todos os séculos dos séculos!» E ouvi os quatro animais que diziam: «Amen!» E os vinte e quatro anciãos prostraram a face em terra e adoraram Aquele que vive em todos os séculos dos séculos!
}\end{paracol}

\paragraphinfo{Gradual}{Sl. 148,1-2}
\begin{paracol}{2}\latim{
\rlettrine{L}{audáte} Dóminum de cœlis: laudáte eum in excélsis. ℣. Laudáte eum, omnes Angeli ejus: laudáte eum, omnes virtútes ejus.
}\switchcolumn\portugues{
\rlettrine{L}{ouvai} o Senhor nos céus! Louvai-O nas alturas celestiais. ℣. Louvai-O vós, Anjos; louvai-O vós, que constituís o seu exército.
}\switchcolumn*\latim{
Allelúja, allelúja. ℣. \emph{Ps. 137, 1-2} In conspéctu Angelórum psallam tibi: adorábo ad templum sanctum tuum, et confitébor nómini tuo. Allelúja.
}\switchcolumn\portugues{
Aleluia, aleluia. ℣. \emph{Sl. 137, 1-2} Cantarei louvores em vossa honra diante dos Anjos; adorar-Vos-ei no vosso santo templo: e glorificarei o vosso nome. Aleluia.
}\end{paracol}

\textit{Depois da Septuagésima omite-se o Aleluia e o Verso que se segue, e diz-se:}

\paragraphinfo{Trato}{Sl. 102, 20}
\begin{paracol}{2}\latim{
\rlettrine{B}{enedícite} Dóminum, omnes Angeli ejus: potentes virtúte, qui fácitis verbum ejus. ℣. \emph{ibid., 21-22} Benedícite Dómino, omnes virtútes ejus: minístri ejus, qui fácitis voluntátem ejus. ℣. Benedicite Dómino, ómnia ópera ejus: in omni loco dominatiónis ejus, bénedic, ánima mea, Dómino.
}\switchcolumn\portugues{
\rlettrine{B}{endizei} o Senhor, ó vós, todos os Anjos, heróis poderosos, executores das suas ordens e sempre fiéis aos seus chamamentos. ℣. \emph{ibid., 21-22}Bendizei o Senhor, vós todos, que formais o seu exército; sois os seus ministros e cumpris a sua vontade! Bendizei o Senhor, ó obras todas do mesmo Senhor! Bendizei o Senhor em todos os lugares do seu domínio, ó minha alma.
}\end{paracol}

\textit{No Tempo Pascal omite-se o Gradual e o Trato, e diz-se:}

\begin{paracol}{2}\latim{
Allelúja, allelúja. ℣. \emph{Ps. 137, 1-2} In conspéctu Angelórum psallam tibi: adorábo ad templum sanctum tuum, et confitébor nómini tuo Allelúja. ℣. \emph{Matth. 28, 2} Angelus Dómini descéndit de cœlo, et accédens revólvit lápidem, et sedébat super eum. Allelúja.
}\switchcolumn\portugues{
Aleluia, aleluia. ℣. \emph{Sl. 137, 1-2} Cantarei louvores em vossa honra diante dos Anjos; adorar-Vos-ei no vosso santo templo; e glorificarei o vosso nome. Aleluia. ℣. \emph{Mt. 28, 2} Um Anjo do Senhor, havendo descido do céu, afastou a pedra e assentou-se sobre ela. Aleluia.
}\end{paracol}

\paragraphinfo{Evangelho}{Jo. 1, 47-51}
\begin{paracol}{2}\latim{
\cruz Sequéntia sancti Evangélii secúndum Joánnem.
}\switchcolumn\portugues{
\cruz Continuação do santo Evangelho segundo S. João.
}\switchcolumn*\latim{
\blettrine{I}{n} illo témpore: Vidit Jesus Nathánaël veniéntem ad se, et dicit de eo: Ecce vere Israëlíta, in quo dolus non est. Dicit ei Nathánaël: Unde me nosti? Respóndit Jesus et dixit ei: Priúsquam te Philíppus vocáret, cum esses sub ficu, vidi te. Respóndit ei Nathánaël et ait: Rabbi, tu es Fílius Dei, tu es Rex Israël. Respóndit Jesus et dixit ei: Quia dixi tibi: Vidi te sub ficu, credis: majus his vidébis. Et dicit ei: Amen, amen, dico vobis, vidébitis cœlum apértum, et Angelos Dei ascendéntes, et descendéntes supra Fílium hóminis.
}\switchcolumn\portugues{
\blettrine{N}{aquele} tempo, Jesus viu Natánael que vinha para Ele, e disse a seu respeito: «Eis aí um verdadeiro Israelita no qual não há dolo». E Natánael disse-Lhe: «Donde me conheceis?». Jesus respondeu-lhe, dizendo: «Antes que Filipe te chamasse, vi-te Eu quando tu estavas debaixo da figueira». Respondeu então Natánael: «Rabi (Mestre) sois o filho de Deus; sois o Rei de Israel?». E Jesus disse-lhe: «Porque te disse: vi-te debaixo da figueira, acreditaste; pois ainda verás cousas maiores». Depois acrescentou: «Em verdade, em verdade te digo: verás o céu aberto e os Anjos de Deus, subindo e descendo sobre o Filho do homem».
}\end{paracol}

\paragraphinfo{Ofertório}{Página \pageref{gabrielarcanjo}}

\paragraphinfo{Secreta}{Página \pageref{aparicaoarcanjomiguel}}

\paragraph{Comúnio}
\begin{paracol}{2}\latim{
\rlettrine{A}{ngeli,} Archángeli, Throni et Dominatiónes, Principátus et Potestátes, Virtútes cœlórum, Chérubim atque Séraphim, Dóminum benedícite in ætérnum. (T. P. Allelúja.)
}\switchcolumn\portugues{
\rlettrine{B}{endizei} o Senhor em todos os séculos, ó Anjos, Arcanjos, Tronos e Dominações, Principados, Potestades, Virtudes do céu, Querubins e Serafins. (T. P. Aleluia.)
}\end{paracol}

\paragraph{Postcomúnio}
\begin{paracol}{2}\latim{
\rlettrine{R}{epléti,} Dómine, benedictióne cœlésti, supplíciter implorámus: ut, quod fragili celebrámus offício, sanctórum Angelórum atque Archangelórum nobis prodésse sentiámus auxílio. Per Dóminum nostrum \emph{\&c.}
}\switchcolumn\portugues{
\rlettrine{S}{enhor,} estando nós, agora, repletos com a bênção celestial, permiti Vos suplicamos, que este sacrifício, que celebramos apesar da nossa fragilidade, nos seja útil com o auxílio dos vossos Santos Anjos e Arcanjos. Por nosso Senhor \emph{\&c.}
}\end{paracol}
