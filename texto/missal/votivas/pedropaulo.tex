\subsectioninfo{Missa dos S. S. Apóstolos Pedro e Paulo}{Para a Quarta-feira}

\paragraphinfo{Intróito}{Página \pageref{andreapostolo}}

\paragraph{Oração}
\begin{paracol}{2}\latim{
\rlettrine{D}{eus,} cujus déxtera beátum Petrum, ambulántem in flúctibus, ne mergerétur, eréxit, et coapóstolum ejus Paulum, tértio naufragántem, de profúndo pélagi liberávit: exáudi nos propítius, et concéde; ut, ambórum méritis, æternitátis glóriam consequámur: Qui vivis et regnas \emph{\&c.}
}\switchcolumn\portugues{
\slettrine{Ó}{} Deus, cuja mão poderosa sustentou o B. Pedro quando caminhava sobre as águas, não deixando que ele se afundasse, e salvou do fundo do mar o seu companheiro de apostolado, Paulo, quando este naufragou pela terceira vez, ouvi-nos propício, a fim de que, pelos méritos destes dous Apóstolos, obtenhamos a glória eterna. Ó Vós, que \emph{\&c.}
}\end{paracol}

\paragraphinfo{Epístola}{Act. 5, 12-16}
\begin{paracol}{2}\latim{
Léctio Actuum Apostolorum.
}\switchcolumn\portugues{
Lição dos Actos dos Apóstolos.
}\switchcolumn*\latim{
\rlettrine{I}{n} diébus illis: Per manus Apostolórum fiébant signa et prodígia multa in plebe. Et erant unanímiter omnes in pórticu Salomónis. Ceterórum autem nemo audébat se conjúngere illis: sed magnificábat eos pópulus. Magis autem augebátur credéntium in Dómino multitúdo virórum ac mulíerum, ita ut in pláteas ejícerent infírmos, et pónerent in léctulis ac grabátis, ut, veniénte Petro, saltem umbra illíus obumbráret quemquam illórum, et liberaréntur ab infirmitátibus suis. Concurrébat autem et multitúdo vicinárum civitátum Jerúsalem, afferéntes ægros et vexátos a spirítibus immúndis: qui curabántur omnes.
}\switchcolumn\portugues{
\rlettrine{N}{aqueles} dias, os Apóstolos praticavam muitos milagres e prodígios no meio do povo, conservando-se todos nas galerias de Salomão com o mesmo pensamento. Nenhum dos outros ousava juntar-se a eles, embora o povo lhes tecesse muitos louvores. E o número daqueles, tantos homens como mulheres, que acreditavam no Senhor, aumentava cada vez mais. E conduziam os doentes para as praças em leitos e macas para que, quando Pedro passasse, ao menos a sua sombra cobrisse alguns deles e ficassem sarados de suas enfermidades. Acorriam até das cidades vizinhas de Jerusalém muitas pessoas, trazendo enfermos, vindo também outros que estavam atormentados pelos espíritos imundos. E todos eram curados.
}\end{paracol}

\paragraphinfo{Gradual}{Sl. 44, 17 \& 18}
\begin{paracol}{2}\latim{
\rlettrine{C}{onstítues} eos príncipes super omnem terram: mémores erunt nóminis tui, Dómine. ℣. Pro pátribus tuis nati sunt tibi fílii: proptérea pópuli confitebúntur tibi.
}\switchcolumn\portugues{
\rlettrine{V}{ós} os instituístes príncipes em todo o universo: e eles perpetuarão a glória do vosso nome, Senhor, em toda a terra. ℣. Para substituir os vossos pais, nascer-vos-ão filhos: pelo que os povos vos louvarão.
}\switchcolumn*\latim{
Allelúja, allelúja. ℣. \emph{Ps. 138, 17} Nimis honoráti sunt amíci tui, Deus: nimis confortátus est principátus eórum. Allelúja.
}\switchcolumn\portugues{
Aleluia, aleluia. ℣. \emph{Sl. 138, 17} Honrais largamente os vossos amigos, ó Deus; o seu poder tem-se fortalecido extraordinariamente. Aleluia.
}\end{paracol}

\textit{Depois da Septuagésima omite-se o Aleluia e o que se segue, e diz-se o:}

\paragraphinfo{Trato}{Sl. 125, 5-6}
\begin{paracol}{2}\latim{
\qlettrine{Q}{ui} séminant in lácrimis, in gáudio metent. ℣. Eúntes ibant et flébant, mitténtes sémina sua. ℣.
Veniéntes autem vénient cum exsultatióne, portántes manípulos suos.
}\switchcolumn\portugues{
\rlettrine{A}{queles} que semeiam com lágrimas, colherão com risos. ℣. Iam chorando e lançando à terra as suas sementes: ℣. Mas, quando regressaram, vinham alegres, transportando feixes do seu trigo.
}\end{paracol}

\paragraphinfo{Evangelho}{Página \pageref{abades}}

\paragraphinfo{Ofertório}{Sl. 18, 5}
\begin{paracol}{2}\latim{
\rlettrine{I}{n} omnem terram exivit sonus eórum: et in fines orbis terræ verba eórum.
}\switchcolumn\portugues{
\rlettrine{O}{} som da sua voz ecoou por toda a terra: e as suas palavras estenderam-se até às extremidades da terra.
}\end{paracol}

\paragraph{Secreta}
\begin{paracol}{2}\latim{
\rlettrine{O}{fférimus} tibi, Dómine, preces et múnera: quæ ut tuo sint digna conspéctu. Apostolórum tuórum Petri et Pauli précibus adjuvémur. Per Dóminum \emph{\&c.}
}\switchcolumn\portugues{
\rlettrine{S}{enhor,} Vos oferecemos as nossas orações e oblatas; e, para que sejam dignas dos vossos olhares, fazei que os vossos Apóstolos Pedro e Paulo as acompanhem com suas preces. Por nosso Senhor \emph{\&c.}
}\end{paracol}

\paragraphinfo{Comúnio}{Mt. 19, 28}
\begin{paracol}{2}\latim{
\rlettrine{V}{os,} qui secuti estis me, sedebitis super sedes, judicantes duodecim tribus Israel.
}\switchcolumn\portugues{
\slettrine{Ó}{} vós, que me acompanhastes, assentar-vos-eis em doze tronos e julgareis as doze tribos de Israel.
}\end{paracol}

\paragraph{Postcomúnio}
\begin{paracol}{2}\latim{
\rlettrine{P}{rótege,} Dómine, pópulum tuum: et Apostolórum tuórum Petri et Pauli patrocínio confidéntem, perpétua defensióne consérva. Per Dóminum nostrum \emph{\&c.}
}\switchcolumn\portugues{
\rlettrine{P}{rotegei} o vosso povo, Senhor, e, visto que ele se coloca sob o patrocínio dos vossos Apóstolos Pedro e Paulo, dignai-Vos defendê-lo e guardá-lo perpetuamente. Por nosso Senhor \emph{\&c.}
}\end{paracol}
