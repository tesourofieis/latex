\subsection{Pela Remissão dos Pecados}

\paragraphinfo{Intróito}{Sb. 11,24, 25 \& 27}
\begin{paracol}{2}\latim{
\rlettrine{M}{iseréris} ómnium, Dómine, et nihil odísti eórum, quæ fecísti: dissímulans peccáta hóminum propter pœniténtiam, et parcens illis: quia tu es Dóminus, Deus noster. (T. P. Allelúja, allelúja) \emph{Ps. 56, 2} Miserére mei, Deus, miserére mei: quóniam in te confídit ánima mea.
℣. Gloria Patri \emph{\&c.}
}\switchcolumn\portugues{
\rlettrine{T}{endes} misericórdia de todos, Senhor, e a nenhum daqueles que criastes, tendes ódio: quando os homens fazem penitência, lançais um véu sobre os seus pecados e perdoais-lhes, porquanto sois o Senhor, nosso Deus. (T. P. Aleluia, aleluia.) \emph{Sl. 56, 2} Tende misericórdia de mim, ó Deus, tende misericórdia de mim, porque a minha alma em Vós confia.
℣. Glória ao Pai \emph{\&c.}
}\end{paracol}

\paragraph{Oração}
\begin{paracol}{2}\latim{
\rlettrine{E}{xáudi,} quǽsumus, Dómine, súpplicum preces, et confiténtium tibi parce peccátis: ut páriter nobis indulgéntiam tríbuas benígnus et pacem. Per Dóminum \emph{\&c.}
}\switchcolumn\portugues{
\rlettrine{S}{enhor,} Vos rogamos, ouvi as preces dos suplicantes e perdoai os pecados destes que se confessam pecadores, a fim de que, benigno, lhes concedais ao mesmo tempo a indulgência e a paz. Por nosso Senhor \emph{\&c.}
}\end{paracol}

\paragraphinfo{Epístola}{Rm. 7, 22-25}
\begin{paracol}{2}\latim{
Léctio Epístolæ beáti Pauli Apóstoli ad Romános.
}\switchcolumn\portugues{
Lição da Ep.ª do B. Ap.º Paulo aos Romanos.
}\switchcolumn*\latim{
\rlettrine{F}{ratres:} Condeléctor legi Dei secúndum interiórem hóminem: video autem áliam legem in membris meis, repugnántem legi mentis meæ, et captivántem me in lege peccáti, quæ est in membris meis. Infélix ego homo, quis me liberábit de córpore mortis hujus? Grátia Dei per Jesum Christum, Dóminum nostrum.
}\switchcolumn\portugues{
\rlettrine{M}{eus} irmãos: Como homem de vida interior, delicio-me; contudo sinto nos meus membros uma outra lei, que repugna à lei dos meus membros. Desditoso de mim, que sou homem! Quem me livrará deste corpo de morte? Graças sejam dadas a Deus, por Jesus Cristo, nosso Senhor.
}\end{paracol}

\paragraphinfo{Gradual}{Sl. 78, 9-10}
\begin{paracol}{2}\latim{
\rlettrine{P}{ropítius} esto, Dómine, peccátis nostris, ne quando dicant gentes: Ubi est Deus eórum? ℣. ibid., 9. Adjuva nos, Deus, salutáris noster: et propter honórem nóminis tui, Dómine, líbera nos.
}\switchcolumn\portugues{
\rlettrine{S}{ede} propício para com os nossos pecados, Senhor, para que não digam os pagãos: onde está o seu Deus? Auxiliai-nos, ó Deus, nosso Salvador: e por causa do vosso nome, Senhor, livrai-nos.
}\switchcolumn*\latim{
Allelúja, allelúja. ℣. \emph{Ps. 7, 12} Deus judex justus, fortis et pátiens: numquid irascátur per síngulos dies? Allelúja.
}\switchcolumn\portugues{
Aleluia, aleluia. Deus é juiz justo, forte e paciente: porventura irar-se-á todos os dias? Aleluia.
}\end{paracol}

\paragraphinfo{Trato}{Sl. 129, 1-4}
\begin{paracol}{2}\latim{
\rlettrine{D}{e} profúndis clamávi ad te, Dómine: Dómine, exáudi vocem meam. ℣. Fiant aures tuæ intendéntes in oratiónem servi tui. ℣. Si iniquitátes observáveris, Dómine: Dómine, quis sustinébit? ℣. Quia apud te propitiátio est: et propter legem tuam sustínui te, Dómine.
}\switchcolumn\portugues{
\rlettrine{D}{as} profundezas do abismo, Senhor, clamo por Vós; ouvi a minha oração, Senhor. ℣. Estejam os vossos ouvidos atentos à oração do vosso servo. ℣. Se guardardes a lembrança dos nossos delitos, Senhor, quem poderá subsistir ante Vós, Senhor? ℣. Pois nas vossas mãos está o perdão: e por causa da vossa lei ousei comparecer ante Vós, Senhor.
}\end{paracol}

\textit{No Tempo Pascal omite-se o Gradual e o Trato, dizendo-se:}

\begin{paracol}{2}\latim{
Allelúja, allelúja. ℣. \emph{Ps. 7, 12} Deus judex justus, fortis et pátiens: numquid irascétur per síngulos dies? Allelúja. ℣. \emph{Ps. 50, 10} Audítui meo dabis gáudium et lætítiam: et exsultábunt ossa humiliáta. Allelúja.
}\switchcolumn\portugues{
Aleluia, aleluia. ℣. \emph{Sl. 7, 12} Deus é juiz justo, forte e paciente. Porventura irar-se-á todos os dias? Aleluia. ℣. \emph{Sl. 50, 10} Dareis gozo e alegria aos meus ouvidos: e meus ossos humilhados exultarão de contentamento. Aleluia.
}\end{paracol}

\paragraphinfo{Evangelho}{Lc. 11, 9-13}
\begin{paracol}{2}\latim{
\cruz Sequéntia sancti Evangélii secúndum Lucam.
}\switchcolumn\portugues{
\cruz Continuação do santo Evangelho segundo S. Lucas.
}\switchcolumn*\latim{
\blettrine{I}{n} illo témpore: Dixit Jesus discípulis suis: Pétite, et dábitur vobis: quǽrite, et inveniétis: pulsáte, et aperiétur vobis. Omnis enim, qui petit, áccipit: et qui quærit, invénit: et pulsánti aperietur, Quis autem ex vobis patrem pétii panem, numquid lápidem dabit illi? Aut piscem: numquid pro pisce serpéntem dabit illi? Aut si petíerit ovum: numquid pórriget illi scorpiónem? Si ergo vos, cum sitis mali, nostis bona data dare fíliis vestris: quanto magis Pater vester de cœlo dabit spíritum bonum peténtibus se?
}\switchcolumn\portugues{
\blettrine{N}{aquele} tempo, disse Jesus aos seus discípulos: «Pedi e recebereis; buscai e encontrareis; batei e abrir-se-vos-á. Porquanto todo aquele que pedir receberá; todo aquele, que procurar achará; todo aquele que bater abrir-se-lhe-á. Se algum de vós pedir um pão a seu pai, porventura este lhe dará uma pedra? Ou, se lhe pedir um peixe, dar-lhe-á uma serpente? Ou, se lhe pedir um ovo, dar-lhe-á um escorpião? Pois se vós, sendo maus, sabeis, contudo, dar coisas boas a vossos filhos, quanto mais vosso Pai celestial dará o Espírito Santo àqueles que lho pedirem».
}\end{paracol}

\paragraphinfo{Ofertório}{Sl. 101, 2}
\begin{paracol}{2}\latim{
\rlettrine{D}{ómine,} exáudi oratiónem meam: et clamor meus ad te pervéniat. (T. P. Allelúja.)
}\switchcolumn\portugues{
\rlettrine{O}{uvi} a minha oração, Senhor: e que meu clamor chegue até Vós. (T. P. Aleluia).
}\end{paracol}

\paragraph{Secreta}
\begin{paracol}{2}\latim{
\rlettrine{H}{óstias} tibi, Dómine, placatiónis et laudis offérimus: ut et delícta nostra miserátus absólvas, et nutántia corda tu dírigas. Per Dóminum nostrum \emph{\&c.}
}\switchcolumn\portugues{
\rlettrine{V}{os} oferecemos, Senhor, estas hóstias de louvor e de paz, a fim que, misericordioso, nos absolvais dos nossos delitos, e amparareis os nossos corações vacilantes. Por nosso Senhor \emph{\&c.}
}\end{paracol}

\paragraphinfo{Comúnio}{Lc. 11, 9-10}
\begin{paracol}{2}\latim{
\rlettrine{P}{etite,} et accipiétis; quǽrite, et inveniétis; pulsáte, et aperiétur vobis. Omnis enim, qui pétii, áccipit; et qui quærit, invénit; et pulsánti aperiétur. (T. P. Allelúja.)
}\switchcolumn\portugues{
\rlettrine{P}{edi} e recebereis; procurai e encontrareis; batei e abrir-se-vos-á. Todo aquele que pede, recebe; e todo aquele que procura, encontra; e todo aquele que bate, abrir-se-lhe-á. (T. P. Aleluia.)
}\end{paracol}

\paragraph{Postcomúnio}
\begin{paracol}{2}\latim{
\rlettrine{P}{ræsta} nobis, ætérne Salvátor: ut, percipiéntes hoc múnere véniam peccatórum, deínceps peccáta vitémus. Per Dóminum \emph{\&c.}
}\switchcolumn\portugues{
\rlettrine{C}{oncedei-nos,} ó eterno Salvador, que, encontrando nós neste dom o perdão dos pecados, doravante evitemos os pecados. Por nosso Senhor \emph{\&c.}
}\end{paracol}
