\subsection{No Dia da Ordenação de Presbítero}

\textit{Como na Missa do dia, juntando-se, o seguinte, sob a mesma conclusão da que antecede:}

\paragraph{Oração}
\begin{paracol}{2}\latim{
\rlettrine{E}{xáudi,} quǽsumus, Dómine, súpplicum preces, et de voto tibi péctore famulántes perpétua defensióne custódi: ut, nullis perturbatiónibus impedíti, líberam servitútem tuis semper exhibeámus offíciis. Per Dóminum \emph{\&c.}
}\switchcolumn\portugues{
\rlettrine{O}{uvi,} Senhor, Vos imploramos, as preces destes vossos servos suplicantes e guardai-nos perpetuamente, a fim de que, livres de todo o temor, exerçamos com toda a liberdade o nosso ministério. Por nosso Senhor \emph{\&c.}
}\end{paracol}

\paragraph{Secreta}
\begin{paracol}{2}\latim{
\rlettrine{T}{uis,} quǽsumus, Dómine, operáre mystériis: ut hæc tibi múnera dignis méntibus offerámus. Per Dóminum \emph{\&c.}
}\switchcolumn\portugues{
\rlettrine{S}{enhor,} Vos suplicamos, fazei que estes mistérios tornem dignos de Vós estes dons, que Vos oferecemos. Por nosso Senhor \emph{\&c.}
}\end{paracol}

\paragraph{Postcomúnio}
\begin{paracol}{2}\latim{
\qlettrine{Q}{uos} tuis, Dómine, réficis sacraméntis, contínuis attólle benígnus auxíliis: ut tuæ redemptiónis efféctum, et mystériis capiámus et móribus: Qui vivis \emph{\&c.}
}\switchcolumn\portugues{
\rlettrine{F}{ortificai,} Senhor, com vossas incessantes graças aqueles que benignamente alimentastes com vossos sacramentos, a fim de que experimentemos o efeito da Redenção, tanto pela virtude destes mistérios, como pela nossa própria conduta. Ó Vós, que, sendo Deus \emph{\&c.}
}\end{paracol}
