\subsectioninfo{Missa de Todos os S. S. Apóstolos}{Para Quarta-feira Fora do Tempo Pascal}

\textit{Tudo como na Missa precedente, excepto o seguinte:}

\paragraphinfo{Oração, Secreta e Postcomúnio}{Página \pageref{simaojudas}}

\paragraphinfo{Epístola}{Ef. 4, 7-13}
\begin{paracol}{2}\latim{
Léctio Epístolæ beáti Pauli Apóstoli ad Ephésios.
}\switchcolumn\portugues{
Lição da do B. Ap.º Paulo aos Efésios.
}\switchcolumn*\latim{
\rlettrine{F}{ratres:} Unicuíque nostrum data est grátia secúndum mensúram donatiónis Christi. Propter quod dicit: Ascéndens in altum captívam duxit captivitátem: dedit dona homínibus. Quod autem ascéndit, quid est, nisi quia et descéndit primum in inferióres partes terræ? Qui descéndit, ipse est et qui ascéndit super omnes cœlos, ut impléret ómnia. Et ipse dedit quosdam quidem apóstolos, quosdam autem prophétas, alios vero vero evangelístas, alios autem pastóres, et doctores ad consummatiónem sanctó rum in opus ministérii, in ædificatiónem córporis Christi: donec occurrámus omnes in unitátem fídei et agnitiónis Fílii Dei, in virum perféctum, in mensúram ætatis plenitúdinis Christi.
}\switchcolumn\portugues{
\rlettrine{M}{eus} irmãos: A graça foi dada a cada um de nós segundo a medida do dom de Jesus Cristo. É por isso que a Escritura diz: «Subindo ao céu, Ele levou consigo muitos cativos e distribuiu dons pelos homens». Mas, porque foi que subiu, senão porque, também, antes descera aos lugares mais baixos da terra? Aquele que desceu, foi o mesmo que subiu acima de todos os céus, para completar todas as coisas. Foi Ele, também, quem destinou uns para apóstolos, outros para profetas, estes para evangelistas, aqueles para pastores e doutores para a perfeição dos santos, para o trabalho do ministério e para a edificação do corpo de Cristo, até que nós todos cheguemos à unidade, da fé e do conhecimento do Filho de Deus, ao estado da perfeição humana, à medida da plenitude de Cristo.
}\end{paracol}

\subsectioninfo{Missa de Todos os S. S. Apóstolos}{Para Quarta-feira Dentro do Tempo Pascal}

\textit{Tudo como na Missa precedente, excepto o seguinte:}

\paragraphinfo{Intróito}{Sl. 63, 3}
\begin{paracol}{2}\latim{
\rlettrine{P}{rotexísti} me, Deus, a convéntu malignántium, allelúja: a multitúdine operántium iniquitátem, allelúja, allelúja. \emph{Ps. ib., 2} Exáudi, Dómine, orationem meam, cum déprecor: a timóre inimíci éripe ánimam meam.
℣. Gloria Patri \emph{\&c.}
}\switchcolumn\portugues{
\slettrine{Ó}{} Deus, livrastes-me da companhia dos maus, aleluia: livrastes-me do meio daqueles que procedem com iniquidade. Aleluia, aleluia. \emph{Sl. ib., 2} Senhor, quando eu Vos invocar, ouvi a minha oração: livrai a minha alma do temor do inimigo.
℣. Glória ao Pai \emph{\&c.}
}\end{paracol}

\textit{Depois da Epistola:}

\begin{paracol}{2}\latim{
Allelúja, allelúja. ℣. \emph{Ps. 88, 6} Confitebúntur cœli mirabília tua, Dómine: étenim veritátem tuam in ecclésia sanctórum. Allelúja. ℣. \emph{Joann. 15, 16} Ego vos elégi de mundo, ut eátis, et fructum afferátis, et fructus vester máneat. Allelúja.
}\switchcolumn\portugues{
Aleluia, aleluia. ℣. \emph{Sl. 88, 6} Senhor, os céus proclamam as vossas maravilhas e a vossa verdade na assembleia dos Santos. Aleluia. ℣. \emph{Jo. 15, 16} Fui Eu quem vos escolheu no mundo, a fim de que possais ir (pelo mundo), alcanceis frutos e esses frutos permaneçam Aleluia.
}\end{paracol}

\paragraphinfo{Ofertório}{Sl. 44, 17-18}
\begin{paracol}{2}\latim{
\rlettrine{C}{onstítues} eos príncipes super omnem terram: mémores erunt nóminis tui, Dómine, in omni progénie et generatióne, allelúja, allelúja.
}\switchcolumn\portugues{
\rlettrine{V}{ós} os instituístes príncipes em toda a terra: eles se recordarão do vosso nome em todas as gerações. Aleluia, aleluia.
}\end{paracol}

\paragraphinfo{Comúnio}{Sl. 18, 5}
\begin{paracol}{2}\latim{
\rlettrine{I}{n} omnem terram exívit sonus eórum: et in fines orbis terræ verba eórum, allelúja, allelúja.
}\switchcolumn\portugues{
\rlettrine{O}{} som da sua voz ecoa por toda a terra, fazendo-se ouvir as suas palavras até às extremidades do mundo, aleluia, aleluia.
}\end{paracol}
