\subsectioninfo{N. S. Jesus Cristo Sumo e Eterno Sacerdote}{Para a Quinta-feira}

\paragraphinfo{Intróito}{Sl. 109, 4}
\begin{paracol}{2}\latim{
\qlettrine{J}{urávit} Dóminus, et non pœnitébit eum: Tu es sacérdos in ætérnum secúndum órdinem Melchísedech. (T. P. Allelúja, allelúja.) \emph{Ps. ibid., 1} Dixit Dóminus Dómino meo: Sede a dextris meis.
℣. Gloria Patri \emph{\&c.}
}\switchcolumn\portugues{
\qlettrine{J}{urou} o Senhor, e não se arrependerá: Tu és sacerdote eternamente, segundo a ordem de Melquisedeque (T. P. Aleluia, aleluia). Disse o Senhor ao meu Senhor: Assenta-te à minha direita.
℣. Glória ao Pai \emph{\&c.}
}\end{paracol}

\paragraph{Oração}
\begin{paracol}{2}\latim{
\rlettrine{D}{eus,} qui, ad majestátis tuæ glóriam et géneris humáni salútem, Unigénitum tuum summum atque ætérnum constituísti Sacerdótem: præsta; ut, quos minístros et mysteriórum suórum dispensatóres elégit, in accépto ministério adimpléndo fidéles inveniántur. Per eúndem Dóminum \emph{\&c.}
}\switchcolumn\portugues{
\slettrine{Ó}{} Deus, que para glória da vossa majestade e salvação do género humano constituístes o vosso Unigénito Filho sumo e eterno sacerdote, fazei que aqueles que Ele escolheu como ministros e dispensadores dos seus mistérios, sejam fiéis no cumprimento do ministério recebido. Pelo mesmo nosso \emph{\&c.}
}\end{paracol}

\paragraphinfo{Epístola}{Heb. 5, 1-11}
\begin{paracol}{2}\latim{
Léctio Epístolæ beáti Pauli Apóstoli ad Hebrǽos.
}\switchcolumn\portugues{
Lição da Ep.ª do B, Ap.º Paulo aos Hebreus.
}\switchcolumn*\latim{
\rlettrine{F}{ratres:} Omnis póntifex ex homínibus assúmptus, pro homínibus constitúitur in iis, quæ sunt ad Deum, ut ófferat dona, et sacrifícia pro peccátis: qui condolére possit iis, qui ígnorant, et errant: quóniam et ipse circúmdatus est infirmitáte: et proptérea debet, quemádmodum pro pópulo, ita étiam et pro semetípso offérre pro peccátis. Nec quisquam sumit sibi honórem, sed qui vocátur a Deo, tamquam Aaron. Sic et Christus non semetípsum clarificávit ut póntifex fíeret, sed qui locútus est ad eum: Fílius meus es tu; ego hódie génui te. Quemádmodum et in alio loco dicit: Tu es sacérdos in ætérnum secúndum órdinem Melchísedech. Qui in diébus carnis suæ preces supplicationésque ad eum, qui possit illum salvum fácere a morte, cum clamóre válido et lácrimis ófferens, exaudítus est pro sua reveréntia. Et quidem, cum esset Fílius Dei, dídicit ex iis, quæ passus est, obœdiéntiam: et consummátus, factus est ómnibus obtemperántibus sibi, causa salútis ætérnæ, appelátus a Deo póntifex juxta órdinem Melchísedech. De quo nobis grandis sermo, et ininterpretábilis ad dicéndum.
}\switchcolumn\portugues{
\rlettrine{M}{eus} irmãos: Todo o pontífice é escolhido entre os homens e estabelecido para os homens no que respeita às suas relações com Deus, a fim de que ofereça dons e sacrifícios pelos pecados, e se compadeça daqueles que pecam por ignorância e por erro, lembrando-se de que também está cheio de fraquezas e deve oferecer sacrifícios de expiação dos pecados por si e pelo povo. Ninguém assuma por si próprio esta honra, mas espere que seja chamado por Deus, como Aarão; pois Cristo não assumiu por si próprio a glória do pontificado, mas recebeu-a d’Aquele que Lhe disse: «Tu és o meu Filho; gerei-te hoje». E também Lhe disse em outra ocasião: «Tu és sacerdote para sempre, segundo a ordem de Melquisedeque», o qual, nos dias da sua Carne, oferecendo com grande clamor e com lágrimas, preces e súplicas a Quem o podia salvar da morte, foi atendido pela sua reverência; e, embora fosse Filho de Deus, aprendeu a obediência por aquilo que sofreu; e pela sua imolação tornou-se a causa da salvação eterna para todos os que Lhe obedecem, sendo chamado por Deus Pontífice segundo a ordem de Melquisedeque: sobre cujo assunto tínhamos muito a dizer a respeito de coisas difíceis de explicar.
}\end{paracol}

\paragraphinfo{Gradual}{Lc. 4, 18}
\begin{paracol}{2}\latim{
\rlettrine{O}{} Espírito do Senhor repousou sobre mim: e ungiu-me. Enviou-me a evangelizar os pobres e a sarar os contritos de coração. Aleluia, aleluia. Jesus, porque permanece para sempre, tem um sacerdócio sempiterno. Aleluia.
}\switchcolumn\portugues{
\rlettrine{S}{píritus} Dómini super me: propter quod unxit me. ℣. Evangelizáre paupéribus misit me, sanáre contrítos corde.
}\end{paracol}

\textit{Após a Septuagésima, omite-se o Aleluia e o seguinte, e diz-se:}

\paragraphinfo{Trato}{Sl. 9, 34 \& 36}
\begin{paracol}{2}\latim{
\rlettrine{E}{xsúrge,} Dómine Deus, exaltétur manus tua: ne obliviscáris páuperum. ℣. Vide quóniam tu labórem et dolórem consíderas: ℣. Tibi derelíctus est pauper: órphano tu eris adjútor.
}\switchcolumn\portugues{
\rlettrine{E}{rguei-Vos,} ó Senhor Deus: elevai a vossa mão: não olvideis os pobres. ℣. Pois Vós apreciais e considerais o trabalho e a dor. ℣. A Vós se abandona o infeliz: sois amparo do órfão.
}\end{paracol}

\textit{No Tempo Pascal omite-se o Gradual e o Trato, e diz-se:}

\begin{paracol}{2}\latim{
Allelúja, allelúja. ℣. \emph{Hebr. 7, 24} Jesus autem eo quod máneat in ætérnum, sempitérnum habet sacerdótium. Allelúja. ℣. \emph{Luc. 4, 18} Spíritus Dómini super me: propter quod unxit me, evangelizáre paupéribus misit me, sanáre contrítos corde. Allelúja.
}\switchcolumn\portugues{
Aleluia, aleluia. ℣. \emph{Heb. 7, 24} Jesus, porque permanece para sempre, tem um sacerdócio sempiterno. Aleluia. ℣. \emph{Lc. 4, 18} O Espírito do Senhor repousou sobre mim: e ungiu-me, mandando-me evangelizar os pobres e sarar os contritos de coração. Aleluia.
}\end{paracol}

\paragraphinfo{Evangelho}{Lc. 22, 14-20}
\begin{paracol}{2}\latim{
\cruz Sequéntia sancti Evangélii secúndum Lucam.
}\switchcolumn\portugues{
\cruz Continuação do santo Evangelho segundo S. Lucas.
}\switchcolumn*\latim{
\blettrine{I}{n} illo témpore: Discúbuit Jesus, et duódecim Apóstoli cum eo. Et ait illis: Desidério desiderávi hoc Pascha manducáre vobíscum, antequam pátiar. Dico enim vobis, quia ex hoc non manducábo illud, donec impleátur in regno Dei. Et accépto cálice, grátias egit, et dixit: Accípite, et divídite inter vos. Dico enim vobis quod non bibam de generatióne vitis, donec regnum Dei véniat. Et accépto pane, grátias egit, et fregit, et dedit eis, dicens: Hoc est Corpus meum, quod pro vobis datur: hoc fácite in meam commemoratiónem. Simíliter et cálicem, postquam cœnávit, dicens: Hic est calix novum testaméntum in sánguine meo, qui pro vobis fundétur.
}\switchcolumn\portugues{
\blettrine{N}{aquele} tempo, assentou-se Jesus à mesa e com Ele os Doze Apóstolos. E disse-lhes Jesus: «Tenho desejado ardentemente comer convosco esta Páscoa antes de morrer; pois, digo-vos, não beberei mais do fruto da videira até que venha o reino de Deus». E, havendo tomado o pão, deu graças, partiu-o e deu-lho, dizendo: «Isto é o meu Corpo, que se dá por vós. Fazei isto em memória de mim». Tomou, também, igualmente o cálice depois de cear e disse: «Este cálice é o Novo Testamento no meu sangue, que será derramado por vós».
}\end{paracol}

\paragraphinfo{Ofertório}{Heb. 10, 12 \& 14}
\begin{paracol}{2}\latim{
\rlettrine{C}{hristus} unam pro peccátis ófferens hóstiam, in sempitérnum sedet in déxtera Dei: una enim oblatióne consummávit in ætérnum sanctificátos. (T. P. Allelúja.)
}\switchcolumn\portugues{
\rlettrine{C}{risto,} tendo oferecido uma hóstia pelos pecados, está assentado para sempre à direita de Deus; porquanto com uma só oblação consumou eternamente os que foram santificados (T. P. Aleluia.)
}\end{paracol}

\paragraph{Secreta}
\begin{paracol}{2}\latim{
\rlettrine{H}{æc} múnera, Dómine, mediátor noster Jesus Christus tibi reddat accépta: et nos, una secum, hóstias tibi gratas exhíbeat: Qui tecum vivit et regnat \emph{\&c.}
}\switchcolumn\portugues{
\qlettrine{Q}{ue} o nosso mediador Jesus Cristo torne agradáveis a Vós estes dons, ó Senhor, e que nos ofereça juntamente consigo como hóstias a Vós agradáveis. O qual convosco vive e reina \emph{\&c.}
}\end{paracol}

\paragraphinfo{Comúnio}{1. Cor. 11, 24 \& 25}
\begin{paracol}{2}\latim{
\rlettrine{H}{oc} Corpus, quod pro vobis tradétur: hic calix novi testaménti est in meo sánguine, dicit Dóminus: hoc fácite, quotiescúmque súmitis, in meam commemoratiónem. (T. P. Allelúja.)
}\switchcolumn\portugues{
\rlettrine{I}{sto} é o meu Corpo, que será entregue por amor de vós: Este cálice é a nova aliança no meu sangue, diz o Senhor; fazei isto em minha memória todas as vezes que o beberdes. (T. P. Aleluia.)
}\end{paracol}

\paragraph{Postcomúnio}
\begin{paracol}{2}\latim{
\rlettrine{V}{ivíficet} nos, quǽsumus, Dómine, divína quam obtúlimus et súmpsimus hóstia: ut, perpétua tibi caritáte conjúncti, fructum, qui semper máneat, afferámus. Per Dóminum nostrum \emph{\&c.}
}\switchcolumn\portugues{
\rlettrine{V}{os} rogamos, Senhor, que, a hóstia divina, que oferecemos e recebemos, nos vivifique, de modo que a Vós unidos pela perpétua caridade, produzamos fruto que sempre permaneça. Por nosso Senhor \emph{\&c.}
}\end{paracol}
