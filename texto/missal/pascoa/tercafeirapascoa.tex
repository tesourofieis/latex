\subsectioninfo{Terça-feira Pascal}{Estaçáo em S. Paulo}

\paragraphinfo{Intróito}{Ecl. 15, 3 \& 4}
\begin{paracol}{2}\latim{
\rlettrine{A}{qua} sapiéntiæ potávit eos, allelúja: firmábitur in illis et non flectétur, allelúja: et exaltábit eos in ætérnum, allelúja, allelúja. \emph{Ps. 104, 1} Confitémini Dómino et invocáte nomen ejus: annuntiáte inter gentes ópera ejus.
℣. Gloria Patri \emph{\&c.}
}\switchcolumn\portugues{
\rlettrine{O}{} Senhor deu-lhes a beber a água da sabedoria, aleluia: Ela permanecerá neles e não vacilarão, aleluia: e ela os exaltará para sempre, aleluia, aleluia. \emph{Sl. 104, 1} Louvai o Senhor e aclamai o seu nome: publicai as suas obras em todos os povos.
℣. Glória ao Pai \emph{\&c.}
}\end{paracol}

\paragraph{Oração}
\begin{paracol}{2}\latim{
\rlettrine{D}{eus,} qui Ecclésiam tuam novo semper fetu multíplicas: concéde fámulis tuis; ut sacraméntum vivéndo téneant, quod fide percepérunt. Per Dóminum \emph{\&c.}
}\switchcolumn\portugues{
\slettrine{Ó}{} Deus, que aumentais incessantemente a vossa Igreja com novos filhos, concedei aos vossos servos a graça de mostrarem, pela sã conduta de sua vida, o efeito do sacramento que receberam pela fé. Por nosso Senhor \emph{\&c.}
}\end{paracol}

\paragraphinfo{Epístola}{Act. 13, 16 \& 26-33}
\begin{paracol}{2}\latim{
Léctio Actuum Apostolórum.
}\switchcolumn\portugues{
Lição dos Actos dos Apóstolos.
}\switchcolumn*\latim{
\rlettrine{I}{n} diébus illis: Surgens Paulus et manu silentium índicens, ait: Viri fratres, fílii generis Abraham, et qui in vobis timent Deum, vobis verbum salútis hujus missum est. Qui enim habitábant Jerúsalem, et príncipes ejus, ignorántes Jesum et voces Prophetárum, quæ per omne sábbatum legúntur, judicántes implevérunt: et nullam causam mortis inveniéntes in eo, petiérunt a Piláto, ut interfícerent eum. Cumque consummássent ómnia, quæ de eo scripta erant, deponéntes eum de ligno, posuérunt eum in monuménto. Deus vero suscitávit eum a mórtuis tértia die: qui visus est per dies multos his, qui simul ascénderant cum eo de Galilǽa in Jerúsalem, qui usque nunc sunt testes ejus ad plebem. Et nos vobis annuntiámus eam, quæ ad patres nostros repromíssio facta est: quóniam hanc Deus adimplévit fíliis nostris, resúscitans Jesum Christum, Dóminum nostrum.
}\switchcolumn\portugues{
\rlettrine{N}{aqueles}dias, levantando-se Paulo e fazendo sinal com a mão, a pedir silêncio, disse: «Varões, meus irmãos, descendentes de Abraão, e aqueles de vós que temem Deus: é a vós que esta palavra de salvação é dirigida. Os habitantes de Jerusalém e seus discípulos, havendo desprezado Jesus e as palavras dos Profetas, que lhes foram lidas em cada sábado, as cumpriram, condenando-O; e, ainda que não tivessem encontrado n’Ele nada que merecesse a morte, pediram a Pilatos que mandasse matá-l’O. Quando se cumpriu tudo quanto estava escrito a seu respeito, desceram-n’O da cruz e depositaram-n’O no sepulcro. Mas Deus ressuscitou-O dos mortos ao terceiro dia; e, durante vários dias, em seguida, foi visto por aqueles que haviam subido com Ele da Galileia para Jerusalém, os quais agora são suas testemunhas junto do povo. Nós, pois, vos anunciamos que a promessa feita a nossos pais, Deus já a cumpriu para os nossos filhos, ressuscitando nosso Senhor Jesus Cristo».
}\end{paracol}

\paragraphinfo{Gradual}{Sl. 117, 24}
\begin{paracol}{2}\latim{
\rlettrine{H}{æc} dies, quam fecit Dóminus: exsultémus et lætémur in ea. ℣. \emph{Ps. 106, 2} Dicant nunc, qui redémpti sunt a Dómino: quos rédemit de manu inimíci, et de regiónibus congregávit eos.
}\switchcolumn\portugues{
\rlettrine{E}{is} o dia que o Senhor fez: exultemos e alegremo-nos nele. ℣. \emph{Sl. 106, 2} Assim devem cantar agora aqueles que o Senhor resgatou e tirou das mãos dos inimigos, e que, sendo de muitas regiões, os congregou em um só povo.
}\switchcolumn*\latim{
Allelúja, allelúja. ℣. Surrexit Dóminus de sepúlcro, qui pro nobis pepéndit in ligno.
}\switchcolumn\portugues{
Aleluia, aleluia. ℣. Ressuscitou do sepulcro o Senhor, que havia sido pregado no madeiro por nossa causa.
}\end{paracol}

\paragraphinfo{Evangelho}{Lc. 24, 36-47}
\begin{paracol}{2}\latim{
\cruz Sequéntia sancti Evangélii secúndum Lucam.
}\switchcolumn\portugues{
\cruz Continuação do santo Evangelho segundo S. Lucas.
}\switchcolumn*\latim{
\blettrine{I}{n} illo témpore: Stetit Jesus in médio discipulórum suórum et dicit eis: Pax vobis: ego sum, nolíte timére. Conturbáti vero et contérriti, existimábant se spíritum vidére. Et dixit eis: Quid turbáti estis, et cogitatiónes ascéndunt in corda vestra? Vidéte manus meas et pedes, quia ego ipse sum: palpáte et vidéte: quia spíritus carnem et ossa non habet, sicut me vidétis habére. Et cum hoc dixísset, osténdit eis manus et pedes. Adhuc autem illis non credéntibus et mirántibus præ gáudio, dixit: Habétis hic aliquid, quod manducétur? At illi obtulérunt ei partem piscis assi et favum mellis. Et cum manducásset coram eis, sumens relíquias, dedit eis. Et dixit ad eos: Hæc sunt verba, quæ locútus sum ad vos, cum adhuc essem vobíscum, quóniam necésse est impléri ómnia, quæ scripta sunt in lege Móysi et Prophétis et Psalmis de me. Tunc apéruit illis sensum, ut intellégerent Scriptúras. Et dixit eis: Quóniam sic scriptum est, et sic oportébat Christum pati, et resúrgere a mórtuis tértia die: et prædicári in nómine ejus pœniténtiam, et remissiónem peccatórum in omnes gentes.
}\switchcolumn\portugues{
\blettrine{N}{aquele} tempo, apareceu Jesus no meio dos discípulos e disse-lhes: «A paz seja convosco! Sou Eu, não tenhais receio». Admirados e atónitos, os discípulos pensavam que viam algum espírito! Mas Ele disse-lhes: «Porque vos perturbais e porque se levantam pensamentos de dúvida nos vossos corações? Vede as minhas mãos e os meus pés; sou Eu mesmo. Apalpai e vede; pois um espírito não tem carne, nem ossos, como Eu tenho». Depois que lhes disse isto, mostrou-lhes as mãos e os pés. Então, como não acreditassem ainda (na cegueira da alegria em que estavam), disse-lhes: «Tendes aí alguma cousa para comer?». Apresentaram-Lhe um bocado de peixe assado e um favo de mel. E, havendo comido perante eles, tomou os restos e deu-lhos. Depois disse-lhes: «Eis o que vos dizia quando ainda estava convosco: que era preciso que se cumprisse, tudo o que estava escrito a meu respeito na lei de Moisés, nos Profetas e nos Salmos». Então iluminou-lhes o espírito, para que compreendessem as Escrituras. Depois disse-lhes: «Está também escrito: «Convinha que Cristo padecesse, ressuscitasse dos mortos ao terceiro dia e em seu nome fosse pregada a penitência e a remissão dos pecados em todas as nações».
}\end{paracol}

\paragraphinfo{Ofertório}{Sl. 17, 14 \& 16}
\begin{paracol}{2}\latim{
\rlettrine{I}{ntónuit} de cœlo Dóminus, et Altíssimus dedit vocem suam: et apparuérunt fontes aquárum, allelúja.
}\switchcolumn\portugues{
\rlettrine{L}{á} no céu o Senhor trovejou e o Altíssimo fez ouvir a sua voz. Então irromperam as fontes das águas, aleluia.
}\end{paracol}

\paragraph{Secreta}
\begin{paracol}{2}\latim{
\rlettrine{S}{úscipe,} Dómine, fidélium preces cum oblatiónibus hostiárum: ut, per hæc piæ devotiónis offícia, ad cœléstem glóriam transeámus. Per Dóminum \emph{\&c.}
}\switchcolumn\portugues{
\rlettrine{R}{ecebei,} Senhor, as preces dos vossos fiéis, unidas às oblações destas hóstias, a fim de que, por meio destes cultos de devoção piedosa, alcancemos glória celestial. Por nosso Senhor \emph{\&c.}
}\end{paracol}

\paragraphinfo{Comúnio}{Cl. 3, 1-2}
\begin{paracol}{2}\latim{
\rlettrine{S}{i} consurrexístis cum Christo, quæ sursum sunt quǽrite, ubi Christus est in déxtera Dei sedens, allelúja: quæ sursum sunt sápite, allelúja.
}\switchcolumn\portugues{
\rlettrine{S}{e} já ressuscitastes com Cristo, procurai as cousas do céu, onde Cristo está assentado à mão direita de Deus, aleluia; meditai nas cousas do céu, aleluia.
}\end{paracol}

\paragraph{Postcomúnio}
\begin{paracol}{2}\latim{
\rlettrine{C}{oncéde,} quǽsumus, omnípotens Deus: ut paschális percéptio sacraménti, contínua in nostris méntibus persevéret. Per Dóminum \emph{\&c.}
}\switchcolumn\portugues{
\slettrine{Ó}{} Deus omnipotente, Vos rogamos, fazei que a virtude do sacramento pascal, que recebemos, permaneça perpetuamente nas nossas almas. Por nosso Senhor \emph{\&c.}
}\end{paracol}
