\subsectioninfo{4.ª Missa - Desde a Páscoa até ao Pentecostes}{Missa Salve, sancta Parens da Virgem Maria}\label{missamaria4}

\textit{Como na Missa Precedente, excepto o seguinte:}

\textit{Depois da Epistola diz-se:}

\begin{paracol}{2}\latim{
Allelúja, allelúja. ℣. \emph{Num. 17, 8} Virga Jesse flóruit: Virgo Deum et hóminem génuit: pacem Deus réddidit, in se reconcílians ima summis. Allelúja. ℣. \emph{Luc. 1, 28} Ave, María, grátia plena; Dóminus tecum: benedícta tu in muliéribus. Allelúja.
}\switchcolumn\portugues{
Aleluia, aleluia. ℣. \emph{Nm. 17, 8} A vara de Jessé floresceu: e a Virgem deu à luz o Homem-Deus: restabeleceu Deus a paz, conciliando na sua pessoa a nossa baixeza com sua suprema grandeza. Aleluia. ℣. \emph{Lc. 1, 28} Ave, Maria, cheia de graça: o Senhor é convosco: bendita sois vós entre as mulheres. Aleluia.
}\end{paracol}

\paragraphinfo{Evangelho}{Jo. 19, 25-27}
\begin{paracol}{2}\latim{
\cruz Sequéntia sancti Evangélii secúndum Joánnem.
}\switchcolumn\portugues{
\cruz Continuação do santo Evangelho segundo S. João.
}\switchcolumn*\latim{
\blettrine{I}{n} illo témpore: Stabant juxta Crucem Jesu Mater ejus, et soror Matris ejus, María Cléophæ, et María Magdaléne. Cum vidísset ergo Jesus Matrem, et discípulum stantem, quem diligébat, dicit Matri suæ: Múlier, ecce fílius tuus. Deinde dicit discípulo: Ecce Mater tua. Et ex illa hora accépit eam discípulus in sua.
}\switchcolumn\portugues{
\blettrine{N}{aquele} tempo, estavam, junto à cruz de Jesus, sua Mãe e a irmã de sua Mãe, Maria, mulher de Cléofas, e Maria Madalena. Vendo Jesus sua Mãe e, perto dela, o discípulo Ele preferia, disse a sua Mãe: «Mulher, eis o vosso filho!». Depois disse ao discípulo: «Eis a tua Mãe!». E desde aquela hora levou-a o discípulo consigo.
}\end{paracol}

\paragraph{Ofertório}
\begin{paracol}{2}\latim{
\rlettrine{B}{eáta} es, Virgo María, quæ ómnium portásti Greatórem: genuísti qui te fecit, et in ætérnum pérmanes Virgo, allelúja.
}\switchcolumn\portugues{
\rlettrine{S}{ois} bem-aventurada, ó Virgem Maria, pois fostes digna de trazer em vosso seio o Criador do mundo. Vós gerastes Aquele que vos criou e permanecestes eternamente Virgem. Aleluia.
}\end{paracol}
