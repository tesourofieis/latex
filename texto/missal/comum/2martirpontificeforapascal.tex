\subsectioninfo{Mártir Pontífice}{Missa Státuit ei Dóminus}\label{martirpontificeforapascal}

\paragraphinfo{Intróito}{Ecl. 45, 30}
\begin{paracol}{2}\latim{
\rlettrine{S}{tátuit} ei Dóminus testaméntum pacis, et príncipem fecit eum: ut sit illi sacerdótii dígnitas in ætérnum. \emph{Ps. 131, 1} Meménto, Dómine, David: et omnis mansuetúdinis ejus.
℣. Gloria Patri \emph{\&c.}
}\switchcolumn\portugues{
\rlettrine{O}{} Senhor fez com ele uma aliança de paz e proclamou-o príncipe, para que a dignidade sacerdotal lhe pertencesse eternamente. \emph{Sl. 131, 1} Lembrai-Vos de David, ó Senhor, e da sua grande solicitude.
℣. Glória ao Pai \emph{\&c.}
}\end{paracol}

\paragraph{Oração}
\begin{paracol}{2}\latim{
\rlettrine{I}{nfirmitátem} nostram réspice, omnípotens Deus: et, quia pondus própriæ actiónis gravat, beáti {\redx N.} Martyris tui atque Pontíficis intercéssio gloriósa nos prótegat. Per Dóminum \emph{\&c.}
}\switchcolumn\portugues{
\rlettrine{O}{lhai} para a nossa fraqueza, ó Deus omnipotente; e, visto que estamos oprimidos sob o peso dos nossos pecados, fazei que nos proteja a gloriosa intercessão do B. {\redx N.}, vosso Pontífice e Mártir. Por nosso Senhor \emph{\&c.}
}\end{paracol}

\paragraphinfo{Epístola}{Tg. 1, 12-18}
\begin{paracol}{2}\latim{
Léctio Epístolæ beáti Jacóbi Apóstoli.
}\switchcolumn\portugues{
Lição da Ep.ª do B. Ap.º Tiago.
}\switchcolumn*\latim{
\rlettrine{C}{aríssimi:} Beátus vir, qui suffert tentatiónem: quóniam, cum probátus fúerit, accípiet corónam vitæ, quam repromísit Deus diligéntibus se. Nemo, cum tentátur, dicat, quóniam a Deo tentátur: Deus enim intentátor malórum est: ipse autem néminem tentat. Unusquísque vero tentátur a concupiscéntia sua abstráctus et illéctus. Deinde Concupiscéntia cum concéperit, parit peccátum: peccátum vero cum consummátum fúerit, génerat mortem. Nolíte itaque erráre, fratres mei dilectíssimi. Omne datum óptimum et omne donum perféctum desúrsum est, descéndens a Patre lúminum, apud quem non est transmutátio nec vicissitúdinis obumbrátio. Voluntárie enim génuit nos verbo veritátis, ut simus inítium aliquod creatúræ ejus.
}\switchcolumn\portugues{
\rlettrine{C}{aríssimos:} bem-aventurado o varão que sofre a tentação, porque, quando acabar a provação, receberá a coroa da vida, que o Senhor prometeu aos que O amam. Ninguém, quando for tentado, diga que é Deus quem o tenta, pois Deus não é tentador que arraste para o mal, nem tenta ninguém. Porém, cada um é tentado pela sua própria concupiscência, que o atrai e solicita; e depois, quando a concupiscência já concebeu, gera o pecado, e o pecado, logo que é consumado, gera a morte. Não vos enganeis, pois, irmãos dilectíssimos. Todo o dom excelente e todo o dom perfeito vêm do alto e derivam do Pai das luzes, em quem não há mudanças, nem sombra de alteração; pois foi Ele quem por sua espontânea vontade nos gerou pela palavra da verdade, a fim de que fôssemos como primícias das suas criaturas.
}\end{paracol}

\paragraphinfo{Gradual}{Sl. 88, 21-23}
\begin{paracol}{2}\latim{
\rlettrine{I}{nvéni} David servum meum, óleo sancto meo unxi eum: manus enim mea auxiliábitur ei, et bráchium meum confortábit eum. ℣. Nihil profíciet inimícus in eo, et fílius iniquitátis non nocébit ei.
}\switchcolumn\portugues{
\rlettrine{E}{ncontrei} o meu servo David e ungi-o com meu óleo sagrado; a minha mão o auxiliará e o meu braço o fortificará. ℣. O inimigo nada poderá contra ele e o filho da iniquidade nenhum mal lhe fará.
}\switchcolumn*\latim{
Allelúja, allelúja. ℣. \emph{Ps. 109, 4} Tu es sacérdos in ætérnum, secúndum órdinem Melchísedech. Allelúja.
}\switchcolumn\portugues{
Aleluia, aleluia. ℣. \emph{Sl. 109, 4} Tu és sacerdote para sempre, segundo a ordem de Melquisedeque. Aleluia.
}\end{paracol}

\textit{Após a Septuagésima omite-se o Aleluia e o seguinte, dizendo-se:}

\paragraphinfo{Trato}{Sl. 20, 3-4}
\begin{paracol}{2}\latim{
\rlettrine{D}{esidérium} ánimæ ejus tribuísti ei: et voluntáte labiórum ejus non fraudásti eum. ℣. Quóniam prævenísti eum in benedictiónibus dulcédinis. ℣. Posuísti in cápite ejus corónam de lápide pretióso.
}\switchcolumn\portugues{
\rlettrine{C}{oncedestes-lhe} o desejo da sua alma: lhe não negastes o que seus lábios Vos pediram. ℣. Premuniste-lo com bênçãos de doçura. ℣. Impusestes na sua cabeça uma coroa de pedras preciosas.
}\end{paracol}

\paragraphinfo{Evangelho}{Lc. 14, 26-33}
\begin{paracol}{2}\latim{
\cruz Sequéntia sancti Evangélii secúndum Lucam.
}\switchcolumn\portugues{
\cruz Continuação do santo Evangelho segundo S. Lucas.
}\switchcolumn*\latim{
\blettrine{I}{n} illo témpore: Dixit Jesus turbis: Si quis venit ad me, et non odit patrem suum, et matrem, et uxórem, et fílios, et fratres, et soróres, adhuc autem et ánimam suam, non potest meus esse discípulus. Et qui non bájulat crucem suam, et venit post me, non potest meus esse discípulus. Quis enim ex vobis volens turrim ædificáre, non prius sedens cómputat sumptus, qui necessárii sunt, si hábeat ad perficiéndum; ne, posteáquam posúerit fundaméntum, et non potúerit perfícere, omnes, qui vident, incípiant illúdere ei, dicéntes: Quia hic homo cœpit ædificáre, et non pótuit consummáre? Aut quis rex iturus commíttere bellum advérsus álium regem, non sedens prius cógitat, si possit cum decem mílibus occúrrere ei, qui cum vigínti mílibus venit ad se? Alióquin, adhuc illo longe agénte, legatiónem mittens, rogat ea, quæ pacis sunt. Sic ergo omnis ex vobis, qui non renúntiat ómnibus, quæ póssidet, non potest meus esse discípulus.
}\switchcolumn\portugues{
\blettrine{N}{aquele} tempo, disse Jesus às turbas: «Se alguém vem a mim e não despreza seu pai, sua mãe, sua mulher e filhos, seus irmãos e irmãs e até mesmo a sua própria vida, não pode ser meu discípulo. E todo aquele que não leva a sua cruz não pode ser meu discípulo. Com efeito, qual é de vós que, querendo edificar uma torre, não calcula primeiramente com cuidado os gastos necessários, para ver se possui meios para a acabar? Pois poderá acontecer que, depois de haver lançado os alicerces e não podendo acabar a torre, comecem a zombar dele aqueles que o vêem, dizendo: «Este homem começou a edificar e não pôde acabar!». Ou qual é o rei que, preparando-se para pelejar com outro rei, não considera primeiramente se com um exército de dez mil homens poderá fazer frente ao inimigo, que avança contra ele com vinte mil homens? Se vê que não pode combater, estando ainda o outro longe, manda-lhe uma embaixada a pedir-lhe a paz. Assim, pois, todo aquele de vós que não renunciar a tudo quanto possui não pode ser meu discípulo».
}\end{paracol}

\paragraphinfo{Ofertório}{Sl. 88, 25}
\begin{paracol}{2}\latim{
\rlettrine{V}{éritas} mea et misericórdia mea cum
ipso: et in nómine meo exaltábitur cornu ejus.
}\switchcolumn\portugues{
\rlettrine{A}{} minha fidelidade e a minha misericórdia estarão com ele: e o seu poder elevar-se-á pelo meu nome.
}\end{paracol}

\paragraph{Secreta}
\begin{paracol}{2}\latim{
\rlettrine{H}{óstias} tibi, Dómine, beáti {\redx N.} Mártyris tui atque Pontíficis dicátas méritis, benígnus assúme: et ad perpétuum nobis tríbue proveníre subsídium. Per Dóminum \emph{\&c.}
}\switchcolumn\portugues{
\rlettrine{R}{ecebei} benigno, Senhor, as hóstias que Vos oferecemos pelos merecimentos do B. {\redx N.}, vosso Mártir e Pontífice, e fazei que elas nos alcancem o vosso perpétuo socorro. Por nosso Senhor \emph{\&c.}
}\end{paracol}

\paragraphinfo{Comúnio}{Sl. 88, 36 \& 37-38}
\begin{paracol}{2}\latim{
\rlettrine{S}{emel} jurávi in sancto meo: Semen ejus in ætérnum manébit: et sedes ejus sicut sol in conspéctu meo, et sicut luna perfécta in ætérnum, et testis in cœlo fidélis.
}\switchcolumn\portugues{
\qlettrine{J}{urei} uma vez por minha santidade: sua descendência durará eternamente e o seu trono brilhará perante mim, como o sol, e como a lua permanecerá para sempre e será testemunho fiel no céu.
}\end{paracol}

\paragraph{Postcomúnio}
\begin{paracol}{2}\latim{
\rlettrine{R}{efécti} participatióne múneris sacri, quǽsumus, Dómine, Deus noster: ut, cujus exséquimur cultum, intercedénte beáto {\redx N.} Mártyre tuo atque Pontífice, sentiámus efféctum. Per Dóminum \emph{\&c.}
}\switchcolumn\portugues{
\rlettrine{F}{ortalecidos} com a participação do dom sacratíssimo, Vos pedimos, Senhor, nosso Deus, que, por intercessão do B. {\redx N.}, vosso Mártir e Pontífice, sintamos o efeito do mystério que hoje celebrámos. Por nosso Senhor \emph{\&c.}
}\end{paracol}
