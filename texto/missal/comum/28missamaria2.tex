\subsectioninfo{2.ª Missa - Desde o Natal até à Purificação}{Missa Vultum tuum da Virgem Maria}\label{missamaria2}

\paragraphinfo{Intróito}{Sl. 44, 13,15 \& 16}
\begin{paracol}{2}\latim{
\rlettrine{V}{ultum} tuum deprecabúntur omnes dívites plebis: adducántur Regi Vírgines post eam: próximæ ejus adducéntur tibi in lætítia et exsultatióne. \emph{Ps. ibid., 2} Eructávit cor meum verbum bonum: dico ego ópera mea Regi.
℣. Gloria Patri \emph{\&c.}
}\switchcolumn\portugues{
\rlettrine{T}{odos} os poderosos da terra imploram o vosso olhar; as virgens serão introduzidas perante o Rei após ela: e as suas companheiras serão apresentadas ao Rei, em transportes de alegria e de júbilo. \emph{Sl. ibid., 2} Meu coração exprimiu uma excelente palavra: Consagro ao Rei as minhas obras!
℣. Glória ao Pai \emph{\&c.}
}\end{paracol}

\paragraph{Oração}
\begin{paracol}{2}\latim{
\rlettrine{D}{eus,} qui salútis ætérnæ, beátæ Maríæ virginitáte fœcúnda, humáno generi prǽmia præstitísti: tríbue, quǽsumus; ut ipsam pro nobis intercédere sentiámus, per quam merúimus auctórem vitæ suscípere, Dóminum nostrum Jesum Christum, Fílium tuum: Qui tecum vivit \emph{\&c.}
}\switchcolumn\portugues{
\slettrine{Ó}{} Deus, que pela virgindade fecunda da B. V. Maria concedestes ao género humano o prémio da salvação eterna, fazei, Vos imploramos, que gozemos os efeitos da intercessão daquela pela qual fomos julgados dignos de receber o autor da vida, N. S. Jesus Cristo, vosso Filho: que convosco Vive e reina \emph{\&c.}
}\end{paracol}

\paragraphinfo{Epístola}{Tt. 3, 4-7.}
\begin{paracol}{2}\latim{
Léctio Epístolæ beáti Pauli Apóstoli ad Titum.
}\switchcolumn\portugues{
Lição da Ep.ª do B. Ap.º Paulo a Tito.
}\switchcolumn*\latim{
\rlettrine{C}{aríssime:} Appáruit benígnitas et humánitas Salvatóris nostri Dei: non ex opéribus justítiæ, quæ fécimus nos, sed secúndum suam misericórdiam salvos nos fecit, per lavácrum regeneratiónis et renovatiónis Spíritus Sancti, quem effúdit in nos abúnde per Jesum Christum, Salvatórem nostrum: ut, justificáti grátia ipsíus, herédes simus secúndum spem vitæ ætérnæ: in Christo Jesu, Dómino nostro.
}\switchcolumn\portugues{
\rlettrine{C}{aríssimo:} A bondade e o amor de Deus, nosso Salvador, se manifestaram. Ele salvou-nos, não por causa das obras de justiça que houvéssemos praticado, mas pela sua misericórdia, lavando-nos em um banho de regeneração e de renovação do Espírito Santo, que lançou copiosamente sobre nós por Jesus Cristo, nosso Salvador, a fim de que, justificados pela sua graça, nos tornemos herdeiros da vida eterna, segundo a esperança que depositamos em Jesus Cristo, nosso Senhor.
}\end{paracol}

\paragraphinfo{Gradual}{Sl. 44, 3 \& 2}
\begin{paracol}{2}\latim{
\rlettrine{S}{peciósus} forma præ fíliis hóminum: diffúsa est grátia in lábiis tuis. ℣. Eructávit cor meum verbum bonum: dico ego ópera mea Regi: lingua mea cálamus scribæ velóciter scribéntis.
}\switchcolumn\portugues{
\rlettrine{S}{ois} mais bela do que todos os filhos dos homens: pois a graça espalhou-se nos vossos lábios. ℣. Meu coração exprimiu uma excelente palavra: Consagro ao Rei as minhas obras. Minha língua é como a pena de um escritor perito.
}\switchcolumn*\latim{
Allelúja, allelúja. ℣. Post partum, Virgo, invioláta permansísti: Dei Génetrix, intercéde pro nobis. Allelúja.
}\switchcolumn\portugues{
Aleluia, aleluia. ℣. Depois de haverdes dado à luz, permanecestes Virgem Imaculada. Aleluia.
}\end{paracol}

\textit{Após a Septuagésima omite-se o Aleluia e o seguinte e diz-se:}

\paragraphinfo{Trato}{}
\begin{paracol}{2}\latim{
\rlettrine{G}{aude,} María Virgo, cunctas hǽreses sola interemísti. ℣. Quæ Gabriélis Archángeli dictis credidísti. ℣. Dum Virgo Deum et hóminem genuísti: et post partum, Virgo, invioláta permansísti. ℣. Dei Génetrix, intercéde pro nobis.
}\switchcolumn\portugues{
\rlettrine{R}{egozijai-vos,} ó Virgem Maria, pois só vós fostes capaz de destruir todas as heresias. ℣. Acreditastes nas palavras do Arcanjo Gabriel. ℣. Sendo Virgem, gerastes o Homem-Deus: e, depois de haverdes dado à luz, permanecestes Virgem Imaculada. ℣. Intercedei por nós, ó Mãe de Deus.
}\end{paracol}

\paragraphinfo{Evangelho}{Lc. 2, 15-20}
\begin{paracol}{2}\latim{
\cruz Sequéntia sancti Evangélii secúndum Lucam.
}\switchcolumn\portugues{
\cruz Continuação do santo Evangelho segundo S. Lucas.
}\switchcolumn*\latim{
\blettrine{I}{n} illo témpore: Pastóres loquebántur ad ínvicem: Transeámus usque Béthlehem, et videámus hoc verbum, quod factum est, quod Dóminus osténdit nobis. Et venérunt festinántes, et invenérunt Maríam, et Joseph, et Infántem pósitum in præsépio. Vidéntes autem cognovérunt de verbo, quod dictum erat illis de Púero hoc. Et omnes, qui audiérunt, miráti sunt: et de his, quæ dicta erant a pastóribus ad ipsos. María autem conservábat ómnia verba hæc, cónferens in corde suo. Et revérsi sunt pastores, glorificántes et laudántes Deum in ómnibus, quæ audíerant et víderant, sicut dictum est ad illos.
}\switchcolumn\portugues{
\blettrine{N}{aquele} tempo, disseram os pastores uns aos outros: «Vamos até Belém e vejamos o que foi isto que aconteceu, que o Senhor nos revelou». Vieram, então, a toda a pressa, e encontraram Maria, José e o Menino deitado no presépio. Vendo isto, conheceram a verdade, do que lhes havia sido revelado acerca deste Menino. E todos quantos ouviam falar os pastores ficavam admirados do que eles diziam. Ora Maria conservava todas estas coisas e meditava-as no seu íntimo. E os pastores retiraram-se, glorificando e louvando Deus pelo que tinham visto e ouvido, segundo o que lhes havia sido revelado.
}\end{paracol}

\paragraph{Ofertório}
\begin{paracol}{2}\latim{
\rlettrine{F}{elix} namque es, sacra Virgo María, et omni laude digníssima: quia ex te ortus est sol justítiæ, Christus, Deus noster.
}\switchcolumn\portugues{
\rlettrine{S}{ois} feliz e digna de todos os louvores, ó Santa Virgem Maria, pois de vós nasceu «o sol da justiça», Cristo, nosso Senhor.
}\end{paracol}

\paragraph{Secreta}
\begin{paracol}{2}\latim{
\rlettrine{D}{ómine,} propitiatióne, et beátæ Maríæ semper Vírginis intercessióne, ad perpétuam atque præséntem hæc oblátio nobis profíciat prosperitátem et pacem. Per Dóminum \emph{\&c.}
}\switchcolumn\portugues{
\rlettrine{P}{ela} vossa misericórdia, Senhor, e por intercessão da B. Maria, sempre Virgem, permiti que esta oferta nos assegure agora e sempre a prosperidade e a paz. Por nosso Senhor \emph{\&c.}
}\end{paracol}

\paragraph{Comúnio}
\begin{paracol}{2}\latim{
\rlettrine{B}{eáta} víscera Maríæ Vírginis, quæ portavérunt ætérni Patris Fílium.
}\switchcolumn\portugues{
\rlettrine{B}{em-aventuradas} as entranhas da Virgem Maria, que trouxeram encerrado o Filho do Pai Eterno.
}\end{paracol}

\paragraph{Postcomúnio}
\begin{paracol}{2}\latim{
\rlettrine{H}{æc} nos commúnio, Dómine, purget a crímine: et, intercedénte beáta Vírgine Dei Genetríce María, cœléstis remédii fáciat esse consórtes. Per eúndem Dóminum nostrum \emph{\&c.}
}\switchcolumn\portugues{
\qlettrine{Q}{ue} esta comunhão, Senhor, nos purifique de nossos crimes; e que, por intercessão da B. Virgem Maria, Mãe de Deus, nos torne participantes do remédio celestial. Por nosso Senhor \emph{\&c.}
}\end{paracol}
