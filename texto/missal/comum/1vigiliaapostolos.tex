\subsectioninfo{Vigília dos Apóstolos}{Missa Ego autem}\label{1vigiliaapostolos}

\paragraphinfo{Intróito}{Sl. 51, 10 \& 11}
\begin{paracol}{2}\latim{
\rlettrine{E}{go} autem, sicut olíva fructífera in domo Dómini, sperávi in misericórdia Dei mei: et exspectábo nomen tuum, quóniam bonum est ante conspéctum sanctórum tuorum. \emph{Ps. ibid., 3} Quid gloriáris in malítia: qui potens es in iniquitáte?
℣. Gloria Patri \emph{\&c.}
}\switchcolumn\portugues{
\rlettrine{E}{u,} porém, sou como urna oliveira fértil na casa do Senhor: confio na misericórdia do meu Deus e esperarei o poder do seu nome, porque sois cheio de bondade diante de vossos santos. \emph{Sl. ibid., 3} Para que te vanglorias com a maldade, ó tu, que és poderoso em iniquidades?
℣. Glória ao Pai \emph{\&c.}
}\end{paracol}

\paragraph{Oração}
\begin{paracol}{2}\latim{
\rlettrine{D}{a,} quǽsumus, omnípotens Deus: ut beáti {\redx N.} Apóstoli tui, quam prævenímus, veneránda sollémnitas, et devotiónem nobis áugeat et salútem. Per Dóminum \emph{\&c.}
}\switchcolumn\portugues{
\rlettrine{C}{oncedei-nos,} ó Deus omnipotente, Vos suplicamos, que a solene festa do vosso B. Apóstolo {\redx N.}, cuja celebração antecipamos, nos aumente a piedade e o desejo da salvação. Por nosso Senhor \emph{\&c.}
}\end{paracol}

Se nesta Missa a Oração Precedente é recitada em honra de outro Santo, substitui-se pela seguinte:

\paragraph{Oração}
\begin{paracol}{2}\latim{
\qlettrine{Q}{uǽsumus,} omnípotens Deus: ut beátus Apóstolus, cujus prævenímus festivitátem, tuum pro nobis implóret auxílium; ut, a nostris reátibus absolúti, a cunctis étiam perículis eruámur. Per Dóminum \emph{\&c.}
}\switchcolumn\portugues{
\rlettrine{V}{os} suplicamos, ó Deus omnipotente, que o B. Apóstolo {\redx N.}, cuja festa antecipamos, implore o vosso socorro em nosso favor, a fim de que, absolvidos de nossas culpas, sejamos também livres de todos os perigos. Por nosso Senhor \emph{\&c.}
}\end{paracol}

\paragraphinfo{Epístola}{Ecl. 44, 25-27; 45, 2-4 \& 6-9}
\begin{paracol}{2}\latim{
Léctio libri Sapiéntiæ.
}\switchcolumn\portugues{
Lição do Livro da Sabedoria.
}\switchcolumn*\latim{
\rlettrine{B}{enedíctio} Dómini super caput justi. Ideo dedit illi Dóminus hereditátem, et divísit illi partem in tríbubus duódecim: et invénit grátiam in conspéctu omnis carnis. Et magnificávit eum in timóre inimicórum, et in verbis suis monstra placávit. Glorificávit illum in conspéctu regum, et jussit illi coram pópulo suo, et osténdit illi glóriam suam. In fide et lenitáte ipsíus sanctum fecit illum, et elégit eum ex omni carne. Et dedit illi coram præcépta, et legem vitæ et disciplínæ, et excélsum fecit illum. Státuit ei testaméntum ætérnum, et circumcínxit eum zona justítiæ: et índuit eum Dóminus corónam glóriæ.
}\switchcolumn\portugues{
\rlettrine{A}{} bênção do Senhor repousa sobre a cabeça do justo. Por isso o Senhor lhe deu a terra em herança e a dividiu entre as doze tribos. Ele achou graça aos olhos de todos os viventes. O Senhor engrandeceu-o e tornou-o admirável diante dos seus inimigos; e com suas palavras aplacou os monstros. O Senhor glorificou-o diante dos reis, deu-lhe os seus mandamentos diante do seu povo e manifestou-lhe a sua glória. Por causa da sua fé e mansidão, santificou-o e escolheu-o entre todos os mortais. Deu-lhe face a face os seus preceitos e a lei da vida e da sabedoria. Estabeleceu com ele uma aliança eterna; cingiu-o com a túnica da justiça; e ornou-o com a coroa da glória.
}\end{paracol}

\paragraphinfo{Gradual}{Sl. 91, 13 \& 14}
\begin{paracol}{2}\latim{
\qlettrine{J}{ustus} ut palma florébit: sicut cedrus Líbani multiplicábitur in domo Dómini. ℣. \emph{ibid., 3} Ad annuntiándum mane misericórdiam tuam, et veritátem tuam per noctem.
}\switchcolumn\portugues{
\rlettrine{O}{} justo florescerá, como a palmeira, e multiplicar-se-á, como o cedro do Líbano, plantado na casa do Senhor. ℣. \emph{ibid., 3} Para publicar de manhã a vossa misericórdia, Senhor, e durante a noite a vossa doutrina.
}\end{paracol}

\paragraphinfo{Evangelho}{Jo. 15. 12-16}
\begin{paracol}{2}\latim{
\cruz Sequéntia sancti Evangélii secúndum Joánnem.
}\switchcolumn\portugues{
\cruz Continuação do santo Evangelho segundo S. João.
}\switchcolumn*\latim{
\blettrine{I}{n} illo témpore: Dixit Jesus discípulis suis: Hoc est præcéptum meum, ut diligátis ínvicem, sicut diléxi vos. Majórem hac dilectiónem nemo habet, ut ánimam suam ponat quis pro amícis suis. Vos amíci mei estis, si fecéritis quæ ego præcípio vobis. Jam non dicam vos servos: quia servus nescit, quid fáciat dóminus ejus. Vos autem dixi amícos: quia ómnia, quæcúmque audivi a Patre meo, nota feci vobis. Non vos me elegístis: sed ego elégi vos, et posui vos, ut eátis, et fructum afferátis: et fructus vester maneat: ut, quodcúmque petiéritis Patrem in nómine meo, det vobis.
}\switchcolumn\portugues{
\blettrine{N}{aquele} tempo, disse Jesus aos discípulos: «Este é o meu mandamento: que vos ameis uns aos outros, como vos amei. Ninguém pode ter maior amor do que dar a sua vida pelos seus amigos. Vós sereis meus amigos, se fizerdes o que vos mando. Já vos não chamarei servos, porque o servo ignora o que faz o seu senhor. Chamo-vos meus amigos porque tudo quanto ouvi a meu Pai vo-lo tenho feito conhecer. Não fostes vós que me escolhestes a mim, mas eu é que vos escolhi e vos estabeleci, para que caminheis e alcanceis fruto. Que este fruto, pois, permaneça, para que meu Pai vos conceda tudo quanto lhe pedirdes em meu nome».
}\end{paracol}

\paragraphinfo{Ofertório}{Sl. 8, 6-7}
\begin{paracol}{2}\latim{
\rlettrine{G}{lória} et honore coronásti eum: et constituísti eum super ópera mánuum tuárum, Dómine.
}\switchcolumn\portugues{
\rlettrine{V}{ós} o coroastes, Senhor, com glória e com honras e lhe concedestes o domínio sobre as obras das vossas mãos.
}\end{paracol}

\paragraph{Secreta}
\begin{paracol}{2}\latim{
\rlettrine{A}{postólici} reveréntia cúlminis offeréntes tibi sacra mystéria, Dómine, quǽsumus: ut beáti {\redx N.} Apóstoli tui suffrágiis, cujus natalícia prævenímus; plebs tua semper et sua vota deprómat, et desideráta percípiat. Per Dóminum nostrum \emph{\&c.}
}\switchcolumn\portugues{
\rlettrine{O}{ferecendo-Vos} estes sagrados mystérios em reverência da dignidade apostólica, Vos suplicamos, Senhor, que, pelos rogos do B. {\redx N.}, vosso Apóstolo, cuja festa antecipamos, o vosso povo possa sempre apresentar-Vos seus votos e alcançar a realização de seus desejos. Por nosso Senhor \emph{\&c.}
}\end{paracol}

\paragraphinfo{Comúnio}{Sl. 20, 6}
\begin{paracol}{2}\latim{
\rlettrine{M}{agna} est glória ejus in salutári tuo: glóriam et magnum decórem impónes super eum, Dómine.
}\switchcolumn\portugues{
\rlettrine{G}{rande} é, Senhor, a sua glória, graças à vossa protecção. Vós o rodeastes de glória e de magnificência.
}\end{paracol}

\paragraph{Postcomúnio}
\begin{paracol}{2}\latim{
\rlettrine{S}{ancti} Apóstoli tui {\redx N.}, quǽsumus. Dómine, supplicatióne placátus: et veniam nobis tríbue, et remédia sempitérna concéde. Per Dóminum \emph{\&c.}
}\switchcolumn\portugues{
\rlettrine{V}{os} suplicamos, Senhor, deixai-Vos aplacar pelas orações do vosso santo Apóstolo {\redx N.}; concedei-nos ainda o perdão das nossas faltas e o remédio sempiterno dos nossos males. Por nosso Senhor \emph{\&c.}
}\end{paracol}
