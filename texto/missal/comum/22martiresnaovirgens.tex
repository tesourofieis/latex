\subsectioninfo{Mártires não Virgens}{Missa Me exspectavérunt}\label{martiresnaovirgens}

\paragraphinfo{Intróito}{Sl. 118, 95-96}
\begin{paracol}{2}\latim{
\rlettrine{M}{e} exspectavérunt peccatóres, ut pérderent me: testimónia tua. Dómine, intelléxi: omnis consummatiónis vidi finem: latum mandátum tuum nimis. (T. P. Allelúja, allelúja.) \emph{Ps. ibid., 1} Beáti immaculáti in via: qui ámbulant in lege Dómini.
℣. Gloria Patri \emph{\&c.}
}\switchcolumn\portugues{
\rlettrine{E}{speraram-me} os pecadores para me perderem; mas eu, Senhor, tinha meditado nos vossos avisos. Encontrei limites em tudo quanto existe: e só os vossos Mandamentos são infinitos. (T. P. Aleluia, aleluia). \emph{Sl. ibid., 1} Bem-aventurados os que são imaculados em seus caminhos e que cumprem a Lei do Senhor.
℣. Glória ao Pai \emph{\&c.}
}\end{paracol}

\paragraph{Oração}
\begin{paracol}{2}\latim{
\rlettrine{D}{eus,} qui inter cétera poténtiæ tuæ mirácula etiam in sexu frágili victóriam martýrii contulísti: concéde propítius;
ut, qui beátæ {\redx N.} Martyris tuæ natalítia cólimus, per ejus ad te exémpla gradiámur. Per Dóminum \emph{\&c.}
}\switchcolumn\portugues{
\slettrine{Ó}{} Deus, que entre outros milagres do vosso poder permitistes que o sexo frágil alcançasse a vitória do martírio, concedei-nos propício que, venerando nós o nascimento no céu da B Mártir {\redx N.}, caminhemos para Vós, imitando os seus exemplos. Por nosso Senhor \emph{\&c.}
}\end{paracol}

\paragraphinfo{Epístola}{Ecl. 51, 1-8 \& 12}
\begin{paracol}{2}\latim{
Lectio Epístolæ beati Pauli Apostoli ad Corinthios.
}\switchcolumn\portugues{
Lição do Livro da Sabedoria.
}\switchcolumn*\latim{
\rlettrine{C}{onfitébor} tibi, Dómine, Rex, et collaudábo te Deum, Salvatórem meum. Confitébor nómini tuo: quóniam adjútor et protéctor factus es mihi, et liberásti corpus meum a perditióne, a láqueo linguæ iníquæ et a lábiis operántium mendácium, et in conspéctu astántium factus es mihi adjútor. Et liberásti me secúndum multitúdinem misericórdiæ nóminis tui a rugiéntibus, præparátis ad escam, de mánibus quæréntium ánimam meam, et de portis tribulatiónum, quæ circumdedérunt me: a pressúra flammæ, quæ circúmdedit me, et in médio ignis non sum æstuáta: de altitúdine ventris ínferi, et a lingua coinquináta, et a verbo mendácii, a rege iníquo, et a lingua injústa: laudábit usque ad mortem ánima mea Dóminum: quóniam éruis sustinéntes te, et líberas eos de mánibus géntium, Dómine, Deus noster.
}\switchcolumn\portugues{
\qlettrine{Q}{uero} glorificar-Vos, ó Senhor e Rei; quero louvar-Vos, ó Deus, meu salvador. Quero glorificar o vosso nome, porque fostes o meu sustentáculo e protector, e livrastes o meu corpo da perdição, do laço da língua iníqua e dos lábios daqueles que tramam a mentira; e na presença dos meus adversários fostes o meu auxílio. Livrastes-me, segundo a grandeza da misericórdia do vosso Nome, dos que rugiam, prestes a devorar-me; das mãos dos que procuravam tirar-me a vida; e das aflições, que me cercavam. Livrastes-me da violência das chamas, que me cercavam, no meio das quais não senti o calor do fogo. Livrastes-me do abysmo profundo do inferno; da língua impura; das palavras mentirosas; do rei iníquo e da língua injusta. Minha alma louvará o Senhor até à morte, porque Vós, Senhor, nosso Deus, livrais dos perigos aqueles que confiam em Vós, salvando-os do poder dos inimigos.
}\end{paracol}

\paragraphinfo{Gradual}{Sl. 44, 8}
\begin{paracol}{2}\latim{
\rlettrine{D}{ilexísti} justítiam, et odísti iniquitátem. ℣. Proptérea unxit te Deus, Deus tuus, óleo lætítiae.
}\switchcolumn\portugues{
\rlettrine{A}{mastes} a justiça e odiastes a iniquidade. Por essa razão, o Senhor, vosso Deus, vos ungiu com o óleo da alegria.
}\switchcolumn*\latim{
Allelúja, allelúja. ℣. \emph{ibid., 5} Spécie tua et pulchritúdine tua inténde, próspere procéde et regna. Allelúja.
}\switchcolumn\portugues{
Aleluia, aleluia. ℣. \emph{ibid., 5} Caminhai, pois, com beleza e com majestade; ide gozar a vitória e reinai. Aleluia.
}\end{paracol}

\textit{Após a Septuagésima omite-se o Aleluia e o seguinte e diz-se:}

\paragraph{Trato}
\begin{paracol}{2}\latim{
\rlettrine{V}{eni,} Sponsa Christi, áccipe corónam, quam tibi Dóminus præparávit in æternum: pro cujus amóre sánguinem tuum fudísti. ℣. \emph{Ps. 44, 8 et 5} Diléxisti justítiam, et odísti iniquitátem: proptérea unxit te Deus, Deus tuus, óleo lætítiae præ consórtibus tuis. ℣. Spécie tua et pulchritúdine tua inténde, próspere procéde et regna.
}\switchcolumn\portugues{
\rlettrine{V}{inde,} ó esposa de Cristo; vinde e recebei a coroa que o Senhor preparou para vós, para a eternidade. Foi por amor dele que derramastes o vosso sangue. ℣. \emph{Sl. 44, 8 et 5} Amastes a justiça e odiastes a iniquidade: eis porque o Senhor, vosso Deus, vos ungiu com o óleo da alegria, de preferência às vossas companheiras. ℣. Caminhai, pois, com beleza e com majestade; ide gozar a vitória e reinai.
}\end{paracol}

\textit{No T. Pascal omite-se o Gradual e o Trato e diz-se:}

\begin{paracol}{2}\latim{
Allelúja, allelúja. ℣. \emph{Ps. 44, 5} Spécie tua et pulchritúdine tua inténde, próspere procéde et regna. Allelúja. ℣. Propter veritátem et mansuetúdinem et justítiam: et dedúcet te mirabíliter déxtera tua. Allelúja.
}\switchcolumn\portugues{
Aleluia, aleluia. ℣. \emph{Sl. 44, 5} Caminhai, pois, com beleza e com majestade; ide gozar a vitória e reinai. Aleluia. ℣. Por causa da vossa verdade, mansidão e justiça, a vossa dextra operará admiráveis prodígios. Aleluia.
}\end{paracol}

\paragraphinfo{Evangelho}{Mt. 13, 44-52}
\begin{paracol}{2}\latim{
\cruz Sequéntia sancti Evangélii secúndum Matthǽum.
}\switchcolumn\portugues{
\cruz Continuação do santo Evangelho segundo S. Mateus
}\switchcolumn*\latim{
\blettrine{I}{n} illo témpore: Dixit Jesus discípulis suis parábolam hanc: Símile est regnum cœlórum thesáuro abscóndito in agro: quem qui invénit homo, abscóndit, et præ gáudio illíus vadit, et vendit univérsa, quæ habet, et emit agrum illum. Iterum símile est regnum cœlórum homini negotiatóri, quærénti bonas margarítas. Invénta autem una pretiósa margaríta, ábiit, et véndidit ómnia, quæ hábuit, et emit eam. Iterum símile est regnum cœlórum sagénæ, missæ in mare et ex omni génere píscium cóngreganti. Quam, cum impléta esset educéntes, et secus litus sedéntes, elegérunt bonos in vasa, malos autem foras misérunt. Sic erit in consummatióne sǽculi: exíbunt Angeli, et separábunt malos de médio justórum, et mittent eos in camínum ignis: ibi erit fletus et stridor déntium. Intellexístis hæc ómnia? Dicunt ei: Etiam. Ait illis: Ideo omnis scriba doctus in regno cœlórum símilis est hómini patrifamílias, qui profert de thesáuro suo nova et vétera.
}\switchcolumn\portugues{
\blettrine{N}{aquele} tempo, disse Jesus aos seus discípulos esta parábola: «O reino dos céus é semelhante a um tesouro escondido no campo, o qual um homem achou e esconde; e, alegre com o achado, vai, vende tudo o que tem e compra o campo. Igualmente o reino dos céus é semelhante a um negociante, que busca pedras preciosas; e, achando uma de grande preço, vai, vende tudo o que tem e compra-a. O reino dos céus é ainda semelhante a uma rede que, lançada ao mar, prende toda a casta de peixes, e, estando já cheia, os pescadores a puxam para a praia, escolhem os bons peixes para os vasos e deitam fora os maus. Assim será no fim dos séculos: virão os Anjos e separarão os maus do meio dos justos e os lançarão na fornalha do fogo, onde não haverá senão fogo e ranger de dentes. Haveis compreendido tudo isto?». «Sim» , responderam eles. E Jesus disse-lhes: «Por esta razão todo o escriba douto, no que diz respeito ao reino dos céus, é semelhante a um pai de família, que tira dos seus tesouros cousas novas e velhas».
}\end{paracol}

\paragraphinfo{Ofertório}{Sl. 44, 3}
\begin{paracol}{2}\latim{
\rlettrine{D}{iffúsa} est grátia in lábiis tuis: proptérea benedíxit te Deus in ætérnum, et in sǽculum sǽculi, allelúja.
}\switchcolumn\portugues{
\rlettrine{A}{} graça espalhou-se nos vossos lábios: eis porque Deus vos abençoou para a eternidade e para todos os séculos dos séculos. (T. P. Aleluia).
}\end{paracol}

\paragraph{Secreta}
\begin{paracol}{2}\latim{
\rlettrine{S}{úscipe,} Dómine, múnera, quæ in beátæ {\redx N.} Martyris tuæ sollemnitáte deférimus: cujus nos confídimus patrocínio liberári. Per Dóminum \emph{\&c.}
}\switchcolumn\portugues{
\rlettrine{R}{ecebei} benigno, Senhor, as ofertas que Vos apresentamos nesta solenidade da vossa B. Mártir {\redx N.}, com o patrocínio da qual esperamos ser livres. Por nosso Senhor \emph{\&c.}
}\end{paracol}

\paragraphinfo{Comúnio}{Sl. 118, 161-162}
\begin{paracol}{2}\latim{
\rlettrine{P}{ríncipes} persecúti sunt me gratis, et a verbis tuis formidávit cor meum: lætábor ego super elóquia tua, quasi qui invénit spólia multa. (T. P. Allelúja.)
}\switchcolumn\portugues{
\rlettrine{O}{s} príncipes perseguiram-me injustamente, mas o meu coração não temeu senão as vossas palavras. Regozijar-me-ei com vossas palavras, como se um homem houvera achado ricos despojos. (T. P. Aleluia.)
}\end{paracol}

\paragraph{Postcomúnio}
\begin{paracol}{2}\latim{
\rlettrine{A}{uxiliéntur} nobis, Dómine, sumpta mystéria: et, intercedénte beáta {\redx N.} Mártyre tua, sempitérna fáciant protectióne gaudére. Per Dóminum nostrum.
\emph{\&c.}
}\switchcolumn\portugues{
\qlettrine{Q}{ue} nos auxiliem os sagrados mystérios que acabámos de receber, Senhor, e que, por intercessão da B. {\redx N.}, vossa Mártir, nos façam gozar continuamente a sua protecção. Por nosso Senhor \emph{\&c.}
}\end{paracol}
