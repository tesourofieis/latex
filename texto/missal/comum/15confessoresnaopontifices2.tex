\subsectioninfo{Confessores não Pontífices}{Missa Justus ut palma}\label{confessoresnaopontifices2}

\paragraphinfo{Intróito}{Sl. 91, 13-14}
\begin{paracol}{2}\latim{
\qlettrine{J}{ustus} ut palma florébit: sicut cedrus Líbani multiplicábitur: plantátus in domo Dómini: in átriis domus Dei nostri. (T. P. Allelúja, allelúja.) \emph{Ps. ibid., 2} Bonum est confitéri Dómino: et psállere nómini tuo, Altíssime.
℣. Gloria Patri \emph{\&c.}
}\switchcolumn\portugues{
\rlettrine{O}{} justo florescerá, como a palmeira, e multiplicar-se-á, como o cedro do Líbano, plantado na casa do Senhor e nos átrios da casa do nosso Deus. (T. P. Aleluia, aleluia.) \emph{Sl. ibid., 2} É bom louvar o Senhor: e cantar hinos em honra do vosso nome, ó Altíssimo!
℣. Glória ao Pai \emph{\&c.}
}\end{paracol}

\paragraph{Oração}
\begin{paracol}{2}\latim{
\rlettrine{A}{désto,} Dómine, supplicatiónibus nostris, quas in beáti {\redx N.} Confessóris tui sollemnitáte deférimus: ut, qui nostræ justítiæ fidúciam non habémus, ejus, qui tibi plácuit, précibus adjuvémur. Per Dóminum \emph{\&c.}
}\switchcolumn\portugues{
\rlettrine{O}{uvi} benigno, Senhor, as súplicas que Vos apresentamos na solenidade do vosso B. Confessor {\redx N.}, a fim de que, já que não podemos ter confiança na nossa justiça, sejamos auxiliados pelas preces daquele que Vos foi agradável neste mundo. Por nosso Senhor \emph{\&c.}
}\end{paracol}

\paragraphinfo{Epístola}{1. Cor. 4, 9-14}
\begin{paracol}{2}\latim{
Léctio Epístolæ beáti Pauli Apóstoli ad Corínthios.
}\switchcolumn\portugues{
Lição da Ep.ª do B. Ap.º Paulo aos Coríntios.
}\switchcolumn*\latim{
\rlettrine{F}{ratres:} Spectáculum facti sumus mundo et Angelis et homínibus. Nos stulti propter Christum, vos autem prudéntes in Christo: nos infírmi, vos autem fortes: vos nóbiles, nos autem ignóbiles. Usque in hanc horam et esurímus, et sitímus, et nudi sumus, et cólaphis cǽdimur, et instábiles sumus, et laborámus operántes mánibus nostris: maledícimur, et benedícimus: persecutiónem pátimur, et sustinémus: blasphemámur, et obsecrámus: tamquam purgaménta hujus mundi facti sumus, ómnium peripséma usque adhuc. Non ut confúndant vos, hæc scribo, sed ut fílios meos caríssimos móneo: in Christo Jesu, Dómino nostro.
}\switchcolumn\portugues{
\rlettrine{M}{eus} irmãos: Tornamo-nos espectáculo do mundo, dos Anjos e dos homens. Somos loucos por amor de Cristo; mas vós sois prudentes em Cristo. Nós somos fracos; mas vós sois fortes. Vós sois honrados; mas nós somos desprezados. Até agora padecemos a fome, a sede e a nudez; fomos maltratados; andamos errantes; trabalhamos penosamente com nossas próprias mãos. Somos amaldiçoados, mas abençoamos; somos perseguidos, e sofremos; somos blasfemados, e correspondemos com orações. Temos sido considerados até ao presente como o refugo deste mundo, como a escória de todos! Escrevo estas coisas, não para vos envergonhar, mas para vos admoestar, como meus filhos caríssimos em N. S. Jesus Cristo.
}\end{paracol}

\paragraphinfo{Gradual}{Sl. 36, 30-31}
\begin{paracol}{2}\latim{
\rlettrine{O}{s} justi meditábitur sapiéntiam, et lingua ejus loquétur judícium. ℣. Lex Dei ejus in corde ipsíus: et non supplantabúntur gressus ejus.
}\switchcolumn\portugues{
\rlettrine{A}{} boca do justo fala com sabedoria e a sua língua proclama a justiça! ℣. A lei do seu Deus está no seu coração e os seus pés não tropeçarão.
}\switchcolumn*\latim{
Allelúja, allelúja. ℣. \emph{Ps. 111, 1} Beátus vir, qui timet Dóminum: in mandátis ejus cupit nimis. Allelúja.
}\switchcolumn\portugues{
Aleluia, aleluia. ℣. \emph{Sl. 111, 1} Bem-aventurado o varão que teme o Senhor e que põe todo seu zelo em obedecer-Lhe. Aleluia.
}\end{paracol}

\textit{Após a Septuagésima omite-se o Aleluia e o seguinte e diz-se:}

\paragraphinfo{Trato}{Sl. 111, 1-3}
\begin{paracol}{2}\latim{
\rlettrine{B}{eátus} vir, qui timet Dóminum: in mandátis ejus cupit nimis. ℣. Potens in terra erit semen ejus: generátio rectórum benedicétur. ℣. Glória et divítiæ in domo ejus: et justítia ejus manet in sǽculum sǽculi.
}\switchcolumn\portugues{
\rlettrine{B}{em-aventurado} o verão que teme o Senhor e que põe todo o zelo em obedecer-Lhe. ℣. Sua descendência será poderosa na terra; pois a geração dos justos será abençoada. ℣. Na sua casa haverá glória e riqueza; e a sua justiça subsistirá em todos os séculos dos séculos.
}\end{paracol}

\textit{No T. Pascal omite-se o Gradual e o Trato e diz-se:}

\begin{paracol}{2}\latim{
Allelúja, allelúja. ℣. \emph{Ps. 111, 1} Beátus vir, qui timet Dóminum: in mandátis ejus cupit nimis. Allelúja. ℣. \emph{Osee 14, 6} Justus germinábit sicut lílium: et florébit in ætérnum ante Dóminum. Allelúja.
}\switchcolumn\portugues{
Aleluia, aleluia. ℣. \emph{Sl. 111, 1} Bem-aventurado o varão que teme o Senhor e que põe todo o zelo em obedecer-Lhe. Aleluia. ℣. \emph{Os. 14, 6} O justo germinará, como o lírio, e florescerá para sempre diante do Senhor. Aleluia.
}\end{paracol}

\paragraphinfo{Evangelho}{Lc. 12, 32-34}
\begin{paracol}{2}\latim{
\cruz Sequéntia sancti Evangélii secúndum Lucam.
}\switchcolumn\portugues{
\cruz Continuação do santo Evangelho segundo S. Lucas.
}\switchcolumn*\latim{
\blettrine{I}{n} illo témpore: Dixit Jesus discípulis suis: Nolíte timére, pusíllus grex, quia complácuit Patri vestro dare vobis regnum. Véndite quæ possidétis, et date eleemósynam. Fácite vobis sácculos, qui non veteráscunt, thesáurum non deficiéntem in cœlis: quo fur non apprópiat, neque tínea corrúmpit. Ubi enim thesáurus vester est, ibi et cor vestrum erit.
}\switchcolumn\portugues{
\blettrine{N}{aquele} tempo, disse Jesus aos seus discípulos: «Não temais, pequeno rebanho, pois agradou ao vosso Pai dar-vos o seu reino. Vendei tudo quanto possuís e dai-o em esmolas. Fazei para vós bolsas que se não estraguem com o tempo; ajuntai um tesouro no céu, onde o ladrão não pode chegar, nem a traça o pode corromper; pois, onde estiver o vosso tesouro, aí estará o vosso coração».
}\end{paracol}

\paragraphinfo{Ofertório}{Sl. 20, 2-3}
\begin{paracol}{2}\latim{
\rlettrine{I}{n} virtúte tua, Dómine, lætábitur justus, et super salutáre tuum exsultábit veheménter: desidérium ánimæ ejus tribuísti ei. (T. P. Allelúja.)
}\switchcolumn\portugues{
\rlettrine{C}{om} o vosso poder, Senhor, se alegrará o justo, o qual exultará de alegria, vendo-se salvo por Vós. Concedestes-lhe, Senhor, o desejo da sua alma. (T. P. Aleluia.)
}\end{paracol}

\paragraph{Secreta}
\begin{paracol}{2}\latim{
\rlettrine{P}{ræsta} nobis, quǽsumus, omnípotens Deus: ut nostræ humilitátis oblátio et pro tuórum tibi grata sit honóre Sanctórum, et nos córpore páriter et mente puríficet. Per Dóminum \emph{\&c.}
}\switchcolumn\portugues{
\rlettrine{D}{ignai-Vos} conceder-nos, Deus omnipotente, que esta oferta da nossa humildade, em honra dos vossos Santos, Vos seja agradável; e permiti que nos purifique ao mesmo tempo o corpo e a alma. Por nosso Senhor \emph{\&c.}
}\end{paracol}

\paragraphinfo{Comúnio}{Mt. 19, 28 \& 29}
\begin{paracol}{2}\latim{
\rlettrine{A}{men,} dico vobis: quod vos, qui reliquístis ómnia et secúti estis me, céntuplum accipiétis, et vitam ætérnam possidébitis, (T. P. Allelúja.)
}\switchcolumn\portugues{
\rlettrine{E}{m} verdade vos digo: vós, que abandonastes tudo e me seguistes, recebereis o cêntuplo e possuireis a vida eterna. (T. P. Aleluia.)
}\end{paracol}

\paragraph{Postcomúnio}
\begin{paracol}{2}\latim{
\qlettrine{Q}{uǽsumus,} omnípotens Deus: ut, qui cœléstia aliménta percépimus, intercedénte beáto {\redx N.} Confessóre tuo, per hæc contra ómnia advérsa muniámur. Per Dóminum nostrum \emph{\&c.}
}\switchcolumn\portugues{
\slettrine{Ó}{} Deus omnipotente, Vos imploramos, havendo nós recebido o alimento celestial, fazei que, por intercessão do B. Confessor {\redx N.} sejamos fortificados contra todas as adversidades. Por nosso Senhor \emph{\&c.}
}\end{paracol}
