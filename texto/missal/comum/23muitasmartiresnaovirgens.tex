\subsectioninfo{Muitas Mártires não Virgens}{Missa Me exspectavérunt}\label{muitasmartiresnaovirgens}

\textit{Como a Missa Precedente, página \pageref{martiresnaovirgens}, excepto o seguinte:}

\paragraph{Oração}
\begin{paracol}{2}\latim{
\rlettrine{D}{a} nobis, quǽsumus, Dómine, Deus noster, sanctárum Mártyrum tuárum {\redx N.} et {\redx N.} palmas incessábili devotióne venerári: ut, quas digna mente non póssumus celebráre, humílibus saltem frequentémus obséquiis. Per Dóminum nostrum \emph{\&c.}
}\switchcolumn\portugues{
\slettrine{Ó}{} Senhor, nosso Deus, dignai-Vos conceder-nos a graça de incessantemente venerarmos com devoção a vitória das vossas santas Mártires {\redx N.} e {\redx N.}, a fim de que, já que não podemos celebrar dignamente os seus méritos, possamos, ao menos, oferecer-lhes as nossas humildes homenagens. Por nosso Senhor \emph{\&c.}
}\end{paracol}

\paragraph{Secreta}
\begin{paracol}{2}\latim{
\rlettrine{I}{nténde,} quǽsumus, Dómine, múnera altáribus tuis pro sanctárum Mártyrum tuárum {\redx N.} et {\redx N.} festivitáte propósita: ut, sicut per hæc beáta mystéria illis glóriam contulísti; ita nobis indulgéntiam largiáris. Per Dóminum \emph{\&c.}
}\switchcolumn\portugues{
\rlettrine{S}{enhor,} dignai-Vos volver os olhares para estas ofertas, que depositamos nos vossos altares para comemorar a festa das vossas santas Mártires {\redx N.} e {\redx N.}, a fim de que, assim como lhes concedestes a glória, assim também nos concedais o perdão dos nossos pecados. Por nosso Senhor \emph{\&c.}
}\end{paracol}

\paragraph{Postcomúnio}
\begin{paracol}{2}\latim{
\rlettrine{P}{ræsta} nobis, quǽsumus, Dómine, intercedéntibus sanctis Martýribus tuis {\redx N.} et {\redx N.}: ut, quod ore contíngimus, pura mente capiámus. Per Dóminum \emph{\&c.}
}\switchcolumn\portugues{
\rlettrine{C}{oncedei-nos,} Senhor, Vos suplicamos, que, por intercessão das vossas santas Mártires {\redx N.} e {\redx N.}, guardemos com o coração puro o que a nossa boca acaba de receber. Por nosso Senhor \emph{\&c.}
}\end{paracol}
