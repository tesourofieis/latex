\subsectioninfo{Mártir Pontífice}{Missa Sacerdótes Dei}\label{martirpontifice}

\paragraphinfo{Intróito}{Dn. 3, 84 \& 87}
\begin{paracol}{2}\latim{
\rlettrine{S}{acerdótes} Dei, benedícite Dóminum: sancti et húmiles corde, laudáte Deum. \emph{ibid., 57} Benedícite, ómnia ópera Dómini, Dómino: laudáte et superexaltáte eum in sǽcula.
℣. Gloria Patri \emph{\&c.}
}\switchcolumn\portugues{
\rlettrine{B}{endizei} o Senhor, ó sacerdotes de Deus: louvai o Senhor, ó vós, santos e humildes de coração! \emph{ibid., 57} Bendizei o Senhor, todas as obras do Senhor: louvai-O e glorificai-O em todos os séculos!
℣. Glória ao Pai \emph{\&c.}
}\end{paracol}

\paragraph{Oração}
\begin{paracol}{2}\latim{
\rlettrine{D}{eus,} qui nos beáti {\redx N.} Mártyris tui atque Pontíficis ánnua sollemnitáte lætíficas: concéde propítius; ut, cujus natalítia cólimus, de ejúsdem étiam protectióne gaudeámus. Per Dóminum \emph{\&c.}
}\switchcolumn\portugues{
\slettrine{Ó}{} Deus, que nos alegrais com a solenidade anual do B. {\redx N.}, vosso Mártir e Pontífice, concedei-nos propício que, assim como celebramos o seu nascimento, assim também gozemos a sua protecção. Por nosso Senhor \emph{\&c.}
}\end{paracol}

\paragraphinfo{Epístola}{2. Cor. 1, 3-7}
\begin{paracol}{2}\latim{
Léctio Epístolæ beáti Pauli Apóstoli ad Corínthios.
}\switchcolumn\portugues{
Lição da Ep.ª do B. Ap.º Paulo aos Coríntios.
}\switchcolumn*\latim{
\rlettrine{F}{ratres:} Benedíctus Deus et Pater Dómini nostri Jesu Christi, Pater misericordiárum, et Deus totíus consolatiónis, qui consolátur nos in omni tribulatióne nostra: ut póssimus et ipsi consolári eos, qui in omni pressúra sunt, per exhortatiónem, qua exhortámur et ipsi a Deo. Quóniam sicut abúndant passiónes Christi in nobis: ita et per Christum abúndat consolátio nostra. Sive autem tribulámur pro vestra exhortatióne et salúte, sive consolámur pro vestra consolatióne, sive exhortámur pro vestra exhortatióne et salúte, quæ operátur tolerántiam earúndem passiónum, quas et nos pátimur: ut spes nostra firma sit pro vobis: sciéntes, quod, sicut sócii passiónum estis, sic éritis et consolatiónis: in Christo Jesu, Dómino nostro.
}\switchcolumn\portugues{
\rlettrine{M}{eus} irmãos: bendito seja Deus, Pai de N. S. Jesus Cristo e das misericórdias e Deus de toda a consolação, que nos consola em todas nossas tribulações, para que pela mesma consolação, que recebemos de Deus, possamos consolar os que estão oprimidos. Pois, assim como abundam em nós as aflições de Cristo, assim também em Cristo abundem as consolações. Se, portanto, somos atribulados, é para vossa consolação e salvação; se somos consolados, é também para vossa consolação; e se somos confortados, é ainda para vosso conforto e salvação, a qual mostra a sua eficácia em suportar os mesmos males que nos afligem. A nossa confiança a vosso respeito é firme, sabendo que, assim como participais das aflições, assim também participareis da consolação em Jesus Cristo, nosso Senhor.
}\end{paracol}

\paragraphinfo{Gradual}{Sl. 8, 6-7}
\begin{paracol}{2}\latim{
\rlettrine{G}{lória} et honóre coronásti eum. ℣. Et constituísti eum super ópera mánuum tuárum, Dómine.
}\switchcolumn\portugues{
\rlettrine{V}{ós} o coroastes, Senhor, com glória e com honras. ℣. Vós lhe destes o domínio sobre as obras das vossas mãos.
}\switchcolumn*\latim{
 Allelúja, allelúja. ℣. Hic est Sacérdos, quem coronávit Dóminus. Allelúja.
}\switchcolumn\portugues{
Aleluia, aleluia. ℣. Eis o sacerdote que o Senhor coroou. Aleluia.
}\end{paracol}

\textit{Após a Septuagésima omite-se o Aleluia e o seguinte e diz-se:}

\paragraphinfo{Trato}{Sl. 111, 1-3}
\begin{paracol}{2}\latim{
\rlettrine{B}{eátus} vir, qui timet Dóminum: in mandátis ejus cupit nimis. ℣. Potens in terra erit semen ejus: generátio rectórum benedicétur. ℣. Glória et divítiæ in domo ejus: et justítia ejus manet in sǽculum sǽculi.
}\switchcolumn\portugues{
\rlettrine{B}{em-aventurado} o varão que teme o Senhor e que põe todo seu zelo em obedecer-Lhe. ℣. Sua descendência será poderosa na terra; pois a geração dos justos será abençoada. ℣. Na sua casa haverá glória e riqueza: e a sua justiça permanecerá em todos os séculos.
}\end{paracol}

\paragraphinfo{Evangelho}{Mt. 16, 24-27}
\begin{paracol}{2}\latim{
\cruz Sequéntia sancti Evangélii secúndum Matthǽum.
}\switchcolumn\portugues{
\cruz Continuação do santo Evangelho segundo S. Mateus.
}\switchcolumn*\latim{
\blettrine{I}{n} illo témpore: Dixit Jesus discípulis suis: Si quis vult post me veníre, ábneget semetípsum, et tollat crucem suam, et sequátur me. Qui enim voluerit ánimam suam salvam fácere, perdet eam: qui autem perdíderit ánimam suam propter me, invéniet eam. Quid enim prodest hómini, si mundum univérsum lucrétur, ánimæ vero suæ detriméntum patiátur? Aut quam dabit homo commutatiónem pro ánima sua? Fílius enim hóminis ventúrus est in glória Patris sui cum Angelis suis: et tunc reddet unicuíque secúndum ópera ejus.
}\switchcolumn\portugues{
\blettrine{N}{aquele} tempo, disse Jesus aos discípulos: «Se alguém quer vir após mim, negue-se a si próprio, tome a sua cruz e siga-me. Porque aquele que quiser salvar a sua vida perdê-la-á; e aquele que tiver perdido a sua vida por mim encontrá-la-á. De que serve ao homem ganhar todo o mundo, se isto vier em prejuízo da sua alma? Ou o que dará um homem em troca de sua alma? Porque o Filho do homem há-de vir na glória de seu Pai com os Anjos, e então dará a cada um segundo as suas obras».
}\end{paracol}

\paragraphinfo{Ofertório}{Sl. 88, 21-22}
\begin{paracol}{2}\latim{
\rlettrine{I}{nvéni} David servum meum, oleo sancto meo unxi eum: manus enim mea auxiliábitur ei, et bráchium meum confortábit eum.
}\switchcolumn\portugues{
\rlettrine{E}{ncontrei} o meu servo David: e ungi-o com meu óleo sagrado: a minha mão o socorrerá e o meu braço o fortalecerá.
}\end{paracol}

\paragraph{Secreta}
\begin{paracol}{2}\latim{
\rlettrine{M}{únera} tibi, Dómine, dicáta sanctífica: et, intercedénte beáto {\redx N.} Mártyre tuo atque Pontífice, per éadem nos placátus inténde. Per Dóminum \emph{\&c.}
}\switchcolumn\portugues{
\rlettrine{S}{antificai,} Senhor, estes dons que Vos são oferecidos, e, por intercessão do B. {\redx N.} vosso Mártir e Pontífice, olhai aplacado para nós. Por nosso Senhor \emph{\&c.}
}\end{paracol}

\paragraphinfo{Comúnio}{Sl. 20, 4}
\begin{paracol}{2}\latim{
\rlettrine{P}{osuísti,} Dómine, in cápite ejus corónam de lápide pretióso.
}\switchcolumn\portugues{
\rlettrine{S}{enhor,} impusestes na sua cabeça uma coroa de pedras preciosas.
}\end{paracol}

\paragraph{Postcomúnio}
\begin{paracol}{2}\latim{
\rlettrine{H}{æc} nos commúnio, Dómine, purget a crímine: et, intercedénte beáto {\redx N.} Mártyre tuo atque Pontífice, cœléstis remédii fáciat esse consórtes. Per Dóminum nostrum \emph{\&c.}
}\switchcolumn\portugues{
\qlettrine{Q}{ue} esta comunhão, Senhor, nos purifique de todos nossos crimes, e, por intercessão do B. {\redx N.}, vosso Mártir e Pontífice, nos torne participante do remédio celestial. Por nosso Senhor \emph{\&c.}
}\end{paracol}
