\subsectioninfo{Mártir não Pontífice}{Missa In virtúte tua}\label{martirnaopontifice1}

\paragraphinfo{Intróito}{Sl. 20, 2-3}
\begin{paracol}{2}\latim{
\rlettrine{I}{n} virtúte tua, Dómine, lætábitur justus: et super salutáre tuum exsultábit veheménter: desidérium ánimæ ejus tribuísti ei. \emph{Ps. ibid., 4} Quóniam prævenísti eum in benedictiónibus dulcédinis: posuísti in cápite ejus corónam de lápide pretióso.
℣. Gloria Patri \emph{\&c.}
}\switchcolumn\portugues{
\rlettrine{O}{} justo rejubilará com vosso poder, Senhor, e exultará de alegria, vendo-se salvo por Vós; pois concedestes-lhe o que seu coração desejava. \emph{Sl. ibid., 4} Com efeito, Vós o premunistes com bênçãos de doçura: e impusestes na sua cabeça uma coroa de pedras preciosas.
℣. Glória ao Pai \emph{\&c.}
}\end{paracol}

\paragraph{Oração}
\begin{paracol}{2}\latim{
\rlettrine{P}{ræsta,} quǽsumus, omnípotens Deus: ut, qui beáti {\redx N.} Mártyris tui natalícia cólimus, intercessióne ejus, in tui nóminis amóre roborémur. Per Dóminum \emph{\&c.}
}\switchcolumn\portugues{
\slettrine{Ó}{} Deus omnipotente, permiti que, celebrando nós o nascimento do B. {\redx N.}, vosso Mártir, e pela sua intercessão, alcancemos a graça de sermos confirmados no amor ao vosso Nome, Por nosso Senhor \emph{\&c.}
}\end{paracol}

\paragraphinfo{Epístola}{Sb. 10, 10-14}
\begin{paracol}{2}\latim{
Léctio libri Sapiéntiæ.
}\switchcolumn\portugues{
Lição do Livro da Sabedoria.
}\switchcolumn*\latim{
\qlettrine{J}{ustum} dedúxit Dóminus per vias rectas, et ostendit illi regnum Dei, et dedit illi sciéntiam sanctórum: honestávit illum in labóribus, et complévit labores illíus. In fraude circumveniéntium illum áffuit illi, et honéstum fecit illum. Custodívit illum ab inimícis, et a seductóribus tutávit illum, et certámen forte dedit illi, ut vínceret et sciret, quóniam ómnium poténtior est sapiéntia. Hæc vénditum jusíum non derelíquit, sed a peccatóribus liberávit eum: descendítque cum illo in fóveam, et in vínculis non derelíquit illum, donec afférret illi sceptrum regni, et poténtiam advérsus eos, qui eum deprimébant: et mendáces osténdit, qui maculavérunt illum, et dedit illi claritátem ætérnam, Dóminus, Deus noster.
}\switchcolumn\portugues{
\rlettrine{O}{} Senhor conduziu o justo por caminhos direitos; mostrou-lhe o reino de Deus; transmitiu-lhe a ciência das cousas santas; enriqueceu-o nos seus trabalhos; e fez frutificar esses seus labores. O Senhor auxiliou-o contra os que queriam enganá-lo com suas fraudes e fê-lo adquirir riquezas. Protegeu-o contra os seus inimigos; defendeu-o de seus sedutores; e alcançou a vitória em um rude combate em seu favor, para lhe ensinar que a sabedoria é a mais poderosa de todas as cousas. O Senhor não abandonou o justo quando este foi vendido, mas até o preservou das mãos dos pecadores; desceu com ele á prisão; e o não abandonou nas cadeias, enquanto lhe não entregou o ceptro do império e o poder sobre os seus opressores. O Senhor, nosso Deus, provou que eram mentirosos aqueles que o desacreditaram e tornou-o ilustre para sempre.
}\end{paracol}

\paragraphinfo{Gradual}{Sl. 111, 1-2}
\begin{paracol}{2}\latim{
\rlettrine{B}{eátus} vir, qui timet Dóminum: in mandátis ejus cupit nimis. ℣. Potens in terra erit semen ejus: generátio rectórum benedicétur.
}\switchcolumn\portugues{
\rlettrine{B}{em-aventurado} o varão que teme o Senhor e que põe todo seu zelo em obedecer-Lhe. ℣. Sua descendência será poderosa na terra; pois a geração dos justos será abençoada.
}\switchcolumn*\latim{
 Allelúja, allelúja. ℣. \emph{Ps. 20, 4} Posuísti, Dómine, super caput ejus corónam de lápide pretióso. Allelúja.
}\switchcolumn\portugues{
Aleluia, aleluia. ℣. \emph{Sl. 20, 4} Senhor, impusestes na sua cabeça uma coroa de pedras preciosas. Aleluia.
}\end{paracol}

\textit{Após a Septuagésima omite-se o Aleluia e o seguinte e diz-se:}

\paragraphinfo{Trato}{Sl. 20, 3-4}
\begin{paracol}{2}\latim{
\rlettrine{D}{esidérium} ánimæ ejus tribuísti ei: et voluntáte labiórum ejus non fraudásti eum. ℣. Quóniam prævenísti eum in benedictiónibus dulcédinis. ℣. Posuísti in cápite ejus corónam de lápide pretióso.
}\switchcolumn\portugues{
\rlettrine{C}{oncedestes-lhe} o desejo da sua alma; lhe não negastes o que seus lábios Vos pediram. ℣. Premuniste-lo com bênçãos de doçura. ℣. Impusestes na sua cabeça uma coroa de pedras preciosas.
}\end{paracol}

\paragraphinfo{Evangelho}{Mt. 10, 34-42}
\begin{paracol}{2}\latim{
\cruz Sequéntia sancti Evangélii secúndum Matthǽum.
}\switchcolumn\portugues{
\cruz Continuação do santo Evangelho segundo S. Mateus.
}\switchcolumn*\latim{
\blettrine{I}{n} illo témpore: Dixit Jesus discípulis suis: Nolíte arbitrári, quia pacem vénerim míttere in terram: non veni pacem míttere, sed gládium. Veni enim separáre hóminem advérsus patrem suum, et fíliam advérsus matrem suam, et nurum advérsus socrum suam: et inimíci hóminis doméstici ejus. Qui amat patrem aut matrem plus quam me, non est me dignus: et qui amat fílium aut fíliam super me, non est me dignus. Et qui non áccipit crucem suam, et séquitur me, non est me dignus. Qui invénit ánimam suam, perdet illam: et qui perdíderit ánimam suam propter me, invéniet eam. Qui récipit vos, me récipit: et qui me récipit, récipit eum, qui me misit. Qui récipit prophétam in nómine prophétæ, mercédem prophétæ accípiet: et qui récipit justum in nómine justi, mercédem justi accípiet. Et quicúmque potum déderit uni ex mínimis istis cálicem aquæ frígidæ tantum in nómine discípuli: amen, dico vobis, non perdet mercédem suam.
}\switchcolumn\portugues{
\blettrine{N}{aquele} tempo, disse Jesus aos discípulos: «Não penseis que vim trazer a paz à terra; não vim trazer a paz, mas o gládio; pois vim separar o homem de seu pai; a filha de sua mãe; e a nora de sua sogra. O homem terá como inimigos os seus próprios criados. Aquele que ama seu pai ou sua mãe mais do que a mim não é digno de mim; e aquele que ama seu filho ou filha mais do que a mim não é digno de mim. Quem não toma a sua cruz e me não segue não é digno de mim. Aquele que conserva a sua vida perdê-la-á; e aquele que por amor de mim a perder achá-la-á. Aquele que vos recebe recebe-me a mim; e o que me recebe, recebe Aquele que me enviou. Aquele que recebe um profeta na qualidade de profeta receberá a recompensa de profeta; e aquele que recebe um justo na qualidade de justo receberá a recompensa de justo. Todo aquele que der de beber, mesmo que seja um copo de água fria, a um destes pequenos, como sendo meu discípulo, eu vos digo, na verdade, que não perderá a recompensa.
}\end{paracol}

\paragraphinfo{Ofertório}{Sl. 8, 6-7}
\begin{paracol}{2}\latim{
\rlettrine{G}{lória} et honóre coronásti eum: et constituísti eum super ópera mánuum tuárum, Dómine.
}\switchcolumn\portugues{
\rlettrine{V}{ós} o coroastes, Senhor, com glória e com honras; Vós lhe concedestes o domínio sobre as obras das vossas mãos.
}\end{paracol}

\paragraph{Secreta}
\begin{paracol}{2}\latim{
\rlettrine{M}{unéribus} nostris, quǽsumus, Dómine, precibúsque suscéptis: et cœléstibus nos munda mystériis, et cleménter exáudi. Per Dóminum nostrum \emph{\&c.}
}\switchcolumn\portugues{
\rlettrine{H}{avendo} Vós, Senhor, aceitado as nossas ofertas e orações, dignai-Vos purificar-nos com vossos celestiais mystérios e ouvir-nos benignamente. Por nosso Senhor \emph{\&c.}
}\end{paracol}

\paragraphinfo{Comúnio}{Mt. 16, 24}
\begin{paracol}{2}\latim{
\qlettrine{Q}{ui} vult veníre post me, ábneget semetípsum, et tollat crucem suam, et sequátur me.
}\switchcolumn\portugues{
\rlettrine{S}{e} alguém quer vir após mim, negue-se a si mesmo, tome a sua cruz e siga-me!
}\end{paracol}

\paragraph{Postcomúnio}
\begin{paracol}{2}\latim{
\rlettrine{D}{a,} quǽsumus, Dómine, Deus noster: ut, sicut tuórum commemoratióne Sanctórum temporáli gratulámur offício; ita perpétuo lætámur aspéctu. Per Dóminum nostrum \emph{\&c.}
}\switchcolumn\portugues{
\rlettrine{P}{ermiti,} ó Senhor, nosso Deus, Vos suplicamos, que, assim como nos alegramos, comemorando nesta vida pelo nosso ministério a memória dos vossos Santos, assim também tenhamos a felicidade de os contemplar na eternidade. Por nosso Senhor \emph{\&c.}
}\end{paracol}
