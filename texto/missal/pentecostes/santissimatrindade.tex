\subsection{Domingo da Santíssima Trindade}\label{domingosantissimatrindade}

\paragraphinfo{Intróito}{Tb. 12, 6}
\begin{paracol}{2}\latim{
\rlettrine{B}{enedícta} sit sancta Trínitas atque indivísa Unitas: confitébimur ei, quia fecit nobíscum misericórdiam suam. \emph{Ps. 8, 2} Dómine, Dóminus noster, quam admirábile est nomen tuum in univérsa terra
℣. Gloria Patri \emph{\&c.}
}\switchcolumn\portugues{
\rlettrine{B}{endita} seja a Santíssima Trindade e a indivisível Unidade! Nós A louvamos; pois foi misericordiosa para connosco. \emph{Sl. 8, 2} Senhor, nosso Deus, como o vosso nome é admirável em todo o universo!
℣. Glória ao Pai \emph{\&c.}
}\end{paracol}

\paragraph{Oração}
\begin{paracol}{2}\latim{
\rlettrine{O}{mnípotens} sempitérne Deus, qui dedísti fámulis tuis in confessióne veræ fídei, ætérnæ Trinitátis glóriam agnóscere, et in poténtia majestátis adoráre Unitátem: quǽsumus; ut, ejúsdem fídei firmitáte, ab ómnibus semper muniámur advérsis. Per Dóminum nostrum \emph{\&c.}
}\switchcolumn\portugues{
\slettrine{Ó}{} omnipotente e eterno Deus, que quisestes que os vossos fiéis, acreditando e confessando a verdadeira fé, conhecessem a glória da sempiterna Trindade e adorassem a Unidade no poder da sua majestade, permiti, Vos imploramos, que, guardando nós firmemente a mesma fé, vençamos sempre todas as adversidades. Por nosso Senhor \emph{\&c.}
}\end{paracol}

\paragraphinfo{Epístola}{Rm. 11, 33-36}
\begin{paracol}{2}\latim{
Léctio Epístolæ beáti Pauli Apóstoli ad Romános
}\switchcolumn\portugues{
Lição da Ep.ª do B. Ap.º Paulo aos Romanos.
}\switchcolumn*\latim{
\rlettrine{O}{} altitúdo divitiárum sapiéntiae et sciéntiæ Dei: quam incomprehensibília sunt judícia ejus, et investigábiles viæ ejus! Quis enim cognovit sensum Dómini? Aut quis consiliárius ejus fuit? Aut quis prior dedit illi, et retribuétur ei? Quóniam ex ipso et per ipsum et in ipso sunt ómnia: ipsi glória in sǽcula. Amen.
}\switchcolumn\portugues{
\slettrine{Ó}{} abysmo das riquezas, da sabedoria e da ciência de Deus! Como são incompreensíveis os seus juízos e impenetráveis os seus caminhos! Pois quem, porventura, conheceu já os desígnios do Senhor? Quem foi seu conselheiro? Quem foi o primeiro a dar-Lhe alguma cousa, para depois receber recompensa? Pois tudo quanto existe é d’Ele, para Ele e n’Ele. A Ele, pois, seja dada glória em todos os séculos. Amen.
}\end{paracol}

\paragraphinfo{Gradual}{Dn. 3, 55-56}
\begin{paracol}{2}\latim{
\rlettrine{B}{enedíctus} es, Dómine, qui intuéris abýssos, et sedes super Chérubim. ℣. Benedíctus es, Dómine, in firmaménto cæli, et laudábilis in sǽcula.
}\switchcolumn\portugues{
\rlettrine{B}{endito} sois, Senhor, que sondais as profundezas dos abysmos e estais assentado sobre os Querubins! ℣. Sim, Senhor, sois bendito no firmamento do céu; e digno de louvor em todos os séculos.
}\switchcolumn*\latim{
Allelúja, allelúja. ℣. \emph{ibid., 52} Benedíctus es, Dómine, Deus patrum nostrórum, et laudábilis in sǽcula. Allelúja.
}\switchcolumn\portugues{
Aleluia, aleluia. ℣. \emph{ibid., 52} Bendito sois, Senhor, Deus de nossos pais, e digno de louvor em todos os séculos. Aleluia.
}\end{paracol}

\paragraphinfo{Evangelho}{Mt. 28, 18-20}
\begin{paracol}{2}\latim{
\cruz Sequéntia sancti Evangélii secúndum Matthǽum.
}\switchcolumn\portugues{
\cruz Continuação do santo Evangelho segundo S. Mateus.
}\switchcolumn*\latim{
\blettrine{I}{n} illo témpore: Dixit Jesus discípulis suis: Data est mihi omnis potéstas in cœlo et in terra. Eúntes ergo docéte omnes gentes, baptizántes eos in nómine Patris, et Fílii, et Spíritus Sancti: docéntes eos serváre ómnia, quæcúmque mandávi vobis. Et ecce, ego vobíscum sum ómnibus diébus usque ad consummatiónem sǽculi.
}\switchcolumn\portugues{
\blettrine{N}{aquele} tempo, disse Jesus aos discípulos: «Todo o poder me foi dado no céu e na terra. Ide, pois; ensinai todos os povos, baptizando-os em nome do Pai, e do Filho, e do Espírito Santo. Ensinai-os a observar tudo o que vos mandei. E sabei que estou convosco todos os dias até à consumação dos séculos».
}\end{paracol}

\paragraphinfo{Ofertório}{Tb. 12, 6}
\begin{paracol}{2}\latim{
\rlettrine{B}{enedíctus} sit Deus Pater, unigenitúsque Dei Fílius, Sanctus quoque Spíritus: quia fecit nobíscum misericórdiam suam.
}\switchcolumn\portugues{
\rlettrine{B}{endito} seja Deus Pai, e o Filho Unigénito de Deus, e também o Espírito Santo: pois foi misericordioso para connosco.
}\end{paracol}

\paragraph{Secreta}
\begin{paracol}{2}\latim{
\rlettrine{S}{anctífica,} quǽsumus, Dómine, Deus noster, per tui sancti nóminis invocatiónem, hujus oblatiónis hóstiam: et per eam nosmetípsos tibi pérfice munus ætérnum. Per Dóminum nostrum \emph{\&c.}
}\switchcolumn\portugues{
\rlettrine{D}{ignai-Vos,} Senhor, nosso Deus, Vos suplicamos, pela invocação do vosso santo nome, santificar esta hóstia, que Vos oferecemos; e que por ela nos convertamos em dom perpétuo de homenagem à vossa majestade. Por nosso Senhor \emph{\&c.}
}\end{paracol}

\paragraphinfo{Comúnio}{Tb. 12, 6}
\begin{paracol}{2}\latim{
\rlettrine{B}{enedícimus} Deum cœli et coram ómnibus vivéntibus confitébimur ei: quia fecit nobíscum misericórdiam suam.
}\switchcolumn\portugues{
\rlettrine{B}{endizemos} Deus do céu e cantamos os seus louvores diante de todos os viventes: pois foi misericordioso para connosco.
}\end{paracol}

\paragraph{Postcomúnio}
\begin{paracol}{2}\latim{
\rlettrine{P}{rofíciat} nobis ad salútem córporis et ánimæ, Dómine, Deus noster, hujus sacraménti suscéptio: et sempitérnæ sanctæ Trinitátis ejusdémque indivíduæ Unitátis conféssio. Per Dóminum \emph{\&c.}
}\switchcolumn\portugues{
\qlettrine{Q}{ue} a recepção deste sacramento, assim como a crença, que confessamos, na santa e eterna Trindade e indivisível Unidade, nos sirvam, Senhor, de proveito para a salvação da alma e do corpo. Por nosso Senhor \emph{\&c.}
}\end{paracol}
