\subsection{Quinto Domingo depois da Epifania}

\paragraphinfo{Intróito}{Jr. 29,11,12 \& 14}
\begin{paracol}{2}\latim{
\rlettrine{D}{icit} Dóminus: Ego cógito cogitatiónes pacis, et non afflictiónis: invocábitis me, et ego exáudiam vos: et redúcam captivitátem vestram de cunctis locis. \emph{Ps. 84, 2} Benedixísti, Dómine, terram tuam: avertísti captivitátem Jacob.
℣. Gloria Patri \emph{\&c.}
}\switchcolumn\portugues{
\rlettrine{D}{isse} o Senhor: tenho pensamentos de paz e não de ira: invocar-me-eis e ouvir-Vos-ei; e farei regressar de todos os países os vossos cativos. \emph{Sl. 84, 2} Abençoastes, Senhor, a vossa terra e livrastes Jacob do cativeiro.
℣. Glória ao Pai \emph{\&c.}
}\end{paracol}

\paragraph{Oração}
\begin{paracol}{2}\latim{
\rlettrine{P}{ræsta,} quǽsumus, omnípotens Deus: ut, semper rationabília meditántes, quæ tibi sunt plácita, et dictis exsequámur et factis. Per Dóminum \emph{\&c.}
}\switchcolumn\portugues{
\rlettrine{P}{ermiti,} Vos imploramos, ó Deus omnipotente, que, meditando nós incessantemente nas cousas santas, pratiquemos sempre, tanto em palavras como em acções, o que seja do vosso agrado. Por nosso Senhor \emph{\&c.}
}\end{paracol}

\paragraphinfo{Epístola}{1 Ts, 1, 2-10}
\begin{paracol}{2}\latim{
Léctio Epístolæ beáti Pauli Apóstoli ad Thessalonicénses.
}\switchcolumn\portugues{
Lição da Ep.ª do B. Ap.º Paulo aos Tessalonicenses.
}\switchcolumn*\latim{
\rlettrine{F}{ratres:} Grátias ágimus Deo semper pro ómnibus vobis, memóriam vestri faciéntes in oratiónibus nostris sine intermissióne, mémores óperis fídei vestræ, et labóris, et caritátis, et sustinéntiæ spei Dómini nostri Jesu Christi, ante Deum et Patrem nostrum: sciéntes, fratres, dilécti a Deo, electiónem vestram: quia Evangélium nostrum non fuit ad vos in sermóne tantum, sed et in virtúte, et in Spíritu Sancto, et in plenitúdine multa, sicut scitis quales fuérimus in vobis propter vos. Et vos imitatóres nostri facti estis, et Dómini, excipiéntes verbum in tribulatióne multa, cum gáudio Spíritus Sancti: ita ut facti sitis forma ómnibus credéntibus in Macedónia et in Achája. A vobis enim diffamátus est sermo Dómini, non solum in Macedónia et in Achája, sed et in omni loco fides vestra, quæ est ad Deum, profécta est, ita ut non sit nobis necésse quidquam loqui. Ipsi enim de nobis annúntiant, qualem intróitum habuérimus ad vos: et quómodo convérsi estis ad Deum a simulácris, servíre Deo vivo et vero, et exspectáre Fílium ejus de cœlis (quem suscitávit ex mórtuis) Jesum, qui erípuit nos ab ira ventúra.
}\switchcolumn\portugues{
\rlettrine{M}{eus} irmãos: Continuamente damos graças a Deus por vós todos, lembrando-nos de vós sem cessar nas nossas orações e recordando diante de Deus, nosso Pai, as obras da vossa fé, os trabalhos da vossa caridade e a constância da vossa esperança em Cristo. Sabemos, irmãos amados de Deus, que fostes escolhidos; pois o nosso Evangelho vos não foi pregado somente com palavras, mas também com milagres, pelo poder do Espírito Santo e com a plenitude de muitos dons. Não ignorais, também, como temos procedido no meio de vós, por causa da vossa salvação. E assim vos fizestes nossos imitadores, e do Senhor, recebendo a sua palavra no meio de muitas tribulações com a alegria do Espírito Santo, de sorte que vos tornastes modelo para todos os fiéis da Macedónia e da Acaia. Pois nem só fostes a causa de que a palavra do Senhor se transmitisse na Macedónia e na Acaia, mas também a vossa fé em Deus se fez conhecer em todo o lugar; e tanto que nem é necessário que falemos em tal. Todos esses povos apregoam o êxito que alcançámos junto de vós, e como vos convertestes, abandonando os ídolos para servir a Deus vivo e verdadeiro e para esperar do céu o seu Filho Jesus, que ressuscitou dos mortos e nos livrou da ira futura.
}\end{paracol}

\paragraphinfo{Gradual}{Sl. 43, 8-9}
\begin{paracol}{2}\latim{
\rlettrine{L}{iberásti} nos, Dómine, ex affligéntibus nos: et eos, qui nos odérunt, confudísti. ℣. In Deo laudábimur tota die, et in nómine tuo confitébimur in sǽcula.
}\switchcolumn\portugues{
\rlettrine{L}{ivrastes-nos,} Senhor, daqueles que nos afligiam: e confundistes os que nos odiavam. Glorificar-nos-emos constantemente em Deus e louvaremos eternamente o vosso nome.
}\switchcolumn*\latim{
Allelúja, allelúja. ℣. \emph{Ps. 129, 12} De profúndis clamávi ad te, Dómine: Dómine, exáudi oratiónem meam. Allelúja.
}\switchcolumn\portugues{
Aleluia, aleluia. ℣. \emph{Sl. 129, 12} Do fundo do abysmo Vos invoquei, Senhor: escutai a minha oração. Aleluia.
}\end{paracol}

\paragraphinfo{Evangelho}{Mt. 13, 31-35}
\begin{paracol}{2}\latim{
\cruz Sequéntia sancti Evangélii secúndum Matthǽum.
}\switchcolumn\portugues{
\cruz Continuação do santo Evangelho segundo S. Mateus.
}\switchcolumn*\latim{
\blettrine{I}{n} illo témpore: Dixit Jesus turbis parábolam hanc: Símile est regnum cœlórum grano sinápis, quod accípiens homo seminávit in agro suo: quod mínimum quidem est ómnibus semínibus: cum autem créverit, majus est ómnibus oléribus, et fit arbor, ita ut vólucres cœli véniant et hábitent in ramis ejus. Aliam parábolam locútus est eis: Símile est regnum cœlórum ferménto, quod accéptum múlier abscóndit in farínæ satis tribus, donec fermentátum est totum. Hæc ómnia locútus est Jesus in parábolis ad turbas: et sine parábolis non loquebátur eis: ut implerétur quod dictum erat per Prophétam dicéntem: Apériam in parábolis os meum, eructábo abscóndita a constitutióne mundi.
}\switchcolumn\portugues{
\blettrine{N}{aquele} tempo, Jesus disse às turbas: O reino dos céus é semelhante a um grão de mostarda que um homem tomou e semeou no seu campo, a qual é a mais Pequena de todas as sementes, mas, quando cresce, torna-se na maior de todas as hortaliças, e forma-se uma árvore de tal modo grande que as aves do céu vêm pousar nos seus ramos. Depois, ainda Jesus lhes disse: O reino dos céus é semelhante ao fermento que uma mulher toma e mistura em três medidas de farinha, até que todas estejam lêvedas. Tudo isto disse Jesus em parábolas; e sem parábolas não falava, para que se cumprisse o que fora dito pelo Profeta: «Abrirei a minha boca em parábolas e revelarei muitas cousas que estão ocultas desde a criação do mundo».
}\end{paracol}

\paragraphinfo{Ofertório}{Sl. 129, 1-2}
\begin{paracol}{2}\latim{
\rlettrine{D}{e} profúndis clamávi ad te, Dómine: Dómine, exáudi oratiónem meam: de profúndis clamávi ad te. Dómine.
}\switchcolumn\portugues{
\rlettrine{D}{as} profundezas dos abysmos Vos invoquei, Senhor; escutai, Senhor, a minha voz: das profundezas dos abysmos Vos invoquei.
}\end{paracol}

\paragraph{Secreta}
\begin{paracol}{2}\latim{
\rlettrine{H}{æc} nos oblátio, Deus, mundet, quǽsumus, et rénovet, gubérnet et prótegat. Per Dóminum \emph{\&c.}
}\switchcolumn\portugues{
\slettrine{Ó}{} Deus, Vos imploramos, que esta oblação nos purifique, restaure, governe e guarde. Por nosso Senhor \emph{\&c.}
}\end{paracol}

\paragraphinfo{Comúnio}{Mc. 11, 24}
\begin{paracol}{2}\latim{
\rlettrine{A}{men,} dico vobis, quidquid orántes pétitis, crédite, quia accipiétis, et fiet vobis.
}\switchcolumn\portugues{
\rlettrine{N}{a} verdade vos digo: «Tudo o que pedirdes nas vossas orações, acreditai que o recebereis; e far-se-á como pedirdes».
}\end{paracol}

\paragraph{Postcomúnio}
\begin{paracol}{2}\latim{
\rlettrine{C}{œléstibus,} Dómine, pasti delíciis: quǽsumus; ut semper éadem, per quæ veráciter vívimus, appétimus. Per Dóminum \emph{\&c.}
}\switchcolumn\portugues{
\rlettrine{A}{limentados} com as celestiais delícias, Senhor, Vos pedimos humildemente, concedei-nos que aspiremos continuamente a este mesmo alimento, pelo qual alcançaremos a verdadeira vida. Por nosso Senhor \emph{\&c.}
}\end{paracol}
