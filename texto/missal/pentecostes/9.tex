\subsection{Nono Domingo depois de Pentecostes}

\paragraphinfo{Intróito}{Sl. 53, 6-7}
\begin{paracol}{2}\latim{
\rlettrine{E}{cce,} Deus adjuvat me, et Dóminus suscéptor est ánimæ meæ: avérte mala inimícis meis, et in veritáte tua dispérde illos, protéctor meus, Dómine. \emph{Ps. ibid., 3} Deus, in nómine tuo salvum me fac: et in virtúte tua libera me.
℣. Gloria Patri \emph{\&c.}
}\switchcolumn\portugues{
\rlettrine{E}{is} que Deus vem em meu auxílio: eis que o Senhor é o sustentáculo da minha alma! Lançai, pois, sobre os meus inimigos, ó Deus, meu protector, o mal que me querem fazer; e, conforme a vossa palavra verdadeira, exterminai-os. \emph{Sl. ibid., 3} Ó Deus, pelo vosso nome, salvai-me; e, pelo vosso poder, livrai-me.
℣. Glória ao Pai \emph{\&c.}
}\end{paracol}

\paragraph{Oração}
\begin{paracol}{2}\latim{
\rlettrine{P}{áteant} aures misericórdiæ tuæ, Dómine, précibus supplicántium: et, ut peténtibus desideráta concédas; fac eos quæ tibi sunt plácita, postuláre. Per Dóminum \emph{\&c.}
}\switchcolumn\portugues{
\qlettrine{Q}{ue} os ouvidos da vossa misericórdia, Senhor, sejam atentos às preces dos vossos suplicantes; e, para que lhes concedais o que Vos suplicam, inspirai-os a que Vos peçam o que seja do vosso agrado. Por nosso Senhor \emph{\&c.}
}\end{paracol}

\paragraphinfo{Epístola}{1. Cor. 10, 6-13}
\begin{paracol}{2}\latim{
Léctio Epístolæ beáti Pauli Apóstoli ad Corinthios.
}\switchcolumn\portugues{
Lição da Ep.ª do B. Ap.º Paulo aos Coríntios.
}\switchcolumn*\latim{
\rlettrine{F}{atres:} Non simus concupiscéntes malórum, sicut et illi concupiérunt. Neque idolólatræ efficiámini, sicut quidam ex ipsis: quemádmodum scriptum est: Sedit pópulus manducáre et bíbere, et surrexérunt lúdere. Neque fornicémur, sicut quidam ex ipsis fornicáti sunt, et cecidérunt una die vigínti tria mília. Neque tentémus Christum, sicut quidam eórum tentavérunt, et a serpéntibus periérunt. Neque murmuravéritis, sicut quidam eórum murmuravérunt, et periérunt ab exterminatóre. Hæc autem ómnia in figúra contingébant illis: scripta sunt autem ad correptiónem nostram, in quos fines sæculórum devenérunt. Itaque qui se exístimat stare, vídeat ne cadat. Tentátio vos non apprehéndat, nisi humána: fidélis autem Deus est, qui non patiétur vos tentári supra id, quod potéstis, sed fáciet étiam cum tentatióne provéntum, ut póssitis sustinére.
}\switchcolumn\portugues{
\rlettrine{M}{eus} irmãos: Não sejamos concupiscentes das cousas más, como nossos antepassados, nem vos torneis idólatras, como alguns deles, a respeito dos quais está escrito: «Assentou-se o povo para comer e beber, e levantou-se para se divertir». Não cometamos, pois, acções impuras, como alguns deles o fizeram, pelo que pereceram em um só dia vinte e três mil. Não tentemos a Cristo, como alguns deles tentaram, pelo que pereceram de mordeduras de serpentes. Não murmureis, como alguns deles murmuraram, pelo que foram feridos mortalmente pelo Anjo exterminador. Todas estas cousas eram como que figuras, e foram escritas para advertência de nós, que viemos no fim dos tempos. Aquele, pois, que crê estar seguro, tenha cuidado, não caia. Nenhuma tentação nos sobreveio que não fosse humana; porém, Deus é fiel e não permitirá que sejais tentados além das vossas forças, mas antes fará que tireis proveito da tentação, a fim de que possais suportá-la.
}\end{paracol}

\paragraphinfo{Gradual}{Sl. 8, 2}
\begin{paracol}{2}\latim{
\rlettrine{D}{ómine,} Dóminus noster, quam admirábile est nomen tuum in universa terra! ℣. Quóniam eleváta est magnificéntia tua super cœlos.
}\switchcolumn\portugues{
\slettrine{Ó}{} Senhor, nosso Deus, como é admirável o vosso nome em todo o universo. ℣. Vossa magnificência eleva-se sobre os céus!
}\switchcolumn*\latim{
Allelúja, allelúja. ℣. \emph{Ps. 58, 2} Eripe me de inimícis meis, Deus meus: et ab insurgéntibus in me líbera me. Allelúja.
}\switchcolumn\portugues{
Aleluia, aleluia. ℣. \emph{Sl. 58, 2} Livrai-me das mãos dos meus inimigos, ó meu Deus; livrai-me daqueles que se insurgem contra mim. Aleluia.
}\end{paracol}

\paragraphinfo{Evangelho}{Lc. 19, 41-47}
\begin{paracol}{2}\latim{
\cruz Sequéntia sancti Evangélii secúndum Lucam.
}\switchcolumn\portugues{
\cruz Continuação do santo Evangelho segundo S. Lucas.
}\switchcolumn*\latim{
\blettrine{I}{n} illo témpore: Cum appropinquáret Jesus Jerúsalem, videns civitátem, flevit super illam, dicens: Quia si cognovísses et tu, et quidem in hac die tua, quæ ad pacem tibi, nunc autem abscóndita sunt ab óculis tuis. Quia vénient dies in te: et circúmdabunt te inimíci tui vallo, et circúmdabunt te: et coangustábunt te úndique: et ad terram prostérnent te, et fílios tuos, qui in te sunt, et non relínquent in te lápidem super lápidem: eo quod non cognóveris tempus visitatiónis tuæ. Et ingréssus in templum, cœpit ejícere vendéntes in illo et eméntes, dicens illis: Scriptum est: Quia domus mea domus oratiónis est. Vos autem fecístis illam speluncam latrónum. Et erat docens cotídie in templo.
}\switchcolumn\portugues{
\blettrine{N}{aquele} tempo, havendo Jesus chegado próximo de Jerusalém e vendo esta cidade, chorou sobre ela, dizendo: «Ah! se tu, ao menos neste dia, que te foi dado, conhecesses o que te pode dar a paz!... Mas, entretanto, tudo está oculto a teus olhos! Pois virão dias infelizes para ti, em que os teus inimigos te cercarão de trincheiras, te sitiarão e te fecharão por todos os lados. Então, te destruirão completamente, assim como aos teus filhos, que estão dentro de ti, não deixando pedra sobre pedra. Porquanto não quiseste conhecer o tempo em que foste visitada». Depois Jesus entrou no templo, expulsando aqueles que lá vendiam ou compravam; e dizia-lhes: Está escrito: «Minha casa é casa de oração, fizestes dela caverna de ladrões!». E Jesus ensinava todos os dias no templo.
}\end{paracol}

\paragraphinfo{Ofertório}{Sl. 18, 9, 10, 11 \& 12}
\begin{paracol}{2}\latim{
\qlettrine{J}{ustítiæ} Dómini rectæ, lætificántes corda, et judícia ejus dulcióra super mel et favum: nam et servus tuus custódit ea.
}\switchcolumn\portugues{
\rlettrine{O}{s} preceitos do Senhor são justos: alegram o coração. Seus juízos são mais doces do que o mel e o favo do mel. Assim, pois, o vosso servo, Senhor, guardá-los-á.
}\end{paracol}

\paragraph{Secreta}
\begin{paracol}{2}\latim{
\rlettrine{C}{oncéde} nobis, quǽsumus, Dómine, hæc digne frequentáre mystéria: quia, quóties hujus hóstiæ commemorátio celebrátur, opus nostræ redemptiónis exercétur. Per Dóminum \emph{\&c.}
}\switchcolumn\portugues{
\rlettrine{C}{oncedei-nos,} Senhor, Vos suplicamos, a graça de recebermos frequente e dignamente estes mystérios, pois cada vez que se celebra este sacrifício opera-se o fruto da nossa redenção. Por nosso Senhor \emph{\&c.}
}\end{paracol}

\paragraphinfo{Comúnio}{Jo. 6, 57}
\begin{paracol}{2}\latim{
\qlettrine{Q}{ui} mandúcat meam carnem et bibit meum sánguinem, in me manet et ego in eo, dicit Dóminus.
}\switchcolumn\portugues{
\rlettrine{A}{quele} que come a minha Carne e bebe o meu Sangue permanece em mim e Eu permaneço nele, diz o Senhor.
}\end{paracol}

\paragraph{Postcomúnio}
\begin{paracol}{2}\latim{
\rlettrine{T}{ui} nobis, quǽsumus, Dómine, commúnio sacraménti, et purificatiónem cónferat, et tríbuat unitátem. Per Dóminum \emph{\&c.}
}\switchcolumn\portugues{
\rlettrine{P}{ermiti,} Senhor, Vos suplicamos, que a nossa comunhão deste sacramento cada vez mais nos purifique e una a Vós. Por nosso Senhor \emph{\&c.}
}\end{paracol}
