\subsectioninfo{Segundo Domingo depois de Pentecostes}{Dentro Oitavária do SS. Corpo do Senhor}\label{2domingopentecostes}

\paragraphinfo{Intróito}{Sl. 17, 19-20}
\begin{paracol}{2}\latim{
\rlettrine{F}{actus} est Dóminus protéctor meus, et edúxit me in latitúdinem: salvum me fecit, quóniam vóluit me. \emph{Ps. ib., 2-3} Díligam te, Dómine, virtus mea: Dóminus firmaméntum meum et refúgium meum et liberátor meus.
℣. Gloria Patri \emph{\&c.}
}\switchcolumn\portugues{
\rlettrine{O}{} Senhor fez-se o meu protector e conduziu-me para o largo: E, porque me amava, salvou-me. \emph{Sl. ib., 2-3} Amar-Vos-ei, Senhor, pois sois a minha fortaleza. Sim, o Senhor é o meu sustentáculo, refúgio e libertador.
℣. Glória ao Pai \emph{\&c.}
}\end{paracol}

\paragraph{Oração}
\begin{paracol}{2}\latim{
\rlettrine{S}{ancti} nóminis tui, Dómine, timórem páriter et amórem fac nos habére perpétuum: quia numquam tua gubernatióne destítuis, quos in soliditáte tuæ dilectiónis instítuis. Per Dóminum \emph{\&c.}
}\switchcolumn\portugues{
\rlettrine{F}{azei,} Senhor, que possuamos perpetuamente o temor e o amor do vosso santo nome, porquanto nunca cessais de inspirar e dirigir aqueles em quem infundistes solidamente o vosso Amor. Por nosso Senhor \emph{\&c.}
}\end{paracol}

\paragraphinfo{Epístola}{1. Jo. 3, 13-18}
\begin{paracol}{2}\latim{
Léctio Epístolæ beáti Joánnis Apóstoli.
}\switchcolumn\portugues{
Lição da Ep.ª do B. Ap.º João.
}\switchcolumn*\latim{
\rlettrine{C}{aríssimi:} Nolíte mirári, si odit vos mundus. Nos scimus, quóniam transláti sumus de morte ad vitam, quóniam dilígimus fratres. Qui non díligit, manet in morte: omnis, qui odit fratrem suum, homícida est. Et scitis, quóniam omnis homícida non habet vitam ætérnam in semetípso manéntem. In hoc cognóvimus caritátem Dei, quóniam ille ánimam suam pro nobis pósuit: et nos debémus pro frátribus ánimas pónere. Qui habúerit substántiam hujus mundi, et víderit fratrem suum necessitátem habére, et cláuserit víscera sua ab eo: quómodo cáritas Dei manet in eo? Filíoli mei, non diligámus verbo neque lingua, sed ópere et veritáte.
}\switchcolumn\portugues{
\rlettrine{C}{aríssimos:} Não vos admireis se porventura o mundo vos odeia. Sabemos que passámos da morte à vida, porque amamos os nossos irmãos. Aquele que não ama, permanece na morte. Aquele que odiar o seu irmão, é homicida; e bem sabeis que em nenhum homicida permanece a vida eterna. Por este sinal conhecemos o amor de Deus: É que, assim como Ele deu a vida por nós, assim também devemos dar a vida pelos nossos irmãos. Se alguém, tendo bens neste mundo e vendo seu irmão com necessidade, lhe fecha o coração, porventura permanece nele o amor de Deus? Meus filhinhos: Não amemos somente com palavras e com desejos, mas com obras e verdade.
}\end{paracol}

\paragraphinfo{Gradual}{Sl. 119, 1-2}
\begin{paracol}{2}\latim{
\rlettrine{A}{d} Dóminum, cum tribulárer, clamávi, et exaudívit me. ℣. Dómine, libera ánimam meam a lábiis iníquis, et a lingua dolósa.
}\switchcolumn\portugues{
\qlettrine{Q}{uando} estava na tribulação, clamei pelo Senhor, que me ouviu. ℣. Livrai, Senhor, a minha alma dos lábios mentirosos e da língua traiçoeira.
}\switchcolumn*\latim{
Allelúja, allelúja. ℣. \emph{Ps. 7, 2} Dómine, Deus meus, in te sperávi: salvum me fac ex ómnibus persequéntibus me et líbera me. Allelúja.
}\switchcolumn\portugues{
Aleluia, aleluia. ℣. \emph{Sl. 7, 2} Senhor, meu Deus, refugio-me em Vós: salvai-me das mãos dos meus perseguidores: livrai-me. Aleluia.
}\end{paracol}

\paragraphinfo{Evangelho}{Lc. 14, 16-24}
\begin{paracol}{2}\latim{
\cruz Sequéntia sancti Evangélii secúndum Lucam.
}\switchcolumn\portugues{
\cruz Continuação do santo Evangelho segundo S. Lucas.
}\switchcolumn*\latim{
\blettrine{I}{n} illo témpore: Dixit Jesus pharisǽis parábolam hanc: Homo quidam fecit cœnam magnam, et vocávit multos. Et misit servum suum hora cœnæ dícere invitátis, ut venírent, quia jam paráta sunt ómnia. Et cœpérunt simul omnes excusáre. Primus dixit ei: Villam emi, et necésse hábeo exíre et vidére illam: rogo te, habe me excusátum. Et alter dixit: Juga boum emi quinque et eo probáre illa: rogo te, habe me excusátum. Et álius dixit: Uxórem duxi, et ídeo non possum veníre. Et revérsus servus nuntiávit hæc dómino suo. Tunc irátus paterfamílias, dixit servo suo: Exi cito in pláteas et vicos civitátis: et páuperes ac débiles et cœcos et claudos íntroduc huc. Et ait servus: Dómine, factum est, ut imperásti, et adhuc locus est. Et ait dóminus servo: Exi in vias et sepes: et compélle intrare, ut impleátur domus mea. Dico autem vobis, quod nemo virórum illórum, qui vocáti sunt, gustábit cœnam meam.
}\switchcolumn\portugues{
\blettrine{N}{aquele} tempo, disse Jesus aos fariseus esta parábola: «Um homem fez uma lauta ceia, convidando para assistir muitas pessoas. À hora da comida, mandou um servo dizer aos convidados que viessem, porque estava tudo preparado. Então, todos, unanimemente, se escusaram. O primeiro disse-lhe: Comprei um campo e tenho necessidade de ir vê-lo; rogo-te, pois, me dês como escusado, segundo disse: Comprei cinco juntas de bois, e vou experimentá-las; peço-te, pois, me escuses. Um outro disse: Eu casei-me; portanto, não posso assistir. Voltando o servo, contou ao senhor todas estas cousas. Então este indignou-se e disse ao servo: Vai depressa por essas praças e ruas da cidade, e conduz para aqui os pobres, os aleijados, os cegos e os coxos. Depois disse o servo: Senhor, fiz o que me mandastes e ainda há lugar. Respondeu ele ao servo: «Vai pelos caminhos e valados e obriga-os a entrar, porque quero que minha casa fique cheia. Eu vos afirmo que nenhum do que haviam sido convidados provará a minha ceia».
}\end{paracol}

\paragraphinfo{Ofertório}{Sl. 6, 5}
\begin{paracol}{2}\latim{
\rlettrine{D}{ómine,} convértere, et éripe ánimam meam: salvum me fac propter misericórdiam tuam.
}\switchcolumn\portugues{
\rlettrine{S}{enhor,} volvei para mim a vossa face: e livrai a minha alma: salvai-me pela vossa misericórdia.
}\end{paracol}

\paragraph{Secreta}
\begin{paracol}{2}\latim{
\rlettrine{O}{blátio} nos, Dómine, tuo nómini dicánda puríficet: et de die in diem ad cœléstis vitæ tránsferat actiónem. Per Dóminum \emph{\&c.}
}\switchcolumn\portugues{
\rlettrine{S}{enhor,} que a oblação que vai ser consagrada ao vosso nome nos purifique; e que dia a dia nos aperfeiçoe na prática duma vida toda celestial. Por nosso Senhor \emph{\&c.}
}\end{paracol}

\paragraphinfo{Comúnio}{Sl. 12, 6}
\begin{paracol}{2}\latim{
\rlettrine{C}{antábo} Dómino, qui bona tríbuit mihi: et psallam nómini Dómini altíssimi.
}\switchcolumn\portugues{
\rlettrine{C}{antarei} hinos ao Senhor, porque me cumulou de benefícios. Cantarei salmos ao nome do altíssimo Senhor.
}\end{paracol}

\paragraph{Postcomúnio}
\begin{paracol}{2}\latim{
\rlettrine{S}{umptis} munéribus sacris, quǽsumus, Dómine: ut cum frequentatióne mystérii, crescat nostræ salútis efféctus. Per Dóminum \emph{\&c.}
}\switchcolumn\portugues{
\rlettrine{H}{avendo} recebido estes dons sacratíssimos. Vos imploramos, Senhor, fazei que pela recepção frequente deste mystério nos sejam aumentados os frutos da salvação. Por nosso Senhor \emph{\&c.}
}\end{paracol}
