\subsection{Sétimo Domingo depois de Pentecostes}

\paragraphinfo{Intróito}{Sl. 46, 2}
\begin{paracol}{2}\latim{
\rlettrine{O}{mnes} gentes, pláudite mánibus: jubiláte Deo in voce exsultatiónis. \emph{Ps. ibid., 3} Quóniam Dóminus excélsus, terríbilis: Rex magnus super omnem terram.
℣. Gloria Patri \emph{\&c.}
}\switchcolumn\portugues{
\qlettrine{Q}{ue} todos os povos aplaudam, batendo palmas; que todos os povos com vozes de júbilo aclamem Deus. \emph{Sl. ibid., 3} Pois o altíssimo Senhor é temível: Ele é o supremo Rei que domina toda a terra!
℣. Glória ao Pai \emph{\&c.}
}\end{paracol}

\paragraph{Oração}
\begin{paracol}{2}\latim{
\rlettrine{D}{eus,} cujus providéntia in sui dispositióne non fállitur: te súpplices exorámus; ut nóxia cuncta submóveas, et ómnia nobis profutúra concédas. Per Dóminum \emph{\&c.}
}\switchcolumn\portugues{
\slettrine{Ó}{} Deus, cuja providência nunca se ilude nos seus desígnios, humildemente Vos suplicamos, afastai de nós tudo o que nos seja prejudicial e concedei-nos tudo o que nos possa ser proveitoso. Por nosso Senhor \emph{\&c.}
}\end{paracol}

\paragraphinfo{Epístola}{Rm. 6, 19-23}
\begin{paracol}{2}\latim{
Léctio Epístolæ beáti Pauli Apóstoli ad Romános.
}\switchcolumn\portugues{
Lição da Ep.ª do B, Ap.º Paulo aos Romanos.
}\switchcolumn*\latim{
\rlettrine{F}{ratres:} Humánum dico, propter infirmitátem carnis vestræ: sicut enim exhibuístis membra vestra servíre immundítiæ et iniquitáti ad iniquitátem, ita nunc exhibéte membra vestra servíre justítiæ in sanctificatiónem. Cum enim servi essétis peccáti, líberi fuístis justítiæ. Quem ergo fructum habuístis tunc in illis, in quibus nunc erubéscitis? Nam finis illórum mors est. Nunc vero liberáti a peccáto, servi autem facti Deo, habétis fructum vestrum in sanctificatiónem, finem vero vitam ætérnam. Stipéndia enim peccáti mors. Grátia autem Dei vita ætérna, in Christo Jesu, Dómino nostro.
}\switchcolumn\portugues{
\rlettrine{M}{eus} irmãos: Falo-vos humanamente, atendendo à fraqueza da vossa carne: Assim como empregastes os vossos membros na imundícia e na impureza para cometerdes iniquidades; assim também agora empregai os vossos membros no serviço da justiça, para a vossa santificação. Quando éreis escravos do pecado, estivestes afastados da justiça. E que fruto tirastes dessas desordens, de que hoje vos envergonhais? Sua finalidade é a morte! Porém, agora, que estais livres do pecado e vos tornastes servos de Deus, tendes, como fruto, a santificação e, como fim, a vida eterna; pois o estipêndio do pecado é a morte, mas a graça de Deus produz a vida eterna, em nosso Senhor Jesus Cristo.
}\end{paracol}

\paragraphinfo{Gradual}{Sl. 33, 12 \& 6}
\begin{paracol}{2}\latim{
\rlettrine{V}{eníte,} fílii, audíte me: timórem Dómini docébo vos. ℣. Accédite ad eum, et illuminámini: et fácies vestræ non confundéntur.
}\switchcolumn\portugues{
\rlettrine{V}{inde,} meus filhos, e escutai-me: Ensinar-vos-ei a temer o Senhor. ℣. Aproximai-Vos d’Ele e ficareis iluminados: então a vossa face não ficará envergonhada.
}\switchcolumn*\latim{
Allelúja, allelúja. ℣. \emph{Ps. 46, 2} Omnes gentes, pláudite mánibus: jubiláte Deo in voce exsultatiónis. Allelúja.
}\switchcolumn\portugues{
Aleluia, aleluia. ℣. \emph{Sl. 46, 2} Que todos os povos aplaudam, batendo palmas; que todos os povos com vozes de júbilo aclamem Deus. Aleluia.
}\end{paracol}

\paragraphinfo{Evangelho}{Mt. 7, 15-21}
\begin{paracol}{2}\latim{
\cruz Sequéntia sancti Evangélii secúndum Matthǽum.
}\switchcolumn\portugues{
\cruz Continuação do santo Evangelho segundo S. Mateus.
}\switchcolumn*\latim{
\blettrine{I}{n} illo témpore: Dixit Jesus discípulis suis: Atténdite a falsis prophétis, qui véniunt ad vos in vestiméntis óvium, intrínsecus autem sunt lupi rapáces: a frúctibus eórum cognoscétis eos. Numquid cólligunt de spinis uvas, aut de tríbulis ficus? Sic omnis arbor bona fructus bonos facit: mala autem arbor malos fructus facit. Non potest arbor bona malos fructus fácere: neque arbor mala bonos fructus fácere. Omnis arbor, quæ non facit fructum bonum, excidétur et in ignem mittétur. Igitur ex frúctibus eórum cognoscétis eos. Non omnis, qui dicit mihi, Dómine, Dómine, intrábit in regnum cœlórum: sed qui facit voluntátem Patris mei, qui in cœlis est, ipse intrábit in regnum cœlórum.
}\switchcolumn\portugues{
\blettrine{N}{aquele} tempo, disse Jesus aos seus discípulos: Tende cuidado com os falsos profetas, que vêm até vós com vestidos de ovelhas, mas que, interiormente, são lobos vorazes. Pelos frutos os conhecereis. Porventura colhem-se uvas nos espinheiros ou figos nos abrolhos? Assim, pois, toda a árvore boa dá frutos bons, e toda a árvore má dá maus frutos. Uma árvore boa não pode dar frutos maus; assim como uma árvore má não pode dar bons frutos. Toda a árvore que não produzir bons frutos será cortada e lançada no fogo. É, pois, pelos frutos que as reconheceis. Nem todo aquele que me diz: «Senhor! Senhor!» entrará no reino dos céus. Só aquele que faz a vontade de meu Pai, que está nos céus, entrará no reino dos céus.
}\end{paracol}

\paragraphinfo{Ofertório}{Dn. 3,40}
\begin{paracol}{2}\latim{
\rlettrine{S}{icut} in holocáustis aríetum et taurórum, et sicut in mílibus agnórum pínguium: sic fiat sacrifícium nostrum in conspéctu tuo hódie, ut pláceat tibi: quia non est confúsio confidéntibus in te, Dómine.
}\switchcolumn\portugues{
\qlettrine{Q}{ue} este nosso sacrifício, que hoje Vos oferecemos, Vos seja agradável, como um holocausto de carneiros e de toiros, ou de mil cordeiros gordos; pois para aqueles que confiam em vós, Senhor, não existe a confusão.
}\end{paracol}

\paragraph{Secreta}
\begin{paracol}{2}\latim{
\rlettrine{D}{eus,} qui legálium differéntiam hostiárum unius sacrifícii perfectione sanxísti: accipe sacrifícium a devótis tibi fámulis, et pari benedictióne, sicut múnera Abel, sanctífica; ut, quod sínguli obtulérunt ad majestátis tuæ honórem, cunctis profíciat ad salútem. Per Dóminum \emph{\&c.}
}\switchcolumn\portugues{
\slettrine{Ó}{} Deus, que unistes as diferentes hóstia da antiga lei em um sacrifício único e Perfeito, recebei este sacrifício, que Vos oferecem com devoção os vossos servos, e santificai-o com a mesma bênção que concedestes aos dons de Abel, a fim de que a oferta feita por cada um, em honra da vossa majestade, sirva de proveito para a salvação de todos. Por nosso Senhor \emph{\&c.}
}\end{paracol}

\paragraphinfo{Comúnio}{Sl. 30, 3}
\begin{paracol}{2}\latim{
\rlettrine{I}{nclína} aurem tuam, accélera, ut erípias me.
}\switchcolumn\portugues{
\rlettrine{I}{nclinai} para mim os vossos ouvidos e apressai-Vos em livrar-me.
}\end{paracol}

\paragraph{Postcomúnio}
\begin{paracol}{2}\latim{
\rlettrine{T}{ua} nos, Dómine, medicinális operátio, et a nostris perversitátibus cleménter expédiat, et ad ea, quæ sunt recta, perdúcat. Per Dóminum nostrum \emph{\&c.}
}\switchcolumn\portugues{
\rlettrine{P}{ermiti,} Senhor, que a acção medicinal nos afaste, clementemente, das nossas perversidades e nos conduza pelos caminhos da justiça. Por nosso Senhor \emph{\&c.}
}\end{paracol}
