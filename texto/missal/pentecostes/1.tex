\subsection{Primeiro Domingo depois de Pentecostes}

\paragraphinfo{Intróito}{Sl. 12, 6. }
\begin{paracol}{2}\latim{
\rlettrine{D}{ómine,} in tua misericórdia sperávi: exsultávit cor meum in salutári tuo: cantábo Dómino, qui bona tríbuit mihi. \emph{Ps. ib., 1} Usquequo, Dómine, oblivísceris me in finem? usquequo avértis fáciem tuam a me?
℣. Gloria Patri \emph{\&c.}
}\switchcolumn\portugues{
\rlettrine{S}{enhor,} esperei na vossa misericórdia: Meu coração exultou de alegria pela salvação que me alcançastes; louvarei o Senhor, que me encheu de benefícios. Até quando, Senhor, Vos esquecereis de mim? \emph{Ps. ib., 1} Esquecer-me-eis para sempre? Até quando afastareis de mim a vossa face?
℣. Glória ao Pai \emph{\&c.}
}\end{paracol}

\paragraph{Oração}
\begin{paracol}{2}\latim{
\rlettrine{D}{eus,} in te sperántium fortitúdo, adésto propítius invocatiónibus nostris: et, quia sine te nihil potest mortális infírmitas, præsta auxílium grátiæ tuæ; ut, in exsequéndis mandátis tuis, et voluntáte tibi et actióne placeámus. Per Dóminum \emph{\&c.}
}\switchcolumn\portugues{
\slettrine{Ó}{} Deus, fortaleza daqueles que em Vós esperam, sede propício às nossas preces; e, visto que sem o vosso auxílio nada pode a fraqueza humana, concedei-nos o socorro da vossa graça, para que, cumprindo os vossos mandamentos, Vos sejamos agradáveis com a nossa vontade e acções. Por nosso Senhor \emph{\&c.}
}\end{paracol}

\paragraphinfo{Epístola}{1. Jo. 4, 8-21}
\begin{paracol}{2}\latim{
Léctio Epístolæ beáti Joánni Apóstoli.
}\switchcolumn\portugues{
Lição da Ep.ª do B. Ap.º João.
}\switchcolumn*\latim{
\rlettrine{C}{aríssimi:} Deus cáritas est. In hoc appáruit cáritas Dei in nobis, quóniam Fílium suum unigénitum misit Deus in mundum, ut vivámus per eum. In hoc est cáritas: non quasi nos dilexérimus Deum, sed quóniam ipse prior diléxit nos, et misit Fílium suum propitiatiónem pro peccátis nostris. Caríssimi, si sic Deus diléxit nos: et nos debémus altérutrum dilígere. Deum nemo vidit umquam. Si diligámus ínvicem, Deus in nobis manet, et cáritas ejus in nobis perfécta est. In hoc cognóscimus, quóniam in eo manémus et ipse in nobis: quóniam de Spíritu suo dedit nobis. Et nos vídimus et testificámur, quóniam Pater misit Fílium suum Salvatórem mundi. Quisquis conféssus fúerit, quóniam Jesus est Fílius Dei, Deus in eo manet et ipse in Deo. Et nos cognóvimus et credídimus caritáti, quam habet Deus in nobis. Deus cáritas est: et qui manet in caritáte, in Deo manet et Deus in eo. In hoc perfécta est cáritas Dei nobíscum, ut fidúciam habeámus in die judicii: quia, sicut ille est, et nos sumus in hoc mundo. Timor non est in caritáte: sed perfécta cáritas foras mittit timórem, quóniam timor pœnam habet. Qui autem timet, non est perféctus in caritáte. Nos ergo diligámus Deum, quóniam Deus prior diléxit nos. Si quis díxerit, quóniam díligo Deum, et fratrem suum óderit, mendax est. Qui enim non díligit fratrem suum, quem videt, Deum, quem non videt, quómodo potest dilígere? Et hoc mandátum habémus a Deo: ut, qui diligit Deum, díligat et fratrem suum.
}\switchcolumn\portugues{
\rlettrine{C}{aríssimos:} Deus é caridade. A caridade de Deus para connosco manifestou-se em Ele ter enviado ao mundo o seu Filho Unigénito, para que vivamos por Ele. E esta caridade consiste em que não fomos nós quem amámos Deus, mas foi Deus quem primeiramente nos amou e enviou o seu Filho, como vítima de propiciação nossos pecados. Meus caríssimos irmãos: Assim como Deus nos amou, assim também devemos amar-nos uns aos outros. Ninguém jamais viu Deus; mas, se nos amarmos uns aos outros, Deus permanece em nós e a sua caridade é perfeita em nós. O que nos faz conhecer que permanecemos em Deus e Deus permanece em nós, é que Ele nos fez participantes do seu Espírito. Nós vimos e damos testemunho de que o Pai mandou o seu Filho como Salvador do mundo. Aquele, pois, que confessar que Jesus é o Filho de Deus, Deus vive nele e ele vive em Deus. Conhecemos o amor que Deus nos consagra, e acreditamos nele, pois Deus é caridade, e todo o que permanece na caridade permanece em Deus e Deus permanece nele. A perfeição da caridade em nós consiste em que devemos ter confiança firme no dia de juízo; pois tal é Jesus Cristo, tais somos, também, nós neste mundo. Na caridade não há temor. A caridade perfeita exclui o temor, visto que o temor é acompanhado de pena e aquele que teme não é perfeito na caridade. Amemos, pois, a Deus, visto que Ele nos amou primeiramente. Se alguém diz: «Eu amo a Deus», mas aborrece o seu irmão, esse é mentiroso; pois aquele que não ama o irmão, que está visível, decerto não pode amar a Deus, que é invisível. Recebemos de Deus este mandamento: «Aquele que ama a Deus deve também amar o seu próximo».
}\end{paracol}

\paragraphinfo{Gradual}{Sl. 40, 5 \& 2}
\begin{paracol}{2}\latim{
\rlettrine{E}{go} dixi: Dómine, miserére mei: sana ánimam meam, quia peccávi tibi. ℣. Beátus, qui intéllegit super egénum et páuperem: in die mala liberábit eum Dóminus.
}\switchcolumn\portugues{
\rlettrine{E}{u} disse: Senhor, tende piedade de mim: curai a minha alma, porque pequei contra Vós. ℣. Bem-aventurado aquele que pensa no pobre e no miserável, pois o Senhor o livrará no dia mau.
}\switchcolumn*\latim{
Allelúja, allelúja. ℣. \emph{Ps. 5, 2} Verba mea áuribus pércipe, Dómine: intéllege clamórem meum. Allelúja.
}\switchcolumn\portugues{
Aleluia, aleluia. ℣. \emph{Sl. 5, 2} Atendei às minhas palavras, Senhor! Ouvi o meu clamor. Aleluia.
}\end{paracol}

\paragraphinfo{Evangelho}{Lc. 6, 36-42}
\begin{paracol}{2}\latim{
\cruz Sequéntia sancti Evangélii secúndum Lucam.
}\switchcolumn\portugues{
\cruz Continuação do santo Evangelho segundo S. Lucas.
}\switchcolumn*\latim{
\blettrine{I}{n} illo témpore: Dixit Jesus discípulis suis: Estóte misericórdes, sicut et Pater vester miséricors est. Nolíte judicáre, et non judicabímini: nolíte condemnáre, et non condemnabímini. Dimíttite, et dimittémini. Date, et dábitur vobis: mensúram bonam et confértam et coagitátam et supereffluéntem dabunt in sinum vestrum. Eadem quippe mensúra, qua mensi fuéritis, remetiétur vobis. Dicébat autem illis et similitúdinem: Numquid potest cæcus cæcum dúcere? nonne ambo in fóveam cadunt? Non est discípulus super magistrum: perféctus autem omnis erit, si sit sicut magister ejus. Quid autem vides festúcam in óculo fratris tui, trabem autem, quæ in óculo tuo est, non consíderas? Aut quómodo potes dícere fratri tuo: Frater, sine, ejíciam festúcam de óculo tuo: ipse in oculo tuo trabem non videns? Hypócrita, ejice primum trabem de oculo tuo: et tunc perspícies, ut edúcas festúcam de óculo fratris tui.
}\switchcolumn\portugues{
\blettrine{N}{aquele} tempo, disse Jesus aos seus discípulos: «Sede misericordiosos, como misericordioso é o vosso Pai. Não julgueis, e não sereis julgados. Não condeneis, e não sereis condenados. Perdoai, e sereis perdoados. Dai aos outros e dar-vos-ão a vós: no vosso regaço colocarão uma boa medida, bem cheia, calcada, acogulada e a transbordar, pois a medida que usardes para com os outros será a medida que usarão para convosco». Depois propôs-lhes esta comparação: Porventura pode um cego guiar outro cego? Acaso não cairão ambos no abismo. O discípulo não é superior ao mestre; mas todo o discípulo será perfeito, se for como o seu mestre. Porque reparas na palha que está no olho do teu irmão e não reparas na tranca que está no teu? Como podes dizer a teu irmão: «Deixa-me tirar essa palha que tens no teu olho», se não vês a tranca que está no teu? Hipócrita, tira primeiramente a tranca do teu olho; depois cuidarás de tirar a palha do olho do teu irmão.
}\end{paracol}

\paragraphinfo{Ofertório}{Sl. 5, 3-4}
\begin{paracol}{2}\latim{
\rlettrine{I}{nténde} voci orationis meæ, Rex meus et Deus meus: quóniam ad te orábo, Dómine.
}\switchcolumn\portugues{
\rlettrine{A}{tendei} à minha súplica, ó meu Rei e meu Deus: pois é a Vós que dirijo a minha oração.
}\end{paracol}

\paragraph{Secreta}
\begin{paracol}{2}\latim{
\rlettrine{H}{óstias} nostras, quǽsumus, Dómine, tibi dicátas placátus assúme: et ad perpétuum nobis tríbue proveníre subsídium. Per Dóminum nostrum \emph{\&c.}
}\switchcolumn\portugues{
\rlettrine{A}{ceitai} benigno, Senhor, as ofertas que Vos oferecemos; e, Vos suplicamos, fazei que nos alcancem o vosso perpétuo auxílio. Por nosso Senhor \emph{\&c.}
}\end{paracol}

\paragraphinfo{Comúnio}{Sl. 9, 2-3}
\begin{paracol}{2}\latim{
\rlettrine{N}{arrábo} ómnia mirabília tua: lætábor et exsultábo in te: psallam nómini tuo, Altíssime.
}\switchcolumn\portugues{
\rlettrine{P}{ublicarei} todas vossas maravilhas: e alegrar-me-ei e rejubilarei em Vós! Cantarei louvores ao vosso nome, ó Altíssimo.
}\end{paracol}

\paragraph{Postcomúnio}
\begin{paracol}{2}\latim{
\rlettrine{T}{antis,} Dómine, repléti munéribus: præsta, quǽsumus; ut et salutária dona capiámus, et a tua numquam laude cessémus. Per Dóminum nostrum \emph{\&c.}
}\switchcolumn\portugues{
\rlettrine{D}{epois} de havermos sido saciados com tantos dons, fazei, Senhor, Vos suplicamos, que nos sejam proveitosos e que nunca cessemos de Vos louvar. Por nosso Senhor \emph{\&c.}
}\end{paracol}
