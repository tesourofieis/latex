\section{Confirmação}

\textit{Logo que é dado o respectivo sinal para começar a Cerimónia da Confirmação (Santo Crisma), aqueles que vão receber o Sacramento ajoelham diante do Bispo, que começa:}

\begin{paracol}{2}\latim{
℣. Spíritus Sanctus supervéniat in vos, et virtus Altíssimi custódiat vos a peccátis.
}\switchcolumn\portugues{
℣. Que o Espírito Santo desça sobre vós e que a virtude do Altíssimo vos livre de todos os pecados.
}\switchcolumn*\latim{
℟. Amen
}\switchcolumn\portugues{
℟. Assim seja.
}\switchcolumn*\latim{
℣. Adjutórium nostrum in nómine Dómini.
}\switchcolumn\portugues{
℣. O nosso auxílio está no Nome do Senhor.
}\switchcolumn*\latim{
℟. Qui fecit cœlum et terram.
}\switchcolumn\portugues{
℟. Que criou o céu e a terra.
}\switchcolumn*\latim{
℣. Dómine, exáudi oratiónem meam.
}\switchcolumn\portugues{
℣. Senhor, ouvi a minha oração.
}\switchcolumn*\latim{
℟. Et clamor meus ad te véniat.
}\switchcolumn\portugues{
℟. E que meu clamor chegue até Vós.
}\switchcolumn*\latim{
℣. Dominus vobíscum.
}\switchcolumn\portugues{
℣. O Senhor seja convosco.
}\switchcolumn*\latim{
℟. Et cum spíritu tuo.
}\switchcolumn\portugues{
℟. E com vosso espírito.
}\end{paracol}

\paragraph{Imposição das Mãos}

\textit{O Bispo estende as mãos sobre as cabeças dos que vão receber a Confirmação, os quais as inclinam mediocremente:}

\begin{paracol}{2}\latim{
\begin{nscenter} Orémus. \end{nscenter}
}\switchcolumn\portugues{
\begin{nscenter} Oremos. \end{nscenter}
}\switchcolumn*\latim{
\rlettrine{O}{mnípotens} sempitérne Deus, qui regeneráre dignátus es hunc famulum tuum (hanc famulam tuam) ex aqua, et Spíritu Sancto, quique dedísti eis remissiónem ómnium peccatórum: emítte in eum (eam) septifórmem Spíritum tuum Sanctum Paráclitum de cælis.
}\switchcolumn\portugues{
\rlettrine{D}{eus} omnipotente e sempiterno, que Vos dignastes regenerar pela água e pelo Espírito Santo os vossos servos aqui presentes e que lhes concedestes a remissão dos seus pecados, enviai-lhes agora do alto dos céus o vosso Espírito Santo Paráclito, que é o Autor dos sete dons.
}\switchcolumn*\latim{
℟. Amen.
}\switchcolumn\portugues{
℟. Amen.
}\switchcolumn*\latim{
℣. Spíritum sapiéntiæ, et intelléctus.
}\switchcolumn\portugues{
℣. O Espírito da Sabedoria e do Entendimento!
}\switchcolumn*\latim{
℟. Amen.
}\switchcolumn\portugues{
℟. Amen.
}\switchcolumn*\latim{
℣. Spíritum consílii, et fortitúdinis.
}\switchcolumn\portugues{
℣. O Espírito do Conselho e da Fortaleza!
}\switchcolumn*\latim{
℟. Amen.
}\switchcolumn\portugues{
℟. Amen.
}\switchcolumn*\latim{
℣. Spíritum sciéntiæ, et pietátis.
}\switchcolumn\portugues{
℣. O Espírito da Ciência e da Piedade!
}\switchcolumn*\latim{
℟. Amen.
}\switchcolumn\portugues{
℟. Amen.
}\switchcolumn*\latim{
\rlettrine{A}{dímple} eum (eam) Spíritu timóris tui, et consígna eum (eam) signo Cru \cruz cis Christi, in vitam propitiátus ætérnam. Per eúmdem Dóminum nostrum Jesum Christum, Fílium tuum: Qui tecum vivit et regnat in unitáte ejúsdem Spíritus Sancti Deus, per ómnia sæcula sæculórum.
}\switchcolumn\portugues{
\rlettrine{E}{nchei-nos} com o Espírito do vosso Temor e marcai-os com o sinal da Cruz \cruz de Cristo, a fim de os auxiliar na posse da vida eterna. Pelo mesmo...
}\end{paracol}

\textit{Todos se erguem, apresentando-se um a um ao Bispo (com ordem e sempre encomendando-se a Deus) para os ungir. Chegados aos pé do Bispo, devem ajoelhar-se e dizer de modo inteligível o seu nome de Baptismo. Bispo diz, fazendo}

\paragraph{A Unção}

\begin{paracol}{2}\latim{
{\redx N.} Signo te signo Cru \cruz cis: et cornfírmo te Chrísmate salútis. In nómine Pa \cruz tris, et Fí \cruz lii, et Spíritus \cruz Sancti.
}\switchcolumn\portugues{
{\redx N.}, eu te marco com o sinal da Cruz \cruz e te confirmo com o Crisma da salvação: em Nome do Pai \cruz e do Filho \cruz e do Espírito \cruz Santo.
}\switchcolumn*\latim{
℟. Amen.
}\switchcolumn\portugues{
℟. Amen.
}\switchcolumn*\latim{
℣. Pax tecum.
}\switchcolumn\portugues{
℣. A paz seja convosco!
}\end{paracol}

\textit{Um dos Clérigos Assistentes ao Bispo limpará a Unção do que recebeu o Sacramento, o qual não poderá retirar-se do Templo sem receber}

\paragraph{A Bênção}

\begin{paracol}{2}\latim{
℣. Confírma hoc, Deus, quod operátus es in nobis, a templo sancto tuo, quod est in Jerúsalem. ℣. Glória Patri, et Fílio, et Spirítui Sancto: Sicut erat in princípio, et nunc, et semper, et in sæcula sæculórum.
}\switchcolumn\portugues{
℣. Confirmai, ó Deus, o que acabais de operar em nós, lá do vosso santo templo que é Jerusalém celestial. ℣. Glória ao Pai e ao Filho e ao Espírito Santo. Assim como era no princípio, agora e sempre e por todos os séculos dos séculos.
}\switchcolumn*\latim{
℟. Amen.
}\switchcolumn\portugues{
℟. Amen.
}\switchcolumn*\latim{
\textit{Et repetitur Antiphona: Confírma hoc, etc.}
}\switchcolumn\portugues{
\textit{Repete-se Confirmai... até celestial.}
}\switchcolumn*\latim{
℣. Osténde nobis, Dómine, misericórdiam tuam.
}\switchcolumn\portugues{
℣. Mostrai, Senhor, a vossa misericórdia.
}\switchcolumn*\latim{
℟. Et salutáre tuum da nobis.
}\switchcolumn\portugues{
℟. E dai-nos a salvação.
}\switchcolumn*\latim{
℣. Dómine, exáudi oratiónem meam.
}\switchcolumn\portugues{
℣. Senhor, ouvi a minha oração.
}\switchcolumn*\latim{
℟. Et clamor meus ad te véniat.
}\switchcolumn\portugues{
℟. E que meu clamor chegue até Vós.
}\switchcolumn*\latim{
℣. Dominus vobíscum.
}\switchcolumn\portugues{
℣. O Senhor seja convosco.
}\switchcolumn*\latim{
℟. Et cum spíritu tuo.
}\switchcolumn\portugues{
℟. E com vosso espírito.
}\end{paracol}

\begin{paracol}{2}\latim{
\begin{nscenter} Orémus. \end{nscenter}
}\switchcolumn\portugues{
\begin{nscenter} Oremos. \end{nscenter}
}\switchcolumn*\latim{
\rlettrine{D}{eus,} qui Apóstolis tuis Sanctum dedísti Spíritum, et per eos eorúmque successóres céteris fidélibus tradéndum esse voluísti: réspice propítius ad humilitátis nostræ famulátum, et præsta; ut eórum corda, quorum frontes sacro Chrísmate delinívimus, et signo sanctæ Crucis signávimus, idem Spíritus Sanctus in eis supervéniens, templum glóriæ suæ dignánter inhabitándo perfíciat: Qui cum Patre, et eódem Spíritu Sancto vivis et regnas Deus, in sæcula sæculórum.
}\switchcolumn\portugues{
\slettrine{Ó}{} Deus, que concedestes o Espírito Santo aos vossos Apóstolos e que quisestes transmiti-los aos outros fiéis pelo seu ministério e pelo dos seus sucessores, dignai-Vos olhar benigno para os vossos humildes servos; e permiti que este mesmo Espírito, descendo aos corações daqueles que ungimos na testa e marcamos com o sinal da Cruz, os faça perfeitos, tornando-os na sua morada e no templo da sua glória: Ó Vós, que viveis e reinais com o Pai e o Espírito Santo pelos séculos dos séculos.
}\switchcolumn*\latim{
℟. Amen.
}\switchcolumn\portugues{
℟. Amen.
}\end{paracol}

\textit{O Bispo continua, estando ainda todos de joelhos:}

\begin{paracol}{2}\latim{
Ecce sic benedicétur omnis homo, qui timet Dóminum.
}\switchcolumn\portugues{
É assim que será abençoado todo o homem que teme o Senhor!
}\switchcolumn*\latim{
Bene \cruz dicat vos Dóminus ex Sion, ut videátis bona Jerúsalem ómnibus diébus vitæ vestræ, et habeátis vitam ætérnam.
}\switchcolumn\portugues{
Que o Senhor vos abençoe \cruz lá do alto de Sião, a fim de que vejais os bens de Jerusalém todos os dias da vossa vida e alcanceis a vida eterna.
}\switchcolumn*\latim{
℟. Amen.
}\switchcolumn\portugues{
℟. Amen.
}\end{paracol}
