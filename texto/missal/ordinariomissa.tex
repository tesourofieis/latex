\section{Ordo Missæ}

\subsection{Primeira Parte - Missa dos Catecúmenos}

\paragraph{Orações ao pé do altar}

\textit{De pé, diante dos degraus do altar, o Sacerdote começa a Missa, fazendo o sinal da cruz e com uma voz clara e audível diz:}

\begin{paracol}{2}\latim{
\cruz In nómine Patris, et Fílii, et Spíritus Sancti.
}\switchcolumn\portugues{
\cruz Em nome do Pai, e do Filho, e do Espírito Santo.
}\switchcolumn*\latim{
℟. Amen.
}\switchcolumn\portugues{
℟. Amen.
}\end{paracol}

\textit{Juntando as mãos no peito, começa a antífona:}

\begin{paracol}{2}\latim{
℣. Introíbo ad altáre Dei.
}\switchcolumn\portugues{
℣. Eu irei até ao altar de Deus.
}\switchcolumn*\latim{
℟. Ad Deum, qui lætíficat juventútem meam.
}\switchcolumn\portugues{
℟. Até Deus, que é a alegria da minha juventude.
}\end{paracol}

\paragraph{Salmo 42}

\textit{O Sacerdote e os Acólitos recitam o salmo alternadamente. Nas Missas de Defuntos e do Tempo da Paixão este salmo omite-se.}

\begin{paracol}{2}\latim{
℣. Júdica me, Deus, et discérne causam meam de gente non sancta: ab hómine iníquo et dolóso érue me.
}\switchcolumn\portugues{
℣. Julgai-me, ó Deus, e defendei a minha causa da gente infiel; livrai-me do homem iníquo e ardiloso.
}\switchcolumn*\latim{
℟. Quia tu es, Deus, fortitudo mea: quare me reppulísti, et quare tristis incédo, dum afflígit me inimícus?
}\switchcolumn\portugues{
℟. Pois que Vós, ó Deus, sois a minha fortaleza, porque me repelistes? E porque ando triste enquanto o meu inimigo me aflige?
}\switchcolumn*\latim{
℣. Emítte lucem tuam et veritátem tuam: ipsa me deduxérunt, et adduxérunt in montem sanctum tuum et in tabernácula tua.
}\switchcolumn\portugues{
℣. Enviai a vossa luz e a vossa verdade; elas me guiarão e conduzirão até ao vosso santo monte, até aos vossos tabernáculos.
}\switchcolumn*\latim{
℟. Et introíbo ad altáre Dei: ad Deum, qui lætíficat juventútem meam.
}\switchcolumn\portugues{
℟. E irei até ao Altar de Deus; até Deus, que é a alegria da minha juventude.
}\switchcolumn*\latim{
℣. Confitébor tibi in cíthara, Deus, Deus meus: quare tristis es, ánima mea, et quare contúrbas me?
}\switchcolumn\portugues{
℣. Ó Deus, ó meu Deus, louvar-Vos-ei com a cítara. Porque estás triste, ó minha alma? Porque te perturbas?
}\switchcolumn*\latim{
℟. Spera in Deo, quóniam adhuc confitébor illi: salutáre vultus mei, et Deus meus.
}\switchcolumn\portugues{
℟. Confia em Deus, pois ainda O louvarei. Ele é a minha salvação e o meu Deus.
}\end{paracol}

\textit{Pequena inclinação da cabeça quando se dá glória ao Pai, ao Filho e ao Espírito Santo.}

\begin{paracol}{2}\latim{
℣. Glória Patri, et Fílio, et Spirítui Sancto.
}\switchcolumn\portugues{
℣. Glória ao Pai, e ao Filho, e ao Espírito Santo.
}\switchcolumn*\latim{
℟. Sicut erat in princípio, et nunc, et semper: et in sǽcula sæculórum. Amen.
}\switchcolumn\portugues{
℟. Assim como era no princípio, e agora, e sempre, por todos os séculos dos séculos. Amen.
}\end{paracol}

\textit{O Sacerdote repete a Antífona:}

\begin{paracol}{2}\latim{
℣. Introíbo ad altáre Dei.
}\switchcolumn\portugues{
℣. Eu irei até ao altar de Deus.
}\switchcolumn*\latim{
℟. Ad Deum, qui lætíficat juventútem meam.
}\switchcolumn\portugues{
℟. Até Deus, que é a alegria da minha juventude.
}\switchcolumn*\latim{
℣. Adjutórium nostrum \cruz in nómine Dómini.
}\switchcolumn\portugues{
℣. O nosso auxílio está \cruz no nome do Senhor.
}\switchcolumn*\latim{
℟. Qui fecit cœlum et terram.
}\switchcolumn\portugues{
℟. Que criou o céu e a terra.
}\end{paracol}

\textit{Profundamente inclinado, o Sacerdote diz o Confíteor, humilhando-se publicamente, e depois dele, os ajudantes.}

\begin{paracol}{2}\latim{
\begin{nscenter} Orémus. \end{nscenter}
}\switchcolumn\portugues{
\begin{nscenter} Oremos. \end{nscenter}
}\switchcolumn*\latim{
℣. Confíteor Deo...
}\switchcolumn\portugues{
℣. Eu me confesso a Deus...
}\switchcolumn*\latim{
℟. Misereátur vestri omnípotens Deus, et, dimíssis peccátis vestris, perdúcat vos ad vitam ætérnam.
}\switchcolumn\portugues{
℟. Compadeça-se de vós o Senhor omnipotente; vos perdoe os pecados e guie até à vida eterna.
}\switchcolumn*\latim{
℣. Amen.
}\switchcolumn\portugues{
℣. Amen.
}\end{paracol}

\textit{Confissão dos Acólitos e dos fiéis:}\label{confiteor}

\begin{paracol}{2}\latim{
℟. Confíteor Deo omnipoténti, beátæ Maríæ semper Vírgini, beáto Michǽli Archángelo, beáto Joánni Baptístæ, sanctis Apóstolis Petro et Paulo, ómnibus Sanctis, et tibi, pater: quia peccávi nimis cogitatióne, verbo et ópere: \textit{(Percutit sibi pectus ter, dicens:)}
}\switchcolumn\portugues{
℟. Eu me confesso a Deus, todo poderoso, à bem-aventurada sempre Virgem Maria, ao bem-aventurado S. Miguel Arcanjo, ao bem-aventurado S. João Baptista, aos Santos Apóstolos S. Pedro e S. Paulo, a todos os santos, e a vós, Padre: que pequei muitas vezes por pensamentos, palavras e obras: \textit{(Baterá no peito três vezes, dizendo:)}
}\switchcolumn*\latim{
\begin{nscenter}\emph{Mea culpa, mea culpa, mea máxima culpa.}\end{nscenter}
}\switchcolumn\portugues{
\begin{nscenter}\emph{Por minha culpa, por minha culpa, por minha tão grande culpa.}\end{nscenter}
}\switchcolumn*\latim{
Ideo precor beátam Maríam semper Vírginem, beátum Michǽlem Archángelum, beátum Joánnem Baptístam, sanctos Apóstolos Petrum et Paulum, omnes Sanctos, et te, pater, orare pro me ad Dóminum, Deum nostrum.
}\switchcolumn\portugues{
Portanto rogo à bem-aventurada sempre Virgem Maria, ao bem-aventurado S. Miguel Arcanjo, ao bem-aventurado S. João Baptista, aos Santos Apóstolos S. Pedro e S. Paulo, a todos os Santos e a vós, Padre, que rogueis a Deus, nosso Senhor, por mim.
}\switchcolumn*\latim{
℣. Misereátur vestri omnípotens Deus, et, dimíssis peccátis vestris, perdúcat vos ad vitam ætérnam.
}\switchcolumn\portugues{
℣. Compadeça-se de vós o Senhor omnipotente; vos perdoe os pecados e guie até à vida eterna.
}\switchcolumn*\latim{
℟. Amen.
}\switchcolumn\portugues{
℟. Amen.
}\switchcolumn*\latim{
℣. Indulgéntiam, \cruz absolutionem et remissiónem peccatórum nostrórum tríbuat nobis omnípotens et miséricors Dóminus.
}\switchcolumn\portugues{
℣. Que o Senhor \cruz omnipotente e misericordioso nos conceda o perdão, a absolvição e a remissão dos nossos pecados.
}\switchcolumn*\latim{
℟. Amen.
}\switchcolumn\portugues{
℟. Amen.
}\switchcolumn*\latim{
℣. Deus, tu convérsus vivificábis nos.
}\switchcolumn\portugues{
℣. Ó Deus, volvei-Vos para nós, e alcançaremos a vida.
}\switchcolumn*\latim{
℟. Et plebs tua lætábitur in te.
}\switchcolumn\portugues{
℟. E o vosso povo se alegrará convosco.
}\switchcolumn*\latim{
℣. Osténde nobis, Dómine, misericórdiam tuam.
}\switchcolumn\portugues{
℣. Senhor, mostrai-nos a vossa misericórdia.
}\switchcolumn*\latim{
℟. Et salutáre tuum da nobis.
}\switchcolumn\portugues{
℟. E concedei-nos a vossa salvação.
}\switchcolumn*\latim{
℣. Dómine, exáudi oratiónem meam.
}\switchcolumn\portugues{
℣. Senhor, atendei à minha oração.
}\switchcolumn*\latim{
℟. Et clamor meus ad te véniat.
}\switchcolumn\portugues{
℟. E que meu clamor chegue até Vós.
}\switchcolumn*\latim{
℣. Dóminus vobíscum
}\switchcolumn\portugues{
℣. O Senhor esteja convosco.
}\switchcolumn*\latim{
℟. Et cum spíritu tuo.
}\switchcolumn\portugues{
℟. E com vosso espírito.
}\end{paracol}

\textit{O Sacerdote sobe ao altar, dizendo:}

\begin{paracol}{2}\latim{
\rlettrine{A}{ufer} a nobis, quǽsumus, Dómine, iniquitátes nostras: ut ad Sancta sanctórum puris mereámur méntibus introíre. Per Christum, Dóminum nostrum. Amen.
}\switchcolumn\portugues{
\rlettrine{A}{fastai} de nós, Senhor, Vos imploramos, as nossas iniquidades, para que mereçamos entrar no santuário com as almas purificadas. Por Cristo, nosso Senhor. Amen.
}\switchcolumn*\latim{
\begin{nscenter} Orémus. \end{nscenter}
}\switchcolumn\portugues{
\begin{nscenter} Oremos. \end{nscenter}
}\end{paracol}

\textit{O Sacerdote, inclinado, diz a seguinte oração:}

\begin{paracol}{2}\latim{
\rlettrine{O}{rámus,} Dómine, per mérita Sanctórum tuórum, quorum relíquiæ hic sunt, et ómnium Sanctórum: ut indulgére dignéris ómnia peccáta mea. Amen.
}\switchcolumn\portugues{
\rlettrine{V}{os} pedimos, Senhor, pelos méritos dos vossos Santos, cujas relíquias estão aqui, e de todos os Santos, que Vos digneis perdoar os nossos pecados. Amen.
}\end{paracol}

\paragraph{Intróito}

\textit{Nas Missas solenes, incensa-se o altar. O Sacerdote vai para o lado da Epístola, e lê o Intróito. Canto solene de entrada, o Intróito como que enuncia o tema geral da Missa ou solenidade do dia. Às primeiras palavras, todos se benzem, ao mesmo tempo que o celebrante.}

\emph{Conforme Missa do dia.}

\paragraph{Kyrie Eleison}

\textit{Os Kyries, são nove clamores dirigidos à Santíssima trindade. O Sacerdote, no meio do altar, diz, alternadamente com os Acólitos:}

\begin{paracol}{2}\latim{
℣. Kýrie eléison.
}\switchcolumn\portugues{
℣. Senhor, tende piedade de nós.
}\switchcolumn*\latim{
℟. Kýrie eléison.
}\switchcolumn\portugues{
℟. Senhor, tende piedade de nós.
}\switchcolumn*\latim{
℣. Kýrie eléison.
}\switchcolumn\portugues{
℣. Senhor, tende piedade de nós.
}\switchcolumn*\latim{
℟. Christe eléison.
}\switchcolumn\portugues{
℟. Cristo, tende piedade de nós.
}\switchcolumn*\latim{
℣. Christe eléison.
}\switchcolumn\portugues{
℣. Cristo, tende piedade de nós.
}\switchcolumn*\latim{
℟. Christe eléison.
}\switchcolumn\portugues{
℟. Cristo, tende piedade de nós.
}\switchcolumn*\latim{
℣. Kýrie eléison.
}\switchcolumn\portugues{
℣. Senhor, tende piedade de nós.
}\switchcolumn*\latim{
℟. Kýrie eléison.
}\switchcolumn\portugues{
℟. Senhor, tende piedade de nós.
}\switchcolumn*\latim{
℣. Kýrie eléison.
}\switchcolumn\portugues{
℣. Senhor, tende piedade de nós.
}\end{paracol}

\paragraph{Glória in Excélsis}

\textit{Canto de alegria, a Glória só se diz nas Missas de carácter festivo: Domingos (fora do Advento, Septuagésima e Quaresma), Tempos do Natal, Tempo Pascal, festas de Nosso Senhor, da Santíssima Virgem, dos Anjos e dos Santos, e Missas votivas solenes. Omite-se em todas as outras Missas.}

\begin{paracol}{2}\latim{
\rlettrine{G}{lória} in excélsis Deo. Et in terra pax homínibus bonæ voluntátis. Laudámus te. Benedícimus te. Adorámus te. Glorificámus te. Grátias ágimus tibi propter magnam glóriam tuam. Dómine Deus, Rex cœléstis, Deus Pater omnípotens. Dómine Fili unigénite, Jesu Christe. Dómine Deus, Agnus Dei, Fílius Patris. Qui tollis peccáta mundi, miserére nobis. Qui tollis peccáta mundi, súscipe deprecatiónem nostram. Qui sedes ad déxteram Patris, miserére nobis. Quóniam tu solus Sanctus. Tu solus Dóminus. Tu solus Altíssimus, Jesu Christe. Cum Sancto Spíritu \cruz in glória Dei Patris.
}\switchcolumn\portugues{
\rlettrine{G}{lória} a Deus nas alturas e paz na terra aos homens de boa vontade. Nós Vos louvamos. Nós Vos bendizemos. Nós Vos adoramos. Nós Vos glorificamos. Nós Vos damos graças pela vossa imensa glória. Ó Senhor Deus, Rei dos céus, Deus Pai todo-o-poderoso. Senhor Jesus Cristo, Filho Unigénito, Senhor Deus, Cordeiro de Deus, Filho de Deus Pai. Vós que tirais os pecados do mundo, tende misericórdia de nós. Vós, que tirais os pecados do mundo, atendei à nossa súplica. Vós, que estais sentado à direita do Pai, tende misericórdia de nós. Só Vós sois o Santo; só Vós, o Senhor; só Vós, o Altíssimo, Jesus Cristo: com o Espírito Santo \cruz na glória de Deus Pai.
}\switchcolumn*\latim{
℟. Amen.
}\switchcolumn\portugues{
℟. Amen.
}\end{paracol}

\textit{O Sacerdote benze-se, beija o altar, volta-se para os fiéis e diz:}

\begin{paracol}{2}\latim{
℣. Dóminus vobíscum.
}\switchcolumn\portugues{
℣. O Senhor esteja convosco.
}\switchcolumn*\latim{
℟. Et cum spíritu tuo.
}\switchcolumn\portugues{
℟. E com vosso espírito.
}\end{paracol}

\paragraph{Colecta}

\textit{O Sacerdote, diante do missal, recita a Colecta. Breve oração que resume e apresenta a Deus os votos de todos os fiéis, votos estes sugeridos pelo mystério ou solenidade do dia.}

\begin{paracol}{2}\latim{
\begin{nscenter} Orémus. \end{nscenter}
}\switchcolumn\portugues{
\begin{nscenter} Oremos. \end{nscenter}
}\end{paracol}

\emph{Conforme Missa do dia.}

\begin{paracol}{2}\latim{
℣. ...per ómnia sǽculua sæculórum.
}\switchcolumn\portugues{
℣. ...por todos os séculos dos séculos.
}\switchcolumn*\latim{
℟. Amen.
}\switchcolumn\portugues{
℟. Amen.
}\end{paracol}

\paragraph{Epístola}

\emph{Conforme Missa do dia.}

\textit{No fim, diz-se:}

\begin{paracol}{2}\latim{
℟. Deo grátias.
}\switchcolumn\portugues{
℟. Graças a Deus.
}\end{paracol}

\paragraph{Gradual}

\textit{A oração curta de acção de graças, consiste geralmente de dous ou três versos retirados dos Salmos ou do Antigo Testamento.}

\emph{Conforme Missa do dia.}

\textit{No Tempo da Septuagésima, o Allelúja é substituído pelo Trato. No Tempo Pascal, omite-se o Gradual, e dizem-se dous Allelúja.}

\textit{Enquanto o Acólito muda o Missal, o Sacerdote inclina-se profundamente no meio do Altar, dizendo:}

\begin{paracol}{2}\latim{
\rlettrine{M}{unda} cor meum ac labia mea, omnípotens Deus, qui labia Isaíæ Prophétæ cálculo mundásti igníto: ita me tua grata miseratióne dignáre mundáre, ut sanctum Evangélium tuum digne váleam nuntiáre. Per Christum, Dóminum nostrum. Amen.
}\switchcolumn\portugues{
\slettrine{Ó}{} omnipotente Deus, assim como purificastes os lábios do Profeta Isaías com uma brasa de fogo, assim também purificais agora o meu coração e os meus lábios. Dignai-Vos, pela vossa benigna misericórdia, purificar-me inteiramente, para que possa dignamente anunciar o vosso Evangelho. Amen.
}\end{paracol}

\textit{Seguidamente recita-se a fórmula da Bênção, a qual só se omite nas Missas de Réquiem e na Sexta-Feira Santa:}

\begin{paracol}{2}\latim{
\qlettrine{J}{ube} Dómine benedícere. Dóminus sit in corde meo, et in lábiis meis: ut dine et competénter annúntiem. Evangélium suum. Amen.
}\switchcolumn\portugues{
\rlettrine{D}{ignai-Vos,} Senhor, abençoar-me. Que o Senhor esteja no meu coração e nos meus lábios, para que possa digna e devidamente anunciar seu Evangelho. Amen.
}\end{paracol}

\textit{Às primeiras palavras - Sequéntia, \&c. faz-se o sinal da cruz na testa, na boca e no peito, declarando a ausência de vergonha na palavra de Deus, a prontidão para confessá-la e o amor, de todo o coração, que a ela têm. Nesta proclamação, ponto culminante desta primeira parte da Missa, a leitura ou canto do Evangelho é revestida da maior solenidade. O respeito para com ele, exige que seja escutado de pé. Nas Missas solenes, o livro é levado honorificamente em procissão. É incensado antes de começar, e, terminada a leitura, é reverentemente beijado pelo Sacerdote.}

\paragraph{Evangelho}

\emph{Conforme Missa do dia.}

\textit{É um momento solene! Toda a assistência está de pé. Procurai o Evangelho próprio da Missa do dia:}

\begin{paracol}{2}\latim{
℣. Dóminus vobíscum.
}\switchcolumn\portugues{
℣. O Senhor esteja convosco.
}\switchcolumn*\latim{
℟. Et cum spíritu tuo.
}\switchcolumn\portugues{
℟. E com vosso espírito.
}\switchcolumn*\latim{
℣. Sequéntia \cruz sancti Evangélii secúndum {\redx N.}
}\switchcolumn\portugues{
℣. Continuação \cruz do santo Evangelho, segundo {\redx N.}
}\switchcolumn*\latim{
℟. Glória tibi, Dómine.
}\switchcolumn\portugues{
℟. Glória a Vós, Senhor.
}\end{paracol}

\textit{O celebrante beija o sagrado texto, dizendo:}

\begin{paracol}{2}\latim{
℣. Per evangélica dicta deleántur nostra delícta.
}\switchcolumn\portugues{
℣. Que pelas palavras do Evangelho nos sejam perdoados os nossos pecados.
}\end{paracol}

\textit{Digamos solenemente:}

\begin{paracol}{2}\latim{
℟. Laus tibi, Christe.
}\switchcolumn\portugues{
℟. Louvores a Vós, ó Cristo.
}\end{paracol}

\paragraph{Homilia}

\textit{Explicaçãodo Evangelho.}

\paragraph{Credo}

\textit{Aos Domingos e certos dias de festa, o celebrante vai ao meio do altar e diz o Credo. Este só se diz aos Domingos, e em algumas festas de 1ª. Classe. É cantado em Missa Solenes.}

\begin{paracol}{2}\latim{
\blettrine{C}{redo} ín unum Deum. Patrem omnipoténtem, factórem cœli et terræ, visibílium ómnium et in visibílium. Et in unum Dóminum Jesum Christum, Fílium Dei unigénitum. Et ex Patre natum ante ómnia sǽcula. Deum de Deo, lumen de lúmine, Deum verum de Deo vero. Génitum, non factum, consubstantiálem Patri: per quem ómnia facta sunt. Qui propter nos hómines et propter nostram salútem descéndit de cœlis. \textit{(Hic genuflectitur)}.
}\switchcolumn\portugues{
\blettrine{C}{reio} em um só Deus. Pai, todo poderoso, criador do Céu e da Terra, de todas as cousas visíveis e invisíveis. E creio em um só Senhor, Jesus Cristo, Filho Unigénito de Deus, nascido do Pai antes de todos os séculos. Deus de Deus, Luz da Luz, Deus verdadeiro de Deus verdadeiro. Gerado, não criado, consubstancial ao Pai: por Ele todas as cousas foram feitas. E por nós, homens, e para nossa salvação desceu dos céus. \textit{(todos se ajoelham)}.
}\switchcolumn*\latim{
\textbf{Et incarnátus est de Spíritu Sancto ex María Vírgine: {\redx Et homo factus est.}}
}\switchcolumn\portugues{
\textbf{E encarnou pelo Espírito Santo, no seio da Virgem Maria: {\redx E foi feito homem.}}
}\switchcolumn*\latim{
Crucifíxus étiam pro nobis: sub Póntio Piláto passus, et sepúltus est. Et resurréxit tértia die, secúndum Scriptúras. Et ascéndit in cœlum: sedet ad déxteram Patris. Et íterum ventúrus est cum glória judicáre vivos et mórtuos: cujus regni non erit finis. Et in Spíritum Sanctum, Dóminum et vivificántem: qui ex Patre Filióque procédit. Qui cum Patre et Fílio simul adorátur et conglorificátur: qui locútus est per Prophétas. Et unam sanctam cathólicam et apostolicam Ecclésiam. Confíteor unum baptísma in remissiónem peccatórum. Et exspécto resurrectiónem mortuórum. Et \cruz vitam ventúri sǽculi.
}\switchcolumn\portugues{
Também por nós foi crucificado sob Pôncio Pilatos, padeceu e foi sepultado. Ressuscitou ao terceiro dia, conforme as Escrituras, e subiu aos céus, onde está sentado à direita do Pai. De novo há-de vir em sua glória, para julgar os vivos e os mortos; e o seu reino não terá fim. Creio no Espírito Santo, Senhor que dá a vida, e procede do Pai e do Filho; e com o Pai e o Filho é adorado e glorificado: Ele que falou pelos Profetas. Creio na Igreja una, santa, católica e apostólica. Confesso um só baptismo para a remissão dos pecados. E espero a ressurreição dos mortos, e \cruz a vida do mundo que há-de vir.
}\switchcolumn*\latim{
℟. Amen.
}\switchcolumn\portugues{
℟. Amen.
}\end{paracol}

\subsection{Segunda Parte - Missa dos Fiéis}

\paragraph{Sacrifício Ofertório}

\textit{Com o Ofertório, começa a segunda parte da Missa ou Sacrifício propriamente dito. O celebrante beija o Altar e voltado para o povo diz:}

\begin{paracol}{2}\latim{
℣. Dóminus vobíscum.
}\switchcolumn\portugues{
℣. O Senhor esteja convosco.
}\switchcolumn*\latim{
℟. Et cum spíritu tuo.
}\switchcolumn\portugues{
℟. E com vosso espírito.
}\end{paracol}

\begin{paracol}{2}\latim{
\begin{nscenter} ℣. Orémus. \end{nscenter}
}\switchcolumn\portugues{
\begin{nscenter} ℣. Oremos. \end{nscenter}
}\end{paracol}

\paragraph{Ofertório}

\emph{Conforme Missa do dia.}

\paragraph{Oferecimento do pão:}

\textit{Terminada esta leitura, o Sacerdote descobre o Cálice e toma nas mãos a patena com o pão, que vai ser consagrado. (O Acólito vai buscar o vinho e a água à credência, levando-os Altar). O Sacerdote oferece-os a Deus,
dizendo:}

\begin{paracol}{2}\latim{
\rlettrine{S}{úscipe,} sancte Pater, omnipotens ætérne Deus, hanc immaculátam hóstiam, quam ego indígnus fámulus tuus óffero tibi Deo meo vivo et vero, pro innumerabílibus peccátis, et offensiónibus, et neglegéntiis meis, et pro ómnibus circumstántibus, sed et pro ómnibus fidélibus christiánis vivis atque defúnctis: ut mihi, et illis profíciat ad salútem in vitam ætérnam. Amen.
}\switchcolumn\portugues{
\rlettrine{R}{ecebei,} ó Pai santo, Deus omnipotente e eterno, esta hóstia imaculada, que eu, vosso indigno servo, Vos ofereço, ó meu Deus vivo e verdadeiro, pelos meus inumeráveis pecados, ofensas e negligências, por todos os assistentes e por todos os cristãos vivos e mortos, a fim de que sirva de proveito para a minha salvação, para a deles e para a vida eterna. Amen.
}\end{paracol}

\textit{Ao lado direito do altar, o celebrante deita vinho no cálice, a que mistura umas gotas de água, dizendo a seguinte oração:}

\begin{paracol}{2}\latim{
\rlettrine{D}{eus,} qui humánæ substántiæ dignitátem mirabíliter condidísti, et mirabílius reformásti: da nobis per hujus aquæ et vini mystérium, ejus divinitátis esse consórtes, qui humanitátis nostræ fíeri dignátus est párticeps, Jesus Christus, Fílius tuus, Dóminus noster: Qui tecum vivit et regnat in unitáte Spíritus Sancti Deus: per ómnia sǽcula sæculórum. Amen.
}\switchcolumn\portugues{
\slettrine{Ó}{} Deus, que maravilhosamente criastes a dignidade da natureza humana e que mais prodigiosamente ainda a reformastes, permiti que pela mistura simbólica desta água e deste vinho sejamos participantes da divindade daquele que quis assumir a nossa humanidade, Jesus Cristo, vosso Filho, nosso Senhor, que, sendo Deus, vive e reina convosco em unidade do Espírito Santo, em todos os séculos dos séculos. Amen.
}\end{paracol}

\textit{No meio do altar, o celebrante faz o oferecimento do cálice:}

\begin{paracol}{2}\latim{
\rlettrine{O}{fférimus} tibi, Dómine, cálicem salutáris, tuam deprecántes cleméntiam: ut in conspéctu divínæ majestátis tuæ, pro nostra et totíus mundi salúte, cum odóre suavitátis ascéndat. Amen.
}\switchcolumn\portugues{
\rlettrine{V}{os} oferecemos, Senhor, o Cálice da salvação; e Vos suplicamos que misericordiosamente o façais subir, como suave perfume, diante da vossa divina majestade, para nossa salvação e de todo o mundo. Amen.
}\end{paracol}

\textit{Depois, inclinando-se diz:}

\begin{paracol}{2}\latim{
\rlettrine{I}{n} spíritu humilitátis et in ánimo contríto suscipiámur a te, Dómine: et sic fiat sacrifícium nostrum in conspéctu tuo hódie, ut pláceat tibi, Dómine Deus.
}\switchcolumn\portugues{
\rlettrine{E}{m} humildade e contrição, sejamos recebidos por Vós, Senhor; e assim este sacrifício, ó Deus, se torne agradável aos vossos olhos.
}\end{paracol}

\textit{Invocação do Espírito Santo:}

\begin{paracol}{2}\latim{
\rlettrine{V}{eni,} sanctificátor omnípotens ætérne Deus: et bene \cruz dic hoc sacrifícium, tuo sancto nómini præparátum.
}\switchcolumn\portugues{
\rlettrine{V}{inde,} ó Santificador omnipotente, Deus eterno, e abençoai \cruz este sacrifício, preparado para a glória do vosso Santo Nome.
}\end{paracol}

\paragraph{Incensão}

\textit{Segue-se, nas Missas solenes, o rito da incensão. Tudo o que é oferecido a Deus é incensado. Logo, são incensados o Pão, o Vinho e todos os fiéis presentes.}

\paragraph{Benção do Incenso:}

\begin{paracol}{2}\latim{
\rlettrine{P}{er} intercessiónem beáti Michǽlis Archángeli, stantis a dextris altáris incénsi, et ómnium electórum suórum, incénsum istud dignétur Dóminus bene \cruz dícere, et in odórem suavitátis accípere. Per Christum, Dóminum nostrum. Amen.
}\switchcolumn\portugues{
\rlettrine{D}{ignai-Vos} Senhor, pela intercessão do bem-aventurado Arcanjo Miguel, que está de pé à direita do altar do incenso, e de todos vossos eleitos, abençoar \cruz este incenso e aceitá-lo como odor de suavidade. Por Cristo, Senhor nosso. Amen.
}\end{paracol}

\textit{O Sacerdote incensa primeiro as oblatas:}

\begin{paracol}{2}\latim{
Incénsum istud a te benedíctum ascéndat ad te, Dómine: et descéndat super nos misericórdia tua.
}\switchcolumn\portugues{
Que este incenso, por Vós abençoado, suba até Vós, Senhor; e desça sobre nós a vossa misericórdia.
}\end{paracol}

\textit{Em seguida incensa a cruz e o altar, dizendo, entretanto, os seguintes versículos, retirados do Salmo 140:}

\begin{paracol}{2}\latim{
\rlettrine{D}{irigátur,} Dómine, orátio mea, sicut incénsum, in conspéctu tuo: elevátio mánuum meárum sacrifícium vespertínum. Pone, Dómine, custódiam ori meo, et óstium circumstántiæ lábiis meis: ut non declínet cor meum in verba malítiæ, ad excusándas excusatiónes in peccátis.
}\switchcolumn\portugues{
\rlettrine{S}{uba} como incenso até Vós, Senhor, a minha oração: a elevação das minhas mãos seja como o sacrifício vespertino. Colocai, Senhor, uma guarda em minha boca, e uma porta em volta de meus lábios. Não deixes que meu coração se deixe arrastar por palavras de maldade, procurando desculpas para pecar.
}\end{paracol}

\textit{O celebrante entrega o turíbulo ao Diácono, dizendo:}

\begin{paracol}{2}\latim{
℟. Accéndat in nobis Dóminus ignem sui amóris, et flammam ætérnæ caritátis. Amen.
}\switchcolumn\portugues{
℟. Que o Senhor acenda em nós o fogo do seu amor e a chama da eterna caridade. Amen.
}\end{paracol}

\textit{O Diácono incensa o Sacerdote, e depois todos os outros por ordem. Nas Missas de defuntos, é incensado só o Sacerdote.}

\paragraph{Lavabo}

\textit{O lavar as mãos simboliza a pureza da alma, necessária para oferecer o santo Sacrifício. O Sacerdote vai à direita do altar e lava as mãos, dizendo, entretanto, os seguintes versículos do salmo 25:}

\begin{paracol}{2}\latim{
\rlettrine{L}{avábo} inter innocéntes manus meas: et circúmdabo altáre tuum. Dómine:
}\switchcolumn\portugues{
\rlettrine{L}{avarei} as minhas mãos, como os inocentes, e rodearei, Senhor o vosso altar:
}\switchcolumn*\latim{
Ut áudiam vocem laudis, et enárrem univérsa mirabília tua. Dómine, diléxi decórem domus tuæ et locum habitatiónis glóriæ tuæ.
}\switchcolumn\portugues{
Para ouvir a voz dos vossos louvores e publicar todas vossas maravilhas. Amei, Senhor, o decoro da vossa casa e o lugar onde reside a vossa glória.
}\switchcolumn*\latim{
Ne perdas cum ímpiis, Deus, ánimam meam, et cum viris sánguinum vitam meam:
}\switchcolumn\portugues{
Não deixeis, ó meu Deus, a minha alma perder-se com os ímpios, nem a minha vida com os homens sanguinários:
}\switchcolumn*\latim{
In quorum mánibus iniquitátes sunt: déxtera eórum repléta est munéribus. Ego autem in innocéntia mea ingréssus sum: rédime me et miserére mei. Pes meus stetit in dirécto: in ecclésiis benedícam te, Dómine.
}\switchcolumn\portugues{
Que têm as mãos manchadas de iniquidades e a mão direita cheia de dádivas. Pois que tenho vivido na inocência, salvai-me e tende misericórdia de mim. Meus pés continuam firmes no caminho direito: e hei-de bendizer-Vos, Senhor, em todas as assembleias.
}\switchcolumn*\latim{
Glória Patri, et Fílio, et Spirítui Sancto. Sicut erat in princípio, et nunc, et semper: et in sǽcula sæculórum. Amen.
}\switchcolumn\portugues{
Glória ao Pai, e ao Filho, e ao Espírito Santo. Assim como era no princípio, e agora, e sempre, por todos os séculos dos séculos. Amen.
}\end{paracol}

\textit{Nas Missas de defuntos e do Tempo da Paixão omite-se o Glória Patri.}

\paragraph{Oração à Santíssima Trindade}

\textit{Inclinado, ao meio do altar, o Sacerdote diz:}

\begin{paracol}{2}\latim{
℣. Súscipe, sancta Trinitas, hanc oblatiónem, quam tibi offérimus ob memóriam passiónis, resurrectiónis, et ascensiónis Jesu Christi, Dómini nostri: et in honórem beátæ Maríæ semper Vírginis, et beáti Joannis Baptistæ, et sanctórum Apostolórum Petri et Pauli, et istórum et ómnium Sanctórum: ut illis profíciat ad honórem, nobis autem ad salútem: et illi pro nobis intercédere dignéntur in cœlis, quorum memóriam ágimus in terris. Per eúndem Christum, Dóminum nostrum. Amen.
}\switchcolumn\portugues{
℣. Recebei, ó Santíssima Trindade, esta oblação, que Vos oferecemos em memória da Paixão, da Ressurreição e da Ascensão de N. S. Jesus Cristo: e em honra da bem-aventurada sempre Virgem Maria, do bem-aventurado João Baptista e dos Santos Apóstolos Pedro e Paulo, e destes, que estão aqui, e de todos os Santos: para que esta oblação lhes sirva de glória e a nós de salvação: e aqueles, cuja memória honramos na terra, se dignem interceder por nós no céu. Pelo mesmo Jesus Cristo, nosso Senhor. Amen.
}\end{paracol}

\textit{Beija o Altar e voltando-se para os fiéis, o Sacerdote convida-os a orar com ele:}

\paragraph{Orate Frates}

\begin{paracol}{2}\latim{
℣. Oráte, fratres: ut meum ac vestrum sacrifícium acceptábile fiat apud Deum Patrem omnipoténtem.
}\switchcolumn\portugues{
℣. Orai, meus irmãos, a fim de que meu sacrifício, que é também vosso, seja recebido por Deus Pai omnipotente.
}\switchcolumn*\latim{
℟. Suscípiat Dóminus sacrifícium de mánibus tuis ad laudem et glóriam nominis sui, ad utilitátem quoque nostram, totiúsque Ecclésiæ suæ sanctæ.
}\switchcolumn\portugues{
℟. Que o Senhor receba por vossas mãos este sacrifício para a honra e glória de seu Nome, e também para a nossa utilidade e de toda sua santa Igreja.
}\end{paracol}

\textit{O Sacerdote responde, em voz baixa:}

\begin{paracol}{2}\latim{
℣. Amen.
}\switchcolumn\portugues{
℣. Amen.
}\end{paracol}

\textit{Em seguida lê a Secreta. À Secreta principal, podem, em certas Missas, ajuntar-se outras, em número igual e segundo as mesmas regras da Colecta.}

\paragraph{Secreta}

\emph{Conforme Missa do dia.}

\begin{paracol}{2}\latim{
℣. ...Per ómnia sǽcula sæculórum.
}\switchcolumn\portugues{
℣. ...Por todos os séculos dos séculos.
}\switchcolumn*\latim{
℟. Amen.
}\switchcolumn\portugues{
℟. Amen.
}\end{paracol}

\subsection{Canon Missæ}

\paragraph{Prefácio}

\textit{O Cânon constitui a parte central da Missa. Com o Prefácio, começa a grande, a solene oração sacerdotal da Igreja e oblação propriamente dita do Sacrifício. Curto diálogo introdutório entre o Sacerdote e os fiéis desperta nas almas os sentimentos de acção de graças que convêm à celebração dos santos mystérios.}

\begin{paracol}{2}\latim{
℣. Dóminus vobíscum.
}\switchcolumn\portugues{
℣. O Senhor esteja convosco.
}\switchcolumn*\latim{
℟. Et cum spíritu tuo.
}\switchcolumn\portugues{
℟. E com vosso espírito.
}\switchcolumn*\latim{
℣. Sursum corda.
}\switchcolumn\portugues{
℣. Corações ao alto.
}\switchcolumn*\latim{
℟. Habémus ad Dóminum.
}\switchcolumn\portugues{
℟. Assim os temos para o Senhor.
}\switchcolumn*\latim{
℣. Grátias agámus Dómino, Deo nostro.
}\switchcolumn\portugues{
℣. Demos graças ao Senhor, Nosso Deus.
}\switchcolumn*\latim{
℟. Dignum et justum est.
}\switchcolumn\portugues{
℟. Assim é digno e justo.
}\end{paracol}

\paragraph{Prefácio}

\textit{O Prefácio da SS. Trindade, página \pageref{prefaciotrindade}, diz-se nas festas e nas Missas votivas da SS. Trindade e em todos os Domingos do ano, excepto nas festas que tiverem próprio (nestas deve-se procurar o Prefácio próprio, página \pageref{prefacios}.):}

\paragraph{Sanctus}

\begin{paracol}{2}\latim{
\rlettrine{S}{anctus,} Sanctus, Sanctus Dóminus, Deus Sábaoth. Pleni sunt cœli et terra glória tua. Hosánna in excélsis.
}\switchcolumn\portugues{
\rlettrine{S}{anto,} Santo, Santo, Senhor Deus do Universo. O céu e a terra proclamam a vossa glória. Hossana nas alturas.
}\switchcolumn*\latim{
Benedíctus, \cruz qui venit in nómine Dómini. Hosánna in excélsis.
}\switchcolumn\portugues{
Bendito \cruz o que vem em nome do Senhor. Hosana nas alturas.
}\end{paracol}

\ilustra{media/pdf/PreciousBloodOurLord}

\paragraph{Cânon}

\textit{O Sacerdote, levanta os olhos para o céu, inclina-se, beija o altar e continua a grande oração sacerdotal.}

\begin{paracol}{2}\latim{
\blettrine{T}{e} Igitur clementíssime Pater, per Jesum Christum, Fílium tuum, Dóminum nostrum, súpplices rogámus, ac pétimus, uti accepta habeas et benedícas, hæc \cruz dona, hæc \cruz múnera, hæc \cruz sancta sacrifícia illibáta, in primis, quæ tibi offérimus pro Ecclésia tua sancta cathólica: quam pacificáre, custodíre, adunáre et régere dignéris toto orbe terrárum: una cum fámulo tuo Papa nostro {\redx N.} et Antístite nostro {\redx N.} et ómnibus orthodóxis, atque cathólicæ et apostólicæ fídei cultóribus.
}\switchcolumn\portugues{
\blettrine{A}{} Vós, pois, rogamos e pedimos, ó clementíssimo Pai, por Jesus Cristo, vosso Filho e nosso Senhor, que estes \cruz dons, estas \cruz ofertas, estes \cruz sacrifícios santos e imaculados Vos sejam agradáveis e os abençoeis, os quais, antes de tudo, Vos oferecemos pela nossa santa Igreja Católica: dignai-Vos conceder-lhe a paz, guardá-la, uni-la e governá-la por toda a terra, em comunhão com vosso servo, nosso Papa {\redx N.} com o nosso Bispo {\redx N.} e com todos os
ortodoxos e os que professam a fé católica e apostólica.
}\end{paracol}

\paragraph{Memento dos vivos}

\begin{paracol}{2}\latim{
\rlettrine{M}{eménto,} Dómine, famulórum famularúmque tuarum {\redx N.} et {\redx N.} et ómnium circumstántium, quorum tibi fides cógnita est et nota devótio, pro quibus tibi offérimus: vel qui tibi ófferunt hoc sacrifícium laudis, pro se suísque ómnibus: pro redemptióne animárum suárum, pro spe salútis et incolumitátis suæ: tibíque reddunt vota sua ætérno Deo, vivo et vero.
}\switchcolumn\portugues{
\rlettrine{L}{embrai-Vos,} Senhor, dos vossos servos {\redx N.} e {\redx N.} e de todos os que aqui estão presentes, cuja fé Vos é conhecida e a piedade é notória: pelos quais Vos oferecemos, ou eles Vos oferecem, este sacrifício de louvor por si próprios e por todos seus, pela redenção das suas almas, pela esperança da sua salvação: é a Vós que oferecem as homenagens, como Deus eterno, vivo e verdadeira, que sois.
}\end{paracol}

\paragraph{Memória dos Santos}

\textit{Para além destes Comunicantes, existem também para o Natal, Epifania, Quinta-Feira Santa, Sábado de Aleluia e Páscoa, Ascensão e Pentecostes.}

\begin{paracol}{2}\latim{
\rlettrine{C}{ommunicántes,} et memóriam venerántes, in primis gloriósæ semper Vírginis Maríæ, Genitrícis Dei et Dómini nostri Jesu Christi: sed et beatórum Apostolórum ac Mártyrum tuórum, Petri et Pauli, Andréæ, Jacóbi, Joánnis, Thomæ, Jacóbi, Philíppi, Bartholomǽi, Matthǽi, Simónis et Thaddǽi: Lini, Cleti, Cleméntis, Xysti, Cornélii, Cypriáni, Lauréntii, Chrysógoni, Joánnis et Pauli, Cosmæ et Damiáni: et ómnium Sanctórum tuórum; quorum méritis precibúsque concédas, ut in ómnibus protectiónis tuæ muniámur auxílio. Per eúndem Christum, Dóminum nostrum. Amen.
}\switchcolumn\portugues{
\rlettrine{U}{nidos} em uma mesma comunhão, primeiramente honramos a memória da gloriosa sempre Virgem Maria, Mãe de Jesus Cristo, nosso Deus e Senhor, e depois também a dos vossos bem-aventurados Apóstolos e Mártires: Pedro e Paulo, André, Tiago, João, Tomé, Tiago, Filipe, Bartolomeu, Mateus, Simão e Tadeu: Lino, Cleto, Clemente, Xisto, Cornélio, Cipriano, Lourenço, Crisógono, João e Paulo, Cosme e Damião: e de todos vossos Santos. Dignai-Vos permitir que por seus méritos e preces gozemos o poderoso auxílio da vossa protecção. Pelo mesmo Cristo, nosso Senhor. Amen.
}\end{paracol}

\paragraph{Orações na Consagração}

\textit{Estendendo as mãos sobre as oblatas, o celebrante diz:}

\begin{paracol}{2}\latim{
\blettrine{H}{anc} igitur oblatiónem servitutis nostræ, sed et cunctæ famíliæ tuæ, quǽsumus, Dómine, ut placátus accípias: diésque nostros in tua pace dispónas, atque ab ætérna damnatióne nos éripi, et in electórum tuórum júbeas grege numerári. Per Christum, Dóminum nostrum. Amen.
}\switchcolumn\portugues{
\blettrine{P}{or} este motivo, Senhor, Vos rogamos que Vos dignais receber favoravelmente esta oferta que eu, vosso indigno servo, e toda vossa família Vos fazemos; concedei-nos o gozo da vossa paz nos nossos dias, livrai-nos da condenação eterna e admiti-nos ao número dos vossos escolhidos. Por Cristo, nosso Senhor. Amen.
}\end{paracol}

\textit{O Sacerdote abençoa as oblatas dizendo:}

\begin{paracol}{2}\latim{
Quam oblatiónem tu, Deus, in ómnibus, quǽsumus, bene \cruz díctam, adscríp \cruz tam, ra \cruz tam, rationábilem, acceptabilémque fácere dignéris: ut nobis Cor \cruz pus, et San \cruz guis fiat dilectíssimi Fílii tui, Dómini nostri Jesu Christi.
}\switchcolumn\portugues{
Que esta oblação, ó Deus, Vos imploramos, seja abençoada, aprovada, confirmada, digna e aceitável, a fim de que se converta para nós no Corpo e no Sangue de vosso amado Filho, nosso Senhor Jesus Cristo.
}\end{paracol}

\paragraph{Consagração e Elevação da Hóstia}

\textit{Inclina-se sobre o altar, e profere as palavras da consagração da Hóstia. Em seguida adora-a, e eleva-a aos olhos dos fiéis, para que todos a adorem em silêncio. O mesmo faz, depois, para a consagração do Cálice.}

\begin{paracol}{2}\latim{
\qlettrine{Q}{ui} prídie quam paterétur, accépit panem in sanctas ac venerábiles manus suas, et elevátis óculis in cœlum ad te Deum, Patrem suum omnipoténtem, tibi grátias agens, bene \cruz dixit, fregit, dedítque discípulis suis, dicens: Accípite, et manducáte ex hoc omnes.
}\switchcolumn\portugues{
\rlettrine{O}{} qual, na véspera da sua paixão tomou o pão em suas santas e veneráveis mãos: e, erguendo os olhos ao céu, a Vós, Deus seu Pai omnipotente, e dando-Vos graças, abençoou-o \cruz, partiu-o e deu-o aos seus discípulos, dizendo: Tomai todos e comei:
}\switchcolumn*\latim{
\begin{nscenter}
\emph{\textbf{Hoc est enim Corpus meum.}}
\end{nscenter}
}\switchcolumn\portugues{
\begin{nscenter}
\emph{\textbf{Pois isto é o meu Corpo.}}
\end{nscenter}
}\end{paracol}

\paragraph{Consagração e Elevação do Cálice:}

\begin{paracol}{2}\latim{
\rlettrine{S}{ímili} modo postquam cœnátum est, accípiens et hunc præclárum Cálicem in sanctas ac venerábiles manus suas: tibi grátias agens, bene \cruz dixit, dedítque discípulis suis, dicens: Accípite, et bíbite ex eo omnes.
}\switchcolumn\portugues{
\rlettrine{D}{o} mesmo modo, Jesus, após a Ceia, tomou em suas santas e veneráveis mãos este precioso Cálice, e, novamente, dando-Vos graças, abençoou-o \cruz e deu-o aos seus discípulos dizendo:
}\switchcolumn*\latim{
\begin{nscenter}
\emph{\textbf{Hic est enim Calix Sánguinis mei, novi et ætérni testaménti: mystérium fídei: qui pro vobis et pro multis effundétur in remissiónem peccatórum.}}
\end{nscenter}
}\switchcolumn\portugues{
\begin{nscenter}
\emph{\textbf{Pois isto é o Cálice do meu Sangue do novo e eterno testamento, mystério da fé, que será derramado por vós e por muitos, para a remissão dos pecados.}}
\end{nscenter}
}\end{paracol}

\textit{Depois da consagração do Cálice, o Sacerdote diz com uma voz baixa:}

\begin{paracol}{2}\latim{
Hæc quotiescúmque fecéritis, in mei memóriam faciétis.
}\switchcolumn\portugues{
Todas as vezes que fizerdes isto, fazei-o em memória de mim.
}\end{paracol}

\paragraph{Oblação da Vítima a Deus}

\textit{O Sacerdote continua depois as orações do Cânon:}

\begin{paracol}{2}\latim{
\rlettrine{U}{nde} et mémores, Dómine, nos servi tui, sed et plebs tua sancta, ejusdem Christi Fílii tui, Dómini nostri, tam beátæ passiónis, nec non et ab ínferis resurrectiónis, sed et in cœlos gloriósæ ascensiónis: offérimus præcláræ majestáti tuæ de tuis donis ac datis, hóstiam \cruz puram, hóstiam \cruz sanctam, hóstiam \cruz immaculátam, Panem \cruz sanctum vitæ ætérnæ, et Calicem \cruz salútis perpétuæ.
}\switchcolumn\portugues{
\rlettrine{P}{or} este motivo, Senhor, nós, vossos servos, e o vosso povo santo, lembrando-nos da bem-aventurada Paixão do mesmo Cristo, vosso Filho e Senhor nosso, assim como também da sua Ressurreição dos mortos e da sua gloriosa Ascensão aos céus, oferecemos à vossa divina majestade os mesmos dons que nos foram dados: a Hóstia \cruz pura, a Hóstia \cruz santa, a Hóstia \cruz imaculada, o Pão \cruz santo da vida eterna e o Cálice \cruz da salvação perpétua.
}\end{paracol}

\textit{Com as mãos afastadas, contínua:}

\begin{paracol}{2}\latim{
\rlettrine{S}{upra} quæ propítio ac seréno vultu respícere dignéris: et accépta habére, sicúti accépta habére dignátus es múnera púeri tui justi Abel, et sacrifícium Patriárchæ nostri Abrahæ: et quod tibi óbtulit summus sacérdos tuus Melchísedech, sanctum sacrifícium, immaculátam hóstiam.
}\switchcolumn\portugues{
\rlettrine{S}{obre} estas ofertas dignai-Vos lançar um olhar propício e benévolo e aceitá-las, como Vos dignastes receber os dons do justo Abel, vosso servo, o sacrifício do nosso Patriarca Abraão e o que Vos ofereceu o Sumo sacerdote Melquisedeque, pois este é um sacrifício santo, uma hóstia imaculada.
}\end{paracol}

\textit{Profundamente inclinado, o Sacerdote diz:}

\begin{paracol}{2}\latim{
\rlettrine{S}{úpplices} te rogámus, omnípotens Deus: jube hæc perférri per manus sancti Ángeli tui in sublíme altáre tuum, in conspéctu divínæ majestátis tuæ: ut, quotquot ex hac altáris participatióne sacrosánctum Fílii tui Cor \cruz pus, et Sán \cruz guinem sumpsérimus, omni benedictióne cœlésti et grátia repleámur. Per eúndem Christum, Dóminum nostrum. Amen.
}\switchcolumn\portugues{
\rlettrine{H}{umildemente} Vos rogamos, ó Deus omnipotente, ordeneis que estas ofertas sejam apresentadas no altar sublime pelas mãos do vosso Santo Anjo, na presença da vossa divina majestade, a fim de que todos aqueles que participam deste Altar pela recepção do Santíssimo Corpo \cruz e Sangue \cruz de vosso Filho sejam repletos de todas as bênçãos do céu e de todas as graças. Pelo mesmo Cristo, nosso Senhor. Amen.
}\end{paracol}

\paragraph{Orações depois da Consagração}

\paragraph{Memento dos defuntos:}

\begin{paracol}{2}\latim{
\rlettrine{M}{eménto} étiam, Dómine, famulórum famularúmque tuárum {\redx N.} et {\redx N.}, qui nos præcessérunt cum signo fídei, et dórmiunt in somno pacis.
}\switchcolumn\portugues{
\rlettrine{L}{embrai-Vos} também Senhor, dos vossos servos e servas {\redx N.} e {\redx N.}, que partiram antes de nós, marcados com o sinal da fé, e agora dormem o sono da paz.
}\switchcolumn*\latim{
Ipsis, Dómine, et ómnibus in Christo quiescéntibus locum refrigérii, lucis pacis ut indúlgeas, deprecámur. Per eúndem Christum, Dóminum nostrum. Amen.
}\switchcolumn\portugues{
Vos suplicamos, Senhor, dignai-Vos conceder a estes, assim como a todos os que descansam em Cristo, um lugar de consolação, de luz e de paz. Pelo mesmo Cristo, nosso Senhor. Amen.
}\end{paracol}

\paragraph{Invocação dos Santos}

\textit{O Sacerdote bate no peito, dizendo:}

\begin{paracol}{2}\latim{
\rlettrine{N}{obis} quoque peccatóribus Extensis manibus ut prius, secrete prosequitur: fámulis tuis, de multitúdine miseratiónum tuárum sperántibus, partem áliquam et societátem donáre dignéris, cum tuis sanctis Apóstolis et Martýribus: cum Joánne, Stéphano, Matthía, Bárnaba, Ignátio, Alexándro, Marcellíno, Petro, Felicitáte, Perpétua, Agatha, Lúcia, Agnéte, Cæcília, Anastásia, et ómnibus Sanctis tuis: intra quorum nos consórtium, non æstimátor mériti, sed véniæ, quǽsumus, largítor admítte. Per Christum, Dóminum nostrum.
}\switchcolumn\portugues{
\rlettrine{E}{} também a nós, pecadores, vossos servos, que esperamos na grandeza das vossas misericórdias, dignai-Vos conceder-nos alguma parte na vossa herança e sociedade com vossos Santos Apóstolos e Mártires: João, Estêvão, Matias, Barnabé, Inácio, Alexandre, Marcelino, Pedro, Felicidade, Perpétua, Águeda, Luzia, Inês, Cecília, Anastácia e com todos os Santos, em cuja companhia, Vos pedimos, nos recebais, não em consideração dos nossos merecimentos, mas segundo a liberalidade da vossa misericórdia. Por Cristo, nosso Senhor.
}\switchcolumn*\latim{
Per quem hæc ómnia, Dómine, semper bona creas, sanctí \cruz ficas, viví \cruz ficas, bene \cruz dícis et præstas nobis.
}\switchcolumn\portugues{
Por quem, Senhor, sempre criais estes bens e os santificais \cruz, vivificais \cruz, abençoais \cruz e no-los concedeis.
}\end{paracol}

\paragraph{Doxologia Final}

\begin{paracol}{2}\latim{
Per ip \cruz sum, et cum ip \cruz so, et in ip \cruz so, est tibi Deo Patri \cruz omnipotenti, in unitáte Spíritus \cruz Sancti, omnis honor, et glória.
}\switchcolumn\portugues{
Por Ele \cruz, com Ele \cruz e n’Ele \cruz, a Vós, ó Deus Pai todo-o-poderoso \cruz, na unidade do Espírito \cruz Santo, pertence, e Vos é dada, toda a honra e glória.
}\end{paracol}

\textit{O Sacerdote termina em voz alta:}

\begin{paracol}{2}\latim{
℣. Per ómnia sǽcula sæculórum.
}\switchcolumn\portugues{
℣. Por todos os séculos dos séculos.
}\switchcolumn*\latim{
℟. Amen.
}\switchcolumn\portugues{
℟. Amen.
}\end{paracol}

\paragraph{Rito da Comunhão}

\textit{Participação no Sacrifício}

\textit{Terminado o Cânon, o Sacerdote diz em voz alta:}

\begin{paracol}{2}\latim{
\begin{nscenter} Orémus. \end{nscenter}
}\switchcolumn\portugues{
\begin{nscenter} Oremos. \end{nscenter}
}\switchcolumn*\latim{
Præcéptis salutáribus móniti, et divína institutióne formáti, audémus dícere:
}\switchcolumn\portugues{
Instruídos com os salutares preceitos do Salvador e dirigidos pelos seus divinos ensinamentos, ousamos dizer:
}\end{paracol}

\paragraph{Pater Noster}

\begin{paracol}{2}\latim{
\blettrine{P}{ater} noster, qui es in cælis: sanctificétur nomen tuum: advéniat regnum tuum: fiat volúntas tua, sicut in cælo, et in terra. Panem nostrum quotidiánum da nobis hódie: et dimítte nobis débita nostra, sicut et nos dimíttimus debitóribus nostris. Et ne nos indúcas in tentatiónem.
}\switchcolumn\portugues{
\blettrine{P}{ai} nosso que estais nos céus; santificado seja o vosso nome; venha a nós o vosso reino; seja feita a vossa vontade, assim na terra como no céu. O pão nosso de cada dia nos dai hoje; perdoai-nos as nossas ofensas, assim como nós perdoamos a quem nos tem ofendido; e não nos deixeis cair em tentação:
}\switchcolumn*\latim{
℟. Sed líbera nos a malo.
}\switchcolumn\portugues{
℟. Mas livrai-nos do mal.
}\end{paracol}

\paragraph{Líbera nos}

\textit{O Sacerdote diz Amen em voz baixa, e continua:}

\begin{paracol}{2}\latim{
\rlettrine{L}{ibera} nos, quǽsumus, Dómine, ab ómnibus malis, prætéritis, præséntibus et futúris: et intercedénte beáta et gloriósa semper Vírgine Dei Genetríce María, cum beátis Apóstolis tuis Petro et Paulo, atque Andréa, et ómnibus Sanctis, da propítius pacem in diébus nostris: ut, ope misericórdiæ tuæ adjúti, et a peccáto simus semper líberi et ab omni perturbatióne secúri.
}\switchcolumn\portugues{
\rlettrine{L}{ivrai-nos,} Senhor, Vos suplicamos, de todos os males passados, presentes e futuros; e, pela intercessão da bem-aventurada e gloriosa sempre Virgem Maria, Mãe de Deus, e dos bem-aventurados Apóstolos Pedro e Paulo e André, e de todos os Santos, dai-nos, benignamente, a paz nos nossos dias, a fim de que auxiliados com vossa misericórdia, sejamos sempre livres do pecado e seguros de toda a perturbação.
}\end{paracol}

\textit{O Sacerdote descobre o Cálice, genuflecte e segura com os dedos polegar e indicador da mão direita a Hóstia, que ergue até ao cimo do cálice.}

\paragraph{Fracção da Hóstia}

\textit{O Sacerdote parte a Hóstia ao meio, de uma das partes tira um pequeno fragmento que deita no preciosíssimo Sangue, traçando antes, com ele, sobre o Cálice, três vezes, o sinal da cruz, e dizendo:}

\begin{paracol}{2}\latim{
Per eúndem Dóminum nostrum Jesum Christum, Fílium tuum. Qui tecum vivit et regnat in unitáte Spíritus Sancti Deus.
}\switchcolumn\portugues{
Pelo mesmo nosso Senhor Jesus Cristo, vosso Filho: que convosco vive e reina em unidade de Deus Espírito Santo.
}\end{paracol}

\textit{Depois, o Sacerdote, tendo entre os dedos da mão direita a Partícula, que fraccionou, coloca-a sobre o Cálice, que segura pelo nós, e acrescenta em voz alta:}

\begin{paracol}{2}\latim{
℣. Per ómnia sæcula sæculórum.
}\switchcolumn\portugues{
℣. Por todos os séculos dos séculos.
}\switchcolumn*\latim{
℟. Amen.
}\switchcolumn\portugues{
℟. Amen.
}\end{paracol}

\textit{O Sacerdote faz três vezes o sinal da Cruz com a Divina Partícula sobre o Cálice:}

\begin{paracol}{2}\latim{
℣. Pax \cruz Dómini sit \cruz semper \cruz vobíscum.
}\switchcolumn\portugues{
℣. Que a paz \cruz do Senhor esteja \cruz sempre \cruz convosco.
}\switchcolumn*\latim{
℟. Et cum spíritu tuo.
}\switchcolumn\portugues{
℟. E com vosso espírito.
}\switchcolumn*\latim{
Hæc commíxtio, et consecrátio Córporis et Sánguinis Dómini nostri Jesu Christi, fiat accipiéntibus nobis in vitam ætérnam. Amen.
}\switchcolumn\portugues{
Que esta mistura e esta consagração do Corpo e do Sangue de nosso Senhor Jesus Cristo sejam penhor de vida eterna para nós que a receberemos. Amen.
}\end{paracol}

\paragraph{Agnus Dei}

\ilustracont{media/pdf/Lambslain}

\textit{O Sacerdote bate três vezes no peito, e diz a oração seguinte. Nas Missas de Defuntos, não se bate no peito e em vez de miserére nobis, diz-se: Dona eis requiem; na terceira parte: dona eis requiem sempiternam.}

\begin{paracol}{2}\latim{
\rlettrine{A}{gnus} Dei, qui tollis peccáta mundi: miserére nobis.
}\switchcolumn\portugues{
\rlettrine{C}{ordeiro} de Deus, que tirais o pecado do mundo, tende piedade de nós.
}\switchcolumn*\latim{
Agnus Dei, qui tollis peccáta mundi: miserére nobis.
}\switchcolumn\portugues{
Cordeiro de Deus, que tirais o pecado do mundo, tende piedade de nós.
}\switchcolumn*\latim{
Agnus Dei, qui tollis peccáta mundi: dona nobis pacem.
}\switchcolumn\portugues{
Cordeiro de Deus, que tirais o pecado do mundo, dai-nos a paz.
}\end{paracol}

\paragraph{Orações para a Comunhão}

\textit{Inclinado, recita as três orações seguintes, pela paz, santificação e graça da Igreja.}

\begin{paracol}{2}\latim{
\rlettrine{D}{ómine} Jesu Christe, qui dixísti Apóstolis tuis: Pacem relínquo vobis, pacem meam do vobis: ne respícias peccáta mea, sed fidem Ecclésiæ tuæ; eámque secúndum voluntátem tuam pacifícáre et coadunáre dignéris: Qui vivis et regnas Deus per ómnia sǽcula sæculórum. Amen.
}\switchcolumn\portugues{
\rlettrine{S}{enhor} Jesus Cristo, que dissestes aos vossos Apóstolos: «Eu vos deixo a paz, dou-vos a minha paz», não olheis para os meus pecados, mas para a fé da vossa Igreja: concedei-lhe paz e união, segundo a vossa vontade: Vós, que, sendo Deus, viveis e reinais em todos os séculos dos séculos. Amen.
}\switchcolumn*\latim{
Dómine Jesu Christe, Fili Dei vivi, qui ex voluntáte Patris, cooperánte Spíritu Sancto, per mortem tuam mundum vivificásti: líbera me per hoc sacrosánctum Corpus et Sánguinem tuum ab ómnibus iniquitátibus meis, et univérsis malis: et fac me tuis semper inhærére mandátis, et a te numquam separári permíttas: Qui cum eódem Deo Patre et Spíritu Sancto vivis et regnas Deus in sǽcula sæculórum. Amen.
}\switchcolumn\portugues{
Senhor Jesus Cristo, Filho de Deus vivo, que, por vontade do Pai, cooperando o Espírito Santo, pela vossa morte, destes a vida ao mundo: livrai-nos de todos os males por este vosso sacrossanto Corpo e Sangue. Permiti que cumpra sempre os vossos preceitos e nunca me afaste de Vós: que sendo Deus, viveis e reinais com o mesmo Deus Pai e Espírito Santo em todos os séculos dos séculos. Amen.
}\switchcolumn*\latim{
Percéptio Córporis tui, Dómine Jesu Christe, quod ego indígnus súmere præsúmo, non mihi provéniat in judícium et condemnatiónem: sed pro tua pietáte prosit mihi ad tutaméntum mentis et córporis, et ad medélam percipiéndam: Qui vivis et regnas cum Deo Patre in unitáte Spíritus Sancti Deus, per ómnia sǽcula sæculórum. Amen.
}\switchcolumn\portugues{
Senhor Jesus Cristo, que este vosso Corpo, que, eu, ainda que indigno, vou receber, não seja para meu juízo e condenação, mas que, pela vossa misericórdia, sirva à minha alma e ao meu corpo de defesa e de remédio salutar: Vós, que, sendo Deus, viveis e reinais com Deus Pai em unidade de Deus Espírito Santo em todos os séculos dos séculos. Amen.
}\end{paracol}

\paragraph{Comunhão do celebrante}

\textit{O Sacerdote genuflecte e pegando depois na sagrada Hóstia.}

\begin{paracol}{2}\latim{
Panem cœléstem accipiam, et nomen Dómini invocábo.
}\switchcolumn\portugues{
Tomarei o Pão do céu e invocarei o nome do Senhor.
}\end{paracol}

\textit{Em seguida bate três vezes no peito.}

\begin{paracol}{2}\latim{
Dómine, non sum dignus, ut intres sub tectum meum: sed tantum dic verbo, et sanábitur ánima mea.
}\switchcolumn\portugues{
Senhor, eu não sou digno de que entreis na minha morada, mas dizei uma só palavra e a minha alma será salva.
}\end{paracol}

\textit{Faz sobre si o sinal da cruz com a sagrada Hóstia, antes de a comungar.}

\begin{paracol}{2}\latim{
Corpus Dómini nostri Jesu Christi custódiat ánimam meam in vitam ætérnam. Amen.
}\switchcolumn\portugues{
Que o Corpo de nosso Senhor Jesus Cristo guarde a minha alma para a vida eterna. Amen.
}\end{paracol}

\textit{Recolhe-se por uns instantes. Toma o preciosíssimo Sangue, fazendo antes sobre si o sinal da cruz.}

\begin{paracol}{2}\latim{
\qlettrine{Q}{uid} retríbuam Dómino pro ómnibus, quæ retríbuit mihi? Cálicem salutáris accípiam, et nomen Dómini invocábo. Laudans invocábo Dóminum, et ab inimícis meis salvus ero.
}\switchcolumn\portugues{
\rlettrine{C}{omo} retribuirei ao Senhor os bens que Ele se dignou dispensar-me? Tomarei o Cálice da salvação e invocarei o nome do Senhor, louvando-O, e ficarei livre dos meus inimigos.
}\end{paracol}

\paragraph{Comunhão dos fiéis}

\textit{Os fiéis, ou o acólito por eles, recitam o Confíteor:}

\begin{paracol}{2}\latim{
℟. Confíteor Deo omnipoténti, beátæ Maríæ semper Vírgini, beáto Michǽli Archángelo, beáto Joánni Baptístæ, sanctis Apóstolis Petro et Paulo, ómnibus Sanctis, et tibi, pater: quia peccávi nimis cogitatióne, verbo et ópere:
}\switchcolumn\portugues{
℟. Eu me confesso a Deus, todo poderoso, à bem-aventurada sempre Virgem Maria, ao bem-aventurado S. Miguel Arcanjo, ao bem-aventurado S. João Baptista, aos Santos Apóstolos S. Pedro e S. Paulo, a todos os santos, e a vós, Padre: que pequei muitas vezes por pensamentos, palavras e obras:
}\end{paracol}

\begin{nscenter}
\textit{Batendo três vezes no peito:}
\end{nscenter}

\begin{paracol}{2}\latim{
Mea culpa, mea culpa, mea máxima culpa.
}\switchcolumn\portugues{
Por minha culpa, por minha culpa, por minha tão grande culpa.
}\switchcolumn*\latim{
Ideo precor beátam Maríam semper Vírginem, beátum Michǽlem Archángelum, beátum Joánnem Baptístam, sanctos Apóstolos Petrum et Paulum, omnes Sanctos, et te, pater, orare pro me ad Dóminum, Deum nostrum.
}\switchcolumn\portugues{
Portanto rogo à bem-aventurada sempre Virgem Maria, ao bem-aventurado S. Miguel Arcanjo, ao bem-aventurado S. João Baptista, aos Santos Apóstolos S. Pedro e S. Paulo, a todos os Santos e a vós, Padre, que rogueis a Deus, nosso Senhor, por mim.
}\end{paracol}

\textit{Voltando-se para os fiéis, o Sacerdote dá a absolvição em voz alta:}

\begin{paracol}{2}\latim{
℣. Misereátur vestri omnípotens Deus, et, dimíssis peccátis vestris, perdúcat vos ad vitam ætérnam.
}\switchcolumn\portugues{
℣. Compadeça-se de vós o Senhor omnipotente; vos perdoe os pecados e guie até à vida eterna.
}\switchcolumn*\latim{
℟. Amen.
}\switchcolumn\portugues{
℟. Amen.
}\end{paracol}

\textit{Fazendo o Sinal da Cruz, diz:}

\begin{paracol}{2}\latim{
℣. Indulgéntiam, \cruz absolutionem et remissiónem peccatórum nostrórum tríbuat nobis omnípotens et miséricors Dóminus.
}\switchcolumn\portugues{
℣. Que o Senhor \cruz omnipotente e misericordioso nos conceda o perdão, a absolvição e a remissão dos nossos pecados.
}\switchcolumn*\latim{
℟. Amen.
}\switchcolumn\portugues{
℟. Amen.
}\end{paracol}

\ilustra{media/pdf/cc/cristorei2}

\textit{O Sacerdote volta-se para o altar, genuflecte e voltando-se para os fiéis ergue a Hóstia, dizendo:}

\begin{paracol}{2}\latim{
℣. Ecce Agnus Dei, ecce qui tollit peccáta mundi.
}\switchcolumn\portugues{
℣. Eis o Cordeiro de Deus; eis Aquele que tira os pecados do mundo.
}\end{paracol}

\textit{E em seguida, três vezes batendo no peito e dizendo:}

\begin{paracol}{2}\latim{
℟. Dómine, non sum dignus, ut intres sub tectum meum: sed tantum dic verbo, et sanábitur ánima mea.
}\switchcolumn\portugues{
℟. Senhor, eu não sou digno de que entreis na minha morada, mas dizei uma só palavra e a minha alma será salva.
}\end{paracol}

\textit{Estando convenientemente preparado aquele que quiser Comungar, aproximar-se-á do Comungatório, ajoelhando e recebendo a Divina Hóstia na língua. O Sacerdote diz a cada um dos comungantes:}

\begin{paracol}{2}\latim{
℣. Corpus Dómini nostri Jesu Christi custódiat ánimam meam in vitam ætérnam. Amen.
}\switchcolumn\portugues{
℣. Que o Corpo de nosso Senhor Jesus Cristo guarde a tua alma para a vida eterna. Amen.
}\end{paracol}

\paragraph{Acção de Graças}

\paragraph{Abluções}

\textit{O Sacerdote purifica primeiro o cálice e depois os dedos, e toma as abluções. Entretanto vai dizendo:}

\begin{paracol}{2}\latim{
\qlettrine{Q}{uod} ore súmpsimus, Dómine, pura mente capiámus: et de munere temporáli fiat nobis remédium sempitérnum.
}\switchcolumn\portugues{
\qlettrine{Q}{ue} conservemos com pureza de coração, Senhor, o que a boca acaba de receber; e que esta dádiva temporal se torne para nós remédio sempiterno.
}\switchcolumn*\latim{
Corpus tuum, Dómine, quod sumpsi, et Sanguis, quem potávi, adhǽreat viscéribus meis: et præsta; ut in me non remáneat scélerum mácula, quem pura et sancta refecérunt sacraménta: Qui vivis et regnas in sǽcula sæculórum.. Amen.
}\switchcolumn\portugues{
Senhor, que o vosso Corpo, que recebi, e o vosso sangue, que bebi, se unam intimamente ás minhas entranhas; dignai-Vos permitir, Senhor, que não fique em mim mancha alguma de pecado, agora que estou confortado com sacramentos tão puros e santos: Vós, que viveis e reinais em todos os séculos. Amen.
}\end{paracol}

\textit{Purifica o cálice e deixa-o, coberto, no meio do altar. Nas Missas solenes, é o subdiácono quem purifica o cálice e o leva para a credencia.}

\paragraph{Antífona da Comunhão}

\textit{O Sacerdote passa para o lado direito do altar, e recita a antífona da Comunhão.}

\emph{Conforme Missa do dia.}

\begin{paracol}{2}\latim{
℣. Dóminus vobíscum.
}\switchcolumn\portugues{
℣. O Senhor esteja convosco.
}\switchcolumn*\latim{
℟. Et cum spíritu tuo.
}\switchcolumn\portugues{
℟. E com vosso espírito.
}\end{paracol}

\paragraph{Pós-Comunhão}

\textit{À Pós-comunhão principal da Missa podem, em certos casos, como para a Colecta, juntar-se outras.}

\emph{Conforme Missa do dia.}

\paragraph{Despedida e Bênção}

\textit{O Sacerdote volta ao meio do altar, beija-o, e, voltando-se para os fiéis saúda-os:}

\begin{paracol}{2}\latim{
℣. Dóminus vobíscum.
}\switchcolumn\portugues{
℣. O Senhor esteja convosco.
}\switchcolumn*\latim{
℟. Et cum spíritu tuo.
}\switchcolumn\portugues{
℟. E com vosso espírito.
}\switchcolumn*\latim{
℣. Ite, Missa est.
}\switchcolumn\portugues{
℣. Ide-vos, acabou a Missa.
}\switchcolumn*\latim{
℟. Deo grátias.
}\switchcolumn\portugues{
℟. Graças a Deus.
}\end{paracol}

\textit{Se alguma acção litúrgica se segue à Missa, diz-se:}

\begin{paracol}{2}\latim{
℣. Benedicámus Dómino.
}\switchcolumn\portugues{
℣. Bendigamos o Senhor.
}\switchcolumn*\latim{
℟. Deo Grátias.
}\switchcolumn\portugues{
℟. Graças a Deus.
}\end{paracol}

\textit{Nas Missas dos Defuntos:}

\begin{paracol}{2}\latim{
℣. Requiéscant in pace.
}\switchcolumn\portugues{
℣. Que descansem em paz.
}\switchcolumn*\latim{
℟. Amen.
}\switchcolumn\portugues{
℟. Amen.
}\end{paracol}

\textit{Voltando-se para o altar, recita a seguinte oração:}

\begin{paracol}{2}\latim{
℣. Pláceat tibi, sancta Trínitas, obséquium servitútis meæ: et præsta; ut sacrifícium, quod óculis tuæ majestátis indígnus óbtuli, tibi sit acceptábile, mihíque et ómnibus, pro quibus illud óbtuli, sit, te miseránte, propitiábile. Per Christum, Dóminum nostrum. Amen.
}\switchcolumn\portugues{
℣. Santíssima Trindade, seja-Vos agradável a homenagem da minha escravidão, a fim de que este sacrifício, que, ainda indignamente, ofereci à vossa divina majestade, seja aceite por Vós, e, pela vossa misericórdia, se torne propiciatório para mim e para todos aqueles por quem o ofereci. Por Cristo, Nosso Senhor. Amen.
}\end{paracol}

\textit{Beija o altar, volta-se para a assistência, e dá a bênção, dizendo:}

\begin{paracol}{2}\latim{
℣. Benedícat vos omnípotens Deus, Pater, et Fílius, \cruz et Spíritus Sanctus.
}\switchcolumn\portugues{
℣. Que desça sobre vós a bênção do omnipotente Deus: Pai, e Filho, \cruz e Espírito Santo.
}\switchcolumn*\latim{
℟. Amen.
}\switchcolumn\portugues{
℟. Amen.
}\end{paracol}

\paragraph{Último Evangelho}

\textit{O Sacerdote passa para o lado esquerdo do altar e recita, como último Evangelho, o princípio do Evangelho de S. João (que se omite na Quinta-feira Santa e na Vigília pascal).}

\begin{paracol}{2}\latim{
℣. Dóminus vobíscum.
}\switchcolumn\portugues{
℣. O Senhor esteja convosco.
}\switchcolumn*\latim{
℟. Et cum spíritu tuo.
}\switchcolumn\portugues{
℟. E com vosso espírito.
}\end{paracol}

\textit{Despois faz o sinal da Cruz na Sacra (ou no Missal) e na sua testa, boca e peito, dizendo:}

\begin{paracol}{2}\latim{
\cruz Initium sancti Evangélii secúndum Joánnem.
}\switchcolumn\portugues{
\cruz Princípio do santo Evangelho segundo S. João.
}\switchcolumn*\latim{
℟. Glória tibi, Dómine.
}\switchcolumn\portugues{
℟. Glória a Vós, Senhor.
}\switchcolumn*\latim{
\blettrine{I}{n} princípio erat Verbum, et Verbum erat apud Deum, et Deus erat Verbum. Hoc erat in princípio apud Deum. Omnia per ipsum facta sunt: et sine ipso factum est nihil, quod factum est: in ipso vita erat, et vita erat lux hóminum: et lux in ténebris lucet, et ténebræ eam non comprehendérunt. Fuit homo missus a Deo, cui nomen erat Joánnes. Hic venit in testimónium, ut testimónium perhibéret de lúmine, ut omnes créderent per illum. Non erat ille lux, sed ut testimónium perhibéret de lúmine. Erat lux vera, quæ illúminat omnem hóminem veniéntem in hunc mundum. In mundo erat, et mundus per ipsum factus est, et mundus eum non cognóvit. In própria venit, et sui eum non recepérunt. Quotquot autem recepérunt eum, dedit eis potestátem fílios Dei fíeri, his, qui credunt in nómine ejus: qui non ex sanguínibus, neque ex voluntáte carnis, neque ex voluntáte viri, sed ex Deo nati sunt. \textit{(Hic genuflectitur)} Et Verbum caro factum est, et habitávit in nobis: et vídimus glóriam ejus, glóriam quasi Unigéniti a Patre, plenum grátiæ et veritátis.
}\switchcolumn\portugues{
\blettrine{N}{o} princípio existia o Verbo, e o Verbo estava com Deus, e o Verbo era Deus. Este estava no princípio com Deus. Todas as cousas foram por Ele criadas, e nada daquilo que foi criado teria sido criado sem Ele. N’Ele havia vida, e a vida era a luz dos homens. A luz resplandeceu nas trevas, mas as trevas a não receberam. Apareceu um homem, mandado por Deus, e o seu nome era João, o qual veio como testemunha, para dar testemunho da luz, a fim de que por ele todos acreditassem. Ele não era a luz, mas aquele que havia de dar testemunho da luz. Existia a luz verdadeira, a luz que ilumina todo o homem que vem a este mundo. Ele estava no mundo, e o mundo, embora houvesse sido criado por Ele, O não conheceu. Veio ao que era seu, e os seus O não receberam. Porém, Ele a todos quantos O receberam e aos que acreditaram no seu nome deu o poder de serem filhos de Deus, os quais não nasceram do sangue, nem do desejo da carne, mas somente da vontade de Deus. E o Verbo fez-se carne \textit{(genuflecte-se)} e habitou entre nós; e contemplamos a sua glória, como era própria do Filho Unigénito do Pai, cheio de graça e de verdade.
}\switchcolumn*\latim{
℟. Deo grátias.
}\switchcolumn\portugues{
℟. Graças a Deus.
}\end{paracol}
