\subsection{17.ª Para implorar o bom tempo}\label{bomtempo}

\paragraph{Oração}
\begin{paracol}{2}\latim{
\rlettrine{A}{d} te nos, Dómine, clamántes exáudi: et áëris serenitátem nobis tríbue supplicántibus; ut, qui juste pro peccátis nostris afflígimur, misericórdia tua præveniénte, cleméntiam sentiámus. Per Dóminum \emph{\&c.}
}\switchcolumn\portugues{
\rlettrine{O}{uvi,} Senhor, aqueles que por Vós clamam, e, Vo-lo pedimos, concedei-nos tempo sereno nos astros, a fim de que nós, que fomos punidos justamente pelos nossos pecados, sejamos remediados pela vossa misericórdia e sintamos a vossa clemência Por nosso Senhor \emph{\&c.}
}\end{paracol}

\paragraph{Secreta}
\begin{paracol}{2}\latim{
\rlettrine{P}{rævéniat} nos, quǽsumus, Dómine, grátia tua semper et subsequátur: et has oblatiónes, quas pro peccátis nostris nómini tuo consecrándas deférimus, benígnus assúme; ut, per intercessiónem Sanctórum tuórum, cunctis nobis profíciant ad salútem. Per Dóminum \emph{\&c.}
}\switchcolumn\portugues{
\rlettrine{S}{enhor,} Vos suplicamos, permiti que a vossa graça nos remedeie e nos acompanhe sempre; e que Vos digneis aceitar benignamente estas oblatas, que vamos consagrar em honra do vosso santo nome, como reparação dos nossos pecados, a fim de que por intercessão dos vossos Santos a todos aproveitem para a salvação. Por nosso Senhor \emph{\&c.}
}\end{paracol}

\paragraph{Postcomúnio}
\begin{paracol}{2}\latim{
\qlettrine{Q}{uǽsumus,} omnípotens Deus, cleméntiam tuam: ut inundántiam coérceas ímbrium, et hilaritátem vultus tui nobis impertíri dignéris. Per Dóminum \emph{\&c.}
}\switchcolumn\portugues{
\slettrine{Ó}{} omnipotente Deus, pedimos à vossa clemência se digne suspender as torrentes da chuva, que nos inunda, e Vos digneis mostrar-nos aspecto agradável. Por nosso Senhor \emph{\&c.}
}\end{paracol}