\subsection{31.ª Pelos inimigos}\label{inimigos}

\paragraph{Oração}
\begin{paracol}{2}\latim{
\rlettrine{D}{eus,} pacis caritatísque amátor et custos: da ómnibus inimícis nostris pacem caritatémque veram; et cunctórum eis remissiónem tríbue peccatórum, nosque ab eórum insídiis poténter éripe. Per Dóminum \emph{\&c.}
}\switchcolumn\portugues{
\slettrine{Ó}{} Deus, que amais e conservais a paz, concedei aos nossos inimigos a paz e a verdadeira caridade, bem como a remissão dos seus pecados; e a nós, Senhor, livrai-nos com vosso poder das suas insídias. Por nosso Senhor \emph{\&c.}
}\end{paracol}

\paragraph{Secreta}
\begin{paracol}{2}\latim{
\rlettrine{O}{blátis,} quǽsumus, Dómine, placáre munéribus: et nos ab inimícis nostris cleménter éripe, eisque indulgéntiam tríbue delictórum. Per Dóminum \emph{\&c.}
}\switchcolumn\portugues{
\rlettrine{V}{os} suplicamos, Senhor, deixai-Vos aplacar com estes dons, que Vos oferecemos; e, pela vossa clemência, livrai-nos das mãos dos nossos inimigos, concedendo-lhes ao mesmo tempo o perdão dos pecados. Por nosso Senhor \emph{\&c.}
}\end{paracol}

\paragraph{Postcomúnio}
\begin{paracol}{2}\latim{
\rlettrine{H}{æc} nos commúnio, Dómine, éruat a delíctis: et ab inimicórum deféndat insídiis. Per Dóminum \emph{\&c.}
}\switchcolumn\portugues{
\qlettrine{Q}{ue} esta comunhão, Senhor, nos livre de todos os delitos e nos defenda das insídias dos nossos inimigos. Por nosso Senhor \emph{\&c.}
}\end{paracol}