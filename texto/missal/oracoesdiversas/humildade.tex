\subsection{27.ª Para pedir a humildade}\label{humildade}

\paragraph{Oração}
\begin{paracol}{2}\latim{
\rlettrine{D}{eus,} qui supérbis resístis et grátiam præstas humílibus: concéde nobis veræ humilitátis virtútem, cujus in se formam fidélibus Unigénitus tuus exhíbuit; ut numquam indignatiónem tuam provocémus elati, sed pótius grátiæ tuæ capiámus dona subjécti. Per eúndem Dóminum nostrum. \emph{\&c.}
}\switchcolumn\portugues{
\slettrine{Ó}{} Deus, que resistis aos soberbos e dais a graça aos humildes, concedei-nos a virtude da verdadeira humildade, da qual o vosso Filho Unigénito deu aos fiéis o exemplo, para que pelo nosso orgulho nunca incorramos na vossa indignação, mas, permanecendo humildes, recebamos os dons da vossa graça. Por nosso Senhor \emph{\&c.}
}\end{paracol}

\paragraph{Secreta}
\begin{paracol}{2}\latim{
\rlettrine{H}{æc} oblátio, Dómine, quǽsumus, veræ nobis humilitátis grátiam obtíneat: simúlque a córdibus nostris concupiscéntiam carnis et oculórum atque ambitiónem sǽculi áuferat; quaténus sóbrie, juste piéque vivéntes, prǽmia consequámur ætérna. Per Dóminum \emph{\&c.}
}\switchcolumn\portugues{
\qlettrine{Q}{ue} esta oblação, Senhor, Vos suplicamos, nos obtenha a graça da verdadeira humildade, e que ao mesmo tempo arranque dos nossos corações a concupiscência da carne e dos olhos, assim como a ambição do espírito do mundo, para que, vivendo nós com sobriedade, justiça e piedade, consigamos os prémios eternos. Por nosso Senhor \emph{\&c.}
}\end{paracol}

\paragraph{Postcomúnio}
\begin{paracol}{2}\latim{
\rlettrine{H}{ujus,} Dómine, sacraménti percéptio peccatórum nostrórum máculas abstérgat: et nos, per humilitátis exhibitiónem, ad cœléstia regna perdúcat. Per Dóminum nostrum \emph{\&c.}
}\switchcolumn\portugues{
\rlettrine{S}{enhor,} que a recepção deste sacramento possa lavar as máculas dos nossos pecados; e, praticando nós a humildade, nos conduza ao reino celestial. Por nosso Senhor \emph{\&c.}
}\end{paracol}