\subsection{1.ª Para implorar os Sufrágios dos Santos}\label{sufragiosantos1}

\paragraph{Oração}
\begin{paracol}{2}\latim{
\rlettrine{C}{oncéde,} quǽsumus, omnípotens Deus: ut intercéssio sanctæ Dei Genetrícis Maríæ, sanctorúmque ómnium Apostolórum, Mártyrum, Confessórum, atque Vírginum, et ómnium electórum tuórum, nos ubíque lætíficet; ut, dum eórum mérita recólimus, patrocínia sentiámus. Per eúndem Dóminum nostrum \emph{\&c.}
}\switchcolumn\portugues{
\rlettrine{C}{oncedei-nos,} Vos suplicamos, ó Deus omnipotente, que a intercessão de Maria, santa Mãe de Deus, de todos os Santos Apóstolos, Mártires, Confessores e Virgens e de todos vossos escolhidos nos alegre sempre e em toda a parte, a fim de que, sempre que recordemos os seus merecimentos, gozemos a sua protecção. Pelo mesmo nosso Senhor \emph{\&c.}
}\end{paracol}

\paragraph{Secreta}
\begin{paracol}{2}\latim{
\rlettrine{O}{blátis,} Dómine, placáre munéribus: et, intercedénte beáta María semper Vírgine cum ómnibus Sanctis tuis, a cunctis nos defénde perículis. Per Dóminum \emph{\&c.}
}\switchcolumn\portugues{
\rlettrine{A}{placai-Vos,} Senhor, com os dons que Vos oferecemos; e, por intercessão da B. Maria, sempre Virgem, e de todos os Santos, defendei-nos de todos os perigos. Por nosso Senhor \emph{\&c.}
}\end{paracol}

\paragraph{Postcomúnio}
\begin{paracol}{2}\latim{
\rlettrine{S}{úmpsimus,} Dómine, beátæ Maríæ semper Vírginis et ómnium Sanctórum tuórum memóriam recoléntes, sacraménta cœléstia: præsta, quǽsumus; ut, quod temporáliter gérimus, ætérnis gáudiis consequámur. Per Dóminum \emph{\&c.}
}\switchcolumn\portugues{
\rlettrine{R}{ecebemos,} Senhor, os dons celestiais em memória da B. Maria, sempre Virgem, e de todos vossos Santos; e, Vos suplicamos, concedei-nos que esta união, começada na terra, possa ser coroada com as alegrias eternas. Por nosso Senhor \emph{\&c.}
}\end{paracol}