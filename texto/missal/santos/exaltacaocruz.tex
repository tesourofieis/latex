\subsectioninfo{Exaltação da Santa Cruz}{14 de Setembro}\label{exaltacaocruz}

\paragraphinfo{Intróito}{Gl. 6, 14}
\begin{paracol}{2}\latim{
\rlettrine{N}{os} autem gloriári opórtet in Cruce Dómini nostri Jesu Christi: in quo est salus, vita et resurréctio nostra: per quem salváti et liberáti sumus. \emph{Ps. 66, 2} Deus misereátur nostri, et benedícat nobis: illúminet vultum suum super nos, et misereátur nostri.
℣. Gloria Patri \emph{\&c.}
}\switchcolumn\portugues{
\rlettrine{C}{onvém} que nos gloriemos na Cruz de nosso Senhor Jesus Cristo, que é a nossa salvação, vida e ressurreição: por quem fomos salvos e livres. \emph{Sl. 66, 2} Que o Senhor tenha misericórdia de nós e nos abençoe; que o Senhor nos ilumine com o brilho que resplandece da sua face, e se compadeça de nós.
℣. Glória ao Pai \emph{\&c.}
}\end{paracol}

\paragraph{Oração}
\begin{paracol}{2}\latim{
\rlettrine{D}{eus,} qui nos hodiérna die Exaltatiónis sanctæ Crucis ánnua sollemnitáte lætíficas: præsta, quǽsumus; ut, cujus mystérium in terra cognóvimus, ejus redemptiónis prǽmia in cœlo mereámur. Per eúndem Dóminum \emph{\&c.}
}\switchcolumn\portugues{
\slettrine{Ó}{} Deus, que neste dia nos regozijais com a festividade anual da Exaltação da Santa Cruz, concedei-nos, Vos imploramos, que obtenhamos no céu as recompensas adquiridas pela redenção operada pelo vosso Filho, cujo mistério na terra conhecemos. Pelo mesmo nosso Senhor \emph{\&c.}
}\end{paracol}

\paragraphinfo{Epístola}{Fl. 2, 5-11}
\begin{paracol}{2}\latim{
Léctio Epístolæ beáti Pauli Apóstoli ad Philippénses.
}\switchcolumn\portugues{
Lição da Ep.ª do B. Ap.º Paulo aos Filipenses.
}\switchcolumn*\latim{
\rlettrine{F}{ratres:} Hoc enim sentíte in vobis, quod et in Christo Jesu: qui, cum in forma Dei esset, non rapinam arbitrátus est esse se æquálem Deo: sed semetípsum exinanívit, formam servi accipiens, in similitudinem hóminum factus, et hábitu inventus ut homo. Humiliávit semetípsum, factus obǿdiens usque ad mortem, mortem autem crucis. Propter quod et Deus exaltávit illum: et donávit illi nomen, quod est super omne nomen: (hic genuflectitur) ut in nomine Jesu omne genu flectátur cœléstium, terréstrium et infernórum: et omnis lingua confiteátur, quia Dóminus Jesus Christus in glória est Dei Patris.
}\switchcolumn\portugues{
\rlettrine{M}{eus} irmãos: Tende os mesmos sentimentos que animaram Jesus Cristo, o qual, embora fosse Deus por natureza (e não era usurpação o julgar-se igual a Deus), contudo humilhou-se a si próprio, reduzindo-se à condição de servo, tornando-se semelhante aos homens e reconhecido como homem pelas aparências. Humilhou-se a si próprio, obedecendo até à morte, e morte na cruz. Por isso Deus O exaltou e deu-Lhe um nome que é superior a todo o nome (devemos genuflectir), para que ao ser proferido o Nome de Jesus todos os joelhos se dobrem nos céus, na terra e até nos infernos; e todas as línguas confessem que nosso Senhor Jesus Cristo está na glória de Deus Pai!
}\end{paracol}

\paragraphinfo{Gradual}{ibid., 8-9}
\begin{paracol}{2}\latim{
\rlettrine{C}{hristus} factus est pro nobis obǿdiens
usque ad mortem, mortem autem crucis. ℣. Propter quod et Deus exaltávit illum, et dedit illi nomen, quod est super omne nomen.
}\switchcolumn\portugues{
\rlettrine{C}{risto} fez-se obediente por nós até à morte, e morte na Cruz. ℣. Eis porque Deus O exaltou e Lhe deu um nome que é superior a todo o nome.
}\switchcolumn*\latim{
Allelúja, allelúja. ℣. Dulce lignum, dulces clavos, dúlcia ferens póndera: quæ sola fuísti digna sustinére Regem cœlórum et Dóminum. Allelúja.
}\switchcolumn\portugues{
Aleluia, aleluia. ℣. Ó doce lenho, ó doces cravos, que segurais um peso mais doce ainda! Só tu fostes digno de segurar o Rei e o Senhor dos céus. Aleluia.
}\end{paracol}

\paragraphinfo{Evangelho}{Jo. 12, 31-36}
\begin{paracol}{2}\latim{
\cruz Sequéntia sancti Evangélii secúndum Joánnem.
}\switchcolumn\portugues{
\cruz Continuação do santo Evangelho segundo S. João.
}\switchcolumn*\latim{
\blettrine{I}{n} illo témpore: Dixit Jesus turbis Judæórum: Nunc judícium est mundi: nunc princeps hujus mundi ejiciátur foras. Et ego si exaltátum fuero a terra, ómnia traham ad meipsum. (Hoc autem dicébat, signíficans, qua morte esset moritúrus.) Respóndit ei turba. Nos audívimus ex lege, quia Christus manet in ætérnum: et quómodo tu dicis: Opórtet exaltári Fílium hóminis? Quis est iste Fílius hóminis? Dixit ergo eis Jesus: Adhuc módicum lumen in vobis est. Ambuláte, dum lucem habétis, ut non vos ténebræ comprehéndant: et qui ámbulat in ténebris, nescit, quo vadat. Dum lucem habétis, crédite in lucem, ut fílii lucis sitis.
}\switchcolumn\portugues{
\blettrine{N}{aquele} tempo, disse Jesus às turbas dos judeus: «É agora que o mundo vai ser julgado; é agora que o príncipe deste mundo vai ser expulso. E Eu, quando for elevado acima da terra, atrairei todos a mim». (Isto dizia Ele para indicar o género de morte de que morreria). As turbas responderam-Lhe: «Aprendemos na Lei que Cristo permanecerá para sempre. Como, pois, dizeis: é necessário que seja elevado o Filho do homem? Quem é este Filho do homem?». E Jesus retorquiu-lhes: «A luz estará no meio de vós, ainda por algum tempo. Caminhai, pois, enquanto tiverdes luz, receando que as trevas vos cerquem. Aquele que caminha nas trevas, não sabe para onde vai. Enquanto tendes a luz, acreditai nela, para que sejais filhos da luz».
}\end{paracol}

\paragraph{Ofertório}
\begin{paracol}{2}\latim{
\rlettrine{P}{rótege,} Dómine, plebem tuam per signum sanctæ Crucis ab ómnibus insídiis inimicórum ómnium: ut tibi gratam exhibeámus servitútem, et acceptábile fiat sacrifícium nostrum, allelúja.
}\switchcolumn\portugues{
\rlettrine{S}{enhor,} com o sinal da Santa Cruz livrai o vosso povo das insídias de todos seus inimigos, a fim de que a nossa servidão Vos seja agradável e aceiteis o nosso sacrifício, aleluia.
}\end{paracol}

\paragraph{Secreta}
\begin{paracol}{2}\latim{
\qlettrine{J}{esu} Christi, Dómini nostri, Córpore et Sánguine saginándi, per quem Crucis est sanctifícátum vexíllum: quǽsumus, Dómine, Deus noster; ut, sicut illud adoráre merúimus, ita perénniter ejus glóriæ salutáris potiámur efféctu. Per eúndem Dóminum \emph{\&c.}
}\switchcolumn\portugues{
\rlettrine{D}{evendo} alimentar-nos com o Corpo e o Sangue de nosso Senhor Jesus Cristo, que santificou o estandarte da Cruz, Vos rogamos, ó Senhor, nosso Deus, que, assim como logramos adorá-l’O na terra, assim alcancemos na eternidade gozar a posse dos efeitos da sua salutar glória. Pelo mesmo nosso Senhor \emph{\&c.}
}\end{paracol}

\paragraph{Comúnio}
\begin{paracol}{2}\latim{
\rlettrine{P}{er} signum Crucis de inimícis nostris líbera nos, Deus noster.
}\switchcolumn\portugues{
\rlettrine{P}{elo} sinal da Santa Cruz, livrai-nos de nossos inimigos, ó nosso Deus!
}\end{paracol}

\paragraph{Postcomúnio}
\begin{paracol}{2}\latim{
\rlettrine{A}{désto} nobis, Dómine, Deus noster: et, quos sanctæ Crucis lætári facis honóre, ejus quoque perpétuis defénde subsídiis. Per Dóminum \emph{\&c.}
}\switchcolumn\portugues{
\rlettrine{P}{rotegei-nos,} ó Senhor, nosso Deus; e, com vosso perpétuo socorro, defendei sempre aqueles a quem proporcionastes a alegria de honrar a Santa Cruz. Por nosso Senhor \emph{\&c.}
}\end{paracol}
