\subsectioninfo{S. Valentim, Presbítero e Mártir}{14 de Fevereiro}

\textit{Como Missa In virtúte tua, página \pageref{martirnaopontifice1}, excepto:}

\paragraph{Oração}
\begin{paracol}{2}\latim{
\rlettrine{P}{ræsta,} quǽsumus, omnípotens Deus: ut, qui beáti Valentíni Mártyris tui natalítia cólimus, a cunctis malis imminéntibus, ejus intercessióne, liberémur. Per Dóminum nostrum \emph{\&c.}
}\switchcolumn\portugues{
\rlettrine{C}{oncedei-nos,} Vos suplicamos, ó Deus omnipotente, que, celebrando o natal do B. Valentim, vosso Mártir, sejamos livres por sua intercessão de todos os males que nos ameaçam. Por nosso Senhor \emph{\&c.}
}\end{paracol}

\paragraph{Secreta}
\begin{paracol}{2}\latim{
\rlettrine{S}{úscipe,} quǽsumus, Dómine, múnera dignánter obláta: et, beáti Valentini Mártyris tui suffragántibus méritis, ad nostræ salútis auxílium proveníre concéde. Per Dóminum \emph{\&c.}
}\switchcolumn\portugues{
\rlettrine{R}{ecebei,} Vos suplicamos, Senhor, os dons que devidamente Vos oferecemos; e pelos méritos e sufrágios do B. Valentim, vosso Mártir, concedei-nos que nos sirvam de auxílio para a salvação. Por nosso Senhor \emph{\&c.}
}\end{paracol}

\paragraph{Postcomúnio}
\begin{paracol}{2}\latim{
\rlettrine{S}{it} nobis, Dómine, reparátio mentis et córporis cœléste mystérium: ut, cujus exséquimur actiónem, intercedénte beáto Valentíno Mártyre tuo, sentiámus efféctum. Per Dóminum \emph{\&c.}
}\switchcolumn\portugues{
\qlettrine{Q}{ue} estes celestiais mystérios, Senhor, restaurem a nossa alma e o nosso corpo, a fim de que, por intercessão do B. Valentim, vosso Mártir, sintamos os efeitos do sacrifício que celebrámos. Por nosso Senhor \emph{\&c.}
}\end{paracol}