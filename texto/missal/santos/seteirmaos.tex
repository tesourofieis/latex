\subsectioninfo{Os 7 Irmãos e S. S. Rufina e Secunda, Márts.}{10 Julho}

\paragraphinfo{Intróito}{Sl. 112, 1 \& 9}
\begin{paracol}{2}\latim{
\rlettrine{L}{audáte,} pueri, Dóminum, laudáte nomen Dómini: qui habitáre facit stérilem in domo, matrem filiórum lætántem. \emph{Ps. ibid., 2} Sit nomen Dómini benedíctum: ex hoc nunc, ei usque in sǽculum.
℣. Gloria Patri \emph{\&c.}
}\switchcolumn\portugues{
\rlettrine{L}{ouvai} o Senhor, ó meninos; louvai o nome do Senhor; pois Ele fez habitar, cheia de alegria, na sua casa, como mãe de numerosos filhos, aquela que antes era estéril. \emph{Sl. ibid., 2} Bendito seja o nome do Senhor, agora e em todos os séculos.
℣. Glória ao Pai \emph{\&c.}
}\end{paracol}

\paragraph{Oração}
\begin{paracol}{2}\latim{
\rlettrine{P}{ræsta,} quǽsumus, omnípotens Deus: ut, qui gloriósos Martyres fortes in sua confessióne cognóvimus, pios apud te in nostra intercessióne sentiámus. Per Dóminum nostrum \emph{\&c.}
}\switchcolumn\portugues{
\rlettrine{C}{oncedei-nos,} Vos pedimos, ó Deus omnipotente, que, reconhecendo nós a fortaleza com que estes gloriosos Mártires confessaram a sua fé, sintamos quanto eles junto de Vós são fervorosos, intercedendo por nós. Por nosso Senhor \emph{\&c.}
}\end{paracol}

\paragraphinfo{Epístola}{Página \pageref{nemvirgensnemmartires}}

\paragraphinfo{Gradual}{Sl. 123, 7-8}
\begin{paracol}{2}\latim{
\rlettrine{A}{nima} nostra, sicut passer, erépta est de láqueo venántium. ℣. Láqueus contrítus est, et nos liberáti sumus: adjutórium nostrum in nómine Dómini, qui fecit cœlum et terram.
}\switchcolumn\portugues{
\rlettrine{A}{} nossa alma livrou-se, como um pássaro, do laço dos caçadores. ℣. O laço quebrou-se e ficamos livres: o nosso auxílio está no nome do Senhor, que criou o céu e a terra.
}\switchcolumn*\latim{
Allelúja, allelúja. ℣. Hæc est vera fratérnitas, quæ vicit mundi crímina: Christum secuta est, ínclita tenens regna cœléstia. Allelúja.
}\switchcolumn\portugues{
Aleluia, aleluia. ℣. Eis a verdadeira fraternidade que venceu os perigos do mundo. Ela seguiu Cristo e possuirá com glória o reino celestial. Aleluia.
}\end{paracol}

\paragraphinfo{Evangelho}{Mt. 12, 46-50}
\begin{paracol}{2}\latim{
\cruz Sequéntia sancti Evangélii secúndum Matthǽum.
}\switchcolumn\portugues{
\cruz Continuação do santo Evangelho segundo S. Mateus.
}\switchcolumn*\latim{
\blettrine{I}{n} illo témpore: Loquente Jesu ad turbas, ecce, Mater ejus et fratres stabant foris, quæréntes loqui ei. Dixit autem ei quidam: Ecce, mater tua et fratres tui foris stant, quæréntes te. At ipse respóndens dicénti sibi, ait: Quæ est mater mea et qui sunt fratres mei? Et exténdens manum in discípulos suos, dixit: Ecce mater mea et fratres mei. Quicúmque enim fécerit voluntátem Patris mei, qui in cœlis est: ipse meus frater et soror et mater est.
}\switchcolumn\portugues{
\blettrine{N}{aquele} tempo, enquanto Jesus falava às turbas, eis que sua mãe e seus irmãos estavam lá fora, procurando falar-Lhe. Disse-Lhe, pois, alguém: «Eis que vossa mãe e vossos irmãos estão lá fora e Vos procuram. Jesus, respondendo àqueles que Lhe falaram, disse: «Quem é a minha mãe e quem são os meus irmãos?». E, estendendo a mão para os seus discípulos, disse: «Eis a minha mãe e os meus irmãos, pois todo aquele que faz a vontade de meu Pai, que está nos céus, é meu irmão, minha irmã e minha mãe».
}\end{paracol}

\paragraphinfo{Ofertório}{Sl. 123, 7}
\begin{paracol}{2}\latim{
\rlettrine{A}{nima} nostra, sicut passer, erépta est de láqueo venántium: láqueus contrítus est, et nos liberáti sumus.
}\switchcolumn\portugues{
\rlettrine{A}{} nossa alma livrou-se, como um pássaro, do laço dos caçadores. O laço quebrou-se e ficámos livres.
}\end{paracol}

\paragraph{Secreta}
\begin{paracol}{2}\latim{
\rlettrine{S}{acrifíciis} præséntibus, quǽsumus, Dómine, inténde placátus: et, intercedéntibus Sanctis tuis, devotióni nostræ profíciant et salúti. Per Dóminum nostrum \emph{\&c.}
}\switchcolumn\portugues{
\rlettrine{D}{ignai-Vos} olhar propício, Senhor, para o presente sacrifício, e que por intercessão dos vossos Santos ele nos sirva para aumentar a nossa devoção e para alcançar a salvação. Por nosso Senhor \emph{\&c.}
}\end{paracol}

\paragraphinfo{Comúnio}{Mt. 12, 50}
\begin{paracol}{2}\latim{
\qlettrine{Q}{uicumque} fecerit voluntátem Patris mei, qui in cælis est: ipse meus frater et soror et mater est, dicit Dóminus.
}\switchcolumn\portugues{
\rlettrine{T}{odo} aquele que faz a vontade de meu Pai, que está nos céus, é meu irmão, minha irmã e minha mãe, diz o Senhor.
}\end{paracol}

\paragraph{Postcomúnio}
\begin{paracol}{2}\latim{
\qlettrine{Q}{uǽsumus,} omnípotens Deus: ut, intercedéntibus Sanctis tuis, illíus salutáris capiámus efféctum; cujus per heec mystéria pignus accépimus. Per Dóminum \emph{\&c.}
}\switchcolumn\portugues{
\rlettrine{V}{os} suplicamos, ó Deus omnipotente, que, pela intercessão dos vossos Santos, Vos digneis dispensar-nos o efeito da salvação, de que já recebemos o penhor nestes mistérios. Por nosso Senhor \emph{\&c.}
}\end{paracol}
