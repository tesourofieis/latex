\subsectioninfo{Comemoração S. Dionísio e Outros, Mártires}{9 de Outubro}

\paragraph{Oração}
\begin{paracol}{2}\latim{
\rlettrine{D}{eus,} qui hodiérna die beátum Dionýsium, Mártyrem tuum atque Pontíficem, virtúte constantiæ in passióne roborásti, quique illi, ad prædicándum géntibus glóriam tuam, Rústicum et Eleuthérium sociáre dignátus es: tríbue nobis, quǽsumus; eórum imitatióne, pro amóre tuo próspera mundi despícere, et nulla ejus advérsa formidáre. Per Dóminum \emph{\&c.}
}\switchcolumn\portugues{
\slettrine{Ó}{} Deus, que neste dia fortalecestes o B. Dionísio, vosso Mártir e Pontífice, com a virtude da constância no martírio e que Vos dignastes associar-lhe Rústico e Eleutério para pregarem a vossa glória aos povos, concedei-nos, Vos suplicamos, que, seguindo os seus exemplos, desprezemos por vosso amor as prosperidades do mundo e nunca temamos as adversidades. Por nosso Senhor \emph{\&c.}
}\end{paracol}

\paragraph{Secreta}
\begin{paracol}{2}\latim{
\rlettrine{O}{bláta} tibi, Dómine, múnera pópuli tui, pro tuórum honóre Sanctórum, súscipe propítius, quǽsumus: et eórum nos intercessióne sanctífica. Per Dóminum \emph{\&c.}
}\switchcolumn\portugues{
\rlettrine{R}{ecebei} propício, Senhor, Vos rogamos, os dons que o vosso povo apresenta em honra dos vossos Santos; e pela sua intercessão santificai-nos. Por nosso Senhor \emph{\&c.}
}\end{paracol}

\paragraph{Postcomúnio}
\begin{paracol}{2}\latim{
\rlettrine{S}{umptis,} Dómine, sacraméntis, quǽsumus: ut, intercedéntibus beátis Martýribus tuis Dionýsio, Rústico et Eleuthério, ad redemptiónis ætérnæ proficiamus augméntum. Per Dóminum nostrum Jesum Christum, Fílium tuum: Qui tecum vivit et regnat \emph{\&c.}
}\switchcolumn\portugues{
\rlettrine{H}{avendo} nós recebido estes sacramentos, Senhor, Vos suplicamos, dignai-Vos permitir que por intercessão dos vossos B. B. Mártires Dionísio, Rústico e Eleutério cada vez mais avancemos no caminho da redenção eterna. Por nosso Senhor \emph{\&c.}
}\end{paracol}