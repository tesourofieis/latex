\subsectioninfo{S. Domingos de Gusmão, Conf.}{4 de Agosto}

\textit{Como na Missa Os justi, página \pageref{confessoresnaopontifices1}, excepto:}

\paragraph{Oração}
\begin{paracol}{2}\latim{
\rlettrine{D}{eus,} qui Ecclésiam tuam beáti Dominici Confessóris tui illumináre dignátus es méritis et doctrinis: concéde; ut ejus intercessióne temporalibus non destituatur auxiliis, et spiritualibus semper profíciat increméntis. Per Dóminum nostrum \emph{\&c.}
}\switchcolumn\portugues{
\slettrine{Ó}{} Deus, que Vos dignastes iluminar a vossa Igreja com os méritos e as lições do B. Domingos, vosso Confessor, concedei-nos pela sua intercessão que ela não seja privada dos auxílios temporais e que sempre tomem incremento os seus bens espirituais. Por nosso Senhor \emph{\&c.}
}\end{paracol}

\paragraphinfo{Epístola}{Página \pageref{doutores}}

\paragraphinfo{Gradual}{Sl. 91, 13 \& 14}
\begin{paracol}{2}\latim{
\qlettrine{J}{ustus} ut palma florébit: sicut cedrus Líbani multiplicábitur in domo Dómini. ℣. \emph{ibid., 3} Ad annuntiándum mane misericórdiam tuam, et veritátem tuam per noctem.
}\switchcolumn\portugues{
\rlettrine{O}{} justo florescerá, como a palmeira, e multiplicar-se-á, como o cedro do Líbano, na casa do Senhor. ℣. \emph{ibid., 3} Para anunciar de manhã a vossa misericórdia e durante a noite a vossa verdade.
}\switchcolumn*\latim{
Allelúja, allelúja. ℣. \emph{Osee 14, 6} Justus germinábit sicut lílium: et florébit in ætérnum ante Dóminum. Allelúja.
}\switchcolumn\portugues{
Aleluia, aleluia. ℣. \emph{Os. 14, 6} O justo germinará, como o lírio, e florescerá perpetuamente na presença do Senhor. Aleluia.
}\end{paracol}

\paragraph{Secreta}
\begin{paracol}{2}\latim{
\rlettrine{M}{únera} tibi, Dómine, dicáta sanctífica: ut, méritis beáti Domínici Confessóris tui, nobis profíciant ad medélam. Per Dóminum \emph{\&c.}
}\switchcolumn\portugues{
\rlettrine{S}{antificai,} Senhor, os dons que Vos são apresentados, a fim de que, pelos méritos do B. Domingos, vosso Confessor, sirvam de remédio, às nossas almas. Por nosso Senhor \emph{\&c.}
}\end{paracol}

\paragraphinfo{Comúnio}{Lc. 12, 42}
\begin{paracol}{2}\latim{
\rlettrine{F}{idélis} servus et prudens, quem constítuit dóminus super famíliam suam: ut det illis in témpore trítici mensúram.
}\switchcolumn\portugues{
\rlettrine{O}{} servo fiel e prudente é destinado pelo Senhor para distribuir oportunamente a cada um na sua família a sua medida de trigo.
}\end{paracol}

\paragraph{Postcomúnio}
\begin{paracol}{2}\latim{
\rlettrine{C}{oncéde,} quǽsumus, omnípotens Deus: ut, qui peccatórum nostrórum póndere prémimur, beáti Domínici Confessóris tui patrocínio sublevémur. Per Dóminum \emph{\&c.}
}\switchcolumn\portugues{
\rlettrine{C}{oncedei-nos,} Deus omnipotente. Vos rogamos, que, estando oprimidos sob o peso dos nossos pecados, sejamos livres deles, pelo patrocínio do B. Domingos, vosso Confessor. Por nosso Senhor \emph{\&c.}
}\end{paracol}
