\subsectioninfo{Santa Cecília, Virgem e Mártir}{22 de Novembro}

\textit{Como na Missa Loquébar, página \pageref{virgensmartires1}, excepto:}

\paragraph{Oração}
\begin{paracol}{2}\latim{
\rlettrine{D}{eus,} qui nos ánnua beátæ Caeciliae Vírginis et Mártyris tuæ sollemnitáte lætíficas: da, ut, quam venerámur offício, étiam piæ conversatiónis sequámur exémplo. Per Dóminum \emph{\&c.}
}\switchcolumn\portugues{
\slettrine{Ó}{} Deus, que nos alegrais com a solenidade anual da B. Cecília, vossa Virgem e Mártir, dignai-Vos permitir que, honrando-a com estes cultos, imitemos também os exemplos da sua piedosa vida. Por nosso Senhor \emph{\&c.}
}\end{paracol}

\paragraphinfo{Epístola}{Página \pageref{virgensmartires2}}

\paragraphinfo{Gradual}{Sl. 44, 11 \& 12}
\begin{paracol}{2}\latim{
\rlettrine{A}{udi,} fília, et vide, et inclína aurem tuam: quia concupívit Rex spéciem tuam. ℣. \emph{ibid., 5} Spécie tua et pulchritúdine tua inténde, próspere procéde et regna.
}\switchcolumn\portugues{
\rlettrine{E}{scutai,} ó minha filha, vede e inclinai o vosso ouvido; o Rei está cheio de amor por vós, por causa da vossa beleza! ℣. \emph{ibid., 5} Com a vossa glória e com vossa formosura caminhai, triunfai e reinai.
}\switchcolumn*\latim{
Allelúja, allelúja. ℣. \emph{Matth. 25, 4 \& 6} Quinque prudéntes vírgines accepérunt óleum in vasis suis cum lampádibus: média autem nocte clamor factus est: Ecce, sponsus venit: exíte óbviam Christo Dómino. Allelúja.
}\switchcolumn\portugues{
Aleluia, aleluia. ℣. \emph{Mt. 25, 4 \& 6} As cinco virgens prudentes tomaram óleo em seus vasos para suas lâmpadas. No meio da noite uma voz soou: eis que chega o esposo: ide ao encontro de Cristo, Senhor. Aleluia.
}\end{paracol}

\paragraph{Secreta}
\begin{paracol}{2}\latim{
\rlettrine{H}{æc} hóstia, Dómine, placatiónis et laudis, quǽsumus: ut, intercedénte beáta Cæcília Vírgine et Mártyre tua, nos propitiatióne tua dignos semper effíciat. Per Dóminum \emph{\&c.}
}\switchcolumn\portugues{
\qlettrine{Q}{ue} esta hóstia de propiciação e de louvor, Senhor, Vos rogamos, nos torne sempre dignos da vossa misericórdia, pela intercessão da B. Cecília, vossa Virgem e Mártir. Por nosso Senhor \emph{\&c.}
}\end{paracol}

\paragraph{Postcomúnio}
\begin{paracol}{2}\latim{
\rlettrine{S}{atiásti,} Dómine, famíliam tuam munéribus sacris: ejus, quǽsumus, semper interventióne nos réfove, cujus sollémnia celebrámus. Per Dóminum \emph{\&c.}
}\switchcolumn\portugues{
\rlettrine{S}{aciastes,} Senhor, a vossa família com vossos sacratíssimos dons; e dignai-Vos favorecer-nos sempre, Vos imploramos, pela intercessão daquela cuja festa celebramos. Por nosso Senhor \emph{\&c.}
}\end{paracol}
