\subsectioninfo{S. Ireneu, B. e Mártir}{28 de Junho}

\paragraphinfo{Intróito}{Ml. 2, 6}
\begin{paracol}{2}\latim{
\rlettrine{L}{ex} veritátis fuit in ore ejus, et iníquitas non est invénta in lábiis ejus: in pace et in æquitáte ambulávit mecum, et multos avértit ab iniquitáte. \emph{Ps. 77, 1} Atténdite, pópule meus, legem meam: inclináte aurem vestram in verba oris mei.
℣. Gloria Patri \emph{\&c.}
}\switchcolumn\portugues{
\rlettrine{A}{} lei da verdade esteve na sua boca, e a iniquidade nunca existiu nos seus lábios: caminhou comigo na paz e na equidade e afastou muitos da iniquidade. \emph{Sl. 77, 1} Ouvi, ó meu povo, a minha lei: inclinai vossos ouvidos para as palavras da minha boca.
℣. Glória ao Pai \emph{\&c.}
}\end{paracol}

\paragraph{Oração}
\begin{paracol}{2}\latim{
\rlettrine{D}{eus,} qui beáto Irenǽo Mártyri tuo atque Pontifici tribuísti, ut et veritate doctrínæ expugnáret hǽreses, et pacem Ecclésiæ felíciter confirmáret: da, quǽsumus, plebi tuæ in sancta religióne constántiam; et pacem tuam nostris concéde tempóribus. Per Dóminum \emph{\&c.}
}\switchcolumn\portugues{
\slettrine{Ó}{} Deus, que concedestes ao B. Ireneu, vosso Mártir e Pontífice, que combatesse as heresias com a verdade da doutrina e que alcançasse felizmente a paz para a Igreja, permiti, Vos suplicamos, que o vosso povo seja constante na santa religião; e, Senhor, concedei a vossa paz aos nossos tempos. Por nosso Senhor \emph{\&c.}
}\end{paracol}

\paragraphinfo{Epístola}{2. Tm. 3, 14-17; 4, 1-5.}
\begin{paracol}{2}\latim{
Léctio Epístolæ beáti Pauli Apóstoli ad Timótheum.
}\switchcolumn\portugues{
Lição da Ep.ª do B. Ap.º Paulo a Timóteo.
}\switchcolumn*\latim{
\rlettrine{C}{aríssime:} Permane in iis, quæ didicísti et crédita sunt tibi: sciens, a quo didíceris; et quia ab infántia sacras; lítteras nosti, quæ te possunt instrúere ad salútem, per fidem, quæ est in Christo Jesu. Omnis Scriptúra divínitus inspiráta útilis est ad docéndum, ad arguéndum, ad corripiéndum, ad erudiéndum in justítia: ut perféctus sit homo Dei, ad omne opus bonum instrúctus. Testíficor coram Deo, et Jesu Christo, qui judicatúrus est vivos et mórtuos, per advéntum ipsíus et regnum ejus: prǽdica verbum, insta opportúne, importúne: árgue, óbsecra, íncrepa in omni patiéntia et doctrína. Erit enim tempus, cum sanam doctrínam non sustinébunt, sed ad sua desidéria coacervábunt sibi magístros, pruriéntes áuribus, et a veritáte quidem audítum avértent, ad fábulas autem converténtur. Tu vero vígila, in ómnibus labóra, opus fac Evangelístæ, ministérium tuum ímpie.
}\switchcolumn\portugues{
\rlettrine{C}{aríssimo:} Permanece firme naquelas coisas que aprendeste, das quais tens a certeza, sabendo de quem as aprendeste. Além de que desde a infância que conheces as sagradas letras que podem instruir-te na salvação pela fé em Jesus Cristo. Toda a Escritura, devidamente inspirada, é útil para ensinar, para argumentar, para corrigir e para educar na justiça, a fim de que o homem de Deus seja perfeito e capaz para toda a obra boa. Eu te conjuro diante de Deus e ele Jesus Cristo, pela sua vinda e pelo seu reino, que deve julgar os vivos e os mortos na sua vinda e no seu reino, que pregues a palavra, instes oportuna e inoportunamente, repreendas, supliques e ameaces com toda a paciência e doutrina; pois virá tempo em que não suportarão a sã doutrina, mas, indo ao sabor dos seus desejos, procurarão para si muitos mestres que lhes preguem o que os ouvidos gostam de ouvir e fechem os ouvidos à verdade, para os abrirem às fábulas. Tu, porém, vigia, trabalha em tudo, faz o trabalho de Evangelista, cumpre o teu ministério.
}\end{paracol}

\paragraphinfo{Gradual}{Sl. 121, 8}
\begin{paracol}{2}\latim{
\rlettrine{P}{ropter} fratres meos et próximos meos loquébar pacem de te. ℣. \emph{Ps. 36, 37} Custódi innocéntiam et vide æquitátem: quóniam sunt relíquiæ hómini pacífico.
}\switchcolumn\portugues{
\rlettrine{P}{or} causa dos meus irmãos e dos meus vizinhos, peço a paz para vós. ℣. \emph{Sl. 36, 37} Guardai a inocência e observai a justiça, pois o homem pacífico terá posteridade.
}\switchcolumn*\latim{
Allelúja, allelúja. ℣. \emph{Eccli. 6, 35} In multitúdine presbyterórum prudéntium sta, et sapiéntiæ illórum ex corde conjúngere, ut omnem narratiónem Dei possis audíre. Allelúja.
}\switchcolumn\portugues{
Aleluia, aleluia. ℣. \emph{Ecl. 6, 35} Permanecei na assembleia dos presbíteros e uni-vos de coração à sua sabedoria, para que possais ouvir tudo quanto digam de Deus. Aleluia.
}\end{paracol}

\paragraphinfo{Evangelho}{Mt. 10, 28-33}
\begin{paracol}{2}\latim{
\cruz Sequéntia sancti Evangélii secúndum Matthǽum.
}\switchcolumn\portugues{
\cruz Continuação do santo Evangelho segundo S. Mateus.
}\switchcolumn*\latim{
\blettrine{I}{n} illo témpore: Dixit Jesus discípulis suis: Nolíte timére eos, qui occídunt corpus, ánimam autem non possunt occídere; sed pótius timéte eum, qui potest et ánimam et corpus pérdere in gehénnam. Nonne duo pásseres asse véneunt: et unus ex illis non cadet super terram sine Patre vestro? Vestri autem capílli cápitis omnes numeráti sunt. Nolíte ergo timére: multis passéribus melióres estis vos. Omnis ergo, qui confitébitur me coram homínibus, confitébor et ego eum coram Patre meo, qui in cœlis est. Qui autem negáverit me coram homínibus, negábo et ego eum coram Patre meo, qui in cœlis est.
}\switchcolumn\portugues{
\blettrine{N}{aquele} tempo, disse Jesus aos seus discípulos: «Não tenhais medo daqueles que matam o corpo e não podem matar a alma; temei antes Aquele que pode condenar a alma e o corpo ao inferno. Porventura se não vendem dois pássaros por um ceitil? E nenhum deles, contudo, cairá no chão sem o consentimento do vosso Pai. Até os cabelos da vossa cabeça estão contados. Nada receeis; pois valeis mais do que muitos pássaros. Portanto, todo aquele que me confessar perante os homens, Eu também o confessarei na presença de meu Pai, que está nos céus».
}\end{paracol}

\paragraphinfo{Ofertório}{Ecl. 24, 44}
\begin{paracol}{2}\latim{
\rlettrine{D}{octrínam} quasi ante lucánum illúmino ómnibus, et enarrábo illam usque ad longínquum.
}\switchcolumn\portugues{
\rlettrine{A}{} doutrina que espalharei em todo o mundo será como a luz matutina que iluminará a todos: e transmiti-la-ei até ao decorrer dos séculos.
}\end{paracol}

\paragraph{Secreta}
\begin{paracol}{2}\latim{
\rlettrine{D}{eus,} qui credéntes in te pópulos nullis sinis cóncuti terróribus: dignáre preces et hóstias dicátæ tibi plebis suscípere; ut pax, a tua pietáte concéssa, christianórum fines ab omni hoste fáciat esse secúros. Per Dóminum nostrum \emph{\&c.}
}\switchcolumn\portugues{
\slettrine{Ó}{} Deus, que conservais a paz nos povos que acreditam em Vós, dignai-Vos aceitar as preces e as hóstias que os vossos fiéis Vos oferecem, para que a paz, que concedeis benignamente, deixe os países cristãos ao abrigo de todos os ataques do inimigo. Por nosso Senhor \emph{\&c.}
}\end{paracol}

\paragraphinfo{Comúnio}{Ecl. 24, 47}
\begin{paracol}{2}\latim{
\rlettrine{V}{idéte,} quóniam non soli mihi laborávi, sed ómnibus exquiréntibus veritátem.
}\switchcolumn\portugues{
\rlettrine{V}{ede} como não tenho trabalhado só para mim, mas para todos aqueles que procuram a verdade.
}\end{paracol}

\paragraph{Postcomúnio}
\begin{paracol}{2}\latim{
\rlettrine{D}{eus,} auctor pacis et amátor, quem nosse vívere, cui servíre regnáre est: prótege ab ómnibus impugnatiónibus súpplices tuos; ut, qui in defensióne tua confídimus, beáti Irenǽi Mártyris tui atque Pontíficis intercessióne, nullius hostilitátis arma timeámus. Per Dóminum \emph{\&c.}
}\switchcolumn\portugues{
\slettrine{Ó}{} Deus, que sois autor e amante da paz, aqueles que Vos conhecem vivem, e aqueles que Vos servem reinam: protegei, pois, contra todas as adversidades os vossos suplicantes, para que, havendo colocado toda a confiança no vosso socorro e na intercessão do B. Ireneu, vosso Mártir e Pontífice, nunca temamos as armas dos nossos inimigos. Por nosso Senhor \emph{\&c.}
}\end{paracol}
