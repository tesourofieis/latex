\subsectioninfo{S. Francisco de Paula, Conf.}{2 de Abril}

\textit{Como Missa Justus ut palma, página \pageref{confessoresnaopontifices2}, excepto:}

\paragraph{Oração}
\begin{paracol}{2}\latim{
\rlettrine{D}{eus,} humílium celsitúdo, qui beátum Francíscum Confessórem Sanctórum tuórum glória sublimásti: tríbue, quǽsumus; ut, ejus méritis et imitatióne, promíssa humílibus prǽmia felíciter consequámur. Per Dóminum \emph{\&c.}
}\switchcolumn\portugues{
\slettrine{Ó}{} Deus, grandeza dos humildes, que coroastes com a glória dos vossos Santos o B. Francisco, Confessor, concedei-nos, pelos seus méritos e pela imitação das suas virtudes, Vos suplicamos, a felicidade de conseguirmos as recompensas prometidas aos humildes. Por nosso Senhor \emph{\&c.}
}\end{paracol}

\paragraphinfo{Epístola}{Fl. 3, 7-12}
\begin{paracol}{2}\latim{
Léctio Epistola; beáti Pauli Apóstoli ad Philippénses.
}\switchcolumn\portugues{
Lição da Ep.ª do B. Ap.º Paulo aos Filipenses.
}\switchcolumn*\latim{
\rlettrine{F}{ratres:} Quæ mihi fuérunt lucra, hæc arbitrátus sum propter Christum detriménta. Verúmtamen exístimo ómnia detriméntum esse propter eminéntem sciéntiam Jesu Christi, Dómini mei: propter quem ómnia detriméntum feci et arbitror ut stércora, ut Christum lucrifáciam, et invéniar in illo, non habens meam justítiam, quæ ex lege est, sed illam, quæ ex fide est Christi Jesu: quæ ex Deo est justítia in fide, ad cognoscéndum illum, et virtútem resurrectiónis ejus, et societátem passiónum illíus: configurátus morti ejus: si quo modo occúrram ad resurrectiónem, quæ est ex mórtuis: non quod jam accéperim aut jam perféctus sim: sequor autem, si quo modo comprehéndam, inquo et comprehénsus sum a Christo Jesu.
}\switchcolumn\portugues{
\rlettrine{M}{eus} irmãos: As coisas que considerava ganho tive-as depois como perda, meditando em Cristo. E, na verdade, considero tudo como perda, pelo melhor conhecimento que tenho de Jesus Cristo, meu Senhor, por amor de quem renunciei a todas as coisas, considerando-as como poeira, a fim de que ganhe Cristo e me encontre com Ele, não com a minha própria justiça (a que vem da Lei) mas com aquela que nasce da fé em Jesus Cristo (a justiça que vem de Deus pela fé), para que o conheça a Ele, assim como ao mistério da sua Ressurreição e tome parte nos seus sofrimentos, havendo-me conformado com sua morte, a fim de que de algum modo possa conseguir a ressurreição dos mortos. Não que eu tenha alcançado o prémio ou me haja tornado já perfeito; mas prossigo no meu caminho para ver se alcanço o destino para que fui predestinado por Jesus Cristo.
}\end{paracol}

\paragraph{Secreta}
\begin{paracol}{2}\latim{
\rlettrine{H}{æc} dona devótæ plebis, Dómine, quibus tua cumulámus altária, beáti Francísci méritis tibi grata nobísque salutária, te miseránte, reddántur. Per Dóminum \emph{\&c.}
}\switchcolumn\portugues{
\qlettrine{Q}{ue} estes dons do vosso povo, que depositamos nos vossos altares, se tornem agradáveis a Vós, Senhor, e salutares para nós por intercessão dos méritos do B. Francisco, e por efeito da vossa misericórdia. Por nosso Senhor \emph{\&c.}
}\end{paracol}

\paragraphinfo{Comúnio}{Mt. 19, 28 \& 29}
\begin{paracol}{2}\latim{
\rlettrine{A}{men,} dico vobis: quod vos, qui reliquístis ómnia et secúti estis me, céntuplum accipiétis, et vitam ætérnam possidébitis. (T.P. Allelúja.)
}\switchcolumn\portugues{
\rlettrine{E}{m} verdade vos digo: «Vós, que abandonastes tudo e me seguistes, recebereis o cêntuplo e possuireis a vida eterna». (T.P. Aleluia.)
}\end{paracol}

\paragraph{Postcomúnio}
\begin{paracol}{2}\latim{
\rlettrine{S}{umpta,} Dómine, sacraménta cœléstia: beáto Francísco Confessóre tuo intercedénte, precámur; ut et temporális vitæ subsídia nobis cónferant et ætérnæ. Per Dóminum nostrum \emph{\&c.}
}\switchcolumn\portugues{
\rlettrine{S}{enhor,} Vos suplicamos, permiti que os celestiais sacramentos, que acabámos de receber, nos consigam, pela intercessão do B. Francisco, vosso Confessor, auxílios para a vida presente e para a eterna. Por nosso Senhor \emph{\&c.}
}\end{paracol}