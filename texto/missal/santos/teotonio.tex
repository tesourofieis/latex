\subsectioninfo{S. Teotónio, Conf.}{18 de Fevereiro}

\paragraphinfo{Intróito}{Sl. 131, 7 \& 121, 2}
\begin{paracol}{2}\latim{
\rlettrine{I}{ntroíbimus} in tabernáculum ejus: adorábimus in loco, ubi stetérunt pedes ejus. Stantes erant pedes nostri in átriis tuis, Jerusalém. \emph{Ps. 121, 1} Lætátus sum in his, quæ dicta sunt mihi: in domum Dómini íbimus.
℣. Gloria Patri \emph{\&c.}
}\switchcolumn\portugues{
\rlettrine{E}{ntraremos} no seu tabernáculo e adoraremos no lugar onde Ele pôs os seus pés. Tínhamos os nossos pés colocados nos teus átrios, ó Jerusalém. \emph{Sl. 121, 1} Senti muita alegria com o que me foi dito: Iremos à casa do Senhor.
℣. Glória ao Pai \emph{\&c.}
}\end{paracol}

\paragraph{Oração}
\begin{paracol}{2}\latim{
\rlettrine{D}{eus,} qui beáti Theotónii et exémplis canónicam disciplínam reparásti: concéde propítius; ut ejus exémplo et intercessióne, arctiórem christiánæ perfectiónis viam ingréssi, facílius vitam ætérnam consequámur. Per Dóminum \emph{\&c.}
}\switchcolumn\portugues{
\slettrine{Ó}{} Deus, que com os salutares conselhos e exemplos do B. Teotónio reformastes a disciplina canónica, concedei-nos propício que, com seu exemplo e intercessão, seguindo nós o apertado caminho da perfeição cristã, mais facilmente alcancemos a vida eterna. Por nosso Senhor \emph{\&c.}
}\end{paracol}

\paragraphinfo{Epístola}{Página \pageref{abades}}

\paragraphinfo{Gradual}{Sl. 118, 46}
\begin{paracol}{2}\latim{
\rlettrine{L}{oquébar} de testimóniis tuis in conspéctu regum: et non confundébar. ℣. \emph{Ps. 19, 10} Dómine, salvum fac regem et exáudi nos in die, qua invocavérimus te.
}\switchcolumn\portugues{
\rlettrine{F}{alava} dos vossos testemunhos na presença dos reis e não era confundido. ℣. \emph{Sl. 19, 10} Senhor, salvai o rei e ouvi-nos no dia em que Vos invocarmos.
}\switchcolumn*\latim{
Allelúja, allelúja. ℣. \emph{Ps. 115, 18-19} Vota mea Dómino redam in átriis domus Dómini, in médio tui Jerusalém. Allelúja.
}\switchcolumn\portugues{
Aleluia, aleluia. ℣. \emph{Sl. 115, 18-19} Apresentarei os meus votos ao Senhor, nos átrios da casa do Senhor e junto de Vós, ó Jerusalém. Aleluia.
}\end{paracol}

% \textit{Após a Septuagésima omite-se o Aleluia e o seguinte, e diz-se o:}

% \paragraphinfo{Trato}{}
% \begin{paracol}{2}\latim{
% \rlettrine{}{}
% }\switchcolumn\portugues{
% \rlettrine{C}{umprirei} os votos que fiz ao Senhor na presença de todo seu povo: aos olhos do Senhor é preciosa a morte dos seus Santos. ℣. Porquanto considerei que fundastes os céus, a lua e as estrelas. Gloriosas cousas têm sido narradas a respeito de vós, ó cidade de Deus.
% }\end{paracol}

\paragraphinfo{Evangelho}{Página \pageref{abades}}

\paragraphinfo{Ofertório}{Gl. 6-14}
\begin{paracol}{2}\latim{
\rlettrine{M}{ihi} autem absit gloriári, nisi in Cruce Dómini nostri Jesu Christi.
}\switchcolumn\portugues{
\rlettrine{L}{onge} esteja de mim gloriar-me nalguma cousa senão na Cruz de nosso Senhor Jesus Cristo.
}\end{paracol}

\paragraph{Secreta}
\begin{paracol}{2}\latim{
\rlettrine{S}{anctæ} Crucis, Dómine, mystéria recoléntes, concéde, ut sacrifícium incruéntum débita veneratióne offerámus: et interveniénte sancto Theotónnio Confessóre tuo, salutáris fructus consequámur efféctum. Per Dóminum \emph{\&c.}
}\switchcolumn\portugues{
\rlettrine{R}{enovando} nós, Senhor, os mystérios da Santa Cruz, concedei-nos que ofereçamos o sacrifício incruento com a devida veneração; e que por intercessão de S. Teotónio, vosso Confessor, alcancemos fruto salutar. Por nosso Senhor \emph{\&c.}
}\end{paracol}

\paragraphinfo{Comúnio}{Rm. 12, 1-2}
\begin{paracol}{2}\latim{
\rlettrine{O}{bsecro} vos, frates, per misericórdiam Dei, ut exhibeátis córpora vestra hóstiam vivéntem, sanctam, Deo placéntem, rationábile obséquium vestrum: et nolíte conformári huic sæculo, sed reformámini.
}\switchcolumn\portugues{
\rlettrine{R}{ogo-vos,} irmãos, que apresenteis vossos corpos em sacrifício vivo, santo e agradável a Deus, pois esta vossa doação é racional. E que vos não conformeis com este mundo, mas antes vos reformeis.
}\end{paracol}

\paragraph{Postcomúnio}
\begin{paracol}{2}\latim{
\rlettrine{P}{er} sancta, Dómine, quæ súmpsimus, sanctus Theotónius intercédat, ne huic sæculo conformémur; sed reformáti, ad perfectiónem semper aspirémus. Per Dóminum \emph{\&c.}
}\switchcolumn\portugues{
\rlettrine{P}{elos} sacrossantos sacramentos que recebemos, Senhor, interceda por nós S. Teotónio, para que nos não conformemos com este mundo, mas, já reformados, aspiremos à perfeição. Por nosso Senhor \emph{\&c.}
}\end{paracol}
