\subsectioninfo{Os Quarenta Mártires}{10 de Março}\label{quarentamartires}

\paragraphinfo{Intróito}{Sl. 33, 18}
\begin{paracol}{2}\latim{
\rlettrine{C}{lamavérunt} justi, et Dóminus exaudívit eos: et ex ómnibus tribulatiónibus eórum liberávit eos. \emph{Ps. ibid., 2} Benedícam Dóminum in omni témpore: semper laus ejus in ore meo.
℣. Gloria Patri \emph{\&c.}
}\switchcolumn\portugues{
\rlettrine{O}{s} justos clamaram por Deus; então o Senhor ouviu-os e livrou-os de todas as tribulações. \emph{Sl. ibid., 2} Bendirei o Senhor em todo o tempo; o seu louvor estará sempre na minha boca.
℣. Glória ao Pai \emph{\&c.}
}\end{paracol}

\paragraph{Oração}
\begin{paracol}{2}\latim{
\rlettrine{P}{ræsta,} quǽsumus, omnípotens Deus: ut, qui gloriósos Mártyres fortes in sua confessióne cognóvimus, pios apud te in nostra intercessióne sentiámus. Per Dóminum nostrum \emph{\&c.}
}\switchcolumn\portugues{
\rlettrine{C}{oncedei-nos,} Vos pedimos, ó Deus omnipotente, que, reconhecendo nós a fortaleza com que estes Santos Mártires confessaram a fé, sintamos também em nosso favor a caridade da sua intercessão junto de Vós. Por nosso Senhor \emph{\&c.}
}\end{paracol}

\paragraphinfo{Epístola}{Página \pageref{fabiaosebastiao}}

\paragraphinfo{Gradual}{Sl. 132, 1-2}
\begin{paracol}{2}\latim{
\rlettrine{E}{cce,} quam bonum et quam jucundum, habitáre fratres in unum! ℣. Sicut unguéntum in cápite, quod descéndit in barbam, barbam Aaron.
}\switchcolumn\portugues{
\rlettrine{O}{h!} como é bom e suave que os irmãos habitem juntos! ℣. É como o perfume suave espalhado na cabeça de Aarão, e que corre pela barba: pela barba de Aarão.
}\end{paracol}

\paragraphinfo{Trato}{Página \pageref{muitosmartires1}}

\paragraphinfo{Evangelho}{Página \pageref{muitosmartires1}}

\paragraphinfo{Ofertório}{Sl. 31, 11}
\begin{paracol}{2}\latim{
\rlettrine{L}{ætámini} in Dómino et exsultáte, justi: et gloriámini, omnes recti corde.
}\switchcolumn\portugues{
\rlettrine{A}{legrai-vos} no Senhor, ó justos! Exultai de júbilo! Todos aqueles que possuem o coração recto serão glorificados.
}\end{paracol}

\paragraph{Secreta}
\begin{paracol}{2}\latim{
\rlettrine{P}{reces,} Dómine, tuórum réspice oblationésque fidélium: ut et tibi gratæ sint pro tuórum festivitáte Sanctórum, et nobis cónferant tuæ propitiatiónis auxílium. Per Dóminum \emph{\&c.}
}\switchcolumn\portugues{
\rlettrine{D}{ignai-Vos} olhar benigno, Senhor, para as preces e ofertas dos vossos fiéis; e fazei que na festa dos vossos Santos Vos sejam agradáveis e nos alcancem o socorro da vossa misericórdia. Por nosso Senhor \emph{\&c.}
}\end{paracol}

\paragraphinfo{Comúnio}{Mt. 12, 50}
\begin{paracol}{2}\latim{
\qlettrine{Q}{uicúmque} fécerit voluntátem Patris mei, qui in cœlis est: ipse meus frater et soror et mater est, dicit Dóminus.
}\switchcolumn\portugues{
\rlettrine{A}{quele} que faz a vontade de meu Pai, que está nos céus, é para mim, meu irmão, minha irmã e minha mãe: diz o Senhor.
}\end{paracol}

\paragraph{Postcomúnio}
\begin{paracol}{2}\latim{
\rlettrine{S}{anctórum} tuórum, Dómine, intercessióne placátus: præsta, quǽsumus; ut, quæ temporáli celebrámus actióne, perpétua salvatióne capiámus. Per Dóminum \emph{\&c.}
}\switchcolumn\portugues{
\rlettrine{A}{placado,} Senhor, com a intercessão dos vossos Santos, permiti, Vos imploramos, que alcancemos a salvação eterna com a celebração desta acção temporal. Por nosso Senhor \emph{\&c.}
}\end{paracol}
