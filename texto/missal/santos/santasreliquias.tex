\subsectioninfo{Sagradas Relíquias}{5 de Novembro}

\paragraphinfo{Intróito}{Sl. 33, 20-21}
\begin{paracol}{2}\latim{
\rlettrine{M}{ultæ} tribulatiónes justórum, et de his ómnibus liberávit eos Dóminus: Dóminus custódit ómnia ossa eórum, unum ex his non conterétur. \emph{Ps. ibid., 2} Benedícam Dóminum in omni témpore: semper laus ejus in ore meo.
℣. Gloria Patri \emph{\&c.}
}\switchcolumn\portugues{
\rlettrine{M}{uitas} são as tribulações dos justos, mas de todas elas o Senhor os livrará. O Senhor guarda todos seus ossos e nem um só deles será quebrado. \emph{Sl. ibid., 2} Bendirei o Senhor em todo o tempo: os seus louvores estarão sempre na minha boca.
℣. Glória ao Pai \emph{\&c.}
}\end{paracol}

\paragraph{Oração}
\begin{paracol}{2}\latim{
\rlettrine{A}{uge} in nobis, Dómine, ressurrectiónis fidem, qui in Sanctórum tuórum mirabília operáris: et fac nos immortális glóriæ partícipes, cujus in eórum cinéribus pígnora venerámur. Per Dóminum \emph{\&c.}
}\switchcolumn\portugues{
\qlettrine{Q}{ue} a nossa fé na ressurreição cresça, Senhor, à vista das maravilhas que operais pelas Relíquias dos vossos Santos; e concedei-nos que participemos da glória imortal, de que são penhor estas cinzas, que veneramos. Por nosso Senhor \emph{\&c.}
}\end{paracol}

\paragraphinfo{Epístola}{Ecl. 44, 10-14}
\begin{paracol}{2}\latim{
Léctio libri Sapiéntiæ.
}\switchcolumn\portugues{
Lição do Livro da Sabedoria.
}\switchcolumn*\latim{
\rlettrine{H}{i} viri misericórdiæ sunt, quorum pietátis non defuérunt: cum sémine eórum pérmanent bona, heréditas sancta nepótes eórum, et in testaméntis stetit semen eórum: et fílii eórum propter illos usque in ætérnum manent: semen eórum et glória eórum non derelinquétur. Córpora ipsórum in pace sepúlta sunt, et nomen eórum vivit in generatioónem. Sapiéntiam ipsórum narrent pópuli, et laudem eórum núntiet Ecclésia.
}\switchcolumn\portugues{
\rlettrine{E}{stes} são homens misericordiosos, cujas virtudes não foram olvidadas. A felicidade transmitiu-se à sua descendência e a sua herança foi assegurada aos netos, mantendo-se sua descendência fiel à aliança com Deus, assim como também os seus filhos. Sua geração permanecerá eternamente e a sua glória nunca desaparecerá. Seus corpos foram sepultados em paz e o seu nome viverá de geração em geração. Que os povos, pois, publiquem a sua sabedoria e que a Igreja cante os seus louvores!
}\end{paracol}

\paragraphinfo{Gradual}{Sl. 149, 5 \& 1}
\begin{paracol}{2}\latim{
\rlettrine{E}{xultábunt} Sancti in glória: lætabúntur in cubílibus suis. ℣. Cantáte Dómino cánticum novum: laus ejus in Ecclésia Sanctórum.
}\switchcolumn\portugues{
\rlettrine{O}{s} Santos exultarão na glória: alegrar-se-ão na sua mansão. ℣. Cantai ao Senhor um cântico novo: que seus louvores permaneçam na assembleia dos Santos.
}\switchcolumn*\latim{
Allelúja, allelúja. ℣. \emph{Ps. 67} Justi epuléntur, et exsúltent in conspéctu Dei: et delecténtur in lætítia. Allelúja
}\switchcolumn\portugues{
Aleluia, aleluia. ℣. \emph{Sl. 67} Inebriem-se os justos e exultem de alegria na presença de Deus! Que se deliciem na alegria. Aleluia.
}\end{paracol}

\paragraphinfo{Evangelho}{Página \pageref{muitosmartires2}}

\paragraphinfo{Ofertório}{Sl. 67, 36}
\begin{paracol}{2}\latim{
\rlettrine{M}{irábilis} Deus in Sanctis suis: Deus Israël ipse dabit virtútem et fortitúdinem plebi suæ: benedíctus Deus, allelúja.
}\switchcolumn\portugues{
\rlettrine{D}{eus} é admirável em seus Santos: Deus de Israel dará ao seu povo a força e a coragem: Bendito seja Deus. Aleluia.
}\end{paracol}

\paragraph{Secreta}
\begin{paracol}{2}\latim{
\rlettrine{I}{mplorámus,} Dómine, cleméntiam tuam: ut Sanctórum tuórum, quorum relíquias venerámur, suffragántibus méritis, hóstia, quam offérimus, nostrórum sit expiátio delictórum. Per Dóminum \emph{\&c.}
}\switchcolumn\portugues{
\rlettrine{I}{mploramos,} Senhor, a vossa clemência, a fim de que, pelos méritos dos vossos Santos, dos quais veneramos as Relíquias, a hóstia, que oferecemos, sirva de expiação dos nossos delitos. Por nosso Senhor \emph{\&c.}
}\end{paracol}

\paragraphinfo{Comúnio}{Sl. 32, 1}
\begin{paracol}{2}\latim{
\rlettrine{G}{audéte} justi in Dómino: rectos decet collaudátio.
}\switchcolumn\portugues{
\rlettrine{A}{legrai-Vos} no Senhor, ó justos: aos que são rectos é que pertence cantar os vossos louvores.
}\end{paracol}

\paragraph{Postcomúnio}
\begin{paracol}{2}\latim{
\rlettrine{M}{ultíplica} super nos, quæsumus, Dómine, per hæc sancta, quæ súmpsimus, misericórdiam tuam: ut sicut in tuórum solemnitáte Sanctórum, quorum relíquias cólimus, pia devotióne lætámur; ita eórum perpétua societáte, te largiénte, fruámur. Per Dóminum \emph{\&c.}
}\switchcolumn\portugues{
\qlettrine{Q}{ue} os sacrossantos mistérios, que recebemos, Senhor, Vos rogamos, multipliquem em nós a vossa misericórdia, a fim de que, assim como nos alegramos com pia devoção nesta solenidade dos vossos Santos, cujas Relíquias veneramos, assim também pela vossa magnanimidade gozemos a sua perpétua companhia. Por nosso Senhor \emph{\&c.}
}\end{paracol}
