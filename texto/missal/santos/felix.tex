\subsectioninfo{S. Félix, Presb. e Márt.}{14 de Janeiro}

\textit{Como na Missa Lætábitur justus, página \pageref{martirnaopontifice2}, excepto:}

\paragraph{Oração}
\begin{paracol}{2}\latim{
\rlettrine{C}{oncéde,} quǽsumus, omnípotens Deus: ut ad meliórem vitam Sanctórum tuórum exémpla nos próvocent; quaténus, quorum sollémnia ágimus, étiam actus imitémur. Per Dóminum \emph{\&c.}
}\switchcolumn\portugues{
\rlettrine{C}{oncedei-nos,} ó omnipotente Deus, que os exemplos dos vossos Santos nos incitem a uma vida melhor, de que modo que imitemos tambe´m as acções daquele cuja solenidade celebramos. Por nosso Senhor \emph{\&c.}
}\end{paracol}

\paragraph{Secreta}
\begin{paracol}{2}\latim{
\rlettrine{H}{óstias} tibi, Dómine, beáti Félicis Mártyris tui dicátas méritis, benígnus assúme: et ad perpétuum nobis tríbue proveníre subsídium. Per Dóminum \emph{\&c.}
}\switchcolumn\portugues{
\rlettrine{A}{ceitai} benignamente, Senhor, as hóstias que Vos oferecemos em honra dos méritos do B. Félix, vosso Mártir, e permiti que nos alcancem o vosso perpétuo auxílio. Por nosso Senhor \emph{\&c.}
}\end{paracol}

\paragraph{Postcomúnio}
\begin{paracol}{2}\latim{
\qlettrine{Q}{uǽsumus,} Dómine, salutáribus repléti mystériis: ut, beáti Félicis Mártyris tui, cujus sollémnia celebrámus, oratiónibus adjuvémur. Per Dóminum \emph{\&c.}
}\switchcolumn\portugues{
\rlettrine{H}{avendo} sido saciados com os salutares dons, dignai-Vos conceder-nos, Senhor, que sejamos auxiliados pelas orações do vosso B. Mártir Félix, cuja solenidade celebrámos. Por nosso Senhor \emph{\&c.}
}\end{paracol}
