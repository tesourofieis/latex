\subsectioninfo{S. Inácio, B. e Márt.}{1 de Fevereiro}

\paragraphinfo{Intróito}{Gl. 6, 14}
\begin{paracol}{2}\latim{
\rlettrine{M}{ihi} autem absit gloriári, nisi in Cruce Dómini nostri Jesu Christi: per quem mihi mundus crucifíxus est, et ego mundo. \emph{Ps. 131, 1} Meménto, Dómine, David: et omnis mansuetúdinis ejus.
℣. Gloria Patri \emph{\&c.}
}\switchcolumn\portugues{
\rlettrine{L}{onge} esteja de mim gloriar-me nalguma coisa senão na Cruz de N. S, Jesus Cristo, pela qual o mundo foi crucificado para mim e eu para o mundo. \emph{Sl. 131, 1} Lembrai-Vos de David, Senhor, assim como da sua mansidão.
℣. Glória ao Pai \emph{\&c.}
}\end{paracol}

\paragraphinfo{Epístola}{Rm. 8, 35-39}
\begin{paracol}{2}\latim{
Léctio Epistolæ beáti Pauli Apóstoli ad Romános.
}\switchcolumn\portugues{
Lição da Ep.ª do B. Ap.º Paulo aos Romanos.
}\switchcolumn*\latim{
\rlettrine{F}{ratres:} Quis nos separábit a cantáte Christi: tribulátio, an angustia, an fames, an núditas, an perículum, an persecútio, an gládius? (sicut scriptum est: Quia propter te mortificámur tota die: æstimáti sumus sicut oves occisiónis). Sed in his ómnibus superámus propter eum, qui diléxit nos. Certus sum enim, quia neque mors, neque vita, neque ángeli, neque principátus, neque virtútes, neque instántia, neque futúra, neque fortitúdo, neque altitúdo, neque profúndum, neque creatúra alia poterit nos separáre a cantáte Dei, quæ est in Christo Jesu, Dómino nostro.
}\switchcolumn\portugues{
\qlettrine{Q}{uem} nos separará do amor de Cristo? A tribulação, a angústia, a fome, a nudez, os perigos, a perseguição, a espada? Segundo o que está escrito: «Por ti somos entregues à morte todos os dias; somos reputados como ovelhas conduzidas à morte». Mas de todas estas coisas saímos vencedores por Aquele que nos amou. Estou certo de que nem a morte, nem a vida, nem os anjos, nem os príncipes, nem as virtudes, nem as coisas presentes, nem as futuras, nem a força, nem a altura, nem a profundidade, nem nenhuma outra criatura poderá separar-nos do amor de Deus, que está em nosso Senhor Jesus Cristo.
}\end{paracol}

\paragraphinfo{Gradual}{Ecl. 44, 16}
\begin{paracol}{2}\latim{
\rlettrine{E}{cce} sacérdos magnus, qui m diébus suis plácuit Deo. ℣. \emph{Ibid , 20} Non est invéntus símilis illi, qui conserváret legem Excélsi.
}\switchcolumn\portugues{
\rlettrine{E}{is} o grande sacerdote que durante a vida agradou a Deus. ℣. \emph{Ibid , 20} Não houve quem, como ele, guardasse a observância da Lei do Altíssimo.
}\switchcolumn*\latim{
Allelúja, allelúja. ℣. \emph{Gal 2, 19-20} Christo confíxus sum Cruci: vivo ego, jam non ego, vivit vero in me Christus. Allelúja.
}\switchcolumn\portugues{
Aleluia, aleluia. ℣. \emph{Gl 2, 19-20} Estou crucificado com Cristo. Vivo, mas já não sou eu que vivo: é Cristo que vive em mim. Aleluia.
}\end{paracol}

\textit{Após a Septuagésima omite-se o Aleluia e o Verso, e diz-se:}

\paragraphinfo{Trato}{Sl. 20, 3-4}
\begin{paracol}{2}\latim{
\rlettrine{D}{esidérium} ánimæ ejus tribuísti ei: et voluntáte labiórum ejus non fraudásti eum. ℣. Quoniam prævenísti eum in benedictiónibus dulcedinis. ℣. Posuísti in cápite ejus coronam de lápide pretióso.
}\switchcolumn\portugues{
\rlettrine{S}{atisfizestes} os desejos do seu coração e o não iludistes nas preces proferidas por seus lábios. ℣. Vós, Senhor, o enchestes com bênçãos faustosas. ℣. Colocastes na sua cabeça o diadema de ouro.
}\end{paracol}

\paragraphinfo{Ofertório}{Sl. 8, 6-7}
\begin{paracol}{2}\latim{
\rlettrine{G}{lória} et honóre coronásti eum: et constituísti eum super ópera mánuum tuárum, Domine.
}\switchcolumn\portugues{
\rlettrine{V}{ós} o coroastes, Senhor, com glória e honras: Vós lhe concedestes domínio sobre as obras das vossas mãos.
}\end{paracol}

\paragraph{Comúnio}{Palavras de S. Inácio}
\begin{paracol}{2}\latim{
\rlettrine{F}{ruméntum} Christi sum: déntibus bestiárum molar, ut panis mundus invéniar.
}\switchcolumn\portugues{
\rlettrine{S}{ou} o trigo de Cristo, que hei-de ser moído nos dentes das feras para me tornar em pão puríssimo.
}\end{paracol}
