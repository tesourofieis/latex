\subsectioninfo{S. Filipe e S. Tiago, Apóstolos}{11 de Maio}\label{filipetiagoapostolos}

\paragraphinfo{Intróito}{Esd. 9, 27}
\begin{paracol}{2}\latim{
\rlettrine{C}{lamavérunt} ad te, Dómine, in témpore afflictiónis suæ, et tu de cœlo exaudísti eos, allelúja, allelúja. \emph{Ps. 32, 1} Exsultáte, justi, in Dómino: rectos decet collaudátio.
℣. Gloria Patri \emph{\&c.}
}\switchcolumn\portugues{
\rlettrine{N}{o} tempo da aflição clamaram por Vós, Senhor; e Vós, do alto do céu, os ouvistes, aleluia, aleluia. \emph{Sl. 32, 1} Alegrai-vos no Senhor, ó justos, pois os que possuem os corações rectos é que devem louvar o Senhor.
℣. Glória ao Pai \emph{\&c.}
}\end{paracol}

\paragraph{Oração}
\begin{paracol}{2}\latim{
\rlettrine{D}{eus,} qui nos ánnua Apostolórum tuórum Philíppi et Jacóbi sollemnitáte lætíficas: præsta, quǽsumus: ut, quorum gaudémus méritis, instruámur exémplis. Per Dóminum \emph{\&c.}
}\switchcolumn\portugues{
\slettrine{Ó}{} Deus, que nos alegrais com a solenidade anual dos vossos Apóstolos Filipe e Tiago, concedei-nos, Vos suplicamos, que, alegrando-nos com seus méritos, aproveitemos com seus exemplos. Por nosso Senhor \emph{\&c.}
}\end{paracol}

\paragraphinfo{Epístola}{Sb. 5, 1-5.}
\begin{paracol}{2}\latim{
Lectio Epístolæ beati Pauli Apostoli ad Corinthios.
}\switchcolumn\portugues{
Lição do Livro da Sabedoria.
}\switchcolumn*\latim{
\rlettrine{S}{tabunt} justi in magna constántia advérsus eos, qui se angustiavérunt et qui abstulérunt labóres eórum. Vidéntes turbabúntur timore horríbili, et mirabúntur in subitatióne insperátæ salútis, dicéntes intra se, pœniténtiam agéntes, et præ angústia spíritus geméntes: Hi sunt, quos habúimus aliquándo in derísum et in similitúdinem impropérii. Nos insensáti vitam illórum æstimabámus insániam, et finem illórum sine honóre: ecce, quómodo computáti sunt inter fílios Dei, et inter Sanctos sors illórum est.
}\switchcolumn\portugues{
\rlettrine{E}{ntão,} os justos erguer-se-ão com grande coragem contra aqueles que os oprimiam e a quem arrebatavam o fruto dos seus trabalhos. Vendo-os assim, os maus perturbar-se-ão, cheios de pavor, e ficarão assombrados com a súbita e inesperada salvação dos justos, dizendo de si para si, arrependidos e angustiados: Estes são aqueles a quem outrora quisemos injuriar com nossas zombarias e insultos. Insensatos que nós fomos! Pareceu-nos que sua vida era uma loucura, e a sua morte uma vergonha; mas eis que os vemos elevados à dignidade de filhos de Deus e compartilhando da glória dos santos!
}\end{paracol}

\begin{paracol}{2}\latim{
Allelúja, allelúja. ℣. \emph{Ps. 88, 6} Confitebúntur cœli mirabília tua, Dómine: etenim veritátem tuam in ecclésia sanctórum. Allelúja. ℣. \emph{Joann. 14, 9} Tanto témpore vobíscum sum, et non cognovístis me? Philíppe, qui videt me, videt et Patrem meum. Allelúja.
}\switchcolumn\portugues{
Aleluia, aleluia. ℣. \emph{Sl. 88, 6} Os céus cantarão as vossas maravilhas, Senhor, assim como a vossa verdade, na assembleia dos Santos. Aleluia. ℣. \emph{Jo. 14, 9} Há tanto tempo que estou convosco e me não conheceis? Filipe, quem me vê, vê também meu Pai! Aleluia.
}\end{paracol}

\paragraphinfo{Evangelho}{Jo. 14, 1-13}
\begin{paracol}{2}\latim{
\cruz Sequéntia sancti Evangélii secúndum Joánnem.
}\switchcolumn\portugues{
\cruz Continuação do santo Evangelho segundo S. João.
}\switchcolumn*\latim{
\blettrine{I}{n} illo témpore: Dixit Jesus discípulis suis: Non turbátur cor vestrum. Creditis in Deum, et in me crédite. In domo Patris mei mansiónes multæ sunt. Si quo minus, dixíssem vobis: Quia vado paráre vobis locum. Et si abíero et præparávero vobis locum: íterum vénio et accípiam vos ad meípsum, ut, ubi sum ego, et vos sitis. Et quo ego vado, scitis, et viam scitis. Dicit ei Thomas: Dómine, nescímus, quo vadis: et quómodo póssumus viam scire? Dicit ei Jesus: Ego sum via et véritas et vita; nemo venit ad Patrem nisi per me. Si cognovissétis me, et Patrem meum útique cognovissétis: et ámodo cognoscátis eum, et vidístis eum. Dicit ei Philíppus: Dómine, osténde nobis Patrem, et sufficit nobis. Dicit ei Jesus: Tanto témpore vobíscum sum, et non cognovístis me? Philíppe, qui videt me, videt et Patrem. Quómodo tu dicis: Osténde nobis Patrem? Non créditis, quia ego in Patre, et Pater in me est? Verba, quæ ego loquor vobis, a meípso non loquor. Pater autem in me manens, ipse facit ópera. Non créditis, quia ego in Patre, et Pater in me est? Alióquin propter ópera ipsa crédite. Amen, amen, dico vobis, qui credit in me, ópera, quæ ego facio, et ipse fáciet, et majóra horum fáciet: quia ego ad Patrem vado. Et quodcúmque petiéritis Patrem in nómine meo, hoc fáciam.
}\switchcolumn\portugues{
\blettrine{N}{aquele} tempo disse Jesus aos discípulos: «Que o vosso coração se não perturbe. Credes em Deus? Crede também em mim. Na casa de meu Pai há muitas moradas. Se não fora assim, já vo-lo tinha dito, pois vou preparar-vos lá um lugar. E, depois de haver ido e de vos preparar o lugar, voltarei outra vez e tomar-vos-ei comigo, a fim de que, onde estiver, estejais vós, também. Sabeis bem para onde vou; assim como conheceis o caminho». Então disse-Lhe Tomé: «Senhor, ignoramos para onde ides. Como podemos, pois, conhecer o caminho?». Respondeu-lhe Jesus: «Eu sou o caminho, a verdade e a vida; ninguém vem ao meu Pai senão por mim! Se me conhecêsseis, certamente conheceríeis, também, meu Pai. Porém vós O conhecereis; até já O haveis visto». Disse-Lhe Filipe: «Mostrai-nos o Pai, Senhor, e isso nos basta». Respondeu-lhe Jesus: «Há tanto tempo que estou convosco e me não conheceis? Filipe, quem me vê, vê também meu Pai! Como dizeis, pois, mostrai-nos o Pai? Não acreditais que estou no Pai, e o Pai está em mim? As palavras que vos digo as não digo de mim próprio; porém o Pai, que está em mim, Ele é que pratica as obras. Não acreditais que Eu estou no Pai, e o Pai está em mim? Ao menos acreditai por causa destas obras. Em verdade, em verdade vos digo: aquele que acredita em mim fará as mesmas obras que faço e fará ainda maiores, porque vou para o Pai. Tudo quanto pedirdes ao Pai em meu nome, vo-lo farei».
}\end{paracol}

\paragraphinfo{Ofertório}{Sl. 88, 6}
\begin{paracol}{2}\latim{
\rlettrine{C}{onfitebúntur} cœli mirabília tua, Dómine: et veritátem tuam in ecclésia sanctórum, allelúja, allelúja.
}\switchcolumn\portugues{
\rlettrine{S}{enhor,} que os céus publiquem as vossas maravilhas; que a vossa verdade seja exaltada na assembleia dos santos. Aleluia.
}\end{paracol}

\paragraph{Secreta}
\begin{paracol}{2}\latim{
\rlettrine{M}{únera,} Dómine, quæ pro Apostolórum tuórum Philippi et Jacóbi sollemnitáte deférimus, propítius súscipe: et mala ómnia, quæ meréraur, avérte. Per Dóminum \emph{\&c.}
}\switchcolumn\portugues{
\rlettrine{R}{ecebei} propício, Senhor, os dons que Vos apresentamos na solenidade dos vossos Apóstolos Filipe e Tiago e afastai de nós todos os males que merecemos. Por nosso Senhor \emph{\&c.}
}\end{paracol}

\paragraphinfo{Comúnio}{Jo. 14, 9 \& 10}
\begin{paracol}{2}\latim{
\rlettrine{T}{anto} témpore vobíscum sum, et non cognovístis me? Philíppe, qui videt me, videt et Patrem meum, allelúja: non credis, quia ego in Patre, et Pater in me est? Allelúja, allelúja.
}\switchcolumn\portugues{
\rlettrine{H}{á} tanto tempo que estou convosco e me não conheceis? Filipe, quem me vê, vê também meu Pai! Aleluia. Não acreditais que estou no Pai, e o Pai está em mim? Aleluia, aleluia.
}\end{paracol}

\paragraph{Postcomúnio}
\begin{paracol}{2}\latim{
\qlettrine{Q}{uǽsumus,} Dómine, salutáribus repléti mystériis: ut, quorum sollémnia celebrámus, eórum oratiónibus adjuvémur. Per Dóminum \emph{\&c.}
}\switchcolumn\portugues{
\rlettrine{S}{aciados} já com estes salutares mystérios, Senhor, Vos suplicamos, permiti que sejamos socorridos com as orações daqueles cuja festa celebramos. Por nosso Senhor \emph{\&c.}
}\end{paracol}
