\subsectioninfo{B. V. Maria, Auxílio dos Cristãos}{24 de Maio}

\textit{Como na Missa Comum das Festas da B. V. M., página \pageref{comumfestasmaria1}, excepto:}

\paragraph{Oração}
\begin{paracol}{2}\latim{
\rlettrine{O}{mnípotens} et miséricors Deus, qui ad defensiónem pópuli christiáni in beatíssima Vírgine María perpétuum auxílium mirabíliter constituísti: concéde propítius; ut, tali præsídio muníti certántes in vita, victóriam de hoste malígno cónsequi valeámus in morte. Per Dóminum \emph{\&c.}
}\switchcolumn\portugues{
\rlettrine{O}{mnipotente} e misericordioso Deus, que para defesa do povo cristão maravilhosamente instituístes a B. V. Maria como sua auxiliadora perpétua, concedei-nos propício que, depois de havermos sido munidos nos combates da vida com uma tão poderosa protecção, mereçamos também alcançar à hora da morte a vitória contra o inimigo maligno. Por nosso Senhor \emph{\&c.}
}\end{paracol}

\paragraph{Secreta}
\begin{paracol}{2}\latim{
\rlettrine{P}{ro} religiónis christánæ triúmpho hóstias placatiónis tibi, Dómine, immolámus: quæ ut nobis profíciant, opem auxiliátrix Virgo præstet; per quam talis perfécta est victória. Per Dóminum \emph{\&c.}
}\switchcolumn\portugues{
\rlettrine{V}{os} oferecemos vítimas de propiciação pelo triunfo da religião cristã, e que elas, Senhor, pela intercessão da Virgem Auxiliadora, pela qual foi assegurada a vitória perfeita, revertam em nosso proveito. Por nosso Senhor \emph{\&c.}
}\end{paracol}

\paragraph{Postcomúnio}
\begin{paracol}{2}\latim{
\rlettrine{A}{désto,} Dómine, pópulis, qui participatióne Córporis et Sánguinis tui reficiúntur: ut, sanctíssima tua Genitríce auxiliánte, ab omni malo et perículo liberéntur, et in omni ópere bono custodiántur: Qui vivis et regnas \emph{\&c.}
}\switchcolumn\portugues{
\rlettrine{A}{colhei} benigno, Senhor, os povos que se alimentam com vosso Corpo e Sangue, a fim de que com o auxílio da vossa Santíssima Mãe sejam livres de todo o mal e de todo o perigo, e perseverem na prática de todas as boas obras. Ó Vós, que viveis e reinais \emph{\&c.}
}\end{paracol}
