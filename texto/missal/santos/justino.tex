\subsectioninfo{S. Justino, Mártir}{14 de Abril}

\paragraphinfo{Intróito}{Sl. 118, 85 \& 46}
\begin{paracol}{2}\latim{
\rlettrine{N}{arravérunt} mihi iníqui fabulatiónes, sed non ut lex tua: ego autem loquébar de testimóniis tuis in conspéctu regum, et non confundébar. (T.P. Allelúja, allelúja.) \emph{Ps. ibid., 1} Beáti immaculáti in via, qui ámbulant in lege Dómini.
℣. Gloria Patri \emph{\&c.}
}\switchcolumn\portugues{
\rlettrine{O}{s} maus narraram-me fábulas, mas não se assemelhavam à vossa lei. Porém, publiquei os vossos ensinos na presença dos reis e não fui confundido. (T.P. Aleluia, aleluia.) \emph{Sl. ibid., 1} Bem-aventurados aqueles que são imaculados na sua vida: e que caminham segundo a lei do Senhor.
℣. Glória ao Pai \emph{\&c.}
}\end{paracol}

\paragraph{Oração}
\begin{paracol}{2}\latim{
\rlettrine{D}{eus,} qui per stultítiam Crucis eminéntem Jesu Christi sciéntiam beátum Justínum Mártyrem mirabíliter docuísti: ejus nobis intercessióne concéde; ut, errórum circumventióne depúlsa, fídei firmitátem consequámur. Per eúndem Dóminum \emph{\&c.}
}\switchcolumn\portugues{
\slettrine{Ó}{} Deus, que pela «loucura da cruz» instruístes admiravelmente o B. Mártir Justino com a eminente ciência de Jesus Cristo, concedei-nos por sua intercessão que, repelindo todos os erros que nos cercam, consigamos possuir uma fé firme. Pelo mesmo nosso Senhor \emph{\&c.}
}\end{paracol}

\paragraphinfo{Oração}{S. Tibúrcio, Mártires}
\begin{paracol}{2}\latim{
\rlettrine{P}{ræsta,} quǽsumus, omnípotens Deus: ut, qui sanctórum Mártyrum tuórum Tibúrtii, Valeriáni et Máximi sollémnia cólimus; eórum étiam virtútes imitémur. Per Dóminum nostrum \emph{\&c.}
}\switchcolumn\portugues{
\qlettrine{J}{á} que celebramos, ó Deus omnipotente, a festa dos vossos Santos Mártires Tibúrcio, Valeriano e Máximo, concedei-nos, Vos suplicamos, que imitemos também as suas virtudes. Por nosso Senhor \emph{\&c.}
}\end{paracol}

\paragraphinfo{Epístola}{1. Cor. 1, 18-25 \& 30}
\begin{paracol}{2}\latim{
Léctio Epístolæ beáti Pauli Apóstoli ad Corínthios.
}\switchcolumn\portugues{
Lição da Ep.ª do B. Ap.º Paulo aos Coríntios.
}\switchcolumn*\latim{
\rlettrine{F}{ratres:} Verbum crueis pereúntibus quidem stultítia est: iis autem, qui salvi fiunt, id est nobis, Dei virtus est. Scriptum est enim: Perdam sapiéntiam sapiéntium et prudéntiam prudéntium reprobábo. Ubi sápiens? ubi scriba? ubi conquisítor hujus séculi? Nonne stultam fecit Deus sapiéntiam hujus mundi? Nam quia in Dei sapiéntia non cognóvit mundus per sapiéntiam Deum: placuit Deo per stultítiam prædicatiónis salvos fácere credéntes. Quóniam et Judǽi signa petunt, et Græci sapiéntiam quærunt: nos autem prædicámus Christum crucifíxum: Judǽis quidem scándalum, géntibus autem stultítiam, ipsis autem vocátis Judǽis, atque Græcis. Christum Dei virtútem et Dei sapiéntiam: quia, quod stultum est Dei, sapiéníius est homínibus: et, quod infírmum est Dei, fórtius est homínibus. Ex ipso autem vos estis in Christo Jesu, qui factus est nobis sapiéntia a Deo et justítia ei sanctificátio et redémptio.
}\switchcolumn\portugues{
\rlettrine{M}{eus} irmãos: A doutrina da cruz é loucura para aqueles que se perdem; mas para aqueles que se salvam, isto é, para nós, é virtude de Deus; pois está escrito: «Eu destruirei a sabedoria dos sábios e reprovarei a prudência dos prudentes». Onde está o sábio? Onde está o doutor da lei? Onde está o investigador deste mundo? Porventura não mostrou Deus que a sabedoria deste mundo é loucura? Na verdade, ainda que na sabedoria de Deus, o mundo não conheceu Deus pela sua sabedoria, contudo aprouve a Deus pela loucura da pregação salvar aqueles que acreditassem n’Ele. Enquanto que os judeus pedem milagres e os gregos procuram a sabedoria, nós pregamos Cristo crucificado, que é, na verdade, escândalo para os judeus e loucura para os pagãos; mas para os que são chamados, quer judeus, quer gregos, é poder de Deus e sabedoria de Deus; pois o que parece estulto em Deus é mais sábio do que a sabedoria dos homens; e o que parece fraco em Deus é mais forte do que os homens. É por Ele que estais em Jesus Cristo, que por Deus se fez nossa sabedoria, justiça, santificação e redenção.
}\end{paracol}

\paragraphinfo{Gradual}{1. Cor. 3, 19 \& 20}
\begin{paracol}{2}\latim{
Allelúja, allelúja. ℣. \emph{1. Cor. 3, 19 \& 20} Sapiéntia hujus mundi stultítia est apud Deum, scriptum est enim: Dóminus novit cogitatiónes sapiéntium, quóniam vanæ sunt. Allelúja. ℣. \emph{Philipp. 3, 8} Verúmtamen exístimo ómnia detriméntum esse propter eminéntem sciéntiam Jesu Christi, Dómini mei. Allelúja.
}\switchcolumn\portugues{
Aleluia, aleluia. ℣. \emph{1. Cor. 3, 19 \& 20} A sabedoria deste mundo é estultícia diante de Deus; pois está escrito: «O Senhor conhece os pensamentos dos sábios e sabe que eles são vãos». Aleluia. ℣. \emph{Fl. 3, 8} Na verdade reputo tudo como perda em comparação da eminente ciência de Jesus Cristo, meu Senhor. Aleluia.
}\end{paracol}

\paragraphinfo{Evangelho}{Lc. 12, 2-8}
\begin{paracol}{2}\latim{
\cruz Sequéntia sancti Evangélii secúndum Lucam.
}\switchcolumn\portugues{
\cruz Continuação do santo Evangelho segundo S. Lucas.
}\switchcolumn*\latim{
\blettrine{I}{n} illo témpore: Dixit Jesus discípulis suis: Nihil opértum csi, quod non revelétur, neque abscónditum, quod non sciátur. Quóniam quæ in ténebris dixístis, in lúmine dicántur: et quod in aurem locúti estis in cubículis, prædicábitur in tectis. Dico autem vobis amícis meis: Ne terreámini ab his, qui occidunt corpus et post hæc non habent ámplius, quid fáciant. Osténdam autem vobis, quem timeátis: timéte eum, qui, postquam occídent, habet potestátem míttere in gehénnam; ita dico vobis, hunc timéte. Nonne quinque pásseres véneunt dipóndio, et unus ex illis non est in oblivióne coram Deo? Sed et capílli capitis vestri omnes numeráti sunt. Nolíte ergo timére: multis passéribus pluris estis vos. Dico autem vobis: Omnis, quicúmque conféssus fúerit me coram homínibus, et Fílius hóminis confitébitur illum coram Angelis Dei.
}\switchcolumn\portugues{
\blettrine{N}{aquele} tempo disse Jesus aos seus discípulos: «Nada há oculto que não chegue a ser descoberto, nem segredo que não venha a ser revelado. Até aquelas coisas que dissestes nas trevas serão publicadas à luz; e o que dissestes ao ouvido, no recôndito dos vossos cubículos, será apregoado sobre os telhados. Digo-vos, porém, a vós, que sois meus amigos: não tenhais receio daqueles que matam o corpo e depois não podem fazer mais nada. Sabeis quem deveis temer? Temei Aquele que, depois de dar a morte, tem ainda o poder de lançar no inferno. Sim; Eu vo-lo digo:
temei Este. Porventura se não vendem cinco pássaros por dois quartos? Contudo nem um só deles fica em esquecimento diante de Deus! Até os cabelos da vossa cabeça estão contados! Não tenhais, pois, receio. Vós valeis mais que muitos pássaros. Também vos digo: todo aquele que me confessar diante dos homens, o Filho do homem o reconhecerá igualmente diante dos Anjos de Deus».
}\end{paracol}

\paragraphinfo{Ofertório}{1. Cor. 2, 2}
\begin{paracol}{2}\latim{
\rlettrine{N}{on} enim judicávi me scire áliquid inter vos, nisi Jesum Christum, et hunc crucifíxum. (T.P. Allelúja.)
}\switchcolumn\portugues{
\rlettrine{E}{u} julgo que não devo conhecer outra coisa entre vós senão Jesus Cristo, e Jesus Cristo crucificado. (T.P. Aleluia.)
}\end{paracol}

\paragraph{Secreta}
\begin{paracol}{2}\latim{
\rlettrine{M}{únera} nostra, Dómine Deus, benígnus súscipe: quorum mirábile mystérium sanctus Martyr Justínus advérsum impiórum calúmnias strénue deféndit. Per Dóminum nostrum \emph{\&c.}
}\switchcolumn\portugues{
\rlettrine{A}{ceitai} benigno, Vos suplicamos, ó Senhor e Deus, as nossas ofertas neste adorável mistério que o Santo Mártir Justino defendeu ardentemente contra as calúnias dos ímpios. Por nosso Senhor \emph{\&c.}
}\end{paracol}

\paragraphinfo{Secreta}{S. Tibúrcio, Mártires}
\begin{paracol}{2}\latim{
\rlettrine{H}{æc} hóstia, quǽsumus, Dómine, quam sanctórum Mártyrum tuórum natalítia recenséntes offérimus: et víncula nostræ pravitátis absolvat, et tuæ nobis misericórdiæ dona concíliet. Per Dóminum \emph{\&c.}
}\switchcolumn\portugues{
\rlettrine{S}{enhor,} Vos suplicamos, permiti que esta hóstia, que Vos oferecemos celebrando o nascimento dos vossos Santos Mártires, nos livre dos laços da nossa perversidade e nos torne merecedores da vossa misericórdia. Por nosso Senhor \emph{\&c.}
}\end{paracol}

\paragraphinfo{Comúnio}{2. Tm. 4, 8}
\begin{paracol}{2}\latim{
\rlettrine{R}{epósita} est mihi coróna justítiæ, quam reddet mihi Dóminus in illa die justus judex. (T.P. Allelúja.)
}\switchcolumn\portugues{
\rlettrine{E}{stá} reservada para mim a coroa da justiça, a qual o Senhor, que é Juiz justo, me entregará no dia da sua vinda. (T.P. Aleluia.)
}\end{paracol}

\paragraph{Postcomúnio}
\begin{paracol}{2}\latim{
\rlettrine{C}{œlésti} alimónia refécti, súpplices te, Dómine, deprecámur: ut, beáti Justíni Mártyris tui mónitis, de accéptis donis semper in gratiárum actióne maneámus. Per Dóminum \emph{\&c.}
}\switchcolumn\portugues{
\rlettrine{S}{ustentados} com o alimento celestial, Vos suplicamos, Senhor, que, seguindo as advertências do B. Justino, vosso Mártir, permaneçamos em contínuas acções de graças pelos dons recebidos. Por nosso Senhor \emph{\&c.}
}\end{paracol}

\paragraphinfo{Postcomúnio}{S. Tibúrcio, Mártires}
\begin{paracol}{2}\latim{
\rlettrine{S}{acro} múnere satiáti, súpplices te, Dómine, deprecámur: ut, quod débitæ servitútis celebrámus offício, salvatiónis tuæ sentiámus augméntum. Per Dóminum \emph{\&c.}
}\switchcolumn\portugues{
\rlettrine{S}{aciados} com vosso sacratíssimo dom, humildemente Vos suplicamos; Senhor, que, rendendo-Vos com a celebração deste ofício as devidas homenagens da nossa servidão, sintamos em nós aumentar os efeitos da vossa redenção. Por nosso Senhor \emph{\&c.}
}\end{paracol}
