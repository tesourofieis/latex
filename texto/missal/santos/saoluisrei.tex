\subsectioninfo{S. Luís, Rei de França}{25 de Agosto}\label{saoluisrei}

\textit{Como na Missa Os justi, página \pageref{confessoresnaopontifices1}, excepto:}

\paragraph{Oração}
\begin{paracol}{2}\latim{
\rlettrine{D}{eus,} qui beátum Ludovícum Confessórem tuum de terréno regno ad cœléstis regni glóriam transtulísti: ejus, quǽsumus, méritis et intercessióne; Regis regum Jesu Christi, Fílii tui, fácias nos esse consórtes: Qui tecum vivit et regnat \emph{\&c.}
}\switchcolumn\portugues{
\slettrine{Ó}{} Deus, que trasladastes o B. Luís, vosso Confessor, do reino terrestre para a glória do reino celestial, humildemente Vos suplicamos, pelos seus méritos e intercessão, que um dia nos façais participar da glória do Rei dos reis, Jesus Cristo, vosso Filho. Que convosco vive e reina \emph{\&c.}
}\end{paracol}

\paragraphinfo{Epístola}{Página \pageref{martirnaopontifice1}}

\paragraphinfo{Evangelho}{Lc. 19, 12-26}
\begin{paracol}{2}\latim{
\cruz Sequéntia sancti Evangélii secúndum Lucam.
}\switchcolumn\portugues{
\cruz Continuação do santo Evangelho segundo S. Lucas.
}\switchcolumn*\latim{
\blettrine{I}{n} illo témpore: Dixit Jesus discípulis suis parábolam hanc: Homo quidam nóbilis ábiit in regionem longínquam accípere sibi regnum, et revérti. Vocátis autem decem servis suis, dedit eis decem mnas, et ait ad illos: Negotiámini, dum vénio. Cives autem ejus óderant eum: et misérunt legatiónem post illum, dicéntes: Nólumus hunc regnáre super nos. Et factum est, ut redíret accépto regno: et jussit vocári servos, quibus dedit pecúniam, ut sciret, quantum quisque negotiátus esset. Venit autem primus, dicens: Dómine, mna tua decem mnas acquisívit. Et ait illi: Euge, bone serve, quia in módico fuísti fidélis, eris potestátem habens super decem civitátes. Et alter venit, dicens: Dómine, mna tua fecit quinque mnas. Et huic ait: Et tu esto super quinque civitátes. Et alter venit, dicens: Dómine, ecce mna tua, quam hábui repósitam in sudário: tímui enim te, quia homo austérus es: tollis, quod non posuísti, et metis, quod non seminásti. Dicit ei: De ore tuo te júdico, serve nequam. Sciébas, quod ego homo austérus sum, tollens, quod non pósui, et metens, quod non seminávi: et quare non dedísti pecúniam meam ad mensam, ut ego véniens cum usúris útique exegíssem illam? Et astántibus dixit: Auferte ab illo mnam et date illi, qui decem mnas habet. Et dixérunt ei: Dómine, habet decem mnas. Dico autem vobis: Quia omni habénti dábitur, et abundábit: ab eo autem, qui non habet, et, quod habet, auferétur ab eo.
}\switchcolumn\portugues{
\blettrine{N}{aquele} tempo, Jesus disse aos seus discípulos esta parábola: «Certo homem de linhagem nobre partiu para um país longínquo, a fim de conseguir a posse desse reino e voltar depois. Chamou, pois, dez dos seus servos, deu-lhes dez moedas e disse-lhes: «Negociai com elas até eu voltar». Porém os seus concidadãos, que o odiavam, enviaram uma embaixada após ele, dizendo: «Não queremos que este homem seja nosso rei”. Ora aconteceu que, quando regressou, revestido da realeza, mandou chamar os servos, a quem havia dado o dinheiro, para saber como haviam negociado. Veio o primeiro e disse-lhe: «Vossa moeda, senhor, rendeu dez moedas». Respondeu-lhe o senhor: «Está bem, servo bom; visto que foste fiel em pouca cousa, recebe o governo de dez cidades». Veio o segundo e disse-lhe: «Vossa moeda, senhor, rendeu cinco moedas». Respondeu-lhe o senhor: «Tu governarás cinco cidades». Veio também o terceiro e disse: «Eis a vossa moeda, senhor, que eu guardei em um pano; pois tive medo de vós, que sois homem austero: tirais o que não depositais e recolheis o que não semeais». Respondeu-lhe o senhor: «Eu te julgarei com tuas próprias palavras, ó servo mau. Tu sabes que sou homem severo, tirando o que não depositei e colhendo o que não semeei; porque, então, não depositaste o meu dinheiro em um banco, para que, quando regressasse, o recolhesse com os juros?». Depois, continuou o senhor, dirigindo-se aos presentes: «Tirai-lhe a moeda e dai-a ao que tem dez». E eles disseram-lhe: «Ele já tem dez moedas». «Eu vos digo, retorquiu o Senhor, dar-se-á àquele que tem, e ficará na abundância; mas àquele que nada tem tirar-se-lhe-á até o que tiver».
}\end{paracol}

\paragraph{Secreta}
\begin{paracol}{2}\latim{
\rlettrine{P}{ræsta,} quǽsumus, omnípotens Deus: ut, sicut beátus Ludovícus Conféssor tuus, spretis mundi oblectaméntis, soli Regi Christo placére stúduit; ita ejus orátio nos tibi reddat accéptos. Per eúndem Dóminum \emph{\&c.}
}\switchcolumn\portugues{
\slettrine{Ó}{} Deus omnipotente, Vos suplicamos, fazei que, assim como o B. Luís, vosso Confessor, desprezando as delícias do mundo só procurou agradar a Cristo-Rei, assim também a sua oração nos torne agradáveis a Vós. Pelo mesmo nosso Senhor \emph{\&c.}
}\end{paracol}

\paragraph{Postcomúnio}
\begin{paracol}{2}\latim{
\rlettrine{D}{eus,} qui beátum Confessórem tuum Ludovícum mirificásti in terris, et gloriósum in cœlis fecísti: eúndem, quǽsumus, Ecclésiæ tuæ constítue defensórem. Per Dóminum nostrum \emph{\&c.}
}\switchcolumn\portugues{
\slettrine{Ó}{} Deus, que engrandecestes na terra e glorificastes no céu o B. Luís, vosso Confessor, constituí-o, Vo-lo pedimos, defensor da vossa Igreja. Por nosso Senhor \emph{\&c.}
}\end{paracol}
