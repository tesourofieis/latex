\subsectioninfo{Aparição da B. V. Maria Imaculada}{11 de Fevereiro}

\paragraphinfo{Intróito}{Ap. 21, 2}
\begin{paracol}{2}\latim{
\rlettrine{V}{idi} civitátem sanctam, Jerúsalem novam, descendéntem de cœlo a Deo, parátam sicut sponsam ornátam viro suo. \emph{Ps. 44,2} Eructávit cor meum verbum bonum: dico ego ópera mea Regi.
℣. Gloria Patri \emph{\&c.}
}\switchcolumn\portugues{
\rlettrine{V}{i} a cidade santa, a nova Jerusalém, que descia do céu, vinda de Deus, adornada como uma esposa que está preparada para receber o seu esposo. \emph{Sl. 44,2} Meu coração exprimiu uma palavra excelente: «Consagro as minhas obras ao Rei».
℣. Glória ao Pai \emph{\&c.}
}\end{paracol}

\paragraph{Oração}
\begin{paracol}{2}\latim{
\rlettrine{D}{eus,} qui per immaculátam Vírginis Conceptiónem dignum Filio tuo habitáculum præparásti: súpplices a te quǽsumus; ut, ejúsdem Vírginis Apparitiónem celebrántes, salútem mentis et córporis consequámur. Per eúndem Dóminum nostrum \emph{\&c.}
}\switchcolumn\portugues{
\slettrine{Ó}{} Deus, que pela Imaculada Conceição da Virgem preparastes para vosso Filho uma morada digna d’Ele, fazei, Vos suplicamos, que, celebrando a Aparição desta mesma Virgem, alcancemos a salvação da alma e do corpo. Pelo mesmo nosso Senhor Jesus Cristo \emph{\&c.}
}\end{paracol}

\paragraphinfo{Epístola}{Ap. 11, 19; 12, 1 \& 10}
\begin{paracol}{2}\latim{
Léctio libri Apocalýpsis beáti Joánnis Apóstoli.
}\switchcolumn\portugues{
Lição do Apocalipse do B. Ap.º João.
}\switchcolumn*\latim{
\rlettrine{A}{pértum} est templum Dei in cœlo: et visa est arca testaménti ejus in templo ejus, et facta sunt fúlgura et voces et terræmótus et grando magna. Et signum magnum appáruit in cœlo: Múlier amícta sole, et luna sub pédibus ejus, et in cápite ejus coróna stellárum duódecim. Et audívi vocem magnam in cœlo dicéntem: Nunc facta est salus et virtus, et regnum Dei nostri et potéstas Christi ejus.
}\switchcolumn\portugues{
\rlettrine{O}{} templo de Deus foi aberto no céu, e a arca da sua aliança foi vista no seu templo. E então houve relâmpagos, vozes, tremor de terra e forte granizo. E apareceu no céu um grande sinal: Uma mulher, revestida com o sol, a lua sob os seus pés e uma coroa de doze estrelas na cabeça. E ouvi no céu uma voz forte clamar: «Agora, foi operada a salvação. A soberania e o domínio estão em Deus e o poder no seu Cristo».
}\end{paracol}

\paragraphinfo{Gradual}{Ct. 2, 12}
\begin{paracol}{2}\latim{
\rlettrine{F}{lores} apparuérunt in terra nostra, tempus putatiónis advénit, vox túrturis audíta est in terra nostra. ℣. \emph{ibid., 10 \& l4} Surge, amíca mea, speciósa mea, et veni: colúmba mea in foramínibus petræ, in cavérna macériæ.
}\switchcolumn\portugues{
\rlettrine{A}{s} flores apareceram na terra! Veio o tempo em que podemos cantar. Os arrulhos da rola ouvem-se já pelos campos! ℣. \emph{ibid., 10 \& l4} Erguei-vos, pois, ó minha amada, que toda sois formosa, e vinde! Ó minha pomba, que viveis nas fendas dos rochedos e nas cavernas escarpadas!
}\switchcolumn*\latim{
Allelúja, allelúja. ℣. Osténde mihi fáciem tuam, sonet vox tua in áuribus meis: vox enim tua dulcis, et fácies tua decóra. Allelúja.
}\switchcolumn\portugues{
Aleluia, aleluia. ℣. Mostrai-me o vosso rosto: fazei que a vossa voz ressoe aos meus ouvidos; pois a vossa voz é doce e o vosso rosto é cheio de beleza. Aleluia.
}\end{paracol}

\textit{Após a Septuagésima omite-se o Aleluia e o Verso e diz-se o:}

\paragraphinfo{Trato}{Jdt 15, 10}
\begin{paracol}{2}\latim{
\rlettrine{T}{u} glória Jerúsalem, tu lætítia Israël, tu honorificéntia pópuli nostri. ℣. \emph{Cant. 4, 7} Tota pulchra es, María: et mácula originális non est in te. ℣. Felix es, sacra Virgo María, et omni laude digníssima, quæ serpéntis caput virgíneo pede contrivísti.
}\switchcolumn\portugues{
\rlettrine{S}{ois} a glória de Jerusalém, a alegria de Israel, a honra do nosso povo. ℣. \emph{Cant. 4, 7} Toda sois formosa, ó Maria; e a mancha original não existe em Vós. ℣. Sois bem-aventurada, Santa Virgem Maria, e digníssima de todo o louvor, pois com vosso pé virginal esmagastes a cabeça da serpente.
}\end{paracol}

\paragraphinfo{Evangelho}{Lc. 1, 26-31}
\begin{paracol}{2}\latim{
\cruz Sequéntia sancti Evangélii secúndum Lucam.
}\switchcolumn\portugues{
\cruz Continuação do santo Evangelho segundo S. Lucas.
}\switchcolumn*\latim{
\blettrine{I}{n} illo témpore: Missus est Angelus Gábriel a Deo in civitátem Galilǽæ, cui nomen Názareth, ad Vírginem desponsátam viro, cui nomen erat Joseph, de domo David, et nomen Vírginis María. Et ingréssus Angelus ad eam dixit: Ave, grátia plena; Dóminus tecum: benedícta tu in muliéribus. Quæ cum audísset, turbáta est in sermóne ejus: et cogitábat, qualis esset ista salutátio. Et ait Angelus ei: Ne tímeas, María, invenísti enim grátiam apud Deum: ecce, concípies in útero et páries fílium, et vocábis nomen ejus Jesum.
}\switchcolumn\portugues{
\blettrine{N}{aquele} tempo, foi mandado por Deus o Anjo Gabriel a uma cidade da Galileia, chamada Nazaré, a uma Virgem, desposada com um varão, cujo nome era José, da casa de David; e o nome da Virgem era Maria. Entrando o Anjo onde ela estava, disse: «Eu te saúdo, cheia de graça: o Senhor é contigo: bendita és tu entre todas as mulheres». Ouvindo ela isto, perturbou-se, e pensava na significação desta saudação. Então, disse-lhe o Anjo: «Não temas, Maria, porquanto alcançaste graça diante do Senhor: eis que conceberás no teu seio e darás à luz um Filho, e o seu nome será Jesus».
}\end{paracol}

\paragraphinfo{Ofertório}{Lc. 1, 28}
\begin{paracol}{2}\latim{
\rlettrine{A}{ve,} grátia plena; Dóminus tecum: benedícta tu in muliéribus.
}\switchcolumn\portugues{
\rlettrine{A}{ve,} ó cheia de graça; o Senhor é convosco; bendita sois entre as mulheres.
}\end{paracol}

\paragraph{Secreta}
\begin{paracol}{2}\latim{
\rlettrine{H}{óstia} laudis, quam tibi, Dómine, per mérita gloriósæ et immaculátæ Vírginis offérimus, sit tibi in odórem suavitátis, et nobis optátam cónferat córporis et ánimæ sanitátem. Per Dóminum \emph{\&c.}
}\switchcolumn\portugues{
\qlettrine{Q}{ue} a hóstia de louvor, que Vos oferecemos, Senhor, seja de agradável odor pata Vós, pelos méritos da gloriosa e Imaculada Virgem, e nos alcance a desejada saúde do corpo e da alma. Por nosso Senhor \emph{\&c.}
}\end{paracol}


\paragraphinfo{Comúnio}{Sl. 64, 10}
\begin{paracol}{2}\latim{
\rlettrine{V}{isitásti} terram et inebriásti eam, multiplicásti locupletáre eam.
}\switchcolumn\portugues{
\rlettrine{V}{isitastes} a terra e inebriaste-la; encheste-la com muitas riquezas.
}\end{paracol}

\paragraph{Postcomúnio}
\begin{paracol}{2}\latim{
\qlettrine{Q}{uos} cœlésti, Dómine, aliménto satiásti, súblevet aextera Genetrícis tuæ immaculátæ: ut ad ætérnam pátriam, ipsa adjuvánte, perveníre mereámur: Qui vivis \emph{\&c.}
}\switchcolumn\portugues{
\rlettrine{S}{aciados} com o alimento celestial, permiti-nos, Senhor, que a mão da vossa Mãe Imaculada nos sustente, para que com a graça da sua protecção mereçamos chegar até à pátria eterna. Ó Vós, que viveis e reinais \emph{\&c.}
}\end{paracol}
