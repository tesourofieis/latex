\subsectioninfo{S. Marcos, Papa e Conf.}{7 de Outubro}

\textit{Como na Missa Si díligis me, página \pageref{sumospontifices}, excepto:}

\paragraph{Oração}
\begin{paracol}{2}\latim{
\rlettrine{E}{xáudi,} Dómine, preces nostras: et, interveniénte beáto Marco Confessóre tuo atque Pontífice, indulgéntiam nobis tríbue placátus et pacem. Per Dóminum. \emph{\&c.}
}\switchcolumn\portugues{
\slettrine{Ó}{} Pastor eterno, atendei propício ao vosso rebanho; e guardai-o com vossa perpétua protecção por intercessão do bem-aventurado Marcos, vosso Sumo Pontífice, o qual escolhestes como pastor de toda a Igreja. Por nosso Senhor \emph{\&c.}
}\end{paracol}

\paragraph{Secreta}
\begin{paracol}{2}\latim{
\rlettrine{A}{ccépta} tibi sit, Dómine, sacrátæ plebis oblátio pro tuórum honóre Sanctórum: quorum se méritis de tribulatióne percepísse cognóscit auxílium. Per Dóminum \emph{\&c.}
}\switchcolumn\portugues{
\rlettrine{C}{om} as ofertas destes dons, Vos suplicamos, Senhor, iluminai benignamente a vossa Igreja, a fim de que não só o vosso rebanho triunfe
em toda a parte, mas também pelo poder do vosso nome os pastores sejam bem acolhidos. Por nosso Senhor \emph{\&c.}
}\end{paracol}

\paragraph{Postcomúnio}
\begin{paracol}{2}\latim{
\rlettrine{D}{a,} quǽsumus, Dómine, fidélibus pópulis Sanctórum tuórum semper veneratióne lætari: et eórum perpétua supplicatióne muníri. Per Dóminum \emph{\&c.}
}\switchcolumn\portugues{
\rlettrine{S}{enhor,} Vos suplicamos, governai com mansidão a vossa Igreja, agora que foi alimentada com a sagrada refeição, a fim de que, dirigida com firme suavidade, alcance o incremento da sua liberdade e persista na integridade da sua doutrina. Por nosso Senhor \emph{\&c.}
}\end{paracol}