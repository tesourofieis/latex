\subsectioninfo{S. Jerónimo Emiliano, Conf.}{20 de Julho}

\paragraphinfo{Intróito}{Lm. 2, 11}
\begin{paracol}{2}\latim{
\rlettrine{E}{ffúsum} est in terra jecur meum super contritióne fíliæ pópuli mei, cum defíceret párvulus et lactens in platéis óppidi. \emph{Ps. 112, 1} Laudáte, pueri, Dóminum: laudáte nomen Dómini.
℣. Gloria Patri \emph{\&c.}
}\switchcolumn\portugues{
\rlettrine{O}{} meu coração ficou ferido ao contemplar as infelicidades da filha do meu povo, quando as criancinhas, e até aquelas que ainda eram amamentadas, caíam desfalecidas nas praças da cidade! \emph{Sl. 112, 1} Louvai o Senhor, ó meninos; louvai o nome do Senhor.
℣. Glória ao Pai \emph{\&c.}
}\end{paracol}

\paragraph{Oração}
\begin{paracol}{2}\latim{
\rlettrine{D}{eus,} misericordiárum pater, per mérita et intercessiónem beáti Hieronymi, quem órphanis adjutorem et patrem esse voluísti: concéde; ut spíritum adoptiónis, quo fílii tui nominámur et sumus, fidéliter custodiámus. Per Dóminum \emph{\&c.}
}\switchcolumn\portugues{
\slettrine{Ó}{} Deus, Pai das misericórdias, pelos méritos e intercessão do B. Jerónimo, que quisestes dar aos órfãos como auxílio e pai, concedei-nos a graça de conservarmos fielmente este espírito de adopção, em virtude do qual somos chamados vossos filhos e realmente o somos. Por nosso Senhor \emph{\&c.}
}\end{paracol}

\paragraphinfo{Epístola}{Is. 58, 7-11}
\begin{paracol}{2}\latim{
Léctio Isaíæ Prophétæ.
}\switchcolumn\portugues{
Lição do Profeta Isaías.
}\switchcolumn*\latim{
\rlettrine{H}{æc} dicit Dóminus: Frange esuriénti panem tuum, et egénos vagósque induc in domum tuam: cum víderis nudum, operi eum, et carnem tuam ne despéxeris. Tunc erúmpet quasi mane lumen tuum, et sánitas tua citius oriétur, et anteíbit fáciem tuam justítia tua, et glória Dómini cólliget te. Tunc invocábis, et Dóminus exáudiet: clamabis, et dicet: Ecce, adsum; si abstúleris de médio tui caténam, et desíeris exténdere dígitum, et loqui quod non prodest. Cum effúderis esuriénti ánimam tuam, et ánimam afflíctam repléveris. Oriétur in tenebris lux tua, et ténebræ tuæ erunt sicut merídies. Et réquiem tibi dabit Dóminus semper, et implébit splendóribus ánimam tuam, et ossa tua liberábit, et eris quasi hortus irríguus, et sicut fons aquárum, cujus non defícient aquæ.
}\switchcolumn\portugues{
\rlettrine{I}{sto} diz o Senhor: «Reparti o vosso pão por aqueles que têm fome e fazei entrar na vossa casa aqueles que não têm asilo. Quando virdes um homem nu, cobri-o e não desprezeis aquele que é da vossa própria carne. Então a vossa luz brilhará, como a aurora; reconquistareis depressa a vossa saúde; a vossa justiça caminhará diante de vós; e a glória do Senhor vos protegerá. Então invocareis o Senhor, que vos ouvirá; chamá-l’O-eis e responderá: «Eis-me aqui». Se tirardes do meio de vós a cadeia com que carregais os vossos irmãos; se deixardes de estender o dedo para eles e de lhes dirigirdes palavras injuriosas; se assistirdes com caridade aos pobres e consolardes as almas aflitas; então a vossa luz despontará nas trevas e as vossas trevas brilharão, como o sol ao meio-dia. E o Senhor vos dará um repouso que não mais terá fim, encherá as vossas almas de esplendor e livrará os vossos ossos da corrupção. Sereis como um jardim sempre regado e como uma fonte cujas águas não secam.
}\end{paracol}

\paragraphinfo{Gradual}{Pr. 5, 16}
\begin{paracol}{2}\latim{
\rlettrine{D}{erivéntur} fontes tui foras, et in platéis aquas tuas divide. ℣. \emph{Ps. 111, 5-6} Jucúndus homo, qui miserétur et cómmodat: dispónet sermónes suos in judício, quia in ætérnum non commovébitur.
}\switchcolumn\portugues{
\qlettrine{Q}{ue} as fontes trasbordem e que as águas se espalhem pelas praças públicas. ℣. \emph{Sl. 111, 5-6} Bem-aventurado o homem que usou de misericórdia; que emprestou ao pobre; e que proferiu as suas palavras com justiça, pois ninguém poderá lançá-lo por terra.
}\switchcolumn*\latim{
Allelúja, allelúja. ℣. \emph{ibid., 9} Dispérsit, dedit paupéribus: justítia ejus manet in sǽculum sǽculi. Allelúja.
}\switchcolumn\portugues{
Aleluia, aleluia. ℣. \emph{ibid., 9} Com liberalidade distribuiu esmolas pelos pobres, pelo que sua justiça permanecerá em todos os séculos dos séculos. Aleluia.
}\end{paracol}

\paragraphinfo{Evangelho}{Mt. 19, 13-21}
\begin{paracol}{2}\latim{
\cruz Sequéntia sancti Evangélii secúndum Matthǽum.
}\switchcolumn\portugues{
\cruz Continuação do santo Evangelho segundo S. Mateus.
}\switchcolumn*\latim{
\blettrine{I}{n} illo témpore: Obláti sunt Jesu párvuli, ut manus eis impóneret et oráret. Discípuli autem increpábant eos. Jesus vero ait eis: Sínite párvulos, et nolíte eos prohibére ad me veníre: tálium est enim regnum cœlórum. Et cum imposuísset eis manus, ábiit inde. Et ecce, unus accedens, ait illi: Magíster bone, quid boni fáciam, ut hábeam vitam ætérnam? Qui dixit ei: Quid me intérrogas de bono? Unus est bonus, Deus. Si autem vis ad vitam íngredi, serva mandáta. Dicit illi: Quæ? Jesus autem dixit: Non homicídium fácies: Non adulterábis: Non fácies furtum: Non falsum testimónium dices: Hónora patrem tuum et matrem tuam, et díliges próximum tuum sicut te ipsum. Dicit illi adoléscens: Omnia hæc custodívi a juventúte mea: quid adhuc mihi deest? Ait illi Jesus: Si vis perféctus esse, vade, vende, quæ habes, et da paupéribus, et habébis thesáurum in cœlo: et veni, séquere me.
}\switchcolumn\portugues{
\blettrine{N}{aquele} tempo, apresentaram a Jesus algumas criancinhas, para que Ele sobre elas impusesse as mãos e orasse por elas. Ora os discípulos afastaram-nas. Então Jesus disse-lhes: «Deixai as criancinhas e as não proibais de vir a mim, pois o reino dos céus é daqueles que se lhes assemelham». E, impondo sobre elas as mãos, se afastou. Mas eis que alguém, aproximando-se d’Ele, disse-Lhe: «Bom Mestre, que tenho a fazer de bom para alcançar a vida eterna?». Jesus respondeu-lhe: «Porque me chamais bom? Só um é bom: Deus! Se, pois, queres entrar na vida eterna, guarda os mandamentos». Disse-Lhe ele: «Quais mandamentos?». Jesus continuou: «Não cometerás homicídio, nem adultério; não furtarás; não levantarás falso testemunho; respeitarás teu pai e tua mãe; e amarás o teu próximo, como a ti próprio». O jovem disse-Lhe: «Tenho observado tudo isso desde a minha juventude; que me falta, pois ainda?». Jesus disse-lhe então: «Se queres ser perfeito, vai, vende tudo o que te pertence e dá-o aos pobres. Então alcançarás um tesouro no céu. Depois vem e segue-me».
}\end{paracol}

\paragraphinfo{Ofertório}{Tb. 12, 12}
\begin{paracol}{2}\latim{
\qlettrine{Q}{uando} orábas cum lácrimis, et sepeliébas mórtuos, et derelinquébas prándium tuum, et mórtuos abscondébas per diem in domo tua, et nocte sepeliébas eos: ego óbtuli oratiónem tuam Dómino.
}\switchcolumn\portugues{
\qlettrine{Q}{uando} rezaste com lágrimas e quando enterraste os mortos, deixando para isso a tua refeição, escondendo os mortos durante o dia na tua casa e enterrando-os durante a noite eu apresentei a tua oração ao Senhor.
}\end{paracol}

\paragraph{Secreta}
\begin{paracol}{2}\latim{
\rlettrine{C}{lementíssime} Deus, qui, véteri homine consúmpto, novum secúndum te in beáto Hierónymo creáre dignátus es: da, per mérita ipsíus; ut nos, páriter renováti, hanc placatiónis hóstiam in odórem tibi suavíssimum offerámus. Per Dóminum \emph{\&c.}
}\switchcolumn\portugues{
\slettrine{Ó}{} Deus clementíssimo, que sobre as ruínas do «homem velho» Vos dignastes criar um novo homem à vossa imagem na pessoa do B. Jerónimo, concedei-nos pelos seus méritos que, sendo renovados, como ele, Vos apresentemos esta hóstia de propiciação, como um perfume de suave odor. Por nosso Senhor \emph{\&c.}
}\end{paracol}

\paragraphinfo{Comúnio}{Tg. 1, 27}
\begin{paracol}{2}\latim{
\rlettrine{R}{elígio} munda et immaculáta apud Deum et Patrem hæc est: Visi
táre pupíllos et víduas in tribulatióne eórum, et immaculátum se custodíre ab hoc sǽculo.
}\switchcolumn\portugues{
\rlettrine{A}{} religião pura e imaculada aos olhos de Deus, nosso Pai, é esta: visitar os órfãos e as viúvas nas suas aflições e conservar-se puro na corrupção deste mundo.
}\end{paracol}

\paragraph{Postcomúnio}
\begin{paracol}{2}\latim{
\rlettrine{A}{ngelórum} pane refécti te, Dómine, supplíciter deprecámur: ut, qui ánnuam beáti Hierónymi Confessóris tui memóriam celebráre gaudémus; ejúsdem étiam et exémplum imitémur, et amplíssimum in regno tuo prǽmium obtinére valeámus. Per Dóminum nostrum \emph{\&c.}
}\switchcolumn\portugues{
\rlettrine{S}{aciados} com o pão dos Anjos, Vos suplicamos humildemente, Senhor, que, celebrando nós com alegria, anualmente, a memória do B. Jerónimo, vosso Confessor, imitemos também os seus exemplos e mereçamos alcançar as liberalíssimas recompensas do vosso reino. Por nosso Senhor \emph{\&c.}
}\end{paracol}
