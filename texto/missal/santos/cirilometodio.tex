\subsectioninfo{S. S. Cirilo e Metódio, Bs. e Cs.}{7 de Julho}

\textit{Como Missa Sacerdótes tui, página \pageref{confessorespontifices2}, excepto:}

\paragraph{Oração}
\begin{paracol}{2}\latim{
\rlettrine{O}{mnípotens} sempitérne Deus, qui Slavóniæ gen
tes per beátos Confessóris tuos atque Pontífices Cyríllum et Methódium ad agnitiónem tui nóminis veníre tribuísti: præsta; ut, quorum festivitáte gloriámur, eórum consórtio copulémur. Per Dóminum. \emph{\&c.}
}\switchcolumn\portugues{
\slettrine{Ó}{} Deus omnipotente e eterno, que Vos dignastes trazer ao conhecimento do vosso nome os povos eslavos pelo ministério dos B. B. Cirilo e Metódio, vossos Confessores e Pontífices, concedei-nos a graça de sermos um dia admitidos na companhia daqueles cuja festa nos gloriamos de celebrar. Por nosso Senhor \emph{\&c.}
}\end{paracol}

\paragraphinfo{Evangelho}{Página \pageref{tito}}

\paragraphinfo{Ofertório}{Sl. 67, 36}
\begin{paracol}{2}\latim{
\rlettrine{M}{irábilis} Deus in Sanctis suis: Deus Israel, ipse dabit virtútem et fortitúdinem plebisuæ: benedíctus Deus.
}\switchcolumn\portugues{
\rlettrine{D}{eus} é admirável em seus Santos. Deus de Israel dará ao seu povo a força e a coragem. Bendito seja Deus.
}\end{paracol}

\paragraph{Secreta}
\begin{paracol}{2}\latim{
\rlettrine{P}{reces} nostras, quæsumus, Dómine, et tuórum réspice oblatiónes fidélium: ut tibi gratæ sint in tuórum festivitáte (commemoratione) Sanctórum, et nobis conferant tuæ propitiatiónis auxílium. Per Dóminum \emph{\&c.}
}\switchcolumn\portugues{
\rlettrine{D}{ignai-Vos} receber benignamente as nossas orações e as oblatas dos fiéis, Senhor, a fim de que estas festas dos vossos Santos Vos sejam agradáveis e nos obtenham o auxílio da vossa propiciação. Por nosso Senhor \emph{\&c.}
}\end{paracol}

\paragraphinfo{Comúnio}{Mt. 10, 27}
\begin{paracol}{2}\latim{
\qlettrine{Q}{uod} dico vobis in tenebris, dícite in lúmine, dicit Dóminus: et quod in aure audítis, prædicáte super tecta.
}\switchcolumn\portugues{
\rlettrine{O}{} que vos digo nas trevas dizei-o às claras, diz o Senhor; e o que vos disse ao ouvido pregai-o em cima dos telhados.
}\end{paracol}

\paragraph{Postcomúnio}
\begin{paracol}{2}\latim{
\qlettrine{Q}{uǽsumus,} omnípotens Deus: ut, qui nobis múnera dignáris præbére cœléstia, intercedéntibus sanctis tuis Cyríllo et Methódio, despícere terréna concédas. Per Dóminum \emph{\&c.}
}\switchcolumn\portugues{
\slettrine{Ó}{} Deus omnipotente, que Vos dignastes cumular-nos com os dons celestiais, Vos suplicamos, concedei-nos por intercessão dos vossos Santos Cirilo e Metódio a graça de desprezarmos as cousas terrenas. Por nosso Senhor \emph{\&c.}
}\end{paracol}
