\subsectioninfo{S. João Capistrano, Conf.}{28 de Março}

\paragraphinfo{Intróito}{Hab. 3, 18-19}
\begin{paracol}{2}\latim{
\rlettrine{E}{go} autem in Dómino gaudébo: et exsultábo in Deo, Jesu meo: Deus Dóminus fortitúdo mea. (T. P. Allelúja, allelúja.) \emph{Ps. 80. 2} Exsultáte Deo, adjutóri nostro, jubiláte Deo Jacob.
℣. Gloria Patri \emph{\&c.}
}\switchcolumn\portugues{
\rlettrine{A}{legrar-me-ei} no Senhor; rejubilarei em Deus, meu Salvador. O Senhor, meu Deus, é a minha fortaleza. (T. P. Aleluia, aleluia.) \emph{Sl. 80. 2} Exultai de alegria, louvando Deus, que é o nosso sustentáculo: aclamai com júbilo Deus de Jacob.
℣. Glória ao Pai \emph{\&c.}
}\end{paracol}

\paragraph{Oração}
\begin{paracol}{2}\latim{
\rlettrine{D}{eus,} qui per beátum Joánnem fidéles tuos in virtúte sanctíssimi nóminis Jesu de Crucis inimícis triumpháre fecísti: præsta, quǽsumus; ut, spirituálium hóstium, ejus intercessióne, superátis insídiis, corónam justítiæ a te accípere mereámur. Per eúndem Dóminum \emph{\&c.}
}\switchcolumn\portugues{
\slettrine{Ó}{} Deus, que pelo B. João permitistes que os vossos fiéis, graças à virtude do Santíssimo Nome de Jesus, triunfassem dos inimigos da Cruz, concedei-nos, Vos suplicamos, que, havendo resistido por sua intercessão às tentações dos inimigos, mereçamos receber de vossas mãos a coroa de justiça. Pelo mesmo nosso Senhor \emph{\&c.}
}\end{paracol}

\paragraphinfo{Epístola}{Página \pageref{martirnaopontifice1}}

\paragraphinfo{Gradual}{Sl. 21, 24-25}
\begin{paracol}{2}\latim{
\qlettrine{Q}{ui} timétis Dóminum, laudáte eum: univérsum semen Jacob, glorificáte eum. ℣. Timeat eum omne semen Israël: quóniam non sprevit, neque despéxit deprecatiónem páuperis.
}\switchcolumn\portugues{
\slettrine{Ó}{} vós, que temeis o Senhor, louvai-O! Vós todos, que sois descendentes de Jacob, glorificai-O! ℣. Tema-O toda a geração de Israel: pois não desprezou, nem desdenhou a oração do pobre.
}\end{paracol}

\paragraphinfo{Trato}{Ex. 15, 2 \& 3}
\begin{paracol}{2}\latim{
\rlettrine{F}{ortitúdo} mea et laus mea Dóminus, et factus est mihi in salútem: iste Deus meus, et glorificábo eum. ℣. Dóminus quasi vir pugnátor, omnípotens nomen ejus. ℣. \emph{Judith 16, 3} Dóminus cónterens bella: Dóminus nomen est illi.
}\switchcolumn\portugues{
\rlettrine{O}{} Senhor é a minha fortaleza e o objecto dos meus louvores. Foi Ele, que é o meu Deus, quem me salvou: eu O glorificarei. ℣. O Senhor mostrou-se um guerreiro invencível: o seu nome é omnipotente. ℣. \emph{Jdt. 16, 3} O Senhor é o vencedor das batalhas. Senhor é o seu nome!
}\end{paracol}

\textit{No Tempo Pascal omite-se o Gradual e o Trato, e diz-se:}

\begin{paracol}{2}\latim{
Allelúja, allelúja. ℣. \emph{Ps. 58, 17} Ego autem cantábo fortitúdinem tuam: et exsultábo mane misericórdiam tuam. Allelúja. ℣. Quia factus es suscéptor meus, et refúgium meum in die tribulatiónis meæ. Allelúja.

}\switchcolumn\portugues{
Aleluia, aleluia. ℣. \emph{Sl. 58, 17} Eu, porém, cantarei a vossa fortaleza e regozijar-me-ei desde manhã com vossa misericórdia. Aleluia. ℣. Pois fostes o meu protector e o meu refúgio no dia da minha tribulação. Aleluia.
}\end{paracol}

\paragraphinfo{Evangelho}{Página \pageref{quintafeirapentecostes}}

\paragraphinfo{Ofertório}{Ecl. 46, 6}
\begin{paracol}{2}\latim{
\rlettrine{I}{nvocávit} Altíssimum poténtem in oppugnándo inimícos úndique, et audívit illum magnus et sanctus Deus.
}\switchcolumn\portugues{
\rlettrine{I}{nvocou} o Omnipotente, o Altíssimo, quando os inimigos o atacaram por todos os lados: e Deus, infinito e santo, escutou-o.
}\end{paracol}

\paragraph{Secreta}
\begin{paracol}{2}\latim{
\rlettrine{S}{acrifícium,} Dómine, quod immolámus, placátus inténde: ut, intercedénte beáto Joánne Confessóre tuo, ad conteréndas inimicórum insídias nos in tuæ protectiónis securitáte constítuat. Per Dóminum nostrum \emph{\&c.}
}\switchcolumn\portugues{
\rlettrine{O}{lhai} aplacado, Senhor, para esta vítima, que imolamos em vossa honra, a fim de que por intercessão do B. João, vosso Confessor, nos acolha com segurança, e sob a vossa protecção possamos repelir as insídias dos inimigos. Por nosso Senhor \emph{\&c.}
}\end{paracol}

\paragraphinfo{Comúnio}{Sb. 10, 20}
\begin{paracol}{2}\latim{
\rlettrine{D}{ecantavérunt,} Dómine, nomen sanctum tuum, et victrícem manum tuam laudavérunt.
}\switchcolumn\portugues{
\rlettrine{C}{elebraram} com seus cânticos o vosso santo nome, ó Senhor, e louvaram a vossa mão vitoriosa.
}\end{paracol}

\paragraph{Postcomúnio}
\begin{paracol}{2}\latim{
\rlettrine{R}{epléti} alimónia cœlésti et spirituáli pópulo recreáti, quǽsumus, omnípotens Deus: ut, intercedénte beáto Joánne Confessóre tuo, nos ab hoste malígno deféndas, et Ecclésiam tuam perpétua pace custódias. Per Dóminum nostrum \emph{\&c.}
}\switchcolumn\portugues{
\rlettrine{S}{aciados} com o alimento celestial e fortalecidos com a bebida espiritual, Vos pedimos, ó Deus omnipotente, por intercessão do B. João, vosso Confessor, defendei-nos contra o inimigo maligno e conservai a vossa Igreja em contínua paz. Por nosso Senhor \emph{\&c.}
}\end{paracol}
