\subsectioninfo{Santa Rosa de Lima, Virgem}{30 de Agosto}

\textit{Como na Missa Dilexísti justitiam, página \pageref{virgemnaomartir1}, excepto:}

\paragraph{Oração}
\begin{paracol}{2}\latim{
\rlettrine{B}{onórum} ómnium largítor, omnípotens Deus, qui beátam Rosam, cœléstis grátiæ rore prævéntam, virginitátis et patiéntiæ decóre Indis floréscere voluísti: da nobis fámulis tuis; ut, in odórem suavitátis ejus curréntes, Christi bonus odor éffici mereámur: Qui tecum \emph{\&c.}
}\switchcolumn\portugues{
\slettrine{Ó}{} Deus omnipotente, dispensador de todos os bens, que antecipadamente enriquecestes com o orvalho celestial da vossa graça a B. Rosa e que a fizestes florescer nas Índias com o brilho da virgindade e da paciência, concedei aos vossos
servos que, correndo após o perfume das suas suaves virtudes, mereçamos tornar-nos o bom odor de Cristo: Que convosco vive e reina \emph{\&c.}
}\end{paracol}

\subsubsection{Comemoração dos S. S. Mártires Félix e Adauto}

\paragraph{Oração}
\begin{paracol}{2}\latim{
\rlettrine{M}{ajestátem} tuam, Dómine, súpplices exorámus: ut, sicut nos júgiter Sanctórum tuórum commemoratióne lætíficas; ita semper supplicatióne deféndas. Per Dóminum nostrum \emph{\&c.}
}\switchcolumn\portugues{
\rlettrine{D}{irigimos} as nossas súplicas à vossa majestade, Senhor, a fim de que, assim como nos dais, na festa dos vossos Santos, perpétuo motivo de alegria, assim também, graças às suas orações, nos defendais perpetuamente. Por nosso Senhor \emph{\&c.}
}\end{paracol}

\paragraph{Secreta}
\begin{paracol}{2}\latim{
\rlettrine{H}{óstias,} Dómine, tuæ plebis inténde: et, quas in honóre Sanctórum tuórum devóta mente célebrat, profícere sibi séntiat ad salútem. Per Dóminum \emph{\&c.}
}\switchcolumn\portugues{
\rlettrine{D}{ignai-Vos} lançar os vossos olhares, Senhor, para as hóstias do vosso povo, e, visto que ele Vo-las oferece com devoção em honra dos vossos Santos, fazei que sejam úteis à sua salvação. Por nosso Senhor \emph{\&c.}
}\end{paracol}

\paragraph{Postcomúnio}
\begin{paracol}{2}\latim{
\rlettrine{R}{epléti,} Dómine, munéribus sacris: quǽsumus: ut, intercedéntibus Sanctis tuis, in gratiárum semper actióne maneámus. Per Dóminum nostrum \emph{\&c.}
}\switchcolumn\portugues{
\rlettrine{S}{aciados} com os sacrossantos dons, Senhor, Vos imploramos, permiti, pela intercessão dos vossos Santos, que permaneçamos sempre em acção de graças. Por nosso Senhor \emph{\&c.}
}\end{paracol}
