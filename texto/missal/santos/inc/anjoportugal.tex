\subsectioninfo{Santo Anjo Custódio de Portugal}{21 de Julho}

\paragraphinfo{Intróito}{}
\begin{paracol}{2}\latim{
\rlettrine{}{}
\emph{}
℣. Gloria Patri \emph{\&c.}
}\switchcolumn\portugues{
\rlettrine{E}{ncomendou-te} Deus aos seus Anjos, para que eles te guardem em todos teus caminhos; em suas mãos te conduzirão, para que o teu Pé não tropece na pedra. Aquele que se abriga sob a protecção do Altíssimo e habita à sombra de Deus do céu. 
\emph{}
℣. Glória ao Pai \emph{\&c.}
}\end{paracol}

\paragraph{Oração}
\begin{paracol}{2}\latim{
\rlettrine{}{} \emph{\&c.}
}\switchcolumn\portugues{
\slettrine{Ó}{} Deus omnipotente e sempiterno, que com inefável providência destinais para cada nação um Anjo, que a guarde, concedei-nos, Vos suplicamos, que, pelas súplicas e pelo patrocínio do Anjo Custódio da nossa Nação, sejamos sempre livres de todas as adversidades. Por nosso Senhor \emph{\&c.}
}\end{paracol}

\paragraphinfo{Epístola}{}
\begin{paracol}{2}\latim{
Lectio Epístolæ beati Pauli Apostoli ad Corinthios.
}\switchcolumn\portugues{
Lição do Livro dos Reis.
}\switchcolumn*\latim{
\rlettrine{F}{ratres:}
}\switchcolumn\portugues{
\rlettrine{N}{aqueles} dias, Isaías, filho de Amós, mandou dizer a Ezequias: «Isto diz o Senhor, a respeito do rei dos Assírios: Ele não entrará nesta cidade, nem nela usará a sua espada, nem ocupará o seu castelo, nem a sitiará com suas paliçadas. Volverá pelo mesmo caminho que veio e não entrará nesta cidade: assim diz o Senhor. Protegerei esta cidade e salvá-la-ei por amor de mim e por amor do meu servo Daniel», Naquela mesma noite, veio o Anjo do Senhor e feriu no acampamento dos Assírios cento e vinte e cinco mil homens; e, ao romper da manhã, viu-se que todos estavam mortos. Então Senaqueribe, rei dos Assírios, partiu e voltou para os seus domínios.
}\end{paracol}

\paragraphinfo{Gradual}{}
\begin{paracol}{2}\latim{
\rlettrine{}{}
}\switchcolumn\portugues{
\rlettrine{O}{} Anjo do Senhor acampará em redor dos que o temem e salvá-los-á. Clamaram os justos e o Senhor ouviu-os: e livrou-os de todas suas tribulações.
}\switchcolumn*\latim{

}\switchcolumn\portugues{
Aleluia, aleluia. Adorai o Senhor, ó vós, todos seus Anjos; porque o Senhor guarda as vidas dos seus servos e salva-os das mãos dos pecadores. Aleluia.
}\end{paracol}

No Tempo Pascal, omite-se o Gradual e diz-se:

\begin{paracol}{2}\latim{

}\switchcolumn\portugues{
Aleluia, aleluia. Diante dos Anjos cantarei Salmos em vosso louvor; adorar-Vos-ei no vosso Templo Sagrado e louvarei o vosso nome. Aleluia. O Anjo do Senhor desceu do céu, e, aproximando-se, revolveu a pedra e assentou-se sobre ela. Aleluia.
}\end{paracol}

Nas Missas Votivas, depois da Septuagésima, omite-se o Aleluia e o Verso seguinte, e diz-se:

\paragraphinfo{Trato}{}
\begin{paracol}{2}\latim{
\rlettrine{}{}
}\switchcolumn\portugues{
\rlettrine{B}{endizei} o Senhor, ó vós, todos seus Anjos, poderosos em força, que obedeceis às suas palavras. Bendizei o Senhor, todos seus exércitos, seus ministros, que executais a sua vontade. Bendizei o Senhor, todas suas obras, em todos os lugares doseu império; bendizei o Senhor, ó minha alma.
}\end{paracol}

\paragraphinfo{Evangelho}{}
\begin{paracol}{2}\latim{
\cruz Sequéntia sancti Evangélii secúndum Lucam.
}\switchcolumn\portugues{
\cruz Continuação do santo Evangelho segundo S. Mateus.
}\switchcolumn*\latim{
\blettrine{I}{n}
}\switchcolumn\portugues{
\blettrine{N}{aquele} tempo, o Anjo do senhor apareceu em sonhos a José e disse-lhe: «Levanta-te, toma o Menino e sua Mãe e foge com eles para o Egipto, permanecendo ali até que te avise, porque Herodes procura o Menino para lhe tirar a vida». E José, então, levantando-se de noite, tomou o Menino e sua Mãe e partiu para o Egipto, onde permaneceu até à morte de Herodes, a fim de se cumprir o que havia sido anunciado pelo Senhor por meio do seu Profeta, quando disse: «Chamarei do Egipto o meu Filho».
}\end{paracol}

\paragraphinfo{Ofertório}{}
\begin{paracol}{2}\latim{
\rlettrine{}{}
}\switchcolumn\portugues{
\rlettrine{B}{endizei} o Senhor, ó vós, todos seus Anjos, seus ministros, executores das suas ordens e sempre fiéis aos seus chamamentos.
}\end{paracol}

\paragraph{Secreta}
\begin{paracol}{2}\latim{
\rlettrine{}{} \emph{\&c.}
}\switchcolumn\portugues{
\rlettrine{S}{enhor} Santo, Pai omnipotente, Deus eterno aceitai estes dons, que Vos oferecemos, suplicando-Vos que, pelos sufrágios do vosso Anjo Custódio, sejamos livres de todas as adversidades. Por nosso Senhor \emph{\&c.}
}\end{paracol}

\paragraphinfo{Comúnio}{}
\begin{paracol}{2}\latim{
\rlettrine{}{}
}\switchcolumn\portugues{
\rlettrine{A}{njos} do Senhor, bendizei todos o Senhor: cantai hinos emseu louvor e exaltai-O em todos os séculos.
}\end{paracol}

\paragraph{Postcomúnio}
\begin{paracol}{2}\latim{
\rlettrine{}{} \emph{\&c.}
}\switchcolumn\portugues{
\rlettrine{S}{ejam-nos} proveitosos, Senhor, para a salvação do corpo e da alma os Sacramentos, que recebemos, a fim de que, pela defesa da angelical custódia, sejamos livres de todos os perigos e dignos de participar dos dons celestiais. Por nosso Senhor \emph{\&c.}
}\end{paracol}
