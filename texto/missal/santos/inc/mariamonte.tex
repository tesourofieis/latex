\subsectioninfo{B. V. Maria do Monte\footnote{Na Diocese de Funchal.}}{9 de Outubro}

\paragraphinfo{Intróito}{}
\begin{paracol}{2}\latim{
\rlettrine{}{}
\emph{}
℣. Gloria Patri \emph{\&c.}
}\switchcolumn\portugues{
\rlettrine{U}{m} grande prodígio apareceu no céu: Uma mulher vestida, como o sol, tendo a lua debaixo de seus pés e sobre a cabeça uma coroa com doze estrelas! Ergui os meus olhos para o Monte, donde me vem o auxílio.
\emph{}
℣. Glória ao Pai \emph{\&c.}
}\end{paracol}

\paragraph{Oração}
\begin{paracol}{2}\latim{
\rlettrine{}{} \emph{\&c.}
}\switchcolumn\portugues{
\slettrine{Ó}{} Deus, que pela Imaculada Conceição da Virgem preparastes digna morada para o vosso Filho, Vos pedimos que, celebrando o celeste patrocínio da mesma Virgem e por sua intercessão e auxílio, mereçamos vencer todas as adversidades. Pelo mesmo nosso Senhor \emph{\&c.}
}\end{paracol}

\paragraphinfo{Epístola}{}
\begin{paracol}{2}\latim{
Lectio Epístolæ beati Pauli Apostoli ad Corinthios.
}\switchcolumn\portugues{
Lição da Ep.ª do B. Ap.º Paulo aos Coríntios.
}\switchcolumn*\latim{
\rlettrine{F}{ratres:}
}\switchcolumn\portugues{
\rlettrine{M}{eus}
}\end{paracol}

\paragraphinfo{Gradual}{}
\begin{paracol}{2}\latim{
\rlettrine{}{}
}\switchcolumn\portugues{
\rlettrine{N}{ão} sofrerão fome nem sede; os não molestará nem a calma nem o sol: pois Aquele que tem compaixão deles conduzi-los-á à fonte das águas e saciá-los-á. Eis que estes vêm de longe: uns vêm do Aquilão, outros do mar e estoutros das bandas do Meio-Dia!
}\switchcolumn*\latim{

}\switchcolumn\portugues{
Aleluia, aleluia. A gloriosíssima Virgem do Monte da santificação olhou para nós, conduziu-nos pelo caminho direito e livrou-nos de todos os perigos. Aleluia.
}\end{paracol}

\paragraphinfo{Evangelho}{}
\begin{paracol}{2}\latim{
\cruz Sequéntia sancti Evangélii secúndum Lucam.
}\switchcolumn\portugues{
\cruz Continuação do santo Evangelho segundo S. Lucas.
}\switchcolumn*\latim{
\blettrine{I}{n}
}\switchcolumn\portugues{
\blettrine{N}{aquele} tempo, levantando-se Maria, foi apressadamente às montanhas de uma cidade de Judá. Aí entrou em casa de Zacarias e saudou Isabel. Logo que Isabel ouviu a saudação de Maria, saltou a criança no seu seio e Isabel ficou cheia do Espírito Santo, exclamando em voz alta: «Bendita sois vós entre todas as mulheres; e bendito é o fruto do vosso ventre. Donde me vem a mim que a Mãe do meu Senhor venha até mim? Pois, desde que a voz da vossa saudação chegou a meus ouvidos, o meu filho exultou de alegria no meu seio! Bem-aventurada sois, porque acreditastes que se hão-de cumprir as cousas que vos foram ditas da parte do Senhor». Maria disse então: «Minha alma glorifica o Senhor: e o meu espírito se alegra em Deus, meu salvador. E, porque Ele se dignou lançar os olhos para a humildade da sua escrava, doravante todas as gerações me proclamarão bem-aventurada; pois o Omnipotente operou em mim maravilhas, e o seu nome é santo».
}\end{paracol}

\paragraphinfo{Ofertório}{}
\begin{paracol}{2}\latim{
\rlettrine{}{}
}\switchcolumn\portugues{
\rlettrine{G}{loriosas} cousas se dirão de vós, ó Maria, porque o Omnipotente em vós operou maravilhas. Aleluia.
}\end{paracol}

\paragraph{Secreta}
\begin{paracol}{2}\latim{
\rlettrine{}{} \emph{\&c.}
}\switchcolumn\portugues{
\rlettrine{A}{ceitai,} Senhor, Vos pedimos, as ofertas que piedosamente Vos apresentamos nesta festividade da B. V. Maria; e permiti que pelos auxílios dos méritos da mesma Virgem sejamos protegidos contra todos os ataques dos nossos inimigos. Por nosso Senhor \emph{\&c.}
}\end{paracol}

\paragraphinfo{Comúnio}{}
\begin{paracol}{2}\latim{
\rlettrine{}{}
}\switchcolumn\portugues{
\rlettrine{S}{ois} a glória de Jerusalém e a alegria de Israel; sois a honra do nosso povo. Aleluia.
}\end{paracol}

\paragraph{Postcomúnio}
\begin{paracol}{2}\latim{
\rlettrine{}{} \emph{\&c.}
}\switchcolumn\portugues{
\rlettrine{C}{onfortados} com o sacrossanto dom, Vos rendemos graças, Senhor, suplicando à vossa misericórdia que pelo patrocínio da B. Maria, sempre Virgem, sejamos livres de todos os males e perigos e mereçamos alcançar a salvação eterna. Por nosso Senhor \emph{\&c.}
}\end{paracol}
