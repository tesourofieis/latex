\subsectioninfo{S. Pedro Claver, Conf.}{9 de Setembro}

\paragraphinfo{Intróito}{}
\begin{paracol}{2}\latim{
\rlettrine{}{}
\emph{}
℣. Gloria Patri \emph{\&c.}
}\switchcolumn\portugues{
\rlettrine{O}{} Senhor saciou a alma exausta: e aqueles que estavam nas trevas e nas sombras da morte; e os que estavam cativos pela indigência e em ferros. Glorificai o Senhor pela sua misericórdia: e pelas suas maravilhas em favor dos filhos dos homens.
\emph{}
℣. Glória ao Pai \emph{\&c.}
}\end{paracol}

\paragraph{Oração}
\begin{paracol}{2}\latim{
\rlettrine{}{} \emph{\&c.}
}\switchcolumn\portugues{
\slettrine{Ó}{} Deus, que, a fim de chamar ao conhecimento do vosso nome os Negritas sujeitos à escravatura, fortalecestes B. Pedro com admirável caridade e paciência para os auxiliar, concedei-nos por sua intercessão que, procurando o que é pertença de Jesus Cristo, amemos o próximo com obras e em verdade. Pelo mesmo nosso Senhor \emph{\&c.}
}\end{paracol}

\paragraphinfo{Epístola}{}
\begin{paracol}{2}\latim{
Lectio Epístolæ beati Pauli Apostoli ad Corinthios.
}\switchcolumn\portugues{
Lição do Profeta Isaías.
}\switchcolumn*\latim{
\rlettrine{F}{ratres:}
}\switchcolumn\portugues{
\rlettrine{I}{sto} diz o Senhor: «Quebrai as algemas da impiedade; tirai do jugo os oprimidos; deixai livres aqueles que estão cativos; desonerai-os de toda a sujeição. Reparti o vosso pão com os que têm fome; hospedai em vossa casa os pobres e os peregrinos; quando virdes um nu, cobri-o, não desprezando a vossa carne. Então, como a aurora, romperá a vossa luz; a vossa saúde depressa surgirá; a vossa justiça raiará na vossa face; e o esplendor do Senhor vos protegerá. E invocareis o Senhor, que vos ouvirá; Chamareis por Ele e vos dirá: «Eis-me aqui!». Quando abrirdes a vossa alma ao faminto e saciardes as suas almas aflitas, nascerá nas trevas a vossa luz e as trevas tornar-se-ão como a luz do meio-dia».
}\end{paracol}

\paragraphinfo{Gradual}{}
\begin{paracol}{2}\latim{
\rlettrine{}{}
}\switchcolumn\portugues{
\rlettrine{L}{ivrará} das mãos do poderoso o pobre e o indigente, desprovido de auxílio: usará de clemência para com o pobre e o desprovido: e salvará as almas dos pobres. Resgatará as suas almas da usura e da iniquidade e seus nomes serão respeitados diante dele.
}\switchcolumn*\latim{

}\switchcolumn\portugues{
Aleluia, aleluia. Erguei-Vos, ó Senhor Deus, elevai a vossa mão; não Olvideis os pobres. A Vós se abandona o pobre: sereis o amparo do órfão. Aleluia.
}\end{paracol}

\paragraphinfo{Evangelho}{}
\begin{paracol}{2}\latim{
\cruz Sequéntia sancti Evangélii secúndum Lucam.
}\switchcolumn\portugues{
\cruz Continuação do santo Evangelho segundo S. Lucas.
}\switchcolumn*\latim{
\blettrine{I}{n}
}\switchcolumn\portugues{
\blettrine{N}{aquele} tempo, um certo doutor da lei, querendo justificar-se, disse a Jesus: «E quem é o meu próximo?». Então Jesus, tomando a palavra, disse: «Caminhando um homem de Jerusalém para Jericó, caiu em poder dos ladrões, que o despojaram e feriram; depois, abandonaram-no Semimorto. Passou por aquele caminho um sacerdote, que viu o ferido e continuou a sua jornada. Passando também por ali um levita, viu igualmente o ferido, aproximou-se um pouco e seguiu a sua viagem. Por fim, viajando por ali um samaritano, passou perto do ferido, viu-o e encheu-se de Compaixão. Logo, aproximou-se dele, deitou óleo e vinho nas feridas, cingiu-as, montou-o no seu próprio jumento, conduziu-o a uma hospedaria e cuidou dele. No outro dia, tirou dois dinheiros da sua bolsa, deu-os ao hospedeiro e disse-lhe: «Cuida deste homem; e tudo o mais que gastares eu te pagarei quando aqui voltar». «Ora», perguntou Jesus ao doutor, «qual destes três te parece ser o próximo do homem que caiu em poder dos ladrões?». O doutor respondeu: «Aquele que teve misericórdia dele». E Jesus terminou: «Pois tu vai e procede semelhantemente».
}\end{paracol}

\paragraphinfo{Ofertório}{}
\begin{paracol}{2}\latim{
\rlettrine{}{}
}\switchcolumn\portugues{
\rlettrine{L}{ivrei} o pobre, que gritava, assim como o órfão, desprovido de auxílio; a bênção do agonizante descia sobre mim; consolei o coração da viúva. Fui olhos para o cego e pés para o coxo: fui o pai dos pobres.
}\end{paracol}

\paragraph{Secreta}
\begin{paracol}{2}\latim{
\rlettrine{}{} \emph{\&c.}
}\switchcolumn\portugues{
\qlettrine{Q}{ue} pela vossa misericórdia, Senhor, nos seja propícia a vítima de caridade, que Vos oferecemos imolando-a, e, pelas preces e méritos do B. Pedro, permiti que ela nos seja eficaz e salutar para obter aumento de paciência e de caridade. Pelo mesmo nosso Senhor \emph{\&c.}
}\end{paracol}

\paragraphinfo{Comúnio}{}
\begin{paracol}{2}\latim{
\rlettrine{}{}
}\switchcolumn\portugues{
\rlettrine{A}{pascentei} as minhas ovelhas e levei-as ao repouso, diz o Senhor Deus. Procurei a que se perdera; conduzi ao redil a que se tresmalhara; sarei as que tinham membros quebrados; e fortaleci as que estavam fracas.
}\end{paracol}

\paragraph{Postcomúnio}
\begin{paracol}{2}\latim{
\rlettrine{}{} \emph{\&c.}
}\switchcolumn\portugues{
\rlettrine{C}{resça} em nós, Senhor, o salutar efeito da vossa piedade, a fim de que, saciados com o alimento celeste, possamos com felicidade alcançar, por intercessão do B. Pedro, a coroa da vida eterna. Por nosso Senhor \emph{\&c.}
}\end{paracol}