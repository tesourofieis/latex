\subsectioninfo{Cinco Chagas de N. S. Jesus Cristo}{13 de Fevereiro}

\textit{Como Missa Votiva da Paixão, na página \pageref{paixao} ,excepto:}

\subsubsection{Oração}
\begin{paracol}{2}\latim{
\rlettrine{}{} \emph{\&c.}
}\switchcolumn\portugues{
\slettrine{Ó}{} Deus, que pela Paixão do vosso Filho Unigénito e pela efusão do sangue das suas Cinco Sagradas Chagas reparastes a natureza humana, perdida pelo pecado, concedei-nos, Vos imploramos, que, venerando na terra as suas Cinco Chagas, mereçamos alcançar no céu o fruto do mesmo preciosíssimo Sangue. Pelo mesmo nosso Senhor \emph{\&c.}
}\end{paracol}

\subsubsection{Secreta}
\begin{paracol}{2}\latim{
\rlettrine{}{} \emph{\&c.}
}\switchcolumn\portugues{
\qlettrine{Q}{ue} a vossa majestade, Senhor, Vos imploramos, aceite as ofertas que Vos apresentamos, nas quais estão as próprias Cinco Chagas do vosso Unigénito, que são o preço da nossa liberdade. Pelo mesmo nosso Senhor \emph{\&c.}
}\end{paracol}

\subsubsection{Postcomúnio}
\begin{paracol}{2}\latim{
\rlettrine{}{} \emph{\&c.}
}\switchcolumn\portugues{
\rlettrine{A}{gora,} que fomos refeitos com os alimentos da vida, Vos suplicamos, ó Senhor, nosso Deus, que, venerando hoje devotamente as Chagas de N. S. Jesus Cristo, mostremos na nossa vida e costumes que as temos impressas nos nossos corações. Pelo mesmo nosso Senhor \emph{\&c.}
}\end{paracol}
