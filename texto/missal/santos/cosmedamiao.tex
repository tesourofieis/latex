\subsectioninfo{S. S. Cosme e Damião, Mártires}{27 de Setembro}

\textit{Como na Missa Sapiéntiam sanctórum, página \pageref{muitosmartires2}, excepto:}

\paragraph{Oração}
\begin{paracol}{2}\latim{
\rlettrine{P}{ræsta,} quǽsumus, omnípotens Deus: ut, qui sanctórum Mártyrum tuórum Cosmæ et Damiáni natalítia cólimus, a cunctis malis imminéntibus, eórum intercessiónibus, liberémur. Per Dóminum nostrum \emph{\&c.}
}\switchcolumn\portugues{
\rlettrine{C}{oncedei-nos,} ó Deus omnipotente, Vos rogamos, que, celebrando o nascimento no céu dos vossos B. B. Mártires Cosme e Damião, sejamos livres, graças à sua intercessão, de todos os males que nos ameaçam. Por nosso Senhor \emph{\&c.}
}\end{paracol}

\paragraphinfo{Gradual}{Sl. 33, 18-19}
\begin{paracol}{2}\latim{
\rlettrine{C}{lamavérunt} justi, et Dóminus exaudívit eos: et ex ómnibus tribulatiónibus eórum liberávit eos. ℣. Juxta est Dóminus his, qui tribuláto sunt corde: et húmiles spíritu salvabit.
}\switchcolumn\portugues{
\rlettrine{C}{lamaram} os justos; então o Senhor ouviu-os e livrou-os de todas suas aflições. ℣. O Senhor está próximo daqueles que têm o coração atribulado; e salvará os que têm o espírito humilhado.
}\switchcolumn*\latim{
Allelúja, allelúja. ℣. Hæc est vera fratérnitas, quæ vicit mundi crímina: Christum secuta est, ínclita tenens regna cœléstia. Allelúja.
}\switchcolumn\portugues{
Aleluia, aleluia. ℣. Esta é a verdadeira fraternidade que venceu os crimes do mundo: Ela seguiu Cristo, possuindo gloriosamente o reino celestial. Aleluia.
}\end{paracol}

\paragraphinfo{Ofertório}{Sl. 5, 12-13}
\begin{paracol}{2}\latim{
\rlettrine{G}{loriabúntur} in te omnes, qui díligunt nomen tuum: quóniam tu, Dómine, benedíces justo: Dómine, ut scuto bonæ voluntátis tuæ coronásti nos.
}\switchcolumn\portugues{
\rlettrine{E}{m} Vós se alegrarão, Senhor, os que amam o vosso nome; pois, Senhor, abençoais o justo. Rodeastes-nos, Senhor, com vosso amor, como se fora um escudo.
}\end{paracol}

\paragraph{Secreta}
\begin{paracol}{2}\latim{
\rlettrine{S}{anctórum} tuórum nobis, Dómine, pia non desit orátio: quæ et múnera nostra concíliet, et tuam nobis indulgéntiam semper obtíneat. Per Dóminum \emph{\&c.}
}\switchcolumn\portugues{
\qlettrine{Q}{ue} a piedosa oração dos vossos Santos, Senhor, nos não falte; e que Vos torne recomendáveis as nossas ofertas e nos obtenha sempre a vossa misericórdia. Por nosso Senhor \emph{\&c.}
}\end{paracol}

\paragraphinfo{Comúnio}{Sl. 78. 2 \& 11}
\begin{paracol}{2}\latim{
\rlettrine{P}{osuérunt} mortália servórum tuórum, Dómine, escas volatilíbus cœli, carnes Sanctórum tuórum béstiis terræ: secúndum magnitúdinem bráchii tui pósside fílios morte punitórum.
}\switchcolumn\portugues{
\rlettrine{D}{eram} os cadáveres dos vossos servos, Senhor, em alimento às aves do céu, e as carnes dos vossos Santos às feras da terra. Com o poder do vosso braço, salvai os filhos daqueles que foram condenados à morte.
}\end{paracol}

\paragraph{Postcomúnio}
\begin{paracol}{2}\latim{
\rlettrine{P}{rótegat,} quǽsumus, Dómine, pópulum tuum et participátio cœléstis indúlta convívii, et deprecátio colláta Sanctórum. Per Dóminum \emph{\&c.}
}\switchcolumn\portugues{
\qlettrine{Q}{ue} o vosso povo, Senhor, Vos suplicamos, seja protegido pela participação, do celestial banquete, que lhe proporcionastes, e pela intercessão, que lhe concedestes, dos vossos Santos. Por nosso Senhor \emph{\&c.}
}\end{paracol}
