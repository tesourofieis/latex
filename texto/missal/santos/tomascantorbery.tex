\subsectioninfo{S. Tomás de Cantorbéry, B. e M.}{29 de Dezembro}\label{tomascantorbery}

\paragraph{Intróito}
\begin{paracol}{2}\latim{
\rlettrine{G}{audeámus} omnes in Dómino, diem festum celebrántes sub honóre beáti Thomæ Mártyris: de cujus passióne gaudent Angeli et colláudant Fílium Dei. \emph{Ps. 32, 1} Exsultáte, justi, in Dómino: rectos decet collaudátio.
℣. Gloria Patri \emph{\&c.}
}\switchcolumn\portugues{
\rlettrine{R}{egozijemo-nos} todos no Senhor, celebrando neste dia a festa em honra do Mártir Tomás, de cujo martírio se alegram os Anjos, que por este motivo louvam o Filho de Deus. \emph{Sl. 32, 1} Ó justos, cantai ao Senhor hinos de alegria! É aos corações rectos que incumbe louvar o Senhor.
℣. Glória ao Pai \emph{\&c.}
}\end{paracol}

\paragraph{Oração}
\begin{paracol}{2}\latim{
\rlettrine{D}{eus,} pro cujus Ecclésia gloriósus Póntifex Thomas gládiis impiórum occúbuit: præsta, quǽsumus; ut omnes, qui ejus implórant auxílium, petitiónis suæ salutárem consequántur efféctum. Per Dóminum \emph{\&c.}
}\switchcolumn\portugues{
\slettrine{Ó}{} Deus, em defesa de cuja Igreja o glorioso Pontífice Tomás morreu sob o gládio dos ímpios, permiti, Vos pedimos, que todos aqueles que imploram o seu auxílio obtenham o efeito salutar de suas súplicas. Por nosso Senhor \emph{\&c.}
}\end{paracol}

\paragraphinfo{Epístola}{Heb. 5, 1-6}
\begin{paracol}{2}\latim{
Léctio Epístolæ beáti Pauli Apóstoli ad Hebrǽos.
}\switchcolumn\portugues{
Lição da Ep.ª do B. Ap.º Paulo aos Hebreus.
}\switchcolumn*\latim{
\rlettrine{F}{ratres:} Omnis póntifex ex homínibus assúmptus, pro homínibus constitúitur in iis, quæ sunt ad Deum: ut ófferat dona, et sacrifícia pro peccátis: qui condolére possit iis, qui ígnorant et errant: quóniam et ipse circúmdatus est infirmitáte: et proptérea debet, quemádmodum pro pópulo, ita étiam et pro semetípso offérre pro peccátis. Nec quisquam sumit sibi honórem, sed qui vocátur a Deo, tamquam Aaron. Sic et Christus non semetípsum clarificávit, ut Póntifex fíeret: sed qui locútus est ad eum: Fílius meus es tu, ego hódie génui te. Quemádmodum et in álio loco dicit: Tu es sacérdos in ætérnum, secúndum órdinem Melchísedech.
}\switchcolumn\portugues{
\rlettrine{M}{eus} irmãos: Todo o pontífice é escolhido entre os homens e estabelecido para os homens no que respeita às suas relações com Deus, a fim de que ofereça dons e sacrifícios pelos pecados e se compadeça daqueles que pecam por ignorância e por erro, lembrando-se de que também está cheio de fraquezas e deve oferecer sacrifícios de expiação dos pecados por si e pelo povo. Ninguém assuma por si próprio esta honra, mas espere que seja chamado por Deus, como Aarão; pois Cristo não assumiu por si próprio a glória do pontificado, mas recebeu-a d’Aquele que lhe disse: «Tu és o meu Filho; gerei-te hoje». E também lhe disse em outra ocasião: «Tu és sacerdote para sempre, segundo a ordem de Melquisedeque».
}\end{paracol}

\paragraphinfo{Gradual}{Ecl. 44, 16}
\begin{paracol}{2}\latim{
\rlettrine{E}{cce} sacérdos magnus, qui in diébus suis plácuit Deo. ℣. \emph{Ib., 20} Non est invéntus símilis illi, qui conserváret legem Excélsi.
}\switchcolumn\portugues{
\rlettrine{E}{is} o sumo sacerdote, que nos dias da sua vida agradou a Deus. ℣. \emph{Ib., 20} Não foi encontrado outrem que o igualasse na observância da lei do Altíssimo.
}\switchcolumn*\latim{
Allelúja, allelúja. ℣. \emph{Joann. 10, 14} Ego sum pastor bonus: et cognósco oves meas, et cognóscunt me meæ. Allelúja.
}\switchcolumn\portugues{
Aleluia, aleluia. ℣. \emph{Jo. 10, 14} Eu sou o bom Pastor: eu conheço as minhas ovelhas e as minhas ovelhas conhecem-me. Aleluia.
}\end{paracol}

\paragraphinfo{Evangelho}{Jo. 10, 11-16}
\begin{paracol}{2}\latim{
\cruz Sequéntia sancti Evangélii secundum Joannem.
}\switchcolumn\portugues{
\cruz Continuação do santo Evangelho segundo S. João.
}\switchcolumn*\latim{
\blettrine{I}{n} illo témpore: Dixit Jesus pharisǽis: Ego sum pastor bonus. Bonus pastor ánimam suam dat pro óvibus suis. Mercennárius autem, et qui non est pastor, cujus non sunt oves própriæ, videt lupum veniéntem, et dimíttit oves et fugit: et lupus rapit et dispérgit oves; mercennárius autem fugit, quia mercennárius est et non pértinet ad eum de óvibus. Ego sum pastor bonus: et cognósco meas et cognóscunt me meæ. Sicut novit me Pater, et ego agnósco Patrem, et ánimam meam pono pro óvibus meis. Et álias oves hábeo, quæ non sunt ex hoc ovíli: et illas opórtet me addúcere, et vocem meam áudient, et fiet unum ovíle et unus pastor.
}\switchcolumn\portugues{
\blettrine{N}{aquele} tempo, disse Jesus aos fariseus: «Eu sou o bom Pastor. O bom Pastor dá a vida pelas ovelhas. Porém, o mercenário, que não é pastor e a quem as ovelhas não pertencem, vê vir o lobo, abandona as ovelhas e foge. E, então, o lobo arrebata as ovelhas e dispersa-as. O mercenário procede assim porque é mercenário e porque não tem cuidado com as ovelhas. Eu sou o bom Pastor: eu conheço as minhas ovelhas, e as minhas ovelhas conhecem-me; assim como meu Pai me conhece e eu conheço meu Pai. Eu dou a minha vida pelas minhas ovelhas. Tenho ainda outras ovelhas que não pertencem a este aprisco; mas é preciso que eu as atraia e ouçam a minha voz, para que não haja senão um só aprisco e um só Pastor».
}\end{paracol}

\paragraphinfo{Ofertório}{Sl. 20, 4-5}
\begin{paracol}{2}\latim{
\rlettrine{P}{osuísti,} Dómine, in capite ejus corónam de lápide pretióso: vitam pétiit a te, et tribuísti ei, allelúja.
}\switchcolumn\portugues{
\rlettrine{I}{mpusestes} na sua cabeça, Senhor, uma coroa de pedras preciosas; e concedestes-lhe a vida que Vos Implorou. Aleluia.
}\end{paracol}

\paragraph{Secreta}
\begin{paracol}{2}\latim{
\rlettrine{M}{únera} tibi, Dómine, dicáta sanctífica: et, intercedénte beáto Thoma Mártyre tuo atque Pontífice, per éadem nos placátus inténde. Per Dóminum \emph{\&c.}
}\switchcolumn\portugues{
\rlettrine{S}{antificai,} Senhor, estes dons que Vos oferecemos, e, por intercessão do B. Tomás, vosso Mártir e Pontífice, e por virtude dos mesmos dons, dignai-Vos olhar aplacado para nós. Por nosso Senhor \emph{\&c.}
}\end{paracol}

\paragraphinfo{Comúnio}{Jo. 10, 14}
\begin{paracol}{2}\latim{
\rlettrine{E}{go} sum pastor bonus: et cognósco oves meas, et cognóscunt me meæ.
}\switchcolumn\portugues{
\rlettrine{E}{u} sou o bom Pastor: conheço as minhas ovelhas e as minhas ovelhas conhecem-me.
}\end{paracol}

\paragraph{Postcomúnio}
\begin{paracol}{2}\latim{
\rlettrine{H}{æc} nos commúnio, Dómine, purget a crímine: et, intercedénte beáto Thoma Mártyre tuo atque Pontífice, cœléstis remédii fáciat esse consórtes. Per Dóminum nostrum \emph{\&c.}
}\switchcolumn\portugues{
\qlettrine{Q}{ue} esta comunhão, Senhor, nos purifique dos nossos crimes, e, intercedendo o B. Tomás, vosso Mártir e Pontífice, nos torne participantes do remédio celestial. Por nosso Senhor \emph{\&c.}
}\end{paracol}