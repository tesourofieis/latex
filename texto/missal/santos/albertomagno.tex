\subsectioninfo{S. Alberto Magno, B. C. e Doutor}{15 de Novembro}

\textit{Como na Missa In médio Ecclésiae, página \pageref{doutores}, excepto:}

\paragraph{Oração}
\begin{paracol}{2}\latim{
\rlettrine{D}{eus,} qui beátum Albértum Pontíficem tuum atque Doctórem in humána sapiéntia divínæ fídei subjiciénda magnum effecísti: da nobis, quǽsumus; ita ejus magistérii inhærére vestígiis, ut luce perfécta fruámur in cœlis. Per Dóminum \emph{\&c.}
}\switchcolumn\portugues{
\slettrine{Ó}{} Deus, que ao B. Alberto, vosso Pontífice e Doutor, tornastes grande na arte de sujeitar a sabedoria humana à fé divina, concedei-nos, Vos suplicamos, que de tal modo sigamos as lições do seu magistério que nos céus gozemos a luz perfeita. Por nosso Senhor \emph{\&c.}
}\end{paracol}

\paragraph{Secreta}
\begin{paracol}{2}\latim{
\rlettrine{S}{acrifíciis} præséntibus, Dómine, quǽsumus, inténde placátus: ut quod Passiónis Fílii tui Dómini nostri mystério gérimus, beáti Alberti intercessióne et exémplo, pio consequámur afféctu. Per eumdem Dóminum \emph{\&c.}
}\switchcolumn\portugues{
\rlettrine{P}{elos} presentes sacrifícios, Senhor, Vos suplicamos, olhai aplacado para nós, a fim de que com o exemplo e intercessão do B. Alberto consigamos alcançar piedosos afectos pelo mistério, que celebramos, da paixão do vosso Filho e nosso Senhor. Pelo mesmo nosso Senhor \emph{\&c.}
}\end{paracol}

\paragraph{Postcomúnio}
\begin{paracol}{2}\latim{
\rlettrine{P}{er} hæc sancta quæ súmpsimus, ab hóstium nos, Dómine, impugnatióne defénde: et intercedénte beáto Albérto Confessóre tuo atque Pontífice, perpétua pace respiráre concéde; Per Dóminum \emph{\&c.}
}\switchcolumn\portugues{
\rlettrine{P}{or} estes sacrossantos sacramentos, que recebemos, Senhor, defendei-nos dos ataques dos nossos inimigos, e, intercedendo o B. Alberto, vosso Confessor e Pontífice, permiti que gozemos a paz perpétua. Por nosso Senhor \emph{\&c.}
}\end{paracol}