\subsectioninfo{S. Inácio de Loiola}{31 de Julho}

\paragraphinfo{Intróito}{Fl. 2, 10-11}
\begin{paracol}{2}\latim{
\rlettrine{I}{n} nómine Jesu omne genu fléctitur, cœléstium, terréstrium et infernórum: et omnis lingua confiteátur, quia Dóminus Jesus Christus in glória est Dei Patris. \emph{Ps. 5, 12-13} Gloriabúntur in te omnes, qui díligunt nomen tuum: quóniam tu benedíces justo.
℣. Gloria Patri \emph{\&c.}
}\switchcolumn\portugues{
\qlettrine{Q}{ue} ao ser pronunciado o nome de Jesus se dobrem todos os joelhos dos que estão no céu, na terra e no inferno: e que toda a língua humana confesse que o Senhor Jesus Cristo está na glória de Deus Pai. \emph{Sl. 5, 12-13} Em Vós serão glorificados todos aqueles que amam o vosso nome, pois abençoais o justo.
℣. Glória ao Pai \emph{\&c.}
}\end{paracol}

\paragraph{Oração}
\begin{paracol}{2}\latim{
\rlettrine{D}{eus,} qui ad majórem tui nóminis glóriam propagándam, novo per beátum Ignátium subsídio militántem Ecclésiam roborásti: concéde; ut, ejus auxílio et imitatióne certántes in terris, coronári cum ipso mereámur in cœlis. Per Dóminum \emph{\&c.}
}\switchcolumn\portugues{
\slettrine{Ó}{} Deus, que, para a propagação da maior glória do vosso nome fortalecestes a vossa Igreja militante com um novo subsídio pelo B. Inácio, permiti que, combatendo nós na terra, como ele e com seu auxílio, mereçamos ser coroados com ele nos céus. Por nosso Senhor \emph{\&c.}
}\end{paracol}

\paragraphinfo{Epístola}{Página \pageref{martirnaopontifice2}}

\paragraphinfo{Gradual}{Página \pageref{confessoresnaopontifices1}}

\paragraphinfo{Evangelho}{Página \pageref{tito}}

\paragraphinfo{Ofertório}{Sl. 88, 25}
\begin{paracol}{2}\latim{
\rlettrine{V}{éritas} mea et misericórdia mea cum ipso: et in nómine meo exaltábitur cornu ejus.
}\switchcolumn\portugues{
\rlettrine{A}{} minha fidelidade e a minha misericórdia estarão com eles, e por virtude do meu nome será exaltado o seu poder.
}\end{paracol}

\paragraph{Secreta}
\begin{paracol}{2}\latim{
\rlettrine{A}{dsint,} Dómine Deus, oblatiónibus nostris sancti Ignátii benígna suffrágia: ut sacrosáncta mystéria, in quibus omnis sanctitátis fontem constituísti, nos quoque in veritáte sanctíficet. Per Dóminum nostrum \emph{\&c.}
}\switchcolumn\portugues{
\qlettrine{Q}{ue} a benigna intercessão de santo Inácio acompanhe as nossas oblatas, ó Senhor, nosso Deus, a fim de que os sacrossantos mystérios, de que fazeis depender a origem de toda a santidade, nos alcancem a verdadeira santificação. Por nosso Senhor \emph{\&c.}
}\end{paracol}

\paragraphinfo{Comúnio}{Lc. 12, 49}
\begin{paracol}{2}\latim{
\rlettrine{I}{gnem} veni míttere in terram: et quid volo, nisi ut accendátur?
}\switchcolumn\portugues{
\rlettrine{E}{u} vim trazer o fogo à terra: e que quero senão que ele se acenda?
}\end{paracol}

\paragraph{Postcomúnio}
\begin{paracol}{2}\latim{
\rlettrine{L}{audis} hóstia, Dómine, quam pro sancto Ignátio grátias agentes obtúlimus: ad perpétuam nos majestátis tuæ laudatiónem, ejus intercessióne, pérducat. Per Dóminum \emph{\&c.}
}\switchcolumn\portugues{
\rlettrine{S}{enhor,} que este sacrifício de louvor, que Vos é oferecido em acção de graças em honra de Santo Inácio, nos alcance por sua intercessão o céu, onde a vossa majestade recebe perpétuo louvor. Por nosso Senhor \emph{\&c.}
}\end{paracol}
