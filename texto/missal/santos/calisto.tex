\subsectioninfo{S. Calisto I, Papa e Mártir}{14 de Outubro}

\textit{Como na Missa Si díligis me, página \pageref{sumospontifices}, excepto:}

\paragraph{Oração}
\begin{paracol}{2}\latim{
\rlettrine{D}{eus,} qui nos cónspicis ex nostra infirmitáte defícere: ad amórem tuum nos misericórditer per Sanctórum tuórum exémpla restáura. Per Dóminum \emph{\&c.}
}\switchcolumn\portugues{
\slettrine{Ó}{} Deus, que nos vedes desfalecer por causa da nossa fraqueza, dignai-Vos pela vossa misericórdia restaurar-nos no vosso amor, segundo os exemplos dos vossos Santos. Por nosso Senhor \emph{\&c.}
}\end{paracol}

\paragraph{Secreta}
\begin{paracol}{2}\latim{
\rlettrine{M}{ýstica} nobis, Dómine, prosit oblátio: quæ nos et a 
reátibus nostris expédiat, et perpétua salvatióne confírmet. Per Dóminum \emph{\&c.}
}\switchcolumn\portugues{
\qlettrine{Q}{ue} esta mística oblação nos aproveite, Senhor; e que nos livre das nossas faltas e nos assegure a salvação eterna. Por nosso Senhor \emph{\&c.}
}\end{paracol}

\paragraph{Postcomúnio}
\begin{paracol}{2}\latim{
\qlettrine{Q}{uǽsumus,} omnípotens Deus: ut reátus nostros múnera sacráta puríficent, et recte vivéndi nobis operéntur efféctum. Per Dóminum \emph{\&c.}
}\switchcolumn\portugues{
\rlettrine{D}{ignai-Vos} permitir, ó Deus omnipotente, que estes dons nos purifiquem das nossas faltas e que, produzindo em nós os seus efeitos, nos façam viver santamente. Por nosso Senhor \emph{\&c.}
}\end{paracol}