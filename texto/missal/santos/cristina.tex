\subsectioninfo{S. Cristina, Virgem e Mártir}{24 de Julho}

\textit{Como na Missa Me exspectavérunt, página \pageref{virgensmartires2}.}

\paragraph{Oração}
\begin{paracol}{2}\latim{
\rlettrine{I}{ndulgéntiam} nobis, quǽsumus, Dómine, beáta Christína Virgo et Martyr implóret: quæ tibi grata semper éxstitit, et merito castitátis, et tuæ professióne virtútis. \emph{\&c.}
}\switchcolumn\portugues{
\rlettrine{C}{oncedei-nos,} Senhor, Vos pedimos que alcancemos o perdão dos nossos pecados pela intercessão da B. Catarina, Virgem e Mártir, que sempre Vos foi agradável não só pelos méritos da castidade, mas também pela prática da vossa virtude. \emph{\&c.}
}\end{paracol}

\paragraph{Secreta}
\begin{paracol}{2}\latim{
\rlettrine{H}{óstias} tibi, Dómine, beátæ Christínæ Vírginis et Martyris tuæ dicátas méritis, benígnus assúme: et ad perpétuum nobis tríbue proveníre subsídium. \emph{\&c.}
}\switchcolumn\portugues{
\rlettrine{A}{ceitai} benignamente, Senhor, as hóstias que Vos oferecemos pelos méritos da B. Cristina, Virgem e Mártir, e dignai-Vos permitir que nos sirvam de perpétuo socorro. \emph{\&c.}
}\end{paracol}

\paragraph{Postcomúnio}
\begin{paracol}{2}\latim{
\rlettrine{D}{ivíni} númeris largitáte satiáti, quǽsumus, Dómine, Deus noster: ut, intercedénte beáta Christína Vírgine et Mártyre tua, in ejus semper participatióne vivámus.
 \emph{\&c.}
}\switchcolumn\portugues{
\rlettrine{S}{aciados} com a liberdade do dom divino, Senhor, nosso Deus, Vos suplicamos, permiti, pela intercessão da B. Catarina, vossa Virgem, que comparticipemos sempre deste dom durante a vida. \emph{\&c.}
}\end{paracol}