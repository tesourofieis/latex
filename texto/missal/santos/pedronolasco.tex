\subsectioninfo{S. Pedro Nolasco, Conf.}{28 de Janeiro}

\textit{Como Missa Justus ut palma, página \pageref{confessoresnaopontifices2}, excepto:}

\paragraph{Oração}
\begin{paracol}{2}\latim{
\rlettrine{D}{eus,} qui in tuæ caritátis exémplum ad fidélium redemptiónem sanctum Petrum Ecclésiam tuam nova prole fœcundáre divínitus docuísti: ipsíus nobis intercessióne concéde; a peccáti servitúte solútis, in cœlésti pátria perpétua libertáte gaudére: Qui vivis et regnas \emph{\&c.}
}\switchcolumn\portugues{
\slettrine{Ó}{} Deus, que em prova da vossa caridade quisestes inspirar sobrenaturalmente S. Pedro a fundar na vossa Igreja uma nova família, destinada à redenção dos fiéis cativos, concedei-nos por sua intercessão que, livres nós do cativeiro do pecado, gozemos perpétua liberdade na pátria celestial. Ó Vós, que, sendo Deus, viveis \emph{\&c.}
}\end{paracol}

\paragraphinfo{Oração}{Comemoração Santa Inês}
\begin{paracol}{2}\latim{
\rlettrine{D}{eus,} qui nos ánnua beátæ Agnetis Vírginis et Martyris tuæ sollemnitáte lætíficas: da, quǽsumus; ut, quam venerámur officio, étiam piæ conversatiónis sequámur exémplo. Per Dóminum \emph{\&c.}
}\switchcolumn\portugues{
\slettrine{Ó}{} Deus, que nos alegrais com a solenidade anual da B. Inês, vossa Virgem e Mártir, concedei-nos a graça, Vos suplicamos, de imitar os exemplos daquela cuja festa celebramos. Por nosso Senhor \emph{\&c.}
}\end{paracol}

\paragraph{Secreta}
\begin{paracol}{2}\latim{
\rlettrine{L}{audis} tibi, Dómine, hóstias immolámus in tuórum commemoratióne Sanctórum quibus nos et præséntibus éxui malis confídimus et futúris. Per Dóminum \emph{\&c.}
}\switchcolumn\portugues{
\rlettrine{V}{os} oferecemos este sacrifício de louvor em memória dos vossos Santos, para que por meio deles nos livremos dos males presentes e futuros. Por nosso Senhor \emph{\&c.}
}\end{paracol}

\paragraphinfo{Secreta}{Comemoração Santa Inês}
\begin{paracol}{2}\latim{
\rlettrine{S}{uper} has, quǽsumus, Dómine, hóstias benedíctio copiósa descéndat: quæ et sanctificatiónem nobis cleménter operétur, et de Mártyrum 
nos sollemnitáte lætíficet. Per Dóminum \emph{\&c.}
}\switchcolumn\portugues{
\qlettrine{Q}{ue} estas hóstias, Senhor, que Vos oferecemos façam descer sobre nós uma bênção abundante, a qual produza em nós por vossa clemência nossa santificação e nos alegre com a solenidade dos vossos Mártires. Por nosso Senhor \emph{\&c.}
}\end{paracol}

\paragraph{Postcomúnio}
\begin{paracol}{2}\latim{
\rlettrine{R}{efécti} cibo potúque cœlésti, Deus noster, te súpplices exorámus: ut, in cujus hæc commemoratióne percépimus, ejus muniámur et précibus Per Dóminum \emph{\&c.}
}\switchcolumn\portugues{
\rlettrine{F}{ortalecidos} com o alimento e com a bebida celestiais, Vos suplicamos humildemente, ó nosso Deus, que nos protejam as preces daquele em cuja memória os recebermos. Por nosso Senhor \emph{\&c.}
}\end{paracol}

\paragraphinfo{Postcomúnio}{Comemoração Santa Inês}
\begin{paracol}{2}\latim{
\rlettrine{S}{úmpsimus,} Dómine, celebritátis ánnuæ votiva sacraménta: præsta, quǽsumus; ut et temporális vitæ nobis remédia prǽbeant et ætérnæ. Per Dóminum \emph{\&c.}
}\switchcolumn\portugues{
\rlettrine{S}{enhor,} havendo recebido os sacramentos que Vos são oferecidos nesta festa anual, concedei-nos, Vos suplicamos, que eles nos alcancem os remédios para a vida presente e para a eterna. Por nosso Senhor \emph{\&c.}
}\end{paracol}