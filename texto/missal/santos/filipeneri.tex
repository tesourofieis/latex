\subsectioninfo{S. Filipe de Néri}{26 de Maio}

\paragraphinfo{Intróito}{Rm. 5, 5}
\begin{paracol}{2}\latim{
\rlettrine{C}{áritas} Dei diffúsa est in córdibus nostris per inhabitántem Spíritum ejus in nobis. (T.P. Allelúja, allelúja.) \emph{Ps. 102. 1} Benedic, ánima mea, Dómino: et ómnia, quæ intra me sunt, nómini sancto ejus.
℣. Gloria Patri \emph{\&c.}
}\switchcolumn\portugues{
\rlettrine{O}{} amor de Deus difundiu-se nos nossos corações pelo seu Espírito que habita em nós. (T.P. Aleluia, aleluia.) \emph{Sl. 102. 1} Bendizei, ó minha alma, o Senhor: que tudo quanto me pertence bendiga o seu santo nome.
℣. Glória ao Pai \emph{\&c.}
}\end{paracol}

\paragraph{Oração}
\begin{paracol}{2}\latim{
\rlettrine{D}{eus,} qui beátum Philippum Confessórem tuum Sanctórum tuórum glória sublimásti: concéde propítius; ut, cujus sollemnitáte lætámur, ejus virtútum proficiámus exémplo. Per Dóminum \emph{\&c.}
}\switchcolumn\portugues{
\slettrine{Ó}{} Deus, que elevastes o B. Filipe, vosso Confessor, à sublime glória dos vossos Santos, concedei-nos propício que, celebrando com alegria esta festa, alcancemos proveito com o exemplo das suas virtudes. Por nosso Senhor \emph{\&c.}
}\end{paracol}

\paragraphinfo{Epístola}{Página \pageref{tomasaquino}}

\paragraphinfo{Gradual}{Sl. 33, 12 \& 6}
\begin{paracol}{2}\latim{
\rlettrine{V}{eníte,} fílii, audíte me: timórem Dómini docébo vos. ℣. Accédite ad eum, et illuminámini: et fácies vestræ non confundéntur.
}\switchcolumn\portugues{
\rlettrine{V}{inde,} meus filhos, e escutai-me: Ensinar-vos-ei a temer o Senhor. ℣. Aproximai-Vos d’Ele e ficareis iluminados: então a vossa face não ficará envergonhada.
}\switchcolumn*\latim{
Allelúja, allelúja. ℣. \emph{Thren. 1, 13} De excélso misit ignem in óssibus meis, et erudívit me. Allelúja.
}\switchcolumn\portugues{
Aleluia, aleluia. ℣. \emph{Lm. 1, 13} Do alto dos céus enviou e fogo sobre os meus ossos e instruiu-me. Aleluia.
}\end{paracol}

\textit{Durante o Tempo Pascal omite-se o Gradual e diz-se a seguinte Aleluia:}

\begin{paracol}{2}\latim{
Allelúja, allelúja. ℣. \emph{Thren. 1, 13} De excélso misit ignem in óssibus meis, et erudívit me. Allelúja. ℣. \emph{Ps. 38, 4} Concáluit cor meum intra me: et in meditatióne mea exardéscet ignis. Allelúja.
}\switchcolumn\portugues{
Aleluia, aleluia. ℣. \emph{Lm. 1, 13} Do alto dos céus enviou e fogo sobre os meus ossos e instruiu-me. Aleluia. ℣. \emph{Sl. 38, 4} Meu coração inflamou-se no meu peito. Enquanto eu meditava, o fogo abrasou-me. Aleluia.
}\end{paracol}

\paragraphinfo{Evangelho}{Página \pageref{confessoresnaopontifices1}}

\paragraphinfo{Ofertório}{Sl. 18, 32}
\begin{paracol}{2}\latim{
\rlettrine{V}{iam} mandatórum tuórum cucúrri, cum dilatásti cor meum. (T.P. Allelúja.)
}\switchcolumn\portugues{
\rlettrine{E}{u} corri pelo caminho dos vossos mandamentos, porquanto dilatastes o meu coração. (T.P. Aleluia.)
}\end{paracol}

\paragraph{Secreta}
\begin{paracol}{2}\latim{
\rlettrine{S}{acrifíciis} præséntibus, quǽsumus, Dómine, inténde placatus: et præsta; ut illo nos igne Spíritus Sanctus inflámmet, quo beáti Phílippi cor mirabíliter penetrávit. Per Dóminum \emph{\&c.}
}\switchcolumn\portugues{
\rlettrine{O}{lhai} aplacado para o presente sacrifício, Senhor, Vos suplicamos, e fazei que o Espírito Santo nos inflame naquele fogo que penetrou maravilhosamente no coração do B. Filipe. Por nosso Senhor \emph{\&c.}
}\end{paracol}

\paragraphinfo{Comúnio}{Sl. 83, 3}
\begin{paracol}{2}\latim{
\rlettrine{C}{or} meum et caro mea exsultavérunt in Deum vivum. (T.P. Allelúja.)
}\switchcolumn\portugues{
\rlettrine{O}{} meu coração e a minha carne exultaram em Deus vivo. (T.P. Aleluia.)
}\end{paracol}

\paragraph{Postcomúnio}
\begin{paracol}{2}\latim{
\rlettrine{C}{œléstibus,} Dómine, pasti delíciis: quǽsumus; ut beáti Philippi Confessóris tui méritis et imitatióne, semper eadem, per quæ veráciter vívimus, appetámus. Per Dóminum nostrum \emph{\&c.}
}\switchcolumn\portugues{
\rlettrine{A}{limentados} com as celestiais delícias, Senhor, Vos pedimos que, pelos méritos do B. Filipe, vosso Confessor, e imitando os seus exemplos, aspiremos sempre a este alimento, que nos dará a verdadeira vida. Por nosso Senhor \emph{\&c.}
}\end{paracol}
