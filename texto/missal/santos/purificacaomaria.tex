\subsectioninfo{Purificação da B. V. Maria}{2 de Fevereiro}

\subsubsection{Bênção das Velas}

\begin{paracol}{2}\latim{
\begin{nscenter} Orémus. \end{nscenter}
}\switchcolumn\portugues{
\begin{nscenter} Oremos. \end{nscenter}
}\switchcolumn*\latim{
\rlettrine{D}{omine} sancte, Pater omnípotens, ætérne Deus, qui ómnia ex níhilo creásti, et jussu tuo per ópera apum hunc liquorem ad perfectionem cérei veníre fecísti: et qui hodiérna die petitiónem justi Simeónis implésti: te humíliter deprecámur; ut has candélas ad usus hóminum et sanitátem córporum et animárum, sive in terra sive in aquis, per invocatiónem tui sanctíssimi nóminis et per intercessiónem beátæ Maríæ semper Vírginis, cujus hódie festa devóte celebrántur, et per preces ómnium Sanctórum tuórum, bene {\cruz} dícere et sancti {\cruz} ficáre dignéris: et hujus plebis tuæ, quæ illas honorífice in mánibus desíderat portare teque cantando laudare, exáudias voces de cœlo sancto tuo et de sede majestátis tuæ: et propítius sis ómnibus clamántibus ad te, quos redemísti pretióso Sánguine Fílii tui: Qui tecum \emph{\&c.}
}\switchcolumn\portugues{
\rlettrine{S}{enhor} santo, Pai omnipotente, Deus eterno, que criastes todas as cousas do nada e por ordem de quem as abelhas compõem a substância para se formarem estas velas, e que neste dia atendestes à petição do justo Simeão: humildemente Vos rogamos, pela invocação do vosso santo nome, pela intercessão da B. Maria, sempre Virgem, cuja festividade hoje celebramos piedosamente, e pelas preces de todos vossos Santos, que Vos digneis benzer {\cruz} e santificar {\cruz} estas velas para uso dos homens e para a saúde dos corpos e das almas, quer na terra, quer no mar. Ouvi, lá do vosso celestial santuário e do trono da vossa majestade, os votos do vosso povo, aqui presente, que deseja levar reverentemente em suas mãos estas velas e louvar-Vos com seus cânticos; enfim, sede propício para com todos que por Vós clamam, os quais resgatastes pelo precioso Sangue do vosso Filho: Que, sendo Deus, convosco vive e \emph{\&c.}
}\switchcolumn*\latim{
℟. Amen.
}\switchcolumn\portugues{
℟. Amen.
}\switchcolumn*\latim{
\begin{nscenter} Orémus. \end{nscenter}
}\switchcolumn\portugues{
\begin{nscenter} Oremos. \end{nscenter}
}\switchcolumn*\latim{
\rlettrine{O}{mnípotens} sempitérne Deus, qui hodiérna die Unigénitum tuum ulnis sancti Simeónis in templo sancto tuo suscipiéndum præsentásti: tuam súpplices deprecámur cleméntiam; ut has candélas, quas nos fámuli tui, in tui nóminis magnificéntiam suscipiéntes, gestáre cúpimus luce accénsas, bene {\cruz} dícere et sancti {\cruz} ficáre atque lúmine supérnæ benedictiónis accéndere dignéris: quaténus eas tibi Dómino, Deo nostro, offeréndo digni, et sancto igne dulcíssimæ caritátis tuæ succénsi, in templo sancto glóriæ tuæ repræsentári mereámur. Per eúndem Dóminum nostrum \emph{\&c.}
}\switchcolumn\portugues{
\rlettrine{D}{eus} omnipotente e eterno, que hoje apresentastes o vosso Filho Unigénito no vosso templo para que fosse recebido nos braços de S. Simeão, incessantemente suplicamos à vossa clemência se digne benzer {\cruz}, santificar {\cruz} e acender com a luz da vossa bênção estas velas que nós, vossos servos, desejamos levar acesas, depois de as havermos recebido para honra do vosso santo Nome, a fim de que, oferecendo-as a Vós, que sois nosso Deus e Senhor, nos tornemos dignos e sejamos abrasados no fogo sagrado da vossa suavíssima caridade, e depois mereçamos ser apresentados no templo sagrado da vossa glória. Pelo mesmo \emph{\&c.}
}\switchcolumn*\latim{
℟. Amen.
}\switchcolumn\portugues{
℟. Amen.
}\switchcolumn*\latim{
\begin{nscenter} Orémus. \end{nscenter}
}\switchcolumn\portugues{
\begin{nscenter} Oremos. \end{nscenter}
}\switchcolumn*\latim{
\rlettrine{D}{ómine} Jesu Christe, lux vera, quæ illúminas omnem hóminem veniéntem in hunc mundum: effúnde bene {\cruz} dictiónem tuam super hos céreos, et sancti {\cruz} fica eos lúmine grátiæ tuæ, et concéde propítius; ut, sicut hæc luminária igne visíbili accénsa noctúrnas depéllunt ténebras; ita corda nostra invisíbili igne, id est, Sancti Spíritus splendóre illustráta, ómnium vitiórum cæcitáte cáreant: ut, purgáto mentis óculo, ea cérnere póssimus, quæ tibi sunt plácita et nostræ salúti utília; quaténus post hujus sǽculi caliginósa discrímina ad lucem indeficiéntem perveníre mereámur. Per te, Christe Jesu, Salvátor mundi, qui in Trinitáte perfécta vivis et regnas Deus, per ómnia sǽcula sæculórum.
}\switchcolumn\portugues{
\rlettrine{S}{enhor} Jesus Cristo, luz verdadeira que ilumina todo o homem que vem a este mundo, lançai a vossa bênção {\cruz} sobre estas velas e santificai-as {\cruz} com a luz da vossa graça; permiti benigno que, assim como estes círios, brilhando com luz visível, afastam as trevas da noite, assim também os nossos corações, iluminados com o fogo invisível, isto é, ilustrados com o esplendor do Espírito Santo sejam livres da cegueira de todos os vícios, a fim de que, purificados os olhos da nossa alma, possamos conhecer o que Vos é agradável e útil à nossa salvação; e, assim, após as perigosas trevas deste mundo, mereçamos alcançar a posse da luz eterna. Por Vós, ó Jesus Cristo, Salvador do mundo, que, sendo Deus, na Trindade perfeita, viveis e reinais por todos os séculos dos séculos.
}\switchcolumn*\latim{
℟. Amen.
}\switchcolumn\portugues{
℟. Amen.
}\switchcolumn*\latim{
\begin{nscenter} Orémus. \end{nscenter}
}\switchcolumn\portugues{
\begin{nscenter} Oremos. \end{nscenter}
}\switchcolumn*\latim{
\rlettrine{O}{mnípotens} sempitérne Deus, qui per Móysen fámulum tuum puríssimum ólei liquórem ad luminária ante conspéctum tuum júgiter concinnánda præparári jussísti: bene {\cruz} dictiónis tuæ grátiam super hos céreos benígnus infúnde; quaténus sic adminístrent lumen extérius, ut, te donánte, lumen Spíritus tui nostris non desit méntibus intérius. Per Dóminum \emph{\&c.}
}\switchcolumn\portugues{
\rlettrine{D}{eus} omnipotente e eterno, que mandastes preparar por Moisés, vosso servo, óleo puríssimo para as lâmpadas, que incessantemente deviam arder na Vasa presença, infundi benigno a graça da vossa bênção {\cruz} nestas velas, de sorte que, dando-nos elas a luz exterior, não falte interiormente às nossas almas pela vossa graça a luz do vosso Espírito Santo. Por nosso Senhor \emph{\&c.}
}\switchcolumn*\latim{
℟. Amen.
}\switchcolumn\portugues{
℟. Amen.
}\switchcolumn*\latim{
\begin{nscenter} Orémus. \end{nscenter}
}\switchcolumn\portugues{
\begin{nscenter} Oremos. \end{nscenter}
}\switchcolumn*\latim{
\rlettrine{D}{ómine} Jesu Christe, qui hodiérna die, in nostræ carnis substántia inter hómines appárens, a paréntibus in templo es præsentátus: quem Símeon venerábilis senex, lúmine Spíritus tui irradiátus, agnóvit, suscépit et benedíxit: præsta propítius; ut, ejúsdem Spíritus Sancti grátia illumináti atque edócti, te veráciter agnoscámus et fidéliter diligámus: Qui cum Deo Patre in unitáte ejúsdem Spíritus Sancti vivis et regnas Deus, per ómnia sǽtula sæculórum.
}\switchcolumn\portugues{
\rlettrine{S}{enhor} Jesus Cristo, que, aparecendo hoje entre os homens na substância da nossa carne, fostes por vossos Pais apresentado no templo, e a quem o venerável Simeão, ilustrado pela luz do vosso Espírito, aceitou em suas mãos, reconheceu e abençoou, concedei-nos propício que, iluminados e ilustrados pela graça do Espírito Santo, verdadeiramente Vos reconheçamos e fielmente Vos amemos. Ó Vós, que com Deus Pai viveis e reinais \emph{\&c.}
}\end{paracol}

Durante a distribuição das velas canta-se:

\paragraphinfo{Antífona}{Lc. 2, 32}
\begin{paracol}{2}\latim{
\rlettrine{L}{umen} ad revelatiónem géntium et glóriam plebis tuæ Israël.
}\switchcolumn\portugues{
\rlettrine{A}{} luz que iluminará as nações e a glória de Israel, vosso povo.
}\end{paracol}

\paragraphinfo{Cântico}{ibid., 29-31}
\begin{paracol}{2}\latim{
\rlettrine{N}{unc} dimíttis servum tuum, Dómine, secúndum verbum tuum in pace.
}\switchcolumn\portugues{
\rlettrine{A}{gora,} Senhor, deixai ir em paz o vosso servo, segundo a vossa palavra.
}\switchcolumn*\latim{
\emph{Antiph} Lumen \emph{\&c.}
}\switchcolumn\portugues{
\emph{Antífona} A luz \emph{\&c.}
}\switchcolumn*\latim{
Quia vidérunt óculi mei salutáre tuum.
}\switchcolumn\portugues{
Pois os meus olhos já viram a vossa salvação.
}\switchcolumn*\latim{
\emph{Antiph} Lumen \emph{\&c.}
}\switchcolumn\portugues{
\emph{Antífona} A luz \emph{\&c.}
}\switchcolumn*\latim{
Quod parásti ante fáciem ómnium populorum.
}\switchcolumn\portugues{
Que preparastes diante dos olhos de todos os povos.
}\switchcolumn*\latim{
\emph{Antiph} Lumen \emph{\&c.}
}\switchcolumn\portugues{
\emph{Antífona} A luz \emph{\&c.}
}\switchcolumn*\latim{
Glória Patri, et Fílio, et Spíritui Sancto.
}\switchcolumn\portugues{
Glória ao Pai, e ao Filho e ao Espírito Santo.
}\switchcolumn*\latim{
\emph{Antiph} Lumen \emph{\&c.}
}\switchcolumn\portugues{
\emph{Antífona} A luz \emph{\&c.}
}\switchcolumn*\latim{
Sicut erat in pricípio, et nunc, et semper, et in sǽcula sæculórum.
}\switchcolumn\portugues{
Assim como era no princípio, agora e sempre, e por todos os séculos dos
séculos.
}\switchcolumn*\latim{
\emph{Antiph} Lumen \emph{\&c.}
}\switchcolumn\portugues{
\emph{Antífona} A luz \emph{\&c.}
}\end{paracol}

\textit{Depois canta-se:}

\paragraphinfo{Antífona}{Sl. 43, 26}
\begin{paracol}{2}\latim{
\rlettrine{E}{xsúrge,} Dómine, ádjuva nos: et líbera nos propter nomen tuum. Ps. ibid., 2. Deus, áuribus nostris audívimus: patres nostri annuntiavérunt nobis. ℣. Glória Patri \emph{\&c.}
}\switchcolumn\portugues{
\rlettrine{E}{rguei-Vos,} Senhor, auxiliai-nos e salvai-nos, pela honra do vosso nome. Ó Deus, ouvimos com nossos ouvidos, nossos pais contaram-nos as vossas maravilhas. ℣. Glória ao Pai \emph{\&c.}
}\switchcolumn*\latim{
Exsúrge, Dómine \emph{\&c.}
}\switchcolumn\portugues{
Erguei-Vos, Senhor \emph{\&c.}
}\end{paracol}

\textit{Se o dia 2 de Fevereiro for depois da Septuagésima, e não ao Domingo, diz-se:}

\paragraph{Oração}
\begin{paracol}{2}\latim{
℣. Flectámus génua.
}\switchcolumn\portugues{
℣. Ajoelhemos!
}\switchcolumn*\latim{
℟. Leváte.
}\switchcolumn\portugues{
℟. Levantai-vos!
}\switchcolumn*\latim{
\begin{nscenter} Orémus. \end{nscenter}
}\switchcolumn\portugues{
\begin{nscenter} Oremos. \end{nscenter}
}\switchcolumn*\latim{
\rlettrine{E}{xáudi,} quǽsumus, Dómine, plebem tuam: et, quæ extrinsécus ánnua tríbuis devotióne venerári, intérius asséqui grátiæ tuæ luce concéde. Per Christum, Dóminum nostrum.
}\switchcolumn\portugues{
\rlettrine{D}{ignai-vos} ouvir o vosso povo, Senhor, e fazei que pela luz da vossa graça realizemos nas nossas almas o mystério que nos permitis celebrar com esta homenagem anual da nossa piedade. Por Cristo, nosso Senhor.
}\switchcolumn*\latim{
℟. Amen.
}\switchcolumn\portugues{
℟. Amen.
}\end{paracol}

\subsection{Missa da Purificação da B. V. Maria}

\paragraphinfo{Intróito}{Sl. 47, 10-11}
\begin{paracol}{2}\latim{
\rlettrine{S}{uscépimus,} Deus, misericórdiam tuam in médio templi tui: secúndum nomen tuum, Deus, ita et laus tua in fines terræ: justítia plena est déxtera tua. \emph{Ps. ibid., 2} Magnus Dóminus, et laudábilis nimis: in civitáte Dei nostri, in monte sancto ejus.
℣. Gloria Patri \emph{\&c.}
}\switchcolumn\portugues{
\rlettrine{R}{ecebemos,} ó Deus, a vossa misericórdia no meio do vosso templo. Assim como o vosso nome, assim os vossos louvores, ó Deus, ressoam até às extremidades da terra: a vossa dextra está cheia de justiça. \emph{Sl. ibid., 2} O Senhor é grande e digno de todos os louvores, na cidade do nosso Deus, na sua montanha sagrada.
℣. Glória ao Pai \emph{\&c.}
}\end{paracol}

\paragraph{Oração}
\begin{paracol}{2}\latim{
\rlettrine{O}{mnípotens} sempitérne Deus, majestátem tuam súpplices exorámus: ut, sicut unigénitus Fílius tuus hodiérna die cum nostræ carnis substántia in templo est præsentátus; ita nos fácias purificátis tibi méntibus præsentári. Per eúndem Dóminum \emph{\&c.}
}\switchcolumn\portugues{
\rlettrine{D}{eus} omnipotente e eterno, humildemente suplicamos à vossa majestade que, assim como o vosso Filho Unigénito foi neste dia apresentado no templo na substância da nossa carne, assim também sejamos apresentados diante de Vós com nossas almas limpas. Pelo mesmo nosso Senhor \emph{\&c.}
}\end{paracol}

\paragraphinfo{Epístola}{Ml. 3, 1-4}
\begin{paracol}{2}\latim{
Léctio Malachíæ Prophétæ.
}\switchcolumn\portugues{
Lição do Profeta Malaquias.
}\switchcolumn*\latim{
\rlettrine{H}{æc} dicit Dóminus Deus: Ecce, ego mitto Angelum meum, et præparábit viam ante fáciem meam. Et statim véniet ad templum suum Dominátor, quem vos quǽritis, et Angelus testaménti, quem vos vultis. Ecce, venit, dicit Dóminus exercítuum: et quis póterit cogitáre diem advéntus ejus, et quis stabit ad vidéndum eum? Ipse enim quasi ignis conflans et quasi herba fullónum: et sedébit conflans et emúndans argéntum, et purgábit fílios Levi et colábit eos quasi aurum et quasi argéntum: et erunt Dómino offeréntes sacrifícia in justítia. Et placébit Dómino sacrifícium Juda et Jerúsalem, sicut dies sýtuli et sicut anni antíqui: dicit Dóminus omnípotens.
}\switchcolumn\portugues{
\rlettrine{O}{} Senhor Deus disse: «Eis que envio o meu Anjo, que preparará o caminho diante da minha face. E logo virá ao seu templo o Dominador, que procurais, e o Anjo da Aliança, que desejais. Eis que Ele vem, diz o Senhor dos Exércitos. Mas quem poderá adivinhar o dia da sua vinda? Quem estará lá para O ver? Pois Ele será como o fogo, que funde os metais, ou como a erva, de que se servem os lavandeiros. Assentar-se-á como um homem que funde e limpa a prata; purificará os filhos de Levi e os tornará limpos, como o ouro ou a prata, e oferecerão ao Senhor sacrifícios de justiça. E o sacrifício de Judá e de Jerusalém será agradável ao Senhor, como nos séculos passados e nos anos antigos: isto diz o omnipotente Senhor».
}\end{paracol}

\paragraphinfo{Gradual}{Sl. 47, 10-11 \& 9}
\begin{paracol}{2}\latim{
\rlettrine{S}{uscépimus,} Deus, misericórdiam tuam in médio templi tui: secúndum nomen tuum, Deus, ita et laus tua in fines terræ. ℣. Sicut audívimus, ita et vídimus m civitáte Dei nostri, in monte sancto ejus.
}\switchcolumn\portugues{
\rlettrine{R}{ecebemos,} ó Deus, a vossa misericórdia no meio do vosso templo. Assim como o vosso nome, ó Deus, assim os vossos louvores ressoam até às extremidades da terra. Aquilo que havia sido anunciado, vimo-lo na cidade do nosso Deus, na sua montanha sagrada.
}\switchcolumn*\latim{
Allelúja, allelúja. ℣. Senex Púerum portábat: Puer autem senem regébat. Allelúja.
}\switchcolumn\portugues{
Aleluia, aleluia. ℣. O ancião segurava o Menino, mas o Menino conduzia o ancião. Aleluia.
}\end{paracol}

\textit{Após a Septuagésima omite-se o Aleluia e o Verso e diz-se:}

\paragraphinfo{Trato}{Lc. 2, 29-32}
\begin{paracol}{2}\latim{
\rlettrine{N}{unc} dimíttis servum tuum, Dómine, secúndum verbum tuum in pace. ℣. Quia vidérunt óculi mei salutáre tuum. ℣. Quod parásti ante fáciem ómnium populórum. ℣. Lumen ad revelatiónem géntium et glóriam plebis tuæ Israël.
}\switchcolumn\portugues{
\rlettrine{A}{gora} deixareis, Senhor, ir em paz o vosso servo. ℣. Pois os meus olhos já viram a vossa salvação: ℣. Que preparastes diante dos olhos de todos os povos. ℣. A luz que iluminará as nações e a glória de Israel, vosso povo.
}\end{paracol}

\paragraphinfo{Evangelho}{Lc. 2, 22-32}
\begin{paracol}{2}\latim{
\cruz Sequéntia sancti Evangélii secúndum Lucam.
}\switchcolumn\portugues{
\cruz Continuação do santo Evangelho segundo S. Lucas.
}\switchcolumn*\latim{
\blettrine{I}{n} illo témpore: Postquam impleti sunt dies purgatiónis Maríæ, secúndum legem Moysi, tulérunt Jesum in Jerúsalem, ut sísterent eum Dómino, sicut scriptum est in lege Dómini: Quia omne masculínum adapériens vulvam sanctum Dómino vocábitur. Et ut darent hóstiam, secúndum quod dictum est in lege Dómini, par túrturum aut duos pullos columbárum. Et ecce, homo erat in Jerúsalem, cui nomen Símeon, et homo iste justus et timorátus, exspéctans consolatiónem Israël, et Spíritus Sanctus erat in eo. Et respónsum accéperat a Spíritu Sancto, non visúrum se mortem, nisi prius vidéret Christum Dómini. Et venit in spíritu in templum. Et cum indúcerent púerum Jesum parentes ejus, ut fácerent secúndum consuetúdinem legis pro eo: et ipse accépit eum in ulnas suas, et benedíxit Deum, et dixit: Nunc dimíttis servum tuum, Dómine, secúndum verbum tuum in pace: Quia vidérunt óculi mei salutáre tuum: Quod parásti ante fáciem ómnium populórum: Lumen ad revelatiónem géntium et glóriam plebis tuæ Israël.
}\switchcolumn\portugues{
\blettrine{N}{aquele} tempo, quando acabaram os dias da purificação de Maria, levaram Jesus a Jerusalém, segundo a lei de Moisés, para O apresentar ao Senhor, como na lei do Senhor está escrito: «Todo o masculino, quando nascer, será consagrado ao Senhor», e para darem a oferta, segundo o que está escrito na lei do Senhor: «Um par de rolas ou dous pombinhos». Ora havia em Jerusalém um homem, chamado Simeão, que era justo, temente a Deus e esperava a consolação de Israel; e no qual habitava o Espírito Santo, que lhe inspirara que não haveria de morrer sem que visse o Ungido do Senhor. Veio, então, ao templo, conduzido pelo Espírito; e, como os Pais conduzissem o Menino Jesus para que n’Ele se cumprisse o que a lei ordenava, tomou Simeão o Menino nos braços e louvou a Deus, dizendo: «Agora, Senhor, deixai ir em paz o vosso servo, segundo a vossa palavra, pois os meus olhos já viram o vosso Salvador, que preparastes ante a face de todos os povos: a luz para iluminar as nações e a glória de Israel, vosso povo».
}\end{paracol}

\paragraphinfo{Ofertório}{Sl. 44, 3}
\begin{paracol}{2}\latim{
\rlettrine{D}{iffúsa} est grátia in lábiis tuis: proptérea benedíxit te Deus in ætérnum, et in sǽculum sǽculi.
}\switchcolumn\portugues{
\rlettrine{A}{} graça espalhou-se nos vossos lábios; por isso abençoou-vos Deus para sempre.
}\end{paracol}

\paragraph{Secreta}
\begin{paracol}{2}\latim{
\rlettrine{E}{xáudi,} Dómine, preces nostras: et, ut digna sint múnera, quæ óculis tuæ majestátis offérimus, subsídium nobis tuæ pietátis impénde. Per Dóminum \emph{\&c.}
}\switchcolumn\portugues{
\rlettrine{O}{uvi} nossas preces, Senhor; e, a fim de que as ofertas que apresentamos diante dos olhos da vossa majestade, sejam dignas, concedei-nos o auxílio da vossa misericórdia. Por nosso Senhor \emph{\&c.}
}\end{paracol}

\paragraphinfo{Comúnio}{Lc. 2, 26}
\begin{paracol}{2}\latim{
\rlettrine{R}{espónsum} accépit Símeon a Spíritu Sancto, non visúrum se mortem, nisi vidéret Christum Dómini.
}\switchcolumn\portugues{
\rlettrine{S}{imeão} recebera do Espírito Santo a revelação de que não morreria sem ver o Ungido do Senhor.
}\end{paracol}

\paragraph{Postcomúnio}
\begin{paracol}{2}\latim{
\qlettrine{Q}{uǽsumus,} Dómine, Deus noster: ut sacrosáncta mystéria, quæ pro reparatiónis nostræ munímine contulísti, intercedénte beáta María semper Vírgine, et præsens nobis remédium esse fácias et futúrum. Per Dóminum nostrum \emph{\&c.}
}\switchcolumn\portugues{
\slettrine{Ó}{} Senhor, nosso Deus, Vos suplicamos, permiti por intercessão da B. Maria, sempre Virgem, que os sacrossantos mystérios que nos concedestes, como salvaguarda da nossa regeneração, nos sirvam de remédio para o presente e para o futuro. Por nosso Senhor \emph{\&c.}
}\end{paracol}
