\subsectioninfo{S. João Crisóstomo, B. C. e Doutor}{27 de Janeiro}

\textit{Como na Missa In médio Ecclésiae, página \pageref{doutores}, excepto:}

\paragraph{Oração}
\begin{paracol}{2}\latim{
\rlettrine{E}{cclésiam} tuam, quǽsumus, Dómine, grátia cœléstis amplíficet: quam beáti Joánnis Chrysóstomi Confessóris tui atque Pontíficis illustráre voluísti gloriósis méritis et doctrínis. Per Dóminum nostrum \emph{\&c.}
}\switchcolumn\portugues{
\rlettrine{V}{os} suplicamos, Senhor, que a graça celestial aumente a vossa Igreja, a qual quisestes ilustrar com os gloriosos méritos e ensinos do B. João Crisóstomo, vosso Confessor e Pontífice. Por nosso Senhor \emph{\&c.}
}\end{paracol}

\paragraphinfo{Gradual}{Ecl. 44, 16}
\begin{paracol}{2}\latim{
\rlettrine{E}{cce} sacérdos magnus, qui in diébus suis plácuit Deo. ℣. \emph{ibid., 20} Non est inventus símilis illi, qui conserváret legem Excélsi.
}\switchcolumn\portugues{
\rlettrine{E}{is} o grande sacerdote que nos dias da sua vida agradou a Deus. ℣. \emph{ibid., 20} Ninguém o igualou na observância das leis do Altíssimo. 
}\switchcolumn*\latim{
Allelúja, allelúja. ℣. \emph{Jac. 1, 12} Beátus vir, qui suffert tentatiónem: quóniam, cum probátus fúerit, accípiet corónam vitæ. Allelúja.
}\switchcolumn\portugues{
Aleluia, aleluia. ℣. \emph{Tg. 1, 12} Bem-aventurado o varão que sofre com paciência a tentação, porque, quando acabar a provação, alcançará a coroa da vida. Aleluia.
}\end{paracol}