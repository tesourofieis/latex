\subsectioninfo{S. Mateus}{21 de Setembro}

\paragraphinfo{Intróito}{Sl. 36, 30-31}
\begin{paracol}{2}\latim{
\rlettrine{O}{s} justi meditábitur sapiéntiam, et lingua ejus loquétur judíci‚ um: lex Dei ejus in corde ipsíus. \emph{Ps. ibid., 1} Noli æmulári in malignántibus: neque zeláveris faciéntes iniquitátem.
℣. Gloria Patri \emph{\&c.}
}\switchcolumn\portugues{
\rlettrine{A}{} boca do justo falará com sabedoria e a sua língua proclamará a justiça. A lei do seu Deus está no seu coração. \emph{Sl. ibid., 1} Não vos irriteis contra os maus, nem tenhais inveja daqueles que cometem iniquidades.
℣. Glória ao Pai \emph{\&c.}
}\end{paracol}

\paragraph{Oração}
\begin{paracol}{2}\latim{
\rlettrine{B}{eáti} Apóstoli et Evangelístæ Matthǽi, Dómine, précibus adjuvémur: ut, quod possibílitas nostra non óbtinet, ejus nobis intercessióne donétur. Per Dóminum \emph{\&c.}
}\switchcolumn\portugues{
\rlettrine{S}{ocorrei-nos,} Senhor, em virtude das preces, do B. Mateus, vosso Apóstolo e Evangelista, a fim de que nos sejam concedidas aquelas graças que não temos possibilidade de obter por nós próprios. Por nosso Senhor \emph{\&c.}
}\end{paracol}

\paragraphinfo{Epístola}{Ez. 1, 10-14}
\begin{paracol}{2}\latim{
Léctio Ezechiélis Prophétæ.
}\switchcolumn\portugues{
Lição do Profeta Ezequiel.
}\switchcolumn*\latim{
\rlettrine{S}{imilitúdo} vultus quátuor animálium: fácies hóminis, et fácies leónis a dextris ipsórum quátuor: fácies autem bovis a sinístris ipsórum quátuor, et fácies áquilæ désuper ipsórum quátuor. Fácies eórum et pénnæ eórum exténtæ désuper: duæ pennæ singulórum jungebántur et duæ tegébant córpora eórum: et unumquódque eórum coram fácie sua ambulábat: ubi erat ímpetus spíritus, illuc gradiebántur, nec revertebántur cum ambulárent. Et similitúdo animálium, aspéctus eórum quasi carbónum ignis ardéntium et quasi aspéctus lampadárum. Hæc erat vísio discúrrens in médio animálium, splendor ignis, et de igne fulgur egrédiens. Et animália ibant et revertebántur in similitúdinem fúlguris coruscántis.
}\switchcolumn\portugues{
\rlettrine{E}{is} a semelhança do rosto dos quatro seres animados: Tinham todos quatro uma face de homem; todos os quatro à direita uma face de leão; todos os quatro à esquerda uma face de touro; e todos os quatro por cima uma face de águia! As faces e as asas mostravam-se estendidas por cima; e estavam unidos uns aos outros por duas asas, cobrindo os corpos com as outras duas. Cada um deles caminhava para a frente do seu rosto, e iam até onde os impelia o espírito, não se voltando enquanto andavam. Estes seres tinham o aspecto de carvões de fogo a arder e de lâmpadas acesas. Viam-se crepitar no meio deles chamas de fogo, saindo do fogo relâmpagos. E eles iam e vinham, semelhante ao fuzilar dos relâmpagos.
}\end{paracol}

\paragraphinfo{Gradual}{Sl. 111, 1-2}
\begin{paracol}{2}\latim{
\rlettrine{B}{eátus} vir, qui timet Dóminum: in mandátis ejus cupit nimis. ℣. Potens in terra erit semen ejus: generátio rectórum benedicétur.
}\switchcolumn\portugues{
\rlettrine{B}{em-aventurado} o varão que teme o Senhor e que põe todo seu zelo em obedecer-Lhe. ℣. Sua descendência será poderosa na terra; pois a geração dos justos será abençoada.
}\switchcolumn*\latim{
Allelúja, allelúja. ℣. Te gloriosus Apostolórum chorus laudat, Dómine. Allelúja.
}\switchcolumn\portugues{
Aleluia, aleluia. ℣. O coro glorioso dos Apóstolos Vos louva, ó Senhor. Aleluia.
}\end{paracol}

\paragraphinfo{Evangelho}{Mt. 9, 9-13}
\begin{paracol}{2}\latim{
\cruz Sequéntia sancti Evangélii secúndum Matthǽum.
}\switchcolumn\portugues{
\cruz Continuação do santo Evangelho segundo S. Mateus.
}\switchcolumn*\latim{
\blettrine{I}{n} illo témpore: Vidit Jesus hóminem sedéntem in telónio, Matthǽum nómine. Et ait illi: Séquere me. Et surgens, secútus est eum. Et factum est, discumbénte eo in domo, ecce, multi publicáni et peccatóres veniéntes discumbébant cum Jesu et discípulis ejus. Et vidéntes pharisǽi, dicébant discípulis ejus: Quare cum publicánis et peccatóribus mánducat Magíster vester? At Jesus áudiens, ait: Non est opus valéntibus médicus, sed male habéntibus. Eúntes autem díscite, quid est: Misericórdiam volo, et non sacrifícium. Non enim veni vocáre justos, sed peccatóres.
}\switchcolumn\portugues{
\blettrine{N}{aquele} tempo, Jesus viu um homem, chamado Levi, assentado ao telónio, e disse-lhe: «Segue-me!». Levantando-se este, seguiu-O. Ora, aconteceu que, estando Jesus assentado à mesa em casa dele aproximaram-se e puseram-se a comer com Ele e com seus discípulos muitos publicanos e pecadores. Vendo isto, os fariseus disseram aos discípulos: «Porque come o vosso Mestre com os publicanos e pecadores?». Jesus, ouvindo-os, disse: «Não são os sãos que necessitam de médico, mas os que estão enfermos. Ide, pois, e aprendei o que significam estas palavras: «Prefiro a misericórdia ao sacrifício»; pois não vim Eu chamar os justos mas os pecadores.
}\end{paracol}

\paragraphinfo{Ofertório}{Sl. 20, 4-5}
\begin{paracol}{2}\latim{
\rlettrine{P}{osuísti,} Dómine, in cápite ejus corónam de lápide pretióso: vitam pétiit a te, et tribuísti ei, allelúja.
}\switchcolumn\portugues{
\rlettrine{V}{ós} o coroastes, Senhor, com glória e com honras: e o colocastes acima das obras das vossas mãos.
}\end{paracol}

\paragraph{Secreta}
\begin{paracol}{2}\latim{
\rlettrine{S}{upplicatiónibus} beáti Matthǽi Apóstoli et Evangelístæ, quǽsumus, Dómine, Ecclésiæ tuæ commendétur oblátio: cujus magníficis prædicatiónibus erúditur. Per Dóminum \emph{\&c.}
}\switchcolumn\portugues{
\rlettrine{P}{ermiti,} Senhor, Vos rogamos, que as súplicas do B. Apóstolo e Evangelista Mateus Vos tornem agradáveis a oblação da vossa Igreja, que foi brilhantemente instruída pelas suas admiráveis pregações. Por nosso Senhor \emph{\&c.}
}\end{paracol}

\paragraphinfo{Comúnio}{Sl. 20, 6}
\begin{paracol}{2}\latim{
\rlettrine{M}{agna} est glória ejus in salutári tuo: glóriam et magnum del córem ímpones super eum, Dómine.
}\switchcolumn\portugues{
\rlettrine{G}{rande} é, Senhor, a sua glória que lhe concedestes na salvação. Vós o rodeareis de glória e de magnificência.
}\end{paracol}

\paragraph{Postcomúnio}
\begin{paracol}{2}\latim{
\rlettrine{P}{ercéptis,} Dómine, sacraméntis, beáto Matthǽo Apóstolo tuo et Evangelísta interveniénte, deprecámur: ut, quæ pro ejus celebráta sunt glória, nobis profíciant ad medélam. Per Dóminum \emph{\&c.}
}\switchcolumn\portugues{
\rlettrine{H}{avendo} recebido os vossos sacramentos, Vos pedimos, Senhor, pela intercessão do B. Mateus, vosso Apóstolo e Evangelista, permitais que este sacrifício, oferecido em sua honra, nos sirva de remédio. Por nosso Senhor \emph{\&c.}
}\end{paracol}
