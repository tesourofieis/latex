\subsectioninfo{S. Úrsula e Outras, Virgens e Mártires}{21 de Outubro}

\paragraph{Oração}
\begin{paracol}{2}\latim{
\rlettrine{D}{a} nobis, quǽsumus, Dómine, Deus noster, sanctárum Vírginum et Mártyrum tuárum Ursulæ et Sociárum ejus palmas incessábili devotióne venerári: ut, quas digna mente non póssumus celebráre, humílibus saltem frequentémus obséquiis. Per Dóminum nostrum \emph{\&c.}
}\switchcolumn\portugues{
\rlettrine{S}{enhor,} nosso Deus, dignai-Vos conceder-nos a graça de incessantemente venerarmos com devoção a vitória das vossas Santas Virgens e Mártires Úrsula e Companheiras, a fim de que, já que não podemos celebrar dignamente os seus méritos, ao menos possamos oferecer-lhes as nossas humildes homenagens. Por nosso Senhor \emph{\&c.}
}\end{paracol}

\paragraph{Secreta}
\begin{paracol}{2}\latim{
\rlettrine{I}{nténde,} quǽsumus, Dómine, múnera altáribus tuis pro sanctárum Virginum et Mártyrum tuárum Ursulæ et Sociárum ejus festivitáte propósita: ut, sicut per hæc beáta mystéria illis glóriam contulísti; ita nobis indulgéntiam largiáris. Per Dóminum \emph{\&c.}
}\switchcolumn\portugues{
\rlettrine{S}{enhor,} dignai-Vos volver os vossos olhares para estas ofertas, que depositamos nos vossos altares para celebrar a festa das vossas Santas Virgens e Mártires Úrsula e Companheiras, a fim de que, assim como lhes concedestes a glória, assim nos concedais o perdão dos nossos pecados. Por nosso Senhor \emph{\&c.}
}\end{paracol}

\paragraph{Postcomúnio}
\begin{paracol}{2}\latim{
\rlettrine{P}{ræsta} nobis, quǽsumus, Dómine: intercedéntibus sanctis Virgínibus et Martýribus tuis Ursula et Sociábus ejus: ut, quod ore contíngimus, pura mente capiámus. Per Dóminum \emph{\&c.}
}\switchcolumn\portugues{
\rlettrine{C}{oncedei-nos,} Senhor, Vos suplicamos, que por intercessão das vossas Santas Virgens e Mártires Úrsula e Companheiras guardemos com o coração puro o que a nossa boca acaba de receber. Por nosso Senhor \emph{\&c.}
}\end{paracol}