\subsectioninfo{Sete Dores da B. V. Maria}{15 de Setembro}\label{setedoressetembro}

\textit{Como na Missa das Sete Dores da B. V. Maria de Março, página \pageref{setedoresmarco}, excepto:}

\paragraph{Oração}
\begin{paracol}{2}\latim{
\rlettrine{D}{eus,} in cujus passióne, secúndum Simeónis prophétiam, dulcíssimam ánimam gloriósæ Vírginis et Matris Maríæ dolóris gladius pertransívit: concéde propítius; ut, qui transfixiónem ejus et passiónem venerándo recólimus, gloriósis méritis et précibus ómnium Sanctórum Cruci fidéliter astántium intercedéntibus, passiónis tuæ efféctum felícem consequámur: Qui vivis \emph{\&c.}
}\switchcolumn\portugues{
\slettrine{Ó}{} Deus, em cuja Paixão, segundo a profecia de Simeão, uma espada de dor traspassou a terníssima alma da Virgem Maria, vossa Mãe, concedei-nos propício que, celebrando devotadamente a memória das suas dores, alcancemos o feliz efeito da vossa Paixão. Ó Vós, que, sendo Deus, viveis e \emph{\&c.}
}\end{paracol}

\paragraph{Gradual}
\begin{paracol}{2}\latim{
\rlettrine{D}{olorósa} et lacrimábilis es, Virgo María, stans juxta Crucem Dómini Jesu, Fílii tui, Redemptóris. ℣. Virgo Dei Génetrix, quem totus non capit orbis, hoc crucis fert supplícium, auctor vitæ factus homo.
}\switchcolumn\portugues{
\rlettrine{C}{heia} de dores e de lágrimas, ó Virgem Maria, estavas junto à Cruz do Senhor Jesus, vosso Filho e Redentor. ℣. Ó Virgem, Mãe de Deus, Aquele a quem o mundo não pode conter o autor da vida feito homem sofre este suplício da cruz!
}\switchcolumn*\latim{
Allelúja, allelúja. ℣. Stabat sancta María, cœli Regína et mundi Dómina, juxta Crucem Dómini nostri Jesu Christi dolorósa.
}\switchcolumn\portugues{
Aleluia, aleluia. ℣. Estava doloroso, junto à Cruz de nosso Senhor Jesus Cristo, a Rainha do céu e Soberana do mundo, Santa Maria.
}\end{paracol}