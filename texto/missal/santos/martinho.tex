\subsectioninfo{S. Martinho}{11 de Novembro}

\paragraphinfo{Intróito}{Página \pageref{martirpontificeforapascal}}

\paragraph{Oração}
\begin{paracol}{2}\latim{
\rlettrine{D}{eus,} qui cónspicis, quia ex nulla nostra virtúte subsístimus: concéde propítius; ut, intercessióne beáti Martíni \emph{\&c.}
}\switchcolumn\portugues{
\slettrine{Ó}{} Deus, que bem conheceis que não é pelo nosso poder que subsistimos, concedei-nos propício que, pela intercessão do B. Martinho, vosso Confessor e Pontífice, sejamos protegidos contra todas as adversidades. Por nosso Senhor \emph{\&c.}
}\end{paracol}

\subsubsectioninfo{Comemoração de S. Mena}{Página \pageref{martirnaopontifice1}}

\paragraphinfo{Epístola}{Página \pageref{confessorespontifices1}}

\paragraphinfo{Gradual}{Ecl. 44, 16}
\begin{paracol}{2}\latim{
\rlettrine{E}{cce} sacérdos magnus, qui in diébus suis plácuit Deo. ℣. \emph{ibid., 20} Non est invéntus símilis illi, qui conserváret legem Excelsi.
}\switchcolumn\portugues{
\rlettrine{E}{is} o grande sacerdote que nos dias da sua vida agradou a Deus. ℣. \emph{ibid., 20} Ninguém o igualou na observância das leis do Altíssimo.
}\switchcolumn*\latim{
Allelúja, allelúja. ℣. Beátus vir, sanctus Martínus, urbis Turónis Epíscopus, requiévit: quem suscéperunt Angeli atque Archángeli, Throni, Dominatiónes et Virtútes. Allelúja.
}\switchcolumn\portugues{
Aleluia, aleluia. ℣. O bem-aventurado varão Martinho, Bispo de Tours, dormiu no Senhor: e os Anjos, os Arcanjos, os Tronos, as Dominações e as Virtudes o acolheram. Aleluia.
}\end{paracol}

\paragraphinfo{Evangelho}{Lc. 11, 33-36}
\begin{paracol}{2}\latim{
\cruz Sequéntia sancti Evangélii secúndum Lucam.
}\switchcolumn\portugues{
\cruz Continuação do santo Evangelho segundo S. Lucas.
}\switchcolumn*\latim{
\blettrine{I}{n} illo témpore: Dixit Jesus discípulis suis: Nemo lucérnam accéndit, et in abscóndito ponit, neque sub módio: sed supra candelábrum, ut, qui ingrediúntur, lumen vídeant. Lucérna córporis tui est óculus tuus. Si óculus tuus fúerit simplex, totum corpus tuum lúcidum erit: si autem nequam fúerit, étiam corpus tuum tenebrósum erit. Vide ergo, ne lumen, quod in te est, ténebræ sint. Si ergo corpus tuum totum lúcidum fúerit, non habens áliquam partem tenebrárum, erit lúcidum totum, et sicut lucérna fulgóris illuminábit te.
}\switchcolumn\portugues{
\blettrine{N}{aquele} tempo, disse Jesus aos seus discípulos: «Ninguém acende uma lâmpada para a colocar num lugar oculto ou sob um alqueire; mas coloca-a sobre o candelabro, para que aqueles que entrarem vejam a luz. A lâmpada do teu corpo é o teu olho. Se o teu olho for simples, todo teu corpo será luminoso; mas se ele for mau, também o teu corpo será tenebroso. Tem, pois, cuidado de que a luz, que está em ti, não se torne em trevas. Se, portanto, todo teu corpo for luminoso sem parte alguma nas trevas, será ele brilhante, como quando uma lâmpada te ilumina com seu brilho».
}\end{paracol}

\paragraphinfo{Ofertório}{Sl. 88, 25}
\begin{paracol}{2}\latim{
\rlettrine{V}{éritas} mea et misericórdia mea cum ipso: et in nómine meo exaltábitur cornu ejus.
}\switchcolumn\portugues{
\rlettrine{A}{} minha fidelidade e a minha misericórdia estarão com ele; e o seu poder elevar-se-á pelo meu nome.
}\end{paracol}

\paragraph{Secreta}
\begin{paracol}{2}\latim{
\rlettrine{S}{anctífica,} quǽsumus, Dómine Deus, hæc múnera, quæ in sollemnitáte sancti Antístitis tui Martíni offérimus: ut per ea vita nostra inter advérsa et próspera ubíque dirigátur. Per Dóminum \emph{\&c.}
}\switchcolumn\portugues{
\slettrine{Ó}{} Senhor e Deus, santificai, Vos rogamos, estes dons que Vos oferecemos na solenidade do Santo Bispo Martinho, vosso Mártir, a fim de que, graças a eles, a nossa vida se regule segundo a vossa vontade, tanto nas adversidades, como nas prosperidades. Por nosso Senhor \emph{\&c.}
}\end{paracol}

\paragraphinfo{Comúnio}{Mt. 24,46-47}
\begin{paracol}{2}\latim{
\rlettrine{B}{eátus} servus, quem, cum vénerit dóminus, invénerit vigilántem: amen, dico vobis, super ómnia bona sua constítuet eum.
}\switchcolumn\portugues{
\rlettrine{B}{em-aventurado} o servo que, quando o Senhor vier, encontrar vigilante. Em verdade vos digo que o encarregará de administrar todos seus bens.
}\end{paracol}

\paragraph{Postcomúnio}
\begin{paracol}{2}\latim{
\rlettrine{P}{ræsta,} quǽsumus, Dómine, Deus noster: ut, quorum festivitáte votíva sunt sacraménta, eórum intercessióne salutária nobis reddántur. Per Dóminum \emph{\&c.}
}\switchcolumn\portugues{
\rlettrine{F}{azei,} Senhor, nosso Deus, que nos sejam salutares estes sacrossantos sacramentos, pela intercessão daqueles em cuja festa Vo-los apresentamos. Por nosso Senhor \emph{\&c.}
}\end{paracol}
