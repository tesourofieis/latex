\subsectioninfo{S. S. Fabião e Sebastião, Márts.}{20 de Janeiro}\label{fabiaosebastiao}

\paragraphinfo{Intróito}{Página \pageref{muitosmartires1}}

\paragraph{Oração}
\begin{paracol}{2}\latim{
\rlettrine{I}{nfirmitátem} nostram réspice, omnípotens Deus: et, quia pondus própriæ actiónis gravat, beatórum Mártyrum tuórum Fabiáni et Sebastiáni intercéssio gloriósa nos prótegat. Per Dóminum \emph{\&c.}
}\switchcolumn\portugues{
\slettrine{Ó}{} Deus omnipotente, olhai para a nossa fraqueza, e, visto que estamos oprimidos sob o peso dos nossos pecados, fazei que sejamos protegidos pela gloriosa intercessão dos vossos B. B. Mártires Fabião e Sebastião. Por nosso Senhor \emph{\&c.}
}\end{paracol}

\paragraphinfo{Epístola}{Heb. 11, 33-39}
\begin{paracol}{2}\latim{
Léctio Epístolæ beáti Pauli Apóstoli ad Hebrǽos.
}\switchcolumn\portugues{
Lição da Epístola do B. Ap.º Paulo aos Hebreus.
}\switchcolumn*\latim{
\rlettrine{F}{ratres:} Sancti per fidem vicérunt regna, operáti sunt justítiam, adépti sunt repromissiónes, obturavérunt ora leónum, exstinxérunt ímpetum ignis, effugérunt áciem gládii, convaluérunt de infirmitáte, fortes facti sunt in bello, castra vertérunt exterórum: accepérunt mulíeres de resurrectióne mórtuos suos: álii autem disténti sunt, non suscipiéntes redemptiónem, ut meliórem invenírent resurrectiónem: alii vero ludíbria et vérbera expérti, ínsuper et víncula et cárceres: lapidáti sunt, secti sunt, tentáti sunt, in occisióne gládii mórtui sunt: circuiérunt in melótis, in péllibus caprínis, egéntes, angustiáti, afflicti: quibus dignus non erat mundus: in solitudínibus errantes, in móntibus et spelúncis et in cavérnis terræ. Et hi omnes testimónio fídei probáti, invénti sunt in Christo Jesu, Dómino nostro.
}\switchcolumn\portugues{
\rlettrine{M}{eus} irmãos: Pela Fé os santos conquistaram os reinos, praticaram a justiça, alcançaram a realização das promessas, fecharam a boca dos leões, apagaram as chamas do fogo, escaparam ao fio das espadas, convalesceram de suas enfermidades, foram fortes nas batalhas e puseram em debandada as forças inimigas. Pela Fé as mulheres receberam, já ressuscitados, os seus mortos, outros aceitaram a tortura sem desejarem livrar-se, a fim de alcançarem uma ressurreição melhor; e outros ainda foram escarnecidos e açoitados e até algemados e presos. Pela Fé muitos foram apedrejados, serrados, tentados, passados à espada, decapitados, levaram vida errante, vestidos com peles de ovelhas e de cabras, despojados de tudo, perseguidos e maltratados. Pela Fé houve homens (de quem o mundo não era digno) que andaram errantes nos desertos e nas montanhas e escondidos nas Covas e nas cavernas da terra. Todos foram provados Para testemunho da sua Fé em N. S. Jesus Cristo.
}\end{paracol}

\paragraphinfo{Gradual}{Ex. 15, 11}
\begin{paracol}{2}\latim{
\rlettrine{G}{loriósus} Deus in Sanctis suis: mirábilis in majestáte, fáciens prodígia. ℣. \emph{ibid., 6} Déxtera tua, Dómine, glorificáta est in virtúte: déxtera manus tua confrégit inimícos.
}\switchcolumn\portugues{
\rlettrine{D}{eus} é glorioso em seus Santos: e admirável na sua majestade, praticando prodígios. ℣. \emph{ibid., 6} Senhor, a vossa mão direita engrandeceu-se pela força: a vossa mão direita esmagou os inimigos.
}\switchcolumn*\latim{
Allelúja, allelúja. ℣. \emph{Ps. 144, 1011} Sancti tui, Dómine, benedícent te: glóriam regni tui dicent. Allelúja.
}\switchcolumn\portugues{
Aleluia, aleluia. ℣. \emph{Sl. 144, 1011} vossos Santos Vos bendirão, Senhor: e proclamarão a glória do vosso reino. Aleluia.
}\end{paracol}

\textit{Depois da Septuagésima, a Aleluia é omitida e é dito o Trato da Missa Intret in, página \pageref{muitosmartires1}.}

\paragraphinfo{Evangelho}{Página \pageref{muitosmartires2}}

\paragraphinfo{Ofertório}{Sl. 31, 11}
\begin{paracol}{2}\latim{
\rlettrine{L}{ætámini} in Dómino et exsultáte, justi: et gloriámini, omnes recti corde.
}\switchcolumn\portugues{
\rlettrine{A}{legrai-vos} no Senhor, ó justos. Todos aqueles que possuem o coração recto serão glorificados.
}\end{paracol}

\paragraph{Secreta}
\begin{paracol}{2}\latim{
\rlettrine{H}{óstias} tibi, Dómine, beatórum Mártyrum tuórum Fabiáni et Sebastiáni dicátas méritis, benígnus assúme: et ad perpétuum nobis tríbue proveníre subsídium. Per Dóminum \emph{\&c.}
}\switchcolumn\portugues{
\rlettrine{R}{ecebei} benigno, Senhor, as hóstias que Vos oferecemos pelos méritos dos vossos B. B. Mártires Fabião e Sebastião e dignai-Vos permitir que em virtude delas alcancemos o vosso perpétuo socorro. Por nosso Senhor \emph{\&c.}
}\end{paracol}

\paragraphinfo{Comúnio}{Lc. 6, 18 \& 19}
\begin{paracol}{2}\latim{
\rlettrine{M}{ultitúdo} languéntium, et qui vexabántur a spirítibus immúndis, veniébant ad eum: quia virtus de illo exíbat, et sanábat omnes.
}\switchcolumn\portugues{
\rlettrine{N}{umerosos} enfermos e possessos de espíritos imundos vinham ter com Ele, pois de si saía uma tal virtude que a todos curava.
}\end{paracol}

\paragraph{Postcomúnio}
\begin{paracol}{2}\latim{
\rlettrine{R}{efécti} participatióne múneris sacri, quǽsumus, Dómine, Deus noster: ut, cujus exséquimur cultum, intercedéntibus sanctis Martýribus tuis Fabiáno et Sebastiáno, sentiámus efféctum. Per Dóminum \emph{\&c.}
}\switchcolumn\portugues{
\rlettrine{F}{ortalecidos} com a participação deste dom sacratíssimo, Vos suplicamos, Senhor, nosso Deus, que, por intercessão dos vossos B. B. Mártires Fabião e Sebastião, sintamos o efeito do mistério que hoje celebrámos. Por nosso Senhor \emph{\&c.}
}\end{paracol}
