\subsectioninfo{Imp. dos Estigmas em S. Francisco}{17 de Setembro}\label{estigmasfrancisco}

\paragraphinfo{Intróito}{Gl. 6, 14}
\begin{paracol}{2}\latim{
\rlettrine{M}{ihi} autem absit gloriári, nisi in Cruce Dómini nostri Jesu Christi: per quem mihi mundus crucifíxus est, et ego mundo. \emph{Ps. 141, 2} Voce mea ad Dóminum clamávi: voce mea ad Dóminum deprecátus sum.
℣. Gloria Patri \emph{\&c.}
}\switchcolumn\portugues{
\rlettrine{N}{unca,} porém, Deus permita que me glorie senão na Cruz de nosso Senhor Jesus Cristo, por quem eu o mundo está crucificado para mim, como eu o estou para o mundo! \emph{Sl. 141, 2} Elevei a minha voz ao Senhor: dirigi ao Senhor a minha voz suplicante!
℣. Glória ao Pai \emph{\&c.}
}\end{paracol}

\paragraph{Oração}
\begin{paracol}{2}\latim{
\rlettrine{D}{ómine} Jesu Christe, qui, frigescénte mundo, ad inflammándum corda nostra tui amóris igne, in carne beatíssimi Francísci passiónis tuæ sacra Stígmata renovásti: concéde propítius; ut ejus méritis et précibus crucem júgiter ferámus, et dignos fructus pœniténtiæ faciámus: Qui vivis \emph{\&c.}
}\switchcolumn\portugues{
\rlettrine{S}{enhor} Jesus Cristo, que, quando a caridade resfriou no mundo, quisestes renovar os Sagrados Estigmas da vossa Paixão na carne do B. Francisco, para inflamar os nossos corações no fogo do vosso amor, concedei-nos, Vos suplicamos, que pelos seus méritos e preces abracemos continuamente a Cruz e pratiquemos dignos frutos de penitência. Ó Vós, que viveis e reinais \emph{\&c.}
}\end{paracol}

\paragraphinfo{Epístola}{Gl. 6, 14-18}
\begin{paracol}{2}\latim{
Léctio Epístolæ beáti Pauli Apóstoli ad Gálatas.
}\switchcolumn\portugues{
Lição da Ep.ª do B. Ap.º Paulo aos Gálatas.
}\switchcolumn*\latim{
\rlettrine{F}{ratres:} Mihi autem absit gloriári, nisi in Cruce Dómini nostri Jesu Christi: per quem mihi mundus crucifíxus est, et ego mundo. In Christo enim Jesu neque circumcísio áliquid valet neque præpútium, sed nova creatúra. Et quicúmque hanc régulam secúti fúerint, pax sin per illos et misericórdia, et super Israël Dei. De cetero nemo mihi moléstus sit: ego enim stígmata Dómini Jesu in córpore meo porto. Grátia Dómini nostri Jesu Christi cum spíritu vestro, fratres. Amen.
}\switchcolumn\portugues{
\rlettrine{M}{eus} irmãos: Nunca Deus permita que me glorie senão na Cruz de nosso Senhor Jesus Cristo, por quem o mundo está crucificado para mim, como eu o estou para o mundo! Na verdade, em Jesus Cristo não é a circuncisão ou a incircuncisão que valem alguma coisa, mas cada um ser uma criatura nova. Que todos aqueles que seguirem esta regra experimentem paz e consolação, e do mesmo modo os que são o Israel (o povo) de Deus. Que doravante ninguém me dê desgosto algum, pois trago no meu coração os Estigmas do Senhor Jesus. Que a graça de nosso Senhor Jesus Cristo, meus irmãos, seja com vosso espírito. Amen.
}\end{paracol}

\paragraphinfo{Gradual}{Sl. 36, 30-31}
\begin{paracol}{2}\latim{
\rlettrine{O}{s} justi meditábitur sapiéntiam, et lingua ejus loquétur judícium. ℣. Lex Dei ejus in corde ipsíus: et non supplantabúntur gressus ejus.
}\switchcolumn\portugues{
\rlettrine{A}{} boca do justo falará com sabedoria e a sua língua proclamará a justiça. ℣. A lei do seu Deus está sempre no seu coração e os seus
pés não tropeçarão.
}\switchcolumn*\latim{
Allelúja, allelúja. ℣. Francíscus pauper et húmilis cœlum dives ingréditur, hymnis cœléstibus honorátur. Allelúja.
}\switchcolumn\portugues{
Aleluia, aleluia. ℣. Francisco, pobre e humilde, entra rico no céu: em sua honra ressoam hinos celestiais. Aleluia.
}\end{paracol}

\paragraphinfo{Evangelho}{Página \pageref{martirpontifice}}

\paragraphinfo{Ofertório}{Sl. 88, 25}
\begin{paracol}{2}\latim{
\rlettrine{V}{éritas} mea et misericórdia mea cum ipso: et in nómine meo exaltábitur cornu ejus.
}\switchcolumn\portugues{
\rlettrine{A}{} minha fidelidade e a minha misericórdia estarão com ele; e por virtude do meu nome será exaltado o seu poder.
}\end{paracol}

\paragraph{Secreta}
\begin{paracol}{2}\latim{
\rlettrine{M}{únera} tibi, Dómine, dicata sanctífica: et, intercedénte beáto Francísco, ab omni nos culpárum labe purífica. Per Dóminum \emph{\&c.}
}\switchcolumn\portugues{
\rlettrine{S}{antificai,} Senhor, os dons que Vos são oferecidos, e pela intercessão do B. Francisco purificai-nos dos nossos pecados. Por nosso Senhor \emph{\&c.}
}\end{paracol}

\paragraphinfo{Comúnio}{Lc. 12, 42}
\begin{paracol}{2}\latim{
\rlettrine{F}{idélis} servus et prudens, quem constítuit dóminus super famíliam suam: ut det illis in témpore trítici mensúram.
}\switchcolumn\portugues{
\rlettrine{E}{is} o servo fiel e prudente, destinado pelo Senhor para distribuir oportunamente a cada um na família a sua medida de trigo.
}\end{paracol}

\paragraph{Postcomúnio}
\begin{paracol}{2}\latim{
\rlettrine{D}{eus,} qui mira Crucis mystéria in beáto Francísco Confessóre tuo multifórmiter demonstrásti: da nobis, quǽsumus; devotiónis suæ semper exémpla sectári, et assídua ejúsdem Crucis meditatióne muníri. Per Dóminum \emph{\&c.}
}\switchcolumn\portugues{
\slettrine{Ó}{} Deus, que sob múltiplas formas manifestastes na pessoa do B. Francisco, vosso Confessor, os admiráveis mistérios da Cruz, concedei-nos, Vos rogamos, que possamos sempre seguir os exemplos da sua devoção e confortar-nos com a contínua meditação desta mesma Cruz. Por nosso \emph{\&c.}
}\end{paracol}
