\subsectioninfo{S. Juliana de Falconéri, Virgem}{19 de Junho}

\textit{Como na Missa Dilexísti justitiam, página \pageref{virgemnaomartir1}, excepto:}

\paragraph{Oração}
\begin{paracol}{2}\latim{
\rlettrine{D}{eus,} qui beátam Juliánam Vírginem tuam extrémo morbo laborántem, pretióso Fílii tui Córpore mirabíliter recreáre dignátus es: concéde, quǽsumus; ut, ejus intercedéntibus méritis, nos quoque eódem in mortis agóne refécti ac roboráti, ad cœléstem pátriam perducámur. Per eúndem Dóminum \emph{\&c.}
}\switchcolumn\portugues{
\slettrine{Ó}{} Deus, que com o preciosíssimo Corpo de vosso Filho Vos dignastes sustentar miraculosamente a B. Juliana, vossa Virgem, quando ela sofreu a última doença, concedei-nos, pelos seus méritos e intercessão, Vos suplicamos, que também na nossa agonia mortal sejamos alimentados e fortificados com este mesmo Corpo e conduzidos à pátria celestial. Pelo mesmo nosso Senhor \emph{\&c.}
}\end{paracol}

\paragraphinfo{Oração}{S. S. Gervásio e Protásio}
\begin{paracol}{2}\latim{
\rlettrine{D}{eus,} qui nos ánnua sanctórum Martyrum tuórum Gervásii et Protásii sollemnitáte lætíficas: concéde propítius; ut, quorum gaudémus méritis, accendámur exémplis. Per Dóminum \emph{\&c.}
}\switchcolumn\portugues{
\slettrine{Ó}{} Deus, que nos alegrais com a solenidade anual dos vossos Santos Mártires Gervásio e Protásio, concedei-nos propício que sejamos inflamados pelos exemplos daqueles cujos méritos nos enchem de alegria celestial. Pelo nosso Senhor \emph{\&c.}
}\end{paracol}

\paragraphinfo{Secreta}{S. S. Mártires}
\begin{paracol}{2}\latim{
\rlettrine{O}{blátis,} quǽsumus, Dómine, placáre munéribus: et, intercedéntibus sanctis Martýribus tuis, a cunctis nos defénde perículis. Per Dóminum nostrum \emph{\&c.}
}\switchcolumn\portugues{
\rlettrine{D}{eixai-Vos} aplacar com os dons que Vos oferecemos, Senhor, e por intercessão dos vossos Santos Mártires defendei-nos de todos os perigos. Por nosso Senhor \emph{\&c.}
}\end{paracol}

\paragraphinfo{Postcomúnio}{S. S. Mártires}
\begin{paracol}{2}\latim{
\rlettrine{H}{æc} nos commúnio, Dómine, purget a crímine: et, intercedéntibus sanctis Martýribus tuis Gervásio et Protásio, cœléstis remédii fáciat esse consórtes. Per Dóminum nostrum \emph{\&c.}
}\switchcolumn\portugues{
\qlettrine{Q}{ue} esta comunhão nos purifique dos nossos crimes, Senhor, e que por intercessão dos vossos Santos Mártires Gervásio e Protásio nos faça participantes do remédio celestial. Por nosso Senhor \emph{\&c.}
}\end{paracol}