\subsectioninfo{S. João Bosco, Conf.}{31 de Janeiro}

\paragraphinfo{Intróito}{3. Rs. 4, 29}
\begin{paracol}{2}\latim{
\rlettrine{D}{edit} illi Deus sapiéntiam, et prudéntiam multam nimis, et latitúdinem cordis, quasi arénam quæ est in líttore maris. \emph{Ps. 112, 1} Laudáte, pueri, Dóminum, laudáte nomen Dómini.
℣. Gloria Patri \emph{\&c.}
}\switchcolumn\portugues{
\rlettrine{D}{eus} deu-lhe a sabedoria, assim como uma admirável prudência e uma imensa magnanimidade, como a areia que há à beira-mar. \emph{Sl. 112, 1} Louvai o Senhor, ó meninos, louvai o nome do Senhor.
℣. Glória ao Pai \emph{\&c.}
}\end{paracol}

\paragraph{Oração}
\begin{paracol}{2}\latim{
\rlettrine{D}{eus,} qui sanctum Joánnem Confessórem tuum adolescentium patrem et magístrum excitásti, ac per eum, auxiliatríce Vírgine María, novas in Ecclésia tua famílias floréscere voluísti: concéde, quǽsumus; ut eódem caritátis igne succénsi, ánimas quǽrere, Ubíque soli servíre valeámus. Per Dóminum \emph{\&c.}
}\switchcolumn\portugues{
\slettrine{Ó}{} Deus, que suscitastes a S. João, vosso Confessor, para pai e mestre dos adolescentes, e por ele com o auxílio da Virgem Maria quisestes florescessem na vossa Igreja novas famílias: concedei, Vos pedimos, que, abrasados no mesmo fogo de caridade, possamos ganhar almas e só Vós servir. Por nosso Senhor \emph{\&c.}
}\end{paracol}

\paragraphinfo{Epístola}{Fl. 4, 4-9 }
\begin{paracol}{2}\latim{
Léctio Epístolæ beáti Pauli Apóstoli ad Philippénses.
}\switchcolumn\portugues{
Lição da Ep.ª do B. Ap.º Paulo aos Filipenses.
}\switchcolumn*\latim{
\rlettrine{F}{ratres:} Gaudéte in Dómino semper: íterum dico, gaudéte. Modéstia vestra nota sit ómnibus homínibus: Dóminus prope est. Nihil sollíciti sitis; sed in omni oratióne et obsecratióne, cum gratiárum actióne, petitiónes vestræ innotéscant apud Deum. Et pax Dei quæ exsúperat omnem sensum, custódiat corda vestra et intelligéntias vestras, in Christo Jesu. De cétero, fratres, quæcúmque sunt vera, quæcúmque púdica, quæcúmque justa, quæcúmque sancta, quæcúmque amabília, quæcúmque bonæ famæ, si qua virtus, si qua laus disciplínæ, hæc cogitáte. Quæ et didicístis, et accepístis, et audístis, et vidístis in me, hæc agite: et Deus pacis erit vobíscum. 
}\switchcolumn\portugues{
\rlettrine{M}{eus} irmãos: Regozijai-vos incessantemente no Senhor. Eu vo-lo repito: regozijai-vos. Que todos os homens vejam a vossa alegria. Não vos inquieteis com coisa alguma; mas mostrai a Deus pelas vossas orações e súplicas todas vossas necessidades. Que a paz de Deus, que ultrapassa toda nossa inteligência, guarde os vossos corações e inteligências, em Jesus Cristo. Quanto ao mais, irmãos, tudo o que ó verdadeiro, tudo o que é honesto, tudo o ,que é justo, tudo o que é santo, tudo o que é amável, tudo o que tem bom nome, qualquer virtude, qualquer coisa digna de louvor da disciplina, seja isto o objecto dos vossos pensamentos. O que aprendestes, recebestes, ouvistes e vistes em mim, praticai-o. E Deus de paz será convosco.
}\end{paracol}

\paragraphinfo{Gradual}{Sl. 36, 3-5}
\begin{paracol}{2}\latim{
\rlettrine{S}{pera} in Dómino, et fac bonitátem, et inhábita terram, et pascéris in divítiis ejus. ℣. Delectáre in Dómino, et dabit tibi petitiónes cordis tui; revéla Dómino viam tuam et spera in eo, et ipse fáciet.
}\switchcolumn\portugues{
\rlettrine{E}{spera} no Senhor e pratica obras boas: então habitarás na terra e te sustentarás com suas riquezas. ℣. Alegra-te no Senhor e conceder-te-á o que teu coração deseja. Expõe ao Senhor a tua situação e tem esperança: Ele te atenderá. 
}\switchcolumn*\latim{
Allelúja, allelúja. ℣. \emph{Ps. 73, 21} Pauper et inops laudábunt nomen tuum. Allelúja.
}\switchcolumn\portugues{
Aleluia, aleluia. ℣. \emph{Sl. 73, 21} O pobre e o desvalido louvarão o vosso Nome. Aleluia.
}\end{paracol}

\textit{Após a Septuagésima omite-se o Aleluia e o Verso, e diz-se o:}

\paragraphinfo{Trato}{Sl. 60, 4-6}
\begin{paracol}{2}\latim{
\rlettrine{F}{actus} es spes mea, Dómine: turris fortitúdinis a fácie inimíci. ℣. Inhabitábo in tabernáculo tuo in sǽcula: prótegar in velaménto alárum tuárum. ℣. Quóniam tu, Deus, exaudísti oratiónem meam: dedísti hereditátem timéntibus nomen tuum.
}\switchcolumn\portugues{
\rlettrine{F}{izestes-Vos} a minha esperança: uma torre sólida contra o inimigo. ℣. Habitarei eternamente no vosso tabernáculo: abrigar-me-ei à sombra das vossas asas. ℣. Pois Vós, meu Deus, ouvistes a minha oração e destes a herança aos que temem o vosso nome.
}\end{paracol}

\textit{Durante o Tempo Pascal o Gradual é omitido e diz-se a seguinte Aleluia:}

\begin{paracol}{2}\latim{
Allelúja, allelúja. ℣. \emph{Ps. 73, 21 } Pauper et inops laudábunt nomen tuum. Allelúja. ℣. \emph{Ps. 35. 9} Inebriabúntur ab ubertáte domus tuæ: et torrénte voluptátis tuæ potábis eos. Allelúja.
}\switchcolumn\portugues{
Aleluia, aleluia. ℣. \emph{Sl. 73, 21 } O pobre e o desvalido louvarão o vosso Nome. Aleluia. ℣. \emph{Sl. 35. 9} Embriagar-se-ão com a abundância da vossa casa, e Vós os fareis beber na torrente das vossas delícias. Aleluia.
}\end{paracol}

\paragraphinfo{Evangelho}{Mt. 18, 1-5}
\begin{paracol}{2}\latim{
\cruz Sequéntia sancti Evangélii secúndum Matthǽum.
}\switchcolumn\portugues{
\cruz Continuação do santo Evangelho segundo S. Mateus.
}\switchcolumn*\latim{
\blettrine{I}{n} illo témpore: Accessérunt discípuli ad Jesum dicéntes: Quis, putas, major est in regno cælórum? Et ad vocans Jesus párvulum, státuit eum in médio eórum, et dixit: Amen dico vobis, nisi convérsi fueritis, et efficiámini sicut párvuli, non intrábitis in regnum cœlórum. Quicúmque ergo humiliáverit se sicut párvulus iste, hic est major in regno cœlórum. Et qui suscéperit unum párvulum talem in nómine meo, me súscipit.
}\switchcolumn\portugues{
\blettrine{N}{aquele} tempo, aproximaram-se de Jesus os discípulos, dizendo-lhe: «Qual pensais Vós que é o maior no reino dos céus?». E Jesus, havendo chamado um pequeno, colocou-o no meio deles e disse: «Em verdade vos digo: se vos não converteis e não vos tornais como os pequenos, não entrareis no reino dos céus. Todo aquele, pois, que se fizer pequeno, como este menino, esse é o maior no reino dos céus; e quem receber em meu nome um pequeno, como este, recebe-me a mim mesmo».
}\end{paracol}

\paragraphinfo{Ofertório}{Sl. 33, 12}
\begin{paracol}{2}\latim{
\rlettrine{V}{eníte,} fílii, audíte me: timórem Dómini docébo vos. 
}\switchcolumn\portugues{
\rlettrine{V}{inde,} filhos; ouvi-me. Ensinar-vos-ei o temor do Senhor.
}\end{paracol}

\paragraph{Secreta}
\begin{paracol}{2}\latim{
\rlettrine{S}{úscipe,} Dómine, oblatiónem mundam salutáris Hóstiæ, et præsta: ut, te in ómnibus et super ómnia diligéntes, in glóriæ tuæ laudem vívere mereámur. Per Dóminum \emph{\&c.}
}\switchcolumn\portugues{
\rlettrine{R}{ecebei,} Senhor, a oblação pura da Hóstia salutar, e fazei que, amando-Vos em tudo e sobretudo, mereçamos viver para louvar a vossa glória. Por nosso Senhor \emph{\&c.}
}\end{paracol}

\paragraphinfo{Comúnio}{Rm. 4, 18}
\begin{paracol}{2}\latim{
\rlettrine{C}{ontra} spem in spem crédidit, ut fíeret pater multárum géntium, secúndum quod dictum est ei.
}\switchcolumn\portugues{
\rlettrine{C}{ontra} toda a esperança, acreditou na esperança de que seria pai de muitas gentes, como o que lhe foi dito.
}\end{paracol}

\paragraph{Postcomúnio}
\begin{paracol}{2}\latim{
\rlettrine{C}{órporis} et Sánguinis tui, Dómine, mystério satiátis, concéde, quǽsumus; ut, intercedénte sancto Joánne Confessóre tuo, in gratiárum semper actióne maneámus: Qui vivis \emph{\&c.}
}\switchcolumn\portugues{
\rlettrine{S}{aciados,} Senhor, com o mistério do vosso Corpo e Sangue, concedei, como pedimos, que, intercedendo por nós S. João, Confessor, permaneçamos sempre em acção de graças. Vós, que \emph{\&c.}
}\end{paracol}