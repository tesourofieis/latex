\subsectioninfo{Nossa Senhora de Guadalupe}{12 de Dezembro}\footnote{No Brasil.}

\textit{Como na Missa Salve, sancta Parens da Virgem Maria, página \pageref{missamaria3}, excepto:}

\paragraph{Oração}
\begin{paracol}{2}\latim{
\rlettrine{D}{eus,} qui sub beatíssimæ Vírginis Maríæ singulári patrocínio constitútos, perpétuis benefíciis nos cumulári voluísti: præsta supplícibus tuis; ut cujus hódie commemoratióne lætámur in terris, ejus conspéctu perfruámur in cœlis. Per Dóminum \emph{\&c.}
}\switchcolumn\portugues{
\slettrine{Ó}{} Deus, que nos colocastes sob a protecção particular da Santíssima Virgem Maria, e nos quisestes cumular com perpétuos benefícios, concedei aos que Vos suplicam, que desfrutemos no céu a visão daquela cuja comemoração hoje na terra festejamos. Por nosso Senhor \emph{\&c.}
}\end{paracol}

\paragraphinfo{Epístola}{Página \pageref{montecarmelo}}

\paragraphinfo{Gradual}{Ct. 6, 9}
\begin{paracol}{2}\latim{
\qlettrine{Q}{uæ} est ista, quæ progréditur quasi auróra consúrgens, pulchra ut luna, elécta ut sol? Quasi arcus refúlgens inter nebulas glóriæ, et quasi flos rosárum in diébus vernis.
}\switchcolumn\portugues{
\qlettrine{Q}{uem} é esta que aparece como a aurora quando desponta, formosa como a lua, eleita, como o sol? É como o arco-íris, que resplandece entre as nuvens transparentes e como a rosa florescente no tempo da primavera.
}\switchcolumn*\latim{
Allelúja, allelúja. ℣. \emph{Cant. 2, 12} Flores apparuérunt in terra nostra, tempus putatiónis advénit. Allelúia.
}\switchcolumn\portugues{
Aleluta, aleluia. ℣. \emph{Ct. 2, 12} As flores apareceram em nossa terra; chegou o tempo da poda. Aleluía.
}\end{paracol}

\paragraphinfo{Evangelho}{Página \pageref{visitacao}}

\paragraphinfo{Ofertório}{2 Cr. 7, 16}
\begin{paracol}{2}\latim{
\rlettrine{E}{légi} et sanctificávi locum istum, ut sit ibi nomen meum, et permáneant óculi mei, et cor meum ibi cunctis diébus.
}\switchcolumn\portugues{
\rlettrine{E}{scolhi} e santifiquei este lugar, a fim de aí estar o meu nome, e estarem fixos nele os meus olhos e o meu coração, em todo o tempo.
}\end{paracol}

\paragraph{Secreta}
\begin{paracol}{2}\latim{
\rlettrine{T}{ua} Dómine, propitiatióne, et beátæ Maríæ semper Vírginis intercessióne, ad perpétuam atque præséntem hæc oblátio nobis profíciat prosperitátem et pacem. Per Dóminum \emph{\&c.}
}\switchcolumn\portugues{
\rlettrine{A}{proveite-nos,} Senhor, esta oblação para nossa perpétua e presente paz e prosperidade por vossa misericórdia e pela íntercessão da B. sempre Virgem Maria. Por nosso Senhor \emph{\&c.}
}\end{paracol}

\paragraphinfo{Comúnio}{Sl. 147, 20}
\begin{paracol}{2}\latim{
\rlettrine{N}{on} fecit táliter omni natióni: et judícia sua non manifestávit eis.
}\switchcolumn\portugues{
\rlettrine{N}{ão} fez assim a todas as nações, nem lhes manifestou os seus desígnios.
}\end{paracol}

\paragraph{Postcomúnio}
\begin{paracol}{2}\latim{
\rlettrine{S}{umptis,} Dómine, salútis nostræ subsídiis: da, quǽsumus, beátæ Maríæ semper Vírginis patrocíniis nos ubíque prótegi: in cujus veneration hæc tuæ obtúlimus majestáti. Per Dóminum nostrum \emph{\&c.}
}\switchcolumn\portugues{
\rlettrine{R}{ecebidos,} Senhor, os auxílios de nossa salvação, concedei que em todo o lugar, nos proteja o patrocínio da B. sempre Virgem Maria, em cuja honra oferecemos êstes santos Místérios à vossa divina Majestade. Por nosso Senhor \emph{\&c.}
}\end{paracol}
