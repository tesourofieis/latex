\subsectioninfo{S. Camilo de Lélis, Conf.}{18 de Julho}

\paragraphinfo{Intróito}{Jo. 15, 13}
\begin{paracol}{2}\latim{
\rlettrine{M}{ajórem} hac dilectiónem nemo habet, ut ánimam suam ponat quis pro amícis suis. \emph{Ps. 40, 2} Beátus, qui intéllegit super egénum et páuperem: in dic mala liberábit eum Dóminus.
℣. Gloria Patri \emph{\&c.}
}\switchcolumn\portugues{
\rlettrine{N}{inguém} pode dar maior prova de amor do que dar a sua vida pelos seus amigos. \emph{Sl. 40, 2} Bem-aventurado aquele que atende às necessidades do pobre e do indigente, pois o Senhor o livrará no dia da aflição.
℣. Glória ao Pai \emph{\&c.}
}\end{paracol}

\paragraph{Oração}
\begin{paracol}{2}\latim{
\rlettrine{D}{eus,} qui sanctum Camíllum, ad animárum in extrémo agóne luctántium subsídium, singulári caritátis prærogatíva decorásti: ejus, quǽsumus, méritis, spíritum nobis tuæ dilectiónis infúnde; ut in hora éxitus nostri hostem víncere, et ad cœléstem mereámur corónam perveníre. Per Dóminum \emph{\&c.}
}\switchcolumn\portugues{
\slettrine{Ó}{} Deus, que dotastes S. Camilo com a prerrogativa duma singular caridade para auxiliar as almas nas derradeiras lutas da agonia, infundi-nos pelos seus méritos, Vos suplicamos, o espírito do vosso amor, a fim de que, na hora da nossa morte, mereçamos vencer o inimigo e alcançar a coroa celestial. Por nosso Senhor \emph{\&c.}
}\end{paracol}

\paragraphinfo{Epístola}{Página \pageref{2domingopentecostes}}

\paragraphinfo{Gradual}{Página \pageref{confessoresnaopontifices2}}

\paragraphinfo{Evangelho}{Jo. 15, 12-16}
\begin{paracol}{2}\latim{
\cruz Sequéntia sancti Evangélii secúndum Joánnem.
}\switchcolumn\portugues{
\cruz Continuação do santo Evangelho segundo S. João.
}\switchcolumn*\latim{
\blettrine{I}{n} illo témpore: Dixit Jesus discípulis suis: Hoc est præcéptum meum, ut diligátis ínvicem, sicut diléxi vos. Majorem hac dilectiónem nemo habet, ut ánimam suam ponat quis pro amícis suis. Vos amíci mei estis, si fecéritis quæ ego præcípio vobis. Jam non dicam vos servos: quia servus nescit, quid fáciat dóminus ejus. Vos autem dixi amícos: quia ómnia, quæcúmque audívi a Patre meo, nota feci vobis. Non vos me elegístis: sed ego elegi vos, et pósui vos, ut eátís, et fructum afferátis: et fructus vester maneat: ut, quodcúmque petiéritis Patrem in nómine meo, det vobis.
}\switchcolumn\portugues{
\blettrine{N}{aquele} tempo, disse Jesus aos seus discípulos: «Este é o meu mandamento: «Que vos ameis uns aos outros, como vos amei». Ninguém pode ter maior amor do que dar a sua vida pelos seus amigos. Vós sereis meus amigos se fizerdes o que vos mando. Já vos não chamarei servos, porque o servo ignora o que faz o seu senhor. Chamo-vos meus amigos, porque tudo quanto ouvi a meu Pai vo-lo tenho feito conhecer. Não fostes vós que me escolhestes a mim, mas Eu é que vos escolhi e vos destinei, para que caminheis e alcanceis fruto. Que este fruto, pois, permaneça, para que meu Pai vos conceda tudo quanto Lhe pedirdes em meu nome».
}\end{paracol}

\paragraphinfo{Ofertório}{Sl. 20, 2-3}
\begin{paracol}{2}\latim{
\rlettrine{I}{n} virtúte tua, Dómine, lætábitur justus, et super salutáre tuum exsultábit veheménter: desidérium ánimæ ejus tribuísti ei.
}\switchcolumn\portugues{
\rlettrine{C}{om} o vosso poder, Senhor, se alegrará o justo, o qual exultará de alegria, vendo-se salvo por Vós. Concedestes-lhe, Senhor, o desejo da sua alma.
}\end{paracol}

\paragraph{Secreta}
\begin{paracol}{2}\latim{
\rlettrine{H}{óstia} immaculáta, qua illud Dómini nostri Jesu Christi imménsæ caritátis opus renovámus, sit, Deus Pater omnípotens, sancto Gamíllo intercedénte, contra omnes córporis et animae infirmitates salutáre remedium, et in extrémo agóne solátium et tutela. Per eúndem Dóminum \emph{\&c.}
}\switchcolumn\portugues{
\qlettrine{Q}{ue} a hóstia imaculada pela qual renovamos esta instituição da imensa caridade de nosso Senhor Jesus Cristo seja para nós, pela intercessão de S. Camilo, remédio salutar contra todas as enfermidades da alma e do corpo e na extrema agonia nos sirva de consolação e de protecção. Por nosso Senhor \emph{\&c.}
}\end{paracol}

\paragraphinfo{Comúnio}{Mt. 25, 36 \& 40}
\begin{paracol}{2}\latim{
\rlettrine{I}{nfírmus} fui, et visitástis me. Amen, amen, dico vobis: Quámdiu fecístis uni ex his frátribus meis minimis, mihi fecístis.
}\switchcolumn\portugues{
\rlettrine{E}{stive} enfermo e visitastes-me. Em verdade, em verdade vos digo: todas as vezes que fizerdes isto mesmo a um destes meus irmãos mais pequeninos, a mim mesmo o fizestes.
}\end{paracol}

\paragraph{Postcomúnio}
\begin{paracol}{2}\latim{
\rlettrine{P}{er} hæc cœléstia aliménta, quæ, sancti Camílli Confessóris tui sollémnia celebrántes, pia devotióne suscépimus: da, quǽsumus. Dómine;
ut, in hora mortis nostræ sacraméntis refécti et culpis ómnibus expiáti, in sinum misericórdiæ tuæ læti súscipi mereámur: Qui vivis \emph{\&c.}
}\switchcolumn\portugues{
\rlettrine{P}{or} estes alimentos celestiais, que recebemos com pia devoção celebrando a festa de S. Camilo, vosso Confessor, concedei-nos, Senhor, Vos suplicamos, que à hora da morte, munidos com os sacramentos e limpos de todas as culpas, mereçamos ser recebidos com alegria no seio da vossa misericórdia. Ó Vós, que, sendo Deus \emph{\&c.}
}\end{paracol}

\subsection{Comemoração de Santa Sinforoza e seus Filhos}

\paragraphinfo{Oração, Secreta e Postcomúnio}{Página \pageref{muitosmartires2}}
