\subsectioninfo{S. Apolinário, B. e Mártir}{23 de Julho}

\paragraphinfo{Intróito}{Página \pageref{martirpontifice}}

\paragraph{Oração}
\begin{paracol}{2}\latim{
\rlettrine{D}{eus,} fidélium remunerátor animárum, qui hunc diem beáti Apollináris Sacerdótis tui martýrio consecrásti: tríbue nobis, quǽsumus, fámulis tuis; ut, cujus venerándam celebrámus festivitátem, précibus ejus indulgéntiam consequámur. Per Dóminum nostrum \emph{\&c.}
}\switchcolumn\portugues{
\slettrine{Ó}{} Deus, remunerador das almas fiéis, que consagrastes este dia com o martírio do B. Apolinário, vosso Sacerdote, fazei que estes vossos servos alcancem a indulgência das suas faltas, pelas preces daquele cuja venerável festa celebramos. Por \emph{\&c.}
}\end{paracol}

\paragraphinfo{Epístola}{1. Pe. 5, 1-11}
\begin{paracol}{2}\latim{
Léctio Epístolæ beáti Petri Apóstoli.
}\switchcolumn\portugues{
Lição da Ep.ª do B. Ap.º Pedro.
}\switchcolumn*\latim{
\rlettrine{C}{aríssimi:} Senióres, qui in vobis sunt, obsecro, consénior et testis Christi passiónum: qui et ejus, quae in futúro revelánda est, glóriæ communicátor: páscite qui in vobis est gregem Dei, providéntes non coácte, sed spontánee secúndum Deum: neque turpis lucri grátia, sed voluntárie: neque ut dominántes in cleris, sed forma facti gregis ex ánimo. Et cum apparúerit princeps pastórum, percipiétis immarcescíbilem glóriæ corónam. Simíliter adolescéntes, súbditi estóte senióribus. Omnes autem ínvicem humilitátem insinuáte: quia Deus supérbis resístit, humílibus autem dat grátiam. Humiliámini ígitur sub poténti manu Dei, ut vos exáltet in témpore visitatiónis: omnem sollicitúdinem vestram projiciéntes in eum, quóniam ipsi cura est de vobis. Sobrii estóte, et vigiláte: quia adversárius vester diábolus tamquam leo rúgiens círcuit, quærens quem dévoret: cui resístite fortes in fide: scientes eándem passiónem ei, quæ in mundo est, vestræ fraternitáti fíeri. Deus autem omnis grátiæ, qui vocávit nos in ætérnam suam glóriam in Christo Jesu, módicum passos ipse perfíciet, confirmábit solidabítque. Ipsi glória et impérium in sǽcula sæculórum. Amen.
}\switchcolumn\portugues{
\rlettrine{C}{aríssimos:} Aos sacerdotes, que estão entre vós, rogo eu, sacerdote, como eles, e testemunha dos sofrimentos de Cristo, e, além disso, participante da glória, que deve ser manifestada: apascentai o rebanho de Deus, que vos foi confiado, vigiando-o não por força, mas espontaneamente, segundo Deus; não por amor de um lucro vergonhoso, mas com zelo; não como querendo ter domínio despótico naqueles que são a herança do Senhor, mas tornando-vos de todo o coração modelos do rebanho. E, quando aparecer o príncipe dos pastores, recebereis na glória a coroa incorruptível. E vós, também, ó jovens, sede submissos aos sacerdotes. Revesti-vos todos de humildade uns para com os outros, porque Deus resiste aos soberbos e dá a graça aos humildes. Humilhai-vos, pois, sob a mão poderosa de Deus, para que Ele vos exalte no tempo da sua visita; lançai no seu seio todas vossas inquietações, pois o demónio, vosso adversário, gira em torno de vós, como um leão a rugir, procurando devorar-vos. Resisti-lhe, portanto, permanecendo firmes na fé e recordando-vos de que os vossos irmãos, espalhados pelo mundo, sofrem as mesmas aflições que vós. Depois de haverdes padecido um pouco, Deus de toda a graça, que nos chamou em Jesus Cristo à eterna glória, vos aperfeiçoará, confirmará e consolidará. A Ele seja dada glória e homenagem em todos os séculos dos séculos. Amen.
}\end{paracol}

\paragraphinfo{Gradual}{Sl. 88, 21-23}
\begin{paracol}{2}\latim{
\rlettrine{I}{nvéni} David servum meum, óleo sancto meo unxi eum: manus enim mea auxiliábitur ei, et bráchium meum confortábit
eum. ℣. Nihil profíciet inimícus in eo, et fílius iniquitátis non nocébit ei.
}\switchcolumn\portugues{
\rlettrine{E}{ncontrei} o meu servo David e ungi-o com meu óleo sagrado: a minha mão o auxiliará e o meu braço o fortalecerá. ℣. O inimigo não terá nele domínio algum; o filho da iniquidade não poderá prejudicá-lo.
}\switchcolumn*\latim{
Allelúja, allelúja. ℣. \emph{Ps. 109, 4} Jurávit Dóminus, et non pœnitébit eum: Tu es sacérdos in ætérnum, secúndum órdinem Melchísedech. Allelúja.
}\switchcolumn\portugues{
Aleluia, aleluia. ℣. \emph{Ps. 109, 4} Jurou o Senhor e não se arrependerá: tu és sacerdote para sempre segundo a ordem de Melquisedeque. Aleluia.
}\end{paracol}

\paragraphinfo{Evangelho}{Lc. 22, 24-30}
\begin{paracol}{2}\latim{
\cruz Sequéntia sancti Evangélii secúndum Lucam.
}\switchcolumn\portugues{
\cruz Continuação do santo Evangelho segundo S. Lucas.
}\switchcolumn*\latim{
\blettrine{I}{n} illo témpore: Facta est conténtio inter discípulos, quis eórum viderétur esse major. Dixit autem eis Jesus: Reges géntium dominántur eórum; et qui potestátem habent super eos, benéfici vocántur. Vos autem non sic: sed qui major est in vobis, fiat sicut minor: et qui præcéssor est, sicut ministrátor. Nam quis major est, qui recúmbit, an qui mínistrat? nonne qui recúmbit? Ego autem in médio vestrum sum, sicut qui mínistrat. Vos autem estis, qui permansístis mecum in tentatiónibus meis: et ego dispóno vobis, sicut dispósuii mihi Pater meus regnum, ut edátis et bibátis super mensam meam in regno meo: et sedeátis super thronos, judicántes duódecim tribus Israël.
}\switchcolumn\portugues{
\blettrine{N}{aquele} tempo, levantou-se entre os discípulos uma contenda acerca de qual deles devia ser considerado o maior. Jesus disse-lhes: «Os reis das nações as dominam com sua autoridade, e aqueles que têm poder nela são chamados benfeitores. Mas entre vós não será assim; pois aquele que é o maior, faça-se como se fosse o menor; e aquele que governa, que se torne como o que serve. Dizei: qual é o maior: o que está à mesa ou o que serve? Não é verdade que é aquele que está à mesa? Ora Eu estou no meio de vós como aquele que serve. Já que ficastes comigo, constantemente, durante as minhas provações, preparo-vos um reino, como meu Pai o preparou para mim, a fim de que possais comer e beber à minha mesa no meu reino, e estejais sentados nos tronos, para julgar as doze tribos de Israel».
}\end{paracol}

\paragraphinfo{Ofertório}{Sl. 88, 25}
\begin{paracol}{2}\latim{
\rlettrine{V}{éritas} mea et misericórdia mea cum ipso: et in nómine meo exaltábitur cornu ejus.
}\switchcolumn\portugues{
\rlettrine{A}{} minha fidelidade e a minha misericórdia estarão com ele: e por virtude de meu nome será exaltado o seu poder.
}\end{paracol}

\paragraph{Secreta}
\begin{paracol}{2}\latim{
\rlettrine{R}{éspice,} Dómine, propítius super hæc múnera: quæ pro beáti Sacerdótis et Martyris tui Apollináris commemoratióne deférimus, et pro nostris offensiónibus immolámus. Per Dóminum \emph{\&c.}
}\switchcolumn\portugues{
\rlettrine{O}{lhai} propício, Senhor, para os dons que vos apresentamos em memória do vosso B. Sacerdote e Mártir Apolinário, os quais Vos oferecemos em sacrifício de expiação pelas nossas ofensas. Por nosso Senhor \emph{\&c.}
}\end{paracol}

\paragraphinfo{Comúnio}{Mt. 25, 20 \& 21}
\begin{paracol}{2}\latim{
\rlettrine{D}{ómine,} quinque talénta tradidísti mihi, ecce, ália quinque superlucrátus sum. Euge, serve bone et fidélis, quia in pauca fuísti fidélis, supra multa te constítuam, intra in gáudium Dómini tui.
}\switchcolumn\portugues{
\rlettrine{E}{ntregastes-me,} Senhor, cinco talentos; eis outros cinco que lucrei. Muito bem, servo fiel e bom. Porque foste fiel em bens de pouca importância, eu te estabelecerei sobre bens mais importantes. Entra no gozo do teu senhor.
}\end{paracol}

\paragraph{Postcomúnio}
\begin{paracol}{2}\latim{
\rlettrine{T}{ua} sancta suméntes, quǽsumus, Dómine, ut beáti Apollináris nos fóveant continuáta præsídia: quia non désinis propítius intuéri, quos tálibus auxíliis concésseris adjuvári. Per Dóminum \emph{\&c.}
}\switchcolumn\portugues{
\rlettrine{T}{endo} participado dos vossos sacrossantos mistérios, Vos suplicamos, Senhor, queirais defender-nos sempre com a protecção do B. Apolinário, pois não cessais de olhar propício para aqueles a quem concedeis tal socorro. Por nosso Senhor \emph{\&c.}
}\end{paracol}
