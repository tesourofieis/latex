\subsectioninfo{Santos Anjos da Guarda}{2 de Outubro}

\textit{Missa da Festa da Aparição de Arcanjo Miguel, página \pageref{aparicaoarcanjomiguel}, excepto:}

\paragraph{Oração}
\begin{paracol}{2}\latim{
\rlettrine{D}{eus,} qui ineffábili providéntia sanctos Angelos tuos ad nostram custódiam míttere dignáris: largíre supplícibus tuis; et eórum semper protectióne deféndi, et ætérna societáte gaudére. Per Dóminum \emph{\&c.}
}\switchcolumn\portugues{
\slettrine{Ó}{} Deus, que por vossa inefável Providência Vos dignastes mandar os vossos Anjos para nos guardarem, concedei aos vossos suplicantes a graça de serem sempre defendidos pela sua protecção e de gozarem eternamente, a sua companhia. Por nosso Senhor \emph{\&c.}
}\end{paracol}

\paragraphinfo{Epístola}{Ex. 23, 20-23}
\begin{paracol}{2}\latim{
Léctio libri Exodi.
}\switchcolumn\portugues{
Lição do Livro Êxodo.
}\switchcolumn*\latim{
\rlettrine{H}{æc} dicit Dóminus Deus: Ecce, ego mittam Angelum meum, qui præcédat te, et custódiat in via, et intróducat in locum, quem parávi. Obsérva eum, et audi vocem ejus, nec contemnéndum putes: quia non dimíttet, cum peccáveris, et est nomen meum in illo. Quod si audiéris vocem ejus et féceris ómnia, quæ loquor, inimícus ero inimícis tuis et affligam affligéntes te: præcedétque te Angelus meus.
}\switchcolumn\portugues{
\rlettrine{I}{sto} diz o Senhor, nosso Deus: «Eis que enviarei o meu Anjo, a fim de que ele vá adiante de vós; vos guarde no caminho; e vos introduza no lugar que vos preparei. Respeitai-o, ouvi a sua voz e não deixeis de atendê-lo, porque vos não perdoará, quando pecardes. Ele vos falará em meu nome. Se escutardes a sua voz e se fizerdes o que vos ordenar, serei inimigo do vosso inimigo e afligirei aqueles que vos afligirem, pois o meu Anjo preceder-vos-á».
}\end{paracol}

\paragraphinfo{Gradual}{Sl. 90,11-12}
\begin{paracol}{2}\latim{
\rlettrine{A}{ngelis} suis Deus mandávit de te, ut custódiant te in ómnibus viis tuis. ℣. In mánibus portábunt te, ne umquam offéndas ad lápidem pedem tuum.
}\switchcolumn\portugues{
\rlettrine{D}{eus} mandou aos seus Anjos que te guardassem em todas as tuas vias. ℣. E eles te conduzirão em cima de suas mãos, para que o teu pé não tropece.
}\switchcolumn*\latim{
Allelúja, allelúja. ℣. \emph{Ps. 102, 21} Benedícite Dómino, omnes virtútes ejus: minístri ejus, qui fácitis voluntátem ejus. Allelúja.
}\switchcolumn\portugues{
Aleluia, aleluia. ℣. \emph{Sl. 102, 21} Bendizei todos o Senhor, ó exércitos do Senhor; pois estais ao seu serviço e desempenhais as suas ordens. Aleluia.
}\end{paracol}

\paragraphinfo{Ofertório}{Sl. 102, 20 \& 21}
\begin{paracol}{2}\latim{
\rlettrine{B}{enedícite} Dóminum, omnes Angeli ejus: minístri ejus, qui fácitis verbum ejus, ad audiéndam vocem sermónum ejus.
}\switchcolumn\portugues{
\rlettrine{B}{endizei} o Senhor, ó vós, Anjos, heróis poderosos, executores das suas ordens e sempre fiéis aos seus chamamentos.
}\end{paracol}

\paragraph{Secreta}
\begin{paracol}{2}\latim{
\rlettrine{S}{úscipe,} Dómine, múnera, quæ pro sanctórum Angelórum tuórum veneratióne deférimus: et concéde propítius; ut, perpétuis eórum præsídiis, a præséntibus perículis liberémur et ad vitam perveniámus ætérnam. Per Dóminum nostrum \emph{\&c.}
}\switchcolumn\portugues{
\rlettrine{R}{ecebei,} Senhor, as ofertas que Vos apresentamos em honra dos vossos Santos Anjos e concedei-nos propício que pela sua contínua protecção sejamos livres dos perigos da vida presente e alcancemos a vida eterna. Por nosso Senhor \emph{\&c.}
}\end{paracol}

\paragraph{Postcomúnio}
\begin{paracol}{2}\latim{
\rlettrine{S}{úmpsimus,} Dómine, divína mystéria, sanctórum Angelórum tuórum festivitáte lætántes: quǽsumus; ut eórum protectióne ab hóstium júgiter liberémur insídiis, et contra ómnia advérsa muniámur. Per Dóminum \emph{\&c.}
}\switchcolumn\portugues{
\rlettrine{H}{avendo} recebido os divinos mystérios enquanto celebramos com júbilo a festa dos vossos Santos Anjos, dignai-Vos permitir, Vos suplicamos, que pela sua protecção sejamos sempre livres das insídias dos nossos inimigos e de todas as adversidades. Por nosso Senhor \emph{\&c.}
}\end{paracol}
