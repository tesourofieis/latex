\subsectioninfo{Sexta-feira da Semana da Paixão}{Estação em Santo Estêvão no Monte Célio}\label{paixaosexta}

\paragraphinfo{Intróito}{Sl. 30, 10, 16 \& 18}
\begin{paracol}{2}\latim{
\rlettrine{}{}Miserére mihi, Dómine, quóniam tríbulor: líbera me, et éripe me de mánibus inimicórum meórum et a persequéntibus me: Dómine, non confúndar, quóniam invocávi te. \emph{Ps. ib., 2} In te, Dómine, sperávi, non confúndar in ætérnum: in justítia tua libera me.
}\switchcolumn\portugues{
\rlettrine{T}{ende} misericórdia de mim, Senhor,
pois estou aflito; salvai-me e livrai-me das mãos dos meus inimigos e perseguidores. Senhor, não serei confundido, porque Vos invoquei. \emph{Sl. ib., 2} Confio em Vós, Senhor, não serei para sempre confundido; livrai-me segundo a vossa justiça.
}\end{paracol}

\paragraph{Oração}
\begin{paracol}{2}\latim{
\rlettrine{C}{órdibus} nostris, quǽsumus, Dómine, grátiam tuam benígnus infúnde: ut peccáta nostra castigatióne voluntária cohibéntes, temporáliter pótius macerémur, quam súppliciis deputémur ætérnis. Per Dóminum \emph{\&c.}
}\switchcolumn\portugues{
\rlettrine{I}{nfundi} benigno, Senhor, Vos suplicamos, a vossa graça em nossos corações, para que, punindo-nos voluntariamente por causa dos nossos pecados, evitemos com estas penas temporais a condenação aos castigos eternos. Por nosso Senhor \emph{\&c.}
}\end{paracol}

\paragraphinfo{Epístola}{Jr. 17, 13-18}
\begin{paracol}{2}\latim{
Léctio Jeremíæ Prophétæ.
}\switchcolumn\portugues{
Lição do Profeta Jeremias.
}\switchcolumn*\latim{
\rlettrine{I}{n} diébus illis: Dixit Jeremías: Dómine, omnes, qui te derelínquunt, confundéntur: recedéntes a te in terra scribéntur: quóniam dereliquérunt venam aquárum vivéntium Dóminum. Sana me. Dómine, et sanábor: salvum me fac, et salvus ero: quóniam laus mea tu es. Ecce, ipsi dicunt ad me: Ubi est verbum Dómini? Véniat. Et ego non sum turbátus, te pastórem sequens: et diem hóminis non desiderávi, tu scis. Quod egréssum est de lábiis meis, rectum in conspéctu tuo fuit. Non sis tu mihi formídini, spes mea tu in die afflictiónis. Confundántur, qui me persequúntur, et non confúndar ego: páveant illi, et non páveam ego. Induc super eos diem afflictiónis, et dúplici contritióne cóntere eos, Dómine, Deus noster.
}\switchcolumn\portugues{
\rlettrine{N}{aqueles} dias, disse Jeremias: «Senhor, todos os que se afastaram de Vós serão confundidos; todos os que se afastaram de Vós ficarão envergonhados, pois abandonaram o Senhor, que é a fonte das águas vivas. Curai-me, Senhor, e serei curado; salvai-me, e serei salvo; pois sois a minha glória. Eis o que eles me dizem: «Onde está a palavra do Senhor? Que ela se cumpra!». Mas não me perturbei, seguindo-Vos, como a um pastor. Não desejei ao infeliz, bem o sabeis, o dia da sua aflição. As palavras que saíram dos meus lábios eram rectas na vossa presença. Não sejais causa de terror para mim, pois sois a minha esperança no dia da aflição. Caiam em confusão os que me perseguem; mas não seja eu confundido. Que eles fiquem assombrados; mas não o seja eu. Fazei que para eles se aproxime o dia da aflição; esmagai-os, sobrecarregando-os com males, ó Senhor, nosso Deus».
}\end{paracol}

\paragraphinfo{Gradual}{Sl. 34, 20 \& 22}
\begin{paracol}{2}\latim{
\rlettrine{P}{acífice} loquebántur mihi inimíci mei: et in ira molésti erant mihi. ℣. Vidísti, Dómine, ne síleas: ne discédas a me.
}\switchcolumn\portugues{
\rlettrine{O}{s} meus inimigos falavam-me aparentemente com palavras de paz; mas na sua ira meditavam perfídias contra mim. Vós bem o vistes, Senhor: não sejais insensível, não Vos afasteis de mim.
}\end{paracol}

\paragraphinfo{Trato}{Página \pageref{tratosemanapaixao}}

\paragraphinfo{Evangelho}{Jo. 11, 47-54}
\begin{paracol}{2}\latim{
\cruz Sequéntia sancti Evangélii secúndum Joánnem.
}\switchcolumn\portugues{
\cruz Continuação do santo Evangelho segundo S. João.
}\switchcolumn*\latim{
\blettrine{I}{n} illo témpore: Collegérunt pontífices et pharisǽi concílium advérsus Jesum, et dicébant: Quid fácimus, quia hic homo multa signa facit? Si dimíttimus eum sic, omnes credent in eum: et vénient Románi, et tollent nostrum locum et gentem. Unus autem ex ipsis, Cáiphas nómine, cum esset póntifex anni illíus, dixit eis: Vos nescítis quidquam, nec cogitátis, quia expédit vobis, ut unus moriátur homo pro pópulo, et non tota gens péreat. Hoc autem a semetípso non dixit: sed cum esset póntifex anni illíus, prophetávit, quod Jesus moritúrus erat pro gente, et non tantum pro gente, sed ut fílios Dei, qui erant dispérsi, congregáret in unum. Ab illo ergo die cogitavérunt, ut interfícerent eum. Jesus ergo jam non in palam ambulábat apud Judǽos: sed ábiit in regiónem juxta desértum, in civitátem, quæ dícitur Ephrem, et ibi morabátur cum discípulis suis.
}\switchcolumn\portugues{
\blettrine{N}{aquele} tempo, reuniram-se os pontífices e os fariseus em assembleia contra Jesus, dizendo: «Que faremos nós a este homem, que opera tantos prodígios? Se o deixarmos livre, todos acreditarão n’Ele; e então os romanos virão e desfruirão a nossa cidade e a nossa nação». Ora, um deles, Caifás, que era naquele ano o sumo sacerdote, disse-lhes: «Vós nada sabeis! Não pensais que é melhor que morra um só homem, por causa do povo, do que padeça toda a nação, por causa d’Ele?». Caifás não disse estas palavras em nome próprio; mas, como sumo sacerdote que era naquele ano, profetizou que Jesus devia morrer pela nação — e não somente pela nação, mas para reunir em um só corpo os filhos de Deus, que estavam dispersos. Desde aquele dia, pois, resolveram matá-l’O. Porém, Jesus já se não mostrava abertamente aos judeus. Retirara-se para um lugar, perto do deserto, na cidade de Efrem, onde permanecia com seus discípulos.
}\end{paracol}

\paragraphinfo{Ofertório}{Sl. 118, 12, 121 \& 42}
\begin{paracol}{2}\latim{
\rlettrine{B}{enedíctus} es, Dómine, doce me justificatiónes tuas: et non tradas calumniántibus me supérbis: et respondébo exprobrántibus mihi verbum.
}\switchcolumn\portugues{
\rlettrine{B}{endito} sois, Senhor. Ensinai-me a conhecer as vossas leis. Não me entregueis àqueles que me perseguem: e eu saberei responder àqueles que me insultam.
}\end{paracol}

\paragraph{Secreta}
\begin{paracol}{2}\latim{
\rlettrine{P}{ræsta} nobis, miséricors Deus: ut digne tuis servíre semper altáribus mereámur; et eórum perpétua participatióne salvári. Per Dóminum nostrum \emph{\&c.}
}\switchcolumn\portugues{
\slettrine{Ó}{} Deus misericordioso, concedei-nos a graça de servirmos sempre dignamente os vossos altares, a fim de que, participando perpetuamente deles, possamos salvar-nos. Por nosso Senhor \emph{\&c.}
}\end{paracol}

\paragraphinfo{Comúnio}{Sl. 26, 12}
\begin{paracol}{2}\latim{
\rlettrine{N}{e} tradíderis me. Dómine, in animas persequéntium me: quóniam insurrexérunt in me testes iníqui, et mentíta est iníquitas sibi.
}\switchcolumn\portugues{
\rlettrine{S}{enhor,} não me abandoneis ao ódio dos que me perseguem, quando se levantam contra mim com testemunhos iníquos; pois a iniquidade contradiz-se a si própria.
}\end{paracol}

\paragraph{Postcomúnio}
\begin{paracol}{2}\latim{
\rlettrine{S}{umpti} sacrifícii, Dómine, perpetua nos tuítio non derelínquat: et nóxia semper a nobis cuncta depéllat. Per Dóminum \emph{\&c.}
}\switchcolumn\portugues{
\qlettrine{Q}{ue} nos não desampare, Senhor, a protecção do sacramento agora recebido, e que para sempre afaste de nós todos os males. Por nosso Senhor \emph{\&c.}
}\end{paracol}

\paragraph{Oração sobre o povo}
\begin{paracol}{2}\latim{
\begin{nscenter} Orémus. \end{nscenter}
}\switchcolumn\portugues{
\begin{nscenter} Oremos. \end{nscenter}
}\switchcolumn*\latim{
Humiliáte cápita vestra Deo.
}\switchcolumn\portugues{
Inclinai as vossas cabeças diante de Deus.
}\switchcolumn*\latim{
Concéde, quǽsumus, omnípotens Deus: ut, qui protectiónis tuæ grátiam quǽrimus, liberáti a malis ómnibus, secúra tibi mente serviámus. Per Dóminum \emph{\&c.}
}\switchcolumn\portugues{
Deus omnipotente, Vos suplicamos, concedei-nos a graça da vossa protecção, que sempre procurámos, a fim de que sejamos livres de todos os males e Vos sirvamos com a alma tranquila. Por nosso Senhor \emph{\&c.}
}\end{paracol}
