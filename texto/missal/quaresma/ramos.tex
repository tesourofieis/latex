\subsection{Domingo dos Ramos}

\subsubsection{Benção dos Ramos}

\paragraph{Antífona}
\begin{paracol}{2}\latim{
\rlettrine{H}{osánna} fílio David: benedíctus, qui venit in nómine Dómini. O Rex Israël: Hosánna in excélsis.
}\switchcolumn\portugues{
\rlettrine{H}{osana} ao filho de David! Bendito seja o que vem em nome do Senhor. Ó Rei de Israel! Hosana no alto dos céus!
}\end{paracol}

\paragraph{Oração}
\begin{paracol}{2}\latim{
\rlettrine{D}{eus,} quem dilígere et amáre justítia est, ineffábilis grátiæ tuæ in nobis dona multíplica: et qui fecísti nos in morte Fílii tui speráre quæ crédimus; fac nos eódem resurgénte perveníre quo téndimus: Qui tecum \emph{\&c.}
}\switchcolumn\portugues{
\slettrine{Ó}{} Deus, a quem devemos amar para sermos justos, multiplicai na nossa alma os dons da vossa inefável graça; e, já que pela morte do vosso Filho quisestes que tivéssemos esperança no que constitui o objecto da nossa Fé, permiti que pela sua ressurreição alcancemos o fim a que aspiramos: O qual, sendo Deus \emph{\&c.}
}\end{paracol}

\paragraphinfo{Epístola}{Ex. 15, 27; 16, 1-7}
\begin{paracol}{2}\latim{
Léctio libri Exodi.
}\switchcolumn\portugues{
Lição do Livro do Êxodo.
}\switchcolumn*\latim{
\rlettrine{I}{n} diébus illis: Venérunt fílii Israël in Elim, ubi erant duódecim fontes aquárum et septuagínta palmæ: et castrametáti
sunt juxta aquas. Profectíque sunt de Elim, et venit omnis multitúdo filiórum Israël in desértum Sin, quod est inter Elim et Sínai: quintodécimo die mensis secúndi, postquam egréssi sunt de terra Ægýpti. Et murmurávit omnis congregátio filiórum Israël contra Móysen et Aaron in solitúdine. Dixerúntque fílii Israël ad eos: Utinam mórtui essëmus per manum Dómini in terra Ægýpti, quando sedebámus super oílas cárnium, et comedebámus panem in saturitáte: cur eduxístis nos in desértum istud, ut occiderétis omnem multitúdinem fame? Dixit autem Dóminus ad Móysen: Ecce, ego pluam vobis panes de cœlo: egrediátur pópulus, et cólligat quæ suffíciunt per síngulos dies: ut tentem eum, utrum ámbulet in lege mea an non. Die autem sexto parent quod ínferant: et sit duplum, quam collígere sciébant per síngulos dies. Dixerúntque Móyses et Aaron ad omnes fílios Israël: Véspere sciétis, quod Dóminus edúxerit vos de terra Ægýpti: et mane vidébitis glóriam Dómini.
}\switchcolumn\portugues{
\rlettrine{N}{aqueles} dias, chegaram os filhos de Israel a Elim, onde havia doze nascentes de água e setenta palmeiras, tendo acampado junto das águas. Partiu, então, de Elim toda a multidão dos filhos de Israel, havendo chegado ao deserto de Sin, que é situado entre Elim e o Sinai, no dia 15 do segundo mês, depois que safram da terra do Egipto. Apenas chegaram, começaram todos a murmurar contra Moisés e Aarão por estarem no deserto, dizendo: «Oxalá tivéssemos sido mortos pela mão do Senhor, quando na terra do Egipto nos assentávamos em frente das caçarolas, cheias de carne, e tínhamos pão em abundância! Porque nos conduzistes a este deserto? Para aqui morrer de fome esta multidão?». Então o Senhor disse a Moisés: «Vou mandar chover pães do céu. Que o povo, pois, saia e recolha a quantidade necessária para cada dia, a fim de que verifique se procede ou não segundo a minha lei; porém, no sexto dia, prepararão o que tiverem recolhido, que será o duplo do que costumam recolher em cada um dos outros dias». Disseram, pois, Moisés e Aarão a todos os filhos de Israel: «Esta tarde reconhecereis que foi o Senhor quem vos livrou da terra do Egipto e amanhã vereis resplandecer a glória do Senhor».
}\switchcolumn*\latim{
℟. \emph{Joann. 11, 47-49, 50 \& 53} Collegérunt pontífices et pharisǽi concílium, et dixérunt: Quid fácimus, quia hic homo multa signa facit? Si dimíttimus eum sic, omnes credent in eum: Et vénient Románi, et tollent nostrum locum et gentem. ℣. Unus autem ex illis, Cáiphas nómine, cum esset póntifex anni illíus, prophetávit dicens: Expedit vobis, ut unus moriátur homo pro pópulo, et non tota gens péreat. Ab illo ergo die cogitavérunt interfícere eum, dicéntes. Et vénient...
}\switchcolumn\portugues{
℟.\emph{Jo. 11, 47-49, 50 \& 53} Os pontífices e os fariseus reuniram-se em conselho e disseram: «Que faremos? Este homem faz muitos prodígios. Se o deixamos andar livremente, todos acreditarão n’Ele. E virão os romanos, e destruirão a nossa terra e o nosso povo». ℣. Mas um deles, chamado Caifás, que era o Pontífice naquele ano, disse profeticamente: «É melhor que morra um só homem pelo povo, do que pereça toda a nação». Desde aquele dia, pois, resolveram matá-l’O, dizendo: «E virão os romanos…»
}\switchcolumn*\latim{
℟.\emph{Mt. 28, 39 \& 41} In monte Olivéti orávit ad Patrem: Pater, si fíeri potest, tránseat a me calix iste. Spíritus quidem promptus est, caro autem infírma: fiat volúntas tua. ℣. Vigiláte et oráte, ut non intrétis in tentatiónem. Spíritus quidem...
}\switchcolumn\portugues{
℟.\emph{Mt. 28, 39 \& 41} No monte das Oliveiras orou a seu Pai: Meu pai, se é possível, afastai de mim este cálice! Na verdade, o espírito está pronto: mas a carne é fraca; contudo, faça-se a vossa vontade. ℣. Vigiai e orai, para não cairdes em tentação. Na verdade...
}\end{paracol}

\paragraphinfo{Evangelho}{Mt. 21, 1-9}
\begin{paracol}{2}\latim{
\cruz Sequéntia sancti Evangélii secúndum Matthǽum.
}\switchcolumn\portugues{
\cruz Continuação do santo Evangelho segundo S. Mateus.
}\switchcolumn*\latim{
\blettrine{I}{n} illo témpore: Cum appropinquásset Jesus Jerosólymis, et venísset Béthphage ad montem Olivéti: tunc misit duos discípulos suos, dicens eis: Ite in castéllum, quod contra vos est, et statim inveniétis ásinam alligátam et pullum cum ea: sólvite et addúcite mihi: et si quis vobis áliquid dixerit, dícite, quia Dóminus his opus habet, et conféstim dimíttet eos. Hoc autem totum factum est, ut adimplerétur, quod dictum est per Prophétam, dicéntem: Dícite fíliae Sion: Ecce, Rex tuus venit tibi mansuétus, sedens super ásinam et pullum, fílium subjugális. Eúntes autem discípuli, fecérunt, sicut præcépit illis Jesus. Et adduxérunt ásinam et pullum: et imposuérunt super eos vestiménta sua, et eum désuper sedére tecérunt. Plúrima autem turba stravérunt vestiménta sua in via: álii autem cædébant ramos de arbóribus, et sternébant in via: turbæ autem, quæ præcedébant et quæ sequebántur, clamábant, dicéntes: Hosánna fílio David: benedíctus, qui venit in nómine Dómini.
}\switchcolumn\portugues{
\blettrine{N}{aquele} tempo, como Jesus se aproximasse de Jerusalém e chegasse a Bétfage, já perto do monte das Oliveiras, mandou dous dos seus discípulos, dizendo-lhes: «Ide à aldeia fronteira e lá encontrareis uma jumenta presa e com ela um jumentinho. Desprendei-a e trazei-os. Se alguém vos disser alguma cousa, respondei: «O Senhor precisa deles». E logo os deixarão trazer». Tudo isto aconteceu para se cumprir o que fora anunciado pelo Profeta: «Dizei à filha de Sião: «Eis o teu Rei, que vem a ti com doçura, montado em uma jumenta e sobre um jumentinho, filho da que está sob o jugo». Foram os discípulos e fizeram tudo como Jesus lhes ordenara, trazendo a jumenta e o jumentinho. Então puseram em cima deles as suas capas e fizeram-n’O montar. Ora a multidão, que era numerosa, estendia as suas capas na estrada e cortava ramos das árvores, com que atapetava o caminho. E os da multidão, tanto os que O precediam, como os que O seguiam, clamavam: «Hosana ao Filho de David! Bendito seja o que vem em nome do Senhor!».
}\end{paracol}

\paragraph{Oração}
\begin{paracol}{2}\latim{
\rlettrine{A}{uge} fidem in te sperántium, Deus, et súpplicum preces cleménter exáudi: véniat super nos múltiplex misericórdia tua: bene \cruz dicántur et hi pálmites palmárum seu olivárum: et sicut in figúra Ecclésiæ multiplicásti Noë egrediéntem de arca, et Móysen exeúntem de Ægýpto cum fíliis Israël: ita nos, portántes palmas et ramos olivárum, bonis áctibus occurrámus óbviam Christo: et per ipsum in gáudium introëámus ætérnum: Qui tecum \emph{\&c.}
}\switchcolumn\portugues{
\slettrine{Ó}{} Deus, aumentai a fé daqueles que esperam em Vós e ouvi clemente as suas súplicas. Permiti que a vossa misericórdia desça sobre nós; dignai-Vos abençoar estes Ramos de palmeira e de oliveira; e, assim como, querendo figurar a Igreja, multiplicastes as vossas graças sobre Noé, saindo da arca, e sobre Moisés, saindo do Egipto com os filhos de Israel, assim também permiti que, levando nós estas palmas e Ramos de oliveira, caminhemos ao encontro de Cristo pelas nossas boas obras; e que com Ele entremos na alegria eterna: Ele que, sendo Deus \emph{\&c.}
}\end{paracol}

\paragraph{Oração}
\begin{paracol}{2}\latim{
\rlettrine{P}{étimus,} Dómine sancte, Pater omnípotens, ætérne Deus: ut hanc creatúram olívæ, quam ex ligni matéria prodíre jussísti, quamque colúmba rédiens ad arcam próprio pértulit ore, bene \cruz dícere et sancti \cruz ficáre dignéris: ut, quicúmque ex ea recéperint, accípiant sibi protectiónem ánimæ et córporis: fiátque, Dómine, nostræ salútis remédium tuæ grátiæ sacraméntum. Per Dóminum nostrum \emph{\&c.}
}\switchcolumn\portugues{
\rlettrine{V}{os} imploramos, Senhor santo, Pai omnipotente, eterno Deus, que Vos digneis \cruz abençoar e \cruz santificar estes Ramos de oliveira, vossa Criatura, que fizestes nascer na árvore, semelhantes ao que a pomba levava no bico quando regressou à arca. Permiti que aqueles que receberam estes Ramos obtenham a vossa protecção na alma e no corpo; e que estes Ramos, Senhor, que são um sinal da vossa graça, se convertam em remédio eficaz para as nossas enfermidades. Por nosso Senhor \emph{\&c.}
}\end{paracol}

\paragraph{Oração}
\begin{paracol}{2}\latim{
\rlettrine{D}{eus,} qui dispérsa cóngregas, et congregáta consérvas: qui pópulis, óbviam Jesu ramos portántibus, benedixísti: béne \cruz dic étiam hos ramos palmæ et olívæ, quos tui fámuli ad honórem nóminis tui fidéliter suscípiunt; ut, in quemcúmque locum introdúcti fúerint, tuam benedictiónem habitatóres loci illíus consequántur: et, omni adversitáte effugáta, déxtera tua prótegat, quos rédemit Jesus Christus, Fílius tuus, Dóminus noster: Qui tecum \emph{\&c.}
}\switchcolumn\portugues{
\slettrine{Ó}{} Deus, que reunis o que está disperso, e, depois de reunido, o conservais, Vós, que abençoastes o povo que saiu com ramos ao encontro de Jesus, abençoai \cruz, também, estes Ramos de palmeira e de oliveira, que os vossos fiéis servos vão receber em honra do vosso nome, a fim de que, em qualquer lugar em que sejam colocados, aqueles que habitarem nesse lugar consigam a vossa bênção, e, afastada toda a adversidade, a vossa dextra proteja os que foram remidos por Jesus Cristo, vosso Filho, nosso Senhor: O qual, sendo Deus \emph{\&c.}
}\end{paracol}

\paragraph{Oração}
\begin{paracol}{2}\latim{
\rlettrine{D}{eus,} qui miro dispositiónis órdine, ex rebus étiam insensibílibus, dispensatiónem nostræ salútis osténdere voluísti: da, quǽsumus; ut devota tuórum corda fidélium salúbriter intéllegant, quid mýstice desígnet in facto, quod hódie, cœlésti lúmine affláta, Redemptóri óbviam procédens, palmárum atque olivárum ramos vestígiis ejus turba substrávit. Palmárum igitur rami de mortis príncipe triúmphos exspéctant; súrculi vero olivárum spirituálem unctiónem advenísse Quodámmodo clamant. Intelléxit enim jam tunc illa hóminum beáta multitúdo præfigurári: quia Redémptor noster, humánis cóndolens misériis, pro totíus mundi vita cum mortis príncipe esset pugnatúrus ac moriéndo triumphatúrus. Et ídeo tália óbsequens administrávit, quæ in illo ei triúmphos victóriæ et misericórdiæ pinguédinem declarárent. Quod nos quoque plena fide, et factum et significátum retinéntes, te, Dómine sancte, Pater omnípotens, ætérne Deus, per eúndem Dóminum nostrum Jesum Christum supplíciter exorámus: ut in ipso atque per ipsum, cujus nos membra fíeri voluísti, de mortis império victóriam reportántes, ipsíus gloriósæ resurrectiónis partícipes esse mereámur: Qui tecum \emph{\&c.}
}\switchcolumn\portugues{
\slettrine{Ó}{} Deus, que por um maravilhoso desígnio da vossa providência quisestes utilizar-Vos das cousas mesmo insensíveis para mostrar a admirável economia da nossa salvação, ilustrai, Vos imploramos, os corações dos vossos fiéis servos, para que compreendam salutarmente o mystério apresentado na acção daquele povo que, levado por inspiração celestial, caminhou neste dia ao encontro do Redentor e atapetou com ramos de palmeira e de oliveira o caminho por onde Ele devia passar. Com efeito, os ramos de palmeira significavam a vitória que ia alcançar sobre o príncipe da morte e os de oliveira publicavam, de certo modo, a união espiritual que ia ser espalhada. Esta feliz multidão de homens pressentiu, então, que o nosso Redentor, comovido com as misérias da humanidade, ia travar combate com o príncipe da morte, para dar a vida ao mundo inteiro, e que Ele triunfaria pela sua própria morte. Por isso, o povo ofereceu ao Senhor a homenagem destes Ramos, dos quais uns significavam a vitória e o triunfo e os outros a efusão da sua misericórdia. Nós, pois, que possuímos a plenitude da fé, vendo neste acontecimento não só o facto mas ainda a significação, Vos pedimos, Senhor santo, Pai omnipotente, Deus eterno, pelo mesmo N. S. Jesus Cristo, de quem houvestes por graça fazer-nos membros, que n’Ele e por Ele triunfemos do império da morte e sejamos dignos de participar da sua gloriosa ressurreição: O qual \emph{\&c.}
}\end{paracol}

\paragraph{Oração}
\begin{paracol}{2}\latim{
\rlettrine{D}{eus,} qui, per olívæ ramum, pacem terris colúmbam nuntiáre jussísti: præsta, quǽsumus; ut hos olívæ ceterarúmque arbórum ramos cœlésti bene \cruz dictióne sanctífices: ut cuncto pópulo tuo profíciant ad salútem. Per Christum, Dóminum nostrum. ℟. Amen.
}\switchcolumn\portugues{
\slettrine{Ó}{} Deus, que quisestes que uma pomba anunciasse a paz ao mundo com um ramo de Oliveira, dignai-Vos santificar com vossa bênção \cruz celestial, Vos pedimos, estes Ramos de oliveira e doutras árvores, a fim de que sirva de proveito a todo vosso povo para sua salvação. Por Cristo, nosso Senhor. ℟. Amen.
}\end{paracol}

\paragraph{Oração}
\begin{paracol}{2}\latim{
\rlettrine{B}{ene} \cruz dic, quǽsumus, Dómine, hos palmárum seu olivárum ramos: et præsta; ut, quod pópulus tuus in tui veneratiónem hodiérna die corporáliter agit, hoc spirituáliter summa devotióne perfíciat, de hoste victóriam reportándo et opus misericórdiæ summópere diligéndo. Per Dóminum \emph{\&c.}
}\switchcolumn\portugues{
\rlettrine{A}{bençoai} \cruz, Senhor, Vos imploramos, estes Ramos de palmeira e de oliveira, e concedei ao vosso povo a graça de realizar espiritualmente com ardente devoção a cerimónia exterior que hoje pratica em vossa honra; e que, triunfando do inimigo por meio dela, corresponda com amor à misericordiosa obra por Vós realizada para sua salvação. Por nosso Senhor \emph{\&c.}
}\end{paracol}

\paragraph{Oração}
\begin{paracol}{2}\latim{
\rlettrine{D}{eus,} qui Fílium tuum Jesum Christum, Dóminum nostrum, pro salute nostra in hunc mundum misísti, ut se humiliáret ad nos et nos revocáret ad te: cui etiam, dum Jerúsalem veniret, ut adimpléret Scripturas, credentium populorum turba, fidelissima devotione, vestimenta sua cum ramis palmarum in via sternébant: præsta, quǽsumus; ut illi fídei viam præparémus, de qua, remoto lápide offensiónis et petra scándali, fróndeant apud te ópera nostra justítiæ ramis: ut ejus vestigia sequi mereámur: Qui tecum \emph{\&c.}
}\switchcolumn\portugues{
\slettrine{Ó}{} Deus, que para nossa salvação enviastes a este mundo o vosso Filho, N. S. Jesus Cristo, a fim de que, humilhando-se Ele até nós, nos faça subir até Vós; e que quisestes, para se cumprirem as Escrituras, que, ao entrar Ele em Jerusalém, uma turba de povo fiel, cheia de sincera piedade, estendesse os seus vestidos e ramos de palmeira à sua passagem, concedei-nos a graça, Vos imploramos, de Lhe prepararmos pela fé um caminho onde não haja pedra, nem de tropeço, nem de escândalo, a fim de que das nossas acções brotem junto de Vós ramos de justiça, de sorte que mereçamos seguir os vestígios d’Aquele que, sendo Deus \emph{\&c.}
}\end{paracol}

\subsubsection{Distribuição dos Ramos}

\paragraphinfo{Antífona}{Jo. 12, 13}
\begin{paracol}{2}\latim{
\rlettrine{P}{ueri} Hebræórum, portántes ramos olivárum, obviavérunt Dómino, clamántes et dicéntes: Hosánna in excélsis.
}\switchcolumn\portugues{
\rlettrine{O}{s} meninos hebreus saíram com ramos de oliveira ao encontro do Senhor, clamando e dizendo: «Hosana no alto dos céus!».
}\end{paracol}

\paragraphinfo{Antífona}{Mt. 21, 8 \& 9}
\begin{paracol}{2}\latim{
\rlettrine{P}{ueri} Hebræórum vestiménta prosternébant in via et clamábant, dicéntes: Hosánna fílio David: benedíctus, qui venit in nómine Dómini.
}\switchcolumn\portugues{
\rlettrine{O}{s} meninos hebreus estendiam os seus vestidos pelos caminhos, clamando e dizendo: «Hosana ao Filho de David! Bendito seja o que vem em nome do Senhor!».
}\end{paracol}

\paragraph{Oração}
\begin{paracol}{2}\latim{
\rlettrine{O}{mnípotens} sempitérne Deus, qui Dóminum nostrum Jesum Christum super pullum ásinæ sedére fecísti, et turbas populórum vestiménta vel ramos arbórum in via stérnere et Hosánna decantáre in laudem ipsíus docuísti: da, quǽsumus; ut illórum innocéntiam imitári póssimus, et eórum méritum cónsequi mereámur. Per eúndem Christum, Dóminum nostrum. ℟. Amen.
}\switchcolumn\portugues{
\rlettrine{O}{mnipotente} e eterno Deus, que enviastes a turba do povo ao encontro de N. S. Jesus Cristo, montado em um jumentinho, e que quisestes que ela estendesse seus vestidos, lançasse ramos de árvores no caminho e cantasse hosanas em seu louvor, concedei-nos a graça, Vos suplicamos, de imitar a inocência dessa turba e de ter parte nos seus merecimentos. Pelo mesmo nosso Senhor Jesus Cristo. ℟. Amen.
}\end{paracol}

\subsubsection{Procissão dos Ramos}

\begin{paracol}{2}\latim{
℣. Procedámus in pace.
}\switchcolumn\portugues{
℣. Caminhemos em paz.
}\switchcolumn*\latim{
℟. In nómine Christi. Amen.
}\switchcolumn\portugues{
℟. Em nome de Cristo. Amen.
}\end{paracol}

\paragraphinfo{Antífona}{Mt. 21, 1-3, 7, 8 \& 9}
\begin{paracol}{2}\latim{
\rlettrine{C}{um} appropinquáret Dóminus Jerosólymam, misit duos ex discípulis suis, dicens: Ite in castéllum, quod contra vos est: et inveniétis pullum ásinæ alligátum, super quem nullus hóminum sedit: sólvite et addúcite mihi. Si quis vos interrogáverit, dícite: Opus Dómino est. Solvéntes adduxérunt ad Jesum: et imposuérunt illi vestiménta sua, et sedit super eum: alii expandébant vestiménta sua in via: alii ramos de arbóribus sternébant: et qui sequebántur, clamábant: Hosánna, benedíctus, qui venit in nómine Dómini: benedíctum regnum patris nostri David: Hosánna in excélsis: miserére nobis, fili David.
}\switchcolumn\portugues{
\qlettrine{Q}{uando} o Senhor se aproximava de Jerusalém, mandou dous discípulos, dizendo-lhes: «Ide à aldeia fronteira e lá encontrareis preso um jumentinho, em o qual ninguém montou ainda. Desprendei-o e trazei-mo. Se alguém vos disser alguma cousa, respondei: «O Senhor precisa dele». Havendo-o desprendido, trouxeram-no a Jesus. Então, puseram as suas capas em cima do jumentinho e fizeram Jesus montá-lo. E alguns da multidão estendiam os vestidos no caminho, outros espalhavam ramos de árvores e todos que O acompanhavam clamavam: «Bendito seja o que vem em nome do Senhor! Bendito seja o reino de David, nosso Pai! Hosana no alto dos céus! Tende piedade de nós, Filho de David!».
}\end{paracol}

\paragraphinfo{Antífona}{Jo. 12, 12 \& 13}
\begin{paracol}{2}\latim{
\rlettrine{C}{um} audísset pópulus, quia Jesus venit Jerosólymam, accepérunt ramos palmárum: et exiérunt ei óbviam, et clamábant púeri, dicéntes: Hic est, qui ventúrus est in salútem pópuli. Hic est salus nostra et redémptio Israël. Quantus est iste, cui Throni et Dominatiónes occúrrunt! Noli timére, fília Sion: ecce, Rex tuus venit tibi, sedens super pullum ásinæ, sicut scriptum est, Salve, Rex, fabricátor mundi, qui venísti redímere nos.
}\switchcolumn\portugues{
\rlettrine{H}{avendo} o povo sabido que Jesus vinha a Jerusalém, empunhou ramos de palmeiras e foi ao seu encontro. Os meninos clamavam, então: «Eis Aquele que vem salvar o seu povo! Este é a nossa salvação e a redenção de Israel! Como é grande Aquele diante de quem os Tronos e as Dominações se curvam para O receberem! Nada temas, ó filha de Sião, eis o teu Rei, que chega montado num jumentinho, como está escrito. Salve, ó Rei, criador do mundo, que viestes à terra para nos resgatar!».
}\end{paracol}

\paragraph{Antífona}
\begin{paracol}{2}\latim{
\rlettrine{A}{nte} sex dies sollémnis Paschæ, quando venit Dóminus in civitátem Jerúsalem, occurrérunt ei pueri: et in mánibus portábant ramos palmárum, et clamábant voce magna, dicéntes: Hosánna in excélsis: benedíctus, qui venísti in multitúdine misericórdiæ tuæ: Hosánna in excélsis.
}\switchcolumn\portugues{
\rlettrine{S}{eis} dias antes da solenidade pascal, quando o Senhor veio à cidade de Jerusalém, saíram-lhe ao encontro os meninos, que empunhavam ramos de palmeiras e clamavam com voz forte: «Hosana no alto dos céus! Sede bendito; pois vindes a nós com a grandeza da vossa misericórdia! Hosana no alto dos céus!».
}\end{paracol}

\paragraph{Antífona}
\begin{paracol}{2}\latim{
\rlettrine{O}{ccúrrunt} turbæ cum flóribus et palmis Redemptóri óbviam: et victóri triumphánti digna dant obséquia: Fílium Dei ore gentes prǽdicant: et in laudem Christi voces tonant per núbila: Hosánna in excélsis.
}\switchcolumn\portugues{
\rlettrine{A}{s} turbas do povo foram com flores e palmas ao encontro do Redentor, prestando-Lhe homenagem digna, como vencedor triunfante. Os povos anunciam hoje a grandeza do Filho de Deus. Reboam pelos ares as aclamações em honra de Cristo! Hosana no alto dos céus!
}\end{paracol}

\paragraph{Antífona}
\begin{paracol}{2}\latim{
\rlettrine{C}{um} Angelis et púeris fidéles inveniántur, triumphatóri mortis damántes: Hosánna in excélsis.
}\switchcolumn\portugues{
\rlettrine{F}{iéis,} unamo-nos aos Anjos e aos meninos e aclamemos o triunfador da morte, dizendo: «Hosana no alto dos céus!».
}\end{paracol}

\paragraph{Antífona}
\begin{paracol}{2}\latim{
\rlettrine{T}{urba} multa, quæ convénerat ad diem festum, clamábat Dómino: Benedíctus, qui venit in nómine Dómini: Hosánna in excélsis.
}\switchcolumn\portugues{
\rlettrine{U}{ma} grande turba de povo, que viera à festa, foi ao encontro do Senhor, clamando: Bendito o que vem em nome do Senhor! Hosana no alto dos céus!
}\end{paracol}

\begin{paracol}{2}\latim{
\rlettrine{G}{lória,} laus et honor tibi sit, Rex Christe, Redémptor: Cui pueríle decus prompsit Hosánna pium.
}\switchcolumn\portugues{
\rlettrine{G}{lória,} louvor e honra Vos sejam dados, ó Cristo, Rei e Redentor! Em honra de quem cantou devotadamente o escol das crianças: Hosana!
}\switchcolumn*\latim{
℟. Glória, laus \&
}\switchcolumn\portugues{
℟. Glória, louvor \&
}\switchcolumn*\latim{
Israël es tu Rex, Davidis et ínclita proles: Nómine qui in Dómini, Rex benedícte, venis.
}\switchcolumn\portugues{
Sois o Rei de Israel, da ínclita prole de David, ó Rei bendito, que vindes em nome do Senhor!
}\switchcolumn*\latim{
℟. Glória, laus \&
}\switchcolumn\portugues{
℟. Glória, louvor \&
}\switchcolumn*\latim{
Cœtus in excélsis te laudat cǽlicus omnis, Et mortális homo, et cuncta creáta simul.
}\switchcolumn\portugues{
A multidão angelical, no alto dos céus, o homem mortal e todas as criaturas cantam em uníssono os vossos louvores.
}\switchcolumn*\latim{
℟. Glória, laus \&
}\switchcolumn\portugues{
℟. Glória, louvor \&
}\switchcolumn*\latim{
Plebs Hebrǽa tibi cum palmis óbvia venit: Cum prece, voto, hymnis, ádsumus ecce tibi.
}\switchcolumn\portugues{
O povo hebreu saiu ao vosso encontro com palmas. E nós vimos diante de Vós com súplicas, votos e hinos.
}\switchcolumn*\latim{
℟. Glória, laus \&
}\switchcolumn\portugues{
℟. Glória, louvor \&
}\switchcolumn*\latim{
Hi tibi passúro solvébaní múnia laudis: Nos tibi regnánti pángimus ecce melos.
}\switchcolumn\portugues{
Quando o povo Vos prestou homenagem, Vós íeis sofrer. E nós Vos oferecemos estes cânticos, agora, que reinais no céu.
}\switchcolumn*\latim{
℟. Glória, laus \&
}\switchcolumn\portugues{
℟. Glória, louvor \&
}\switchcolumn*\latim{
Hi placuére tibi, pláceat devótio nostra: Rex bone, Rex clemens, cui bona cuncta placent.
}\switchcolumn\portugues{
Tais votos foram aceites. Que a nossa devoção o seja também, ó Rei de bondade, ó Rei de clemência, a quem agrada tudo quanto é bom.
}\switchcolumn*\latim{
℟. Glória, laus \&
}\switchcolumn\portugues{
℟. Glória, louvor \&
}\end{paracol}

\paragraph{Responsório}
\begin{paracol}{2}\latim{
℟. Ingrediénte Dómino in sanctam civitátem, Hebræórum púeri resurrectiónem vitæ pronuntiántes. Cum ramis palmárum: Hosánna, clamábant, in excélsis. ℣. Cum audísset pópulus, quod Jesus veníret Jerosólymam, exiérunt óbviam ei. Cum ramis palmárum: Hosánna, clamábant, in excélsis.
}\switchcolumn\portugues{
℟. Quando o Senhor entrava na cidade santa, os meninos hebreus anunciavam antecipadamente a ressurreição d’Aquele que é a vida.
* Empunhando ramos de palmeira, clamavam: «Hosana no alto dos céus!». ℣. E, tendo o povo notícia de que Jesus vinha a Jerusalém, saiu ao seu encontro. Empunhando ramos de palmeira, clamava: «Hosana no alto dos céus!».
}\end{paracol}

\subsection{Missa de Domingo de Ramos}

\paragraphinfo{Intróito}{Sl. 21, 20 \& 22}
\begin{paracol}{2}\latim{
\rlettrine{D}{ómine,} ne longe fácias auxílium tuum a me, ad defensiónem meam áspice: líbera me de ore leonis, et a córnibus unicórnium humilitátem meam. \emph{Ps. ibid., 2} Deus, Deus meus, réspice in me: quare me dereliquísti? longe a salúte mea verba delictórum meórum
}\switchcolumn\portugues{
\rlettrine{S}{enhor,} não afasteis de mim o vosso auxílio; apressai-Vos em defender-me. Livrai-me da boca do leão e das pontas dos unicórnios, pois sou fraco. \emph{Sl. ibid., 2} Meu Deus, meu Deus, lançai para mim vossos olhares. Porque me abandonastes? O clamor dos meus pecados afastou de mim a salvação.
}\end{paracol}

\paragraph{Oração}
\begin{paracol}{2}\latim{
\rlettrine{O}{mnípotens} sempitérne Deus, qui humáno generi, ad imitandum humilitátis exémplum, Salvatórem nostrum carnem súmere et crucem subíre fecísti: concéde propítius; ut et patiéntiæ ipsíus habére documénta et resurrectiónis consórtia mereámur. Per eúndem Dóminum nostrum \emph{\&c.}
}\switchcolumn\portugues{
\rlettrine{D}{eus} omnipotente e eterno, que, para dar ao género humano exemplo de humildade a imitar, quisestes que o Salvador assumisse a nossa carne e sofresse o suplício da Cruz, concedei-nos benigno a graça de seguirmos as lições da sua paciência para merecermos comparticipar da sua ressurreição. Pelo mesmo \emph{\&c.}
}\end{paracol}

\paragraphinfo{Epístola}{Fl. 2, 5-11}
\begin{paracol}{2}\latim{
Léctio Epístolæ beáti Pauli Apóstoli ad Philippénses.
}\switchcolumn\portugues{
Lição da Ep.ª do B. Ap.º Paulo aos Filipenses.
}\switchcolumn*\latim{
\rlettrine{F}{ratres:} Hoc enim sentíte in vobis, quod et in Christo Jesu: qui, cum in forma Dei esset, non rapínam arbitrátus est esse se æqualem Deo: sed semetípsum exinanívit, formam servi accípiens, in similitúdinem hóminum factus, et hábitu invéntus ut homo. Humiliávit semetípsum, factus obœdiens usque ad mortem, mortem autem crucis. Propter quod et Deus exaltávit illum: ei donávit illi nomen, quod est super omne nomen: \emph{(hic genuflectitur)} ut in nómine Jesu omne genu flectátur cœléstium, terréstrium et inférno rum: et omnis lingua confiteátur, quia Dóminus Jesus Christus in glória est Dei Patris.
}\switchcolumn\portugues{
\rlettrine{M}{eus} irmãos: Tende os mesmos sentimentos que animaram Jesus Cristo, o qual, embora fosse Deus por natureza (e não era usurpação o julgar-se igual a Deus), contudo humilhou-se a si próprio, reduzindo-se à condição de servo, tornando-se semelhante aos homens e reconhecido como homem pelas aparências. Humilhou-se a si próprio, obedecendo até à morte, e morte na cruz. Por isso Deus O exaltou e deu-Lhe um nome que é superior a todo o nome \emph{(devemos genuflectir)}, para que ao ser proferido o nome de Jesus todos os joelhos se dobrem nos céus, na terra e até nos infernos; e todas as línguas confessem que Nosso Senhor Jesus Cristo está na glória de Deus Pai!
}\end{paracol}

\paragraphinfo{Gradual}{Sl. 72, 24 et 1-3}
\begin{paracol}{2}\latim{
\rlettrine{T}{enuísti} manum déxteram meam: et in voluntáte tua deduxísti me: et cum glória assumpsísti me. ℣. Quam bonus Israël Deus rectis corde! mei autem pæne moti sunt pedes: pæne effúsi sunt gressus mei: quia zelávi in peccatóribus, pacem peccatórum videns.
}\switchcolumn\portugues{
\rlettrine{A}{poiastes-me} com vossa dextra; conduzistes-me segundo a vossa vontade; e elevastes-me com glória. ℣. Como o Deus de Israel é bom para os que possuem coração recto! Meus pés estiveram vacilantes; estive quase a cair, pois eu olhava com indignação para os ímpios, vendo a paz que gozavam os pecadores.
}\end{paracol}

\paragraphinfo{Trato}{Sl. 21, 2-9, 18, 19, 22, 24 \& 32}
\begin{paracol}{2}\latim{
\rlettrine{D}{eus,} Deus meus, réspice in me: quare me dereliquísti? ℣. Longe a salúte mea verba delictórum meórum. ℣. Deus meus, clamábo per diem, nec exáudies: in nocte, et non ad insipiéntiam mihi. ℣. Tu autem in sancto hábitas, laus Israël. ℣. In te speravérunt patres nostri: speravérunt, et liberásti eos. ℣. Ad te clamavérunt, et salvi facti sunt: in te speravérunt, et non sunt confusi. ℣. Ego autem sum vermis, et non homo: oppróbrium hóminum et abjéctio plebis. ℣. Omnes, qui vidébant me, aspernabántur me: locúti sunt lábiis et movérunt caput. ℣. Sperávit in Dómino, erípiat eum: salvum fáciat eum, quóniam vult eum. ℣. Ipsi vero consideravérunt et conspexérunt me: divisérunt sibi vestiménta mea, et super vestem meam misérunt mortem. ℣. Líbera me de ore leónis: et a córnibus unicórnium humilitátem meam. ℣. Qui timétis Dóminum, laudáte eum: univérsum semen Jacob, magnificáte eum. ℣. Annuntiábitur Dómino generátio ventúra: et annuntiábunt cœli justítiam ejus. ℣. Pópulo, qui nascétur, quem fecit Dóminus.
}\switchcolumn\portugues{
\rlettrine{M}{eu} Deus, meu Deus, olhai para mim: Porque me abandonastes? ℣. O clamor dos meus pecados afastou de mim a salvação. ℣. Meu rei durante o dia e não me ouvi Deus, clama reis; clamarei durante a noite e não acharei consolação. ℣. Contudo, sois a glória de Israel e habitais no santuário. ℣. Os nossos pais esperaram em Vós: esperaram e fostes o seu libertador! ℣. Clamaram por Vós, e foram salvos; confiaram em Vós e não foram iludidos. ℣. Porém sou um verme e não um homem; sou o opróbrio dos homens, a irrisão do povo! ℣. Todos quantos me vêem, enchem-me de injúrias, murmuram de mim, abanando a cabeça e dizendo: ℣. «Esperou no Senhor; pois que venha salvá-l’O, já que O ama». ℣. Olharam para mim e desprezaram-me; repartiram entre si os meus vestidos e lançaram sortes sobre a minha túnica. ℣. Livrai-me, Senhor, da boca do leão e das pontas dos unicórnios, pois sou fraco. ℣. Louvai o Senhor, ó Vós, que O temeis! Cantai louvores em sua honra, ó filhos de Jacob. ℣. À geração futura será anunciado o Senhor; os céus proclamarão a sua justiça. ℣. Ao povo, que há-de nascer, ensinarão que o Senhor O fez aparecer.
}\end{paracol}

\paragraphinfo{Evangelho}{Mt. 26, 1-75; 27, 1-66}
\begin{paracol}{2}\latim{
\cruz Passio Dómini nostri Jesu Christi secúndum Matthǽum.
}\switchcolumn\portugues{
\cruz Paixão de Nosso Senhor Jesus Cristo, segundo S. Mateus.
}\switchcolumn*\latim{
\blettrine{I}{n} illo témpore: Dixit Jesus discípulis suis: \cruz Scitis, quid post bíduum Pascha fiet, et Fílius hóminis tradétur, ut crucifigátur. {\redx C.} Tunc congregáti sunt príncipes sacerdótum et senióres pópuli in átrium príncipis sacerdótum, qui dicebátur Cáiphas: et consílium fecérunt, ut Jesum dolo tenérent et occíderent. Dicébant autem: {\redx S.} Non in die festo, ne forte tumúltus fíeret in pópulo. {\redx C.} Cum autem Jesus esset in Bethánia in domo Simónis leprósi, accéssit ad eum múlier habens alabástrum unguénti pretiósi, et effúdit super caput ipsíus recumbéntis. Vidéntes autem discípuli, indignáti sunt, dicéntes: {\redx S.} Ut quid perdítio hæc? pótuit enim istud venúmdari multo, et dari paupéribus. {\redx C.} Sciens autem Jesus, ait illis: \cruz Quid molésti estis huic mulíeri? opus enim bonum operáta est in me. Nam semper páuperes habétis vobíscum: me autem non semper habétis. Mittens enim hæc unguéntum hoc in corpus meum, ad sepeliéndum me fecit. Amen, dico vobis, ubicúmque prædicátum fúerit hoc Evangélium in toto mundo, dicétur et, quod hæc fecit, in memóriam ejus. {\redx C.} Tunc ábiit unus de duódecim, qui dicebátur Judas Iscariótes, ad príncipes sacerdótum, et ait illis: {\redx S.} Quid vultis mihi dare, et ego vobis eum tradam? {\redx C.} At illi constituérunt ei trigínta argénteos. Et exínde quærébat opportunitátem, ut eum tráderet.
}\switchcolumn\portugues{
\blettrine{N}{aquele} tempo, disse Jesus aos discípulos: \cruz «Sabeis que, passados dous dias, se celebrará a Páscoa e que o Filho do homem será entregue, para O crucificarem». {\redx C.} Então, reuniram-se os príncipes dos sacerdotes e os anciãos na sala do sumo Pontífice, que era chamado Caifás, e deliberaram prender Jesus, insidiosamente, e matarem-n’O. Mas diziam: {\redx S.} «Que isso, porém, não seja no dia da festa, para que o povo não faça tumulto». {\redx C.} Então, estando Jesus em Betânia, em casa de Simão, o leproso, aproximou-se d’Ele uma mulher, trazendo um vaso de alabastro, cheio de perfumes preciosos, derramando-os sobre a cabeça de Jesus, que estava assentado à mesa. Vendo isto, indignaram-se os discípulos, dizendo: {\redx S.} «Para que serve tal desperdício? Pois poderia ter-se vendido por elevada quantia este perfume e dar aos pobres o seu preço». {\redx C.} Conhecendo Jesus isto, disse-lhes: \cruz «Para que causais pena a esta mulher? Foi uma boa obra para comigo, que ela praticou; pois pobres sempre os tereis convosco; porém a mim nem sempre me tereis. Espalhando este perfume sobre o meu corpo, ungiu-me para Eu ser sepultado. Em verdade vos digo: onde quer que seja pregado este Evangelho (no mundo inteiro) contar-se-á também em sua memória a acção que praticou». {\redx C.} Então um dos Doze, chamado Judas Iscariotes, foi ter com o príncipe dos sacerdotes e disse-lhe: {\redx S.} «Quanto quereis dar-me para que eu vo-l’O entregue?». {\redx C.} E combinaram dar-lhe trinta moedas de prata. Desde logo, procurou ele oportunidade para O entregar.
}\switchcolumn*\latim{
Prima autem die azymórum accessérunt discípuli ad Jesum, dicéntes: {\redx S.} Ubi vis parémus tibi comédere pascha? {\redx C.} At Jesus dixit: \cruz Ite in civitátem ad quendam, et dícite ei: Magíster dicit: Tempus meum prope est, apud te fácio pascha cum discípulis meis. {\redx C.} Et fecérunt discípuli, sicut constítuit illis Jesus, et paravérunt pascha. Véspere autem facto, discumbébat cum duódecim discípulis suis. Et edéntibus illis, dixit: \cruz Amen, dico vobis, quia unus vestrum me traditúrus est. {\redx C.} Et contristáti valde, cœpérunt sínguli dícere: {\redx S.} Numquid ego sum, Dómine? {\redx C.} At ipse respóndens, ait: \cruz Qui intíngit mecum manum in parópside, hic me tradet. Fílius quidem hóminis vadit, sicut scriptum est de illo: væ autem hómini illi, per quem Fílius hóminis tradétur: bonum erat ei, si natus non fuísset homo ille. {\redx C.} Respóndens autem Judas, qui trádidit eum, dixit: {\redx S.} Numquid ego sum, Rabbi? {\redx C.} Ait illi: \cruz Tu dixísti. {\redx C.} Cenántibus autem eis, accépit Jesus panem, et benedíxit, ac fregit, dedítque discípulis suis, et ait: \cruz Accípite et comédite: hoc est corpus meum. {\redx C.} Et accípiens cálicem, grátias egit: et dedit illis, dicens: \cruz Bíbite ex hoc omnes. Hic est enim sanguis meus novi Testaménti, qui pro multis effundétur in remissiónem peccatórum. Dico autem vobis: non bibam ámodo de hoc genímine vitis usque in diem illum, cum illud bibam vobíscum novum in regno Patris mei. {\redx C.} Et hymno dicto, exiérunt in montem Olivéti. Tunc dicit illis Jesus: \cruz Omnes vos scándalum patiémini in me in ista nocte. Scriptum est enim: Percútiam pastórem, et dispergéntur oves gregis. Postquam autem resurréxero, præcédam vos in Galilǽam. {\redx C.} Respóndens autem Petrus, ait illi: {\redx S.} Et si omnes scandalizáti fúerint in te, ego numquam scandalizábor. {\redx C.} Ait illi Jesus: \cruz Amen, dico tibi, quia in hac nocte, antequam gallus cantet, ter me negábis. {\redx C.} Ait illi Petrus: {\redx S.} Etiam si oportúerit me mori tecum, non te negábo. {\redx C.} Simíliter et omnes discípuli dixérunt.
}\switchcolumn\portugues{
No primeiro dia dos ázimos, vieram os discípulos ter com Jesus, dizendo-Lhe: {\redx S.} «Onde quereis que preparemos o que é necessário para comer a Páscoa?» {\redx C.} Jesus disse-lhes: \cruz «Ide à cidade, a casa dum tal, e dizei-lhe: «O Mestre diz: «Meu tempo está próximo; quero celebrar a Páscoa com meus discípulos em tua casa». {\redx C.} Os discípulos fizeram o que Jesus lhes ordenara e prepararam a Páscoa. Chegada, pois, a tarde, achava-se Jesus à mesa com seus Doze Discípulos. E, estando eles a comer, disse-lhes: \cruz «Em verdade vos digo que um de vós me trairá». {\redx C.} Então, cheios de profunda tristeza, começaram, individualmente, a dizer: {\redx S.} «Serei eu, Senhor?» {\redx C.} Ele disse: \cruz «O que me há-de trair é aquele que mete comigo a mão no prato! Com efeito, o Filho do homem vai ser traído, segundo o que está escrito a seu respeito, mas infeliz daquele que O tiver traído! Melhor lhe fora não haver nascido!». {\redx C.} Ora Judas, o discípulo que O traiu, disse: {\redx S.} «Serei eu, porventura, Senhor?» {\redx C.} Jesus disse-lhe: \cruz «Tu o disseste!». {\redx C.} Enquanto ceavam, tomou Jesus o pão, benzeu-o, partiu-o e deu-o aos discípulos, dizendo: \cruz «Tomai e comei: Isto é o meu corpo». {\redx C.} E, pegando no cálice, deu graças e entregou-lho, dizendo-lhes: \cruz «Bebei dele vós todos. Pois este é o meu sangue do Novo Testamento, que será derramado por muitos para remissão dos pecados. Digo-vos, porém, que não mais tornarei a beber deste fruto da videira, até ao dia em que o hei-de beber, novamente, convosco no reino de meu Pai». {\redx C.} Havendo dito o hino de graças, saíram para o monte das Oliveiras, Então, disse-lhes Jesus: \cruz «Esta noite serei para vós todos motivo de escândalo; porque está escrito: «Ferirei o pastor e as ovelhas do rebanho serão dispersas». Mas, depois de ressuscitar, irei adiante de vós para a Galileia» {\redx C.} Porém, respondendo Pedro, disse-Lhe: {\redx S.} «Ainda que todos se escandalizem, eu nunca me escandalizarei». {\redx C.} E Jesus retorquiu-lhe: \cruz «Em verdade te digo: esta noite, antes de o galo cantar, negar-me-ás três vezes». {\redx C.}, pedro disse-Lhe: {\redx S.} «Ainda que tenha de morrer convosco, não Vos negarei!». {\redx C.} O mesmo afirmaram todos os discípulos.
}\switchcolumn*\latim{
Tunc venit Jesus cum illis in villam, quæ dícitur Gethsémani, et dixit discípulis suis: \cruz Sedéte hic, donec vadam illuc et orem. {\redx C.} Et assúmpto Petro et duóbus fíliis Zebedǽi, cœpit contristári et mæstus esse. Tunc ait illis: \cruz Tristis est ánima mea usque ad mortem: sustinéte hic, et vigilate mecum. {\redx C.} Et progréssus pusíllum, prócidit in fáciem suam, orans et dicens: \cruz Pater mi, si possíbile est, tránseat a me calix iste: Verúmtamen non sicut ego volo, sed sicut tu. {\redx C.} Et venit ad discípulos suos, et invénit eos dormiéntes: et dicit Petro: \cruz Sic non potuístis una hora vigiláre mecum? Vigiláte et oráte, ut non intrétis in tentatiónem. Spíritus quidem promptus est, caro autem infírma. {\redx C.} Iterum secúndo ábiit et orávit, dicens: \cruz Pater mi, si non potest hic calix transíre, nisi bibam illum, fiat volúntas tua. {\redx C.} Et venit íterum, et invenit eos dormiéntes: erant enim óculi eórum graváti. Et relíctis illis, íterum ábiit et orávit tértio, eúndem sermónem dicens. Tunc venit ad discípulos suos, et dicit illis: \cruz Dormíte jam et requiéscite: ecce, appropinquávit hora, et Fílius hóminis tradétur in manus peccatórum. Súrgite, eámus: ecce, appropinquávit, qui me tradet.
}\switchcolumn\portugues{
Então, foi Jesus com eles para um sítio chamado Getsémani, e disse aos discípulos: \cruz «Assentai-vos aqui, enquanto vou, ali, orar». {\redx C.} E, levando consigo Pedro e os filhos de Zebedeu, começou a entristecer-se e a angustiar-se. E disse-lhes: \cruz «Minha alma está triste até à morte! Ficai aqui e vigiai comigo». {\redx C.} Depois, avançou um pouco e prostrou-se com o rosto no chão, orando e dizendo: \cruz «Meu Pai, se é possível, fazei que este cálice se afaste de mim; contudo faça-se a vossa vontade e não a minha». {\redx C.} Em seguida, veio ter com os discípulos, encontrando-os a dormir. Disse, então, a Pedro: \cruz «Pois não pudestes vigiar uma hora comigo?! Vigiai e orai para não entrardes em tentação. Na verdade, o espírito está pronto, porém a carne é fraca». {\redx C.} De novo se retirou Jesus e, pela segunda vez orou, dizendo: \cruz «Meu Pai, se este cálice não pode passar sem que Eu o beba, faça-se a vossa vontade». {\redx C.} Depois veio outra vez ter com os discípulos, que achou dormindo (até tinham os olhos colados de sono!), e, deixando-os, foi pela terceira vez orar, repetindo as mesmas palavras. Depois, veio ter com os discípulos e disse-lhes: \cruz «Dormi agora e repousai: eis que se aproxima a hora em que o Filho do homem será entregue às mãos dos pecadores. Erguei-vos; vamos! Eis que está próximo o que me trairá».
}\switchcolumn*\latim{
{\redx C.} Adhuc eo loquénte, ecce, Judas, unus de duódecim, venit, et cum eo turba multa cum gládiis et fústibus, missi a princípibus sacerdótum et senióribus pópuli. Qui autem trádidit eum, dedit illis signum, dicens: {\redx S.} Quemcúmque osculátus fúero, ipse est, tenéte eum. {\redx C.} Et conféstim accédens ad Jesum, dixit: {\redx S.} Ave, Rabbi. {\redx C.} Et osculátus est eum. Dixítque illi Jesus: \cruz Amíce, ad quid venísti?
}\switchcolumn\portugues{
{\redx C.} Ainda Jesus falava, quando Judas, um dos Doze, chegou e com ele numerosa turba, armada com espadas e paus, que fora enviada pelos príncipes dos sacerdotes e anciãos do povo. Ora, aquele que O traíra, havia dado este sinal à turba: {\redx S.} «Aquele que eu beijar, é Esse; prendei-O». {\redx C.} Logo que Judas chegou, aproximou-se do Mestre e disse-Lhe: {\redx S.} «Salve, ó Mestre». {\redx C.} E osculou-O. Jesus disse-lhe: \cruz «Amigo, a que vieste?».
}\switchcolumn*\latim{
{\redx C.} Tunc accessérunt, et manus injecérunt in Jesum et tenuérunt eum. Et ecce, unus ex his, qui erant cum Jesu, exténdens manum, exémit gládium suum, et percútiens servum príncipis sacerdótum, amputávit aurículam ejus. Tunc ait illi Jesus: \cruz Convérte gládium tuum in locum suum. Omnes enim, qui accéperint gládium, gládio períbunt. An putas, quia non possum rogáre Patrem meum, et exhibébit mihi modo plus quam duódecim legiónes Angelórum? Quómodo ergo implebúntur Scripturae, quia sic oportet fíeri? {\redx C.} In illa hora dixit Jesus turbis: \cruz Tamquam ad latrónem exístis cum gládiis et fústibus comprehéndere me: cotídie apud vos sedébam docens in templo, et non me tenuístis. {\redx C.} Hoc autem totum factum est, ut adimpleréntur Scripturae Prophetárum. Tunc discípuli omnes, relícto eo, fugérunt.
}\switchcolumn\portugues{
{\redx C.} Chegaram-se, então, a Ele os outros, lançaram-Lhe as mãos e prenderam-n’O. Mas, eis que um dos que estavam com Jesus, lançando mão da espada, desembainhou-a e acutilou um servo do príncipe dos sacerdotes, cortando-lhe uma orelha. Entretanto Jesus disse-lhe: \cruz «Mete a espada no seu lugar; pois todos quantos se servem da espada morrerão pela espada. Acaso pensas que não posso rogar auxílio a meu Pai, que logo me enviaria mais de doze legiões de Anjos? Mas, como se cumpririam as Escrituras que anunciam que assim deveria suceder?». {\redx C.} Ao mesmo tempo, Jesus disse às turbas: \cruz «Viestes com espadas e paus para me prender, como se Eu fora um ladrão?! Todos os dias estava assentado convosco, ensinando no templo, e não me prendestes?! {\redx C.} Tudo isto aconteceu assim, para que se cumprissem as Escrituras dos Profetas. E, naquela hora, todos os discípulos, havendo-O abandonado, fugiram.
}\switchcolumn*\latim{
At illi tenéntes Jesum, duxérunt ad Cáipham, príncipem sacerdótum, ubi scribæ et senióres convénerant. Petrus autem sequebátur eum a longe, usque in átrium príncipis sacerdótum. Et ingréssus intro, sedébat cum minístris, ut vidéret finem. Príncipes autem sacerdótum et omne concílium quærébant falsum testimónium contra Jesum, ut eum morti tráderent: et non invenérunt, cum multi falsi testes accessíssent. Novíssime autem venérunt duo falsi testes et dixérunt: {\redx S.} Hic dixit: Possum destrúere templum Dei, et post tríduum reædificáre illud. {\redx C.} Et surgens princeps sacerdótum, ait illi: {\redx S.} Nihil respóndes ad ea, quæ isti advérsum te testificántur? {\redx C.} Jesus autem tacébat. Et princeps sacerdótum ait illi: {\redx S.} Adjúro te per Deum vivum, ut dicas nobis, si tu es Christus, Fílius Dei. {\redx C.} Dicit illi Jesus: \cruz Tu dixísti. Verúmtamen dico vobis, ámodo vidébitis Fílium hóminis sedéntem a dextris virtútis Dei, et veniéntem in núbibus cœli. {\redx C.} Tunc princeps sacerdótum scidit vestiménta sua, dicens: {\redx S.} Blasphemávit: quid adhuc egémus téstibus? Ecce, nunc audístis blasphémiam: quid vobis vidétur? {\redx C.} At illi respondéntes dixérunt: {\redx S.} Reus est mortis. {\redx C.} Tunc exspuérunt in fáciem ejus, et cólaphis eum cecidérunt, álii autem palmas in fáciem ejus dedérunt, dicéntes: {\redx S.} Prophetíza nobis, Christe, quis est, qui te percússit?
}\switchcolumn\portugues{
Tendo Jesus sido preso, foi conduzido a casa de Caifás, príncipe dos sacerdotes, onde estavam reunidos os escribas e os anciãos. Pedro foi seguindo Jesus ao longe, até ao pátio dos príncipes dos sacerdotes, havendo entrado e tomado lugar, junto com os criados, para ver o resultado. Entretanto, os príncipes dos sacerdotes e todos os do conselho buscavam algum falso testemunho contra Jesus para O condenarem à morte; mas o não achavam, ainda que se tivessem apresentado muitas testemunhas falsas. Por fim, vieram duas testemunhas falsas, que declararam: {\redx S.} «Ele disse: «Posso destruir o templo de Deus e reedificá-lo em três dias». {\redx C.} Logo se levantou o príncipe dos sacerdotes e disse: {\redx S.} «Nada respondeis ao que estes dizem contra Vós?». {\redx C.} Jesus, porém, nada dizia; pelo que o príncipe dos sacerdotes O instou: {\redx S.} «Conjuro-Vos, por Deus vivo, que nos digais se sois Cristo, Filho de Deus!». {\redx C.} Jesus respondeu: \cruz «Tu o disseste; contudo digo-vos que haveis de ver daqui a pouco o Filho do homem assentar-se à direita do poder de Deus, caminhando sobre as nuvens do céu». {\redx C.} Então o príncipe dos sacerdotes rasgou os seus vestidos, dizendo: {\redx S.} «Blasfemou! Para que são precisas ainda testemunhas? Eis que acabais de ouvir uma blasfémia! Que vos parece?» {\redx C.} Eles responderam: {\redx S.} «É réu de morte». {\redx C.} Então, cuspiram-Lhe no rosto e deram-Lhe bofetadas, dizendo: {\redx S.} «Adivinhai, ó Cristo, quem Vos bateu?».
}\switchcolumn*\latim{
{\redx C.} Petrus vero sedébat foris in átrio: et accéssit ad eum una ancílla, dicens: {\redx S.} Et tu cum Jesu Galilǽo eras. {\redx C.} At ille negávit coram ómnibus, dicens: {\redx S.} Néscio, quid dicis. {\redx C.} Exeúnte autem illo jánuam, vidit eum ália ancílla, et ait his, qui erant ibi: {\redx S.} Et hic erat cum Jesu Nazaréno. {\redx C.} Et íterum negávit cum juraménto: Quia non novi hóminem. Et post pusíllum accessérunt, qui stabant, et dixérunt Petro: {\redx S.} Vere et tu ex illis es: nam et loquéla tua maniféstum te facit. {\redx C.} Tunc cœpit detestári et juráre, quia non novísset hóminem. Et contínuo gallus cantávit. Et recordátus est Petrus verbi Jesu, quod díxerat: Priúsquam gallus cantet, ter me negábis. Et egréssus foras, flevit amáre.
}\switchcolumn\portugues{
{\redx C.} Durante este tempo, continuava Pedro no pátio. Aproximou-se dele uma criada e disse-lhe: {\redx S.} «Tu também estavas com Jesus, o Galileu». {\redx C.} Pedro negou logo, diante de todos, dizendo: {\redx S.} «Não sei o que dizes». {\redx C.} Saindo, então, ele a porta, viu-o outra criada, que logo disse para os que estavam ali: {\redx S.}, «Este também estava com Jesus Nazareno». {\redx C.}, Pedro negou segunda vez com juramento, afirmando: «Não conheço tal homem». Pouco depois chegaram os que ali estavam e disseram a Pedro: {\redx S.} «Verdadeiramente tu também és deles, pois, o teu modo de falar, manifestamente, o dá a conhecer». {\redx C.} Então, começou a proferir imprecações e a jurar que não conhecia tal homem. Subitamente, cantou o galo. E logo Pedro se recordou de que Jesus lhe dissera: «Antes de o galo cantar, negar-me-ás três vezes». Saiu, pois, para fora e chorou amargamente!...
}\switchcolumn*\latim{
Mane autem facto, consílium iniérunt omnes príncipes sacerdótum et senióres pópuli advérsus Jesum, ut eum morti tráderent. Et vinctum adduxérunt eum, et tradidérunt Póntio Piláto prǽsidi. Tunc videns Judas, qui eum trádidit, quod damnátus esset, pæniténtia ductus, réttulit trigínta argénteos princípibus sacerdótum et senióribus, dicens: {\redx S.} Peccávi, tradens sánguinem justum. {\redx C.} At illi dixérunt: {\redx S.} Quid ad nos? Tu vidéris. {\redx C.} Et projéctis argénteis in templo, recéssit: et ábiens, láqueo se suspéndit. Príncipes autem sacerdótum, accéptis argénteis, dixérunt: {\redx S.} Non licet eos míttere in córbonam: quia prétium sánguinis est. {\redx C.} Consílio autem ínito, emérunt ex illis agrum fíguli, in sepultúram peregrinórum. Propter hoc vocátus est ager ille Hacéldama, hoc est, ager sánguinis, usque in hodiérnum diem. Tunc implétum est, quod dictum est per Jeremíam Prophétam, dicéntem: Et accepérunt trigínta argénteos prétium appretiáti, quem appretiavérunt a fíliis Israël: et dedérunt eos in agrum fíguli, sicut constítuit mihi Dóminus.
}\switchcolumn\portugues{
Havendo rompido a manhã, todos os príncipes dos sacerdotes e os anciãos se reuniram em conselho contra Jesus, para O condenarem à morte. E, levando-O, conduziram-n’O e entregaram-n’O ao Governador Pôncio Pilatos. Então Judas, tendo atraiçoado Jesus e vendo que este havia sido condenado, foi logo, cheio de arrependimento, levar as trinta moedas de prata aos príncipes dos sacerdotes e aos anciãos, dizendo: {\redx S.} «Pequei, entregando-vos o sangue inocente!». {\redx C.} Mas eles disseram: {\redx S.} «Que nos importa isso? Tu poderias pensar no que fazias!». {\redx C.} Ele, então, arrojou as moedas para o templo, afastou-se e foi enforcar-se! Os príncipes dos sacerdotes recolheram o dinheiro e disseram: {\redx S.} «Não é lícito deitá-lo no cofre sagrado, pois é o preço do sangue». {\redx C.} E, havendo reunido o conselho a respeito disto, compraram com esse dinheiro o campo dum oleiro, para servir de cemitério dos peregrinos; por isso aquele campo é ainda hoje chamado «Hacéldama», isto é, campo do sangue. Com isto se cumpriu o que fora anunciado pelo Profeta Jeremias, quando dissera: «Recolheram as trinta moedas de prata, preço d’Aquele que foi posto a preço por alguns filhos de Israel, comprando com elas o campo dum oleiro, como o Senhor me ordenou».
}\switchcolumn*\latim{
Jesus autem stetit ante prǽsidem, et interrogávit eum præses, dicens: {\redx S.} Tu es Rex Judæórum? {\redx C.} Dicit illi Jesus: \cruz Tu dicis. {\redx C.} Et cum accusarétur a princípibus sacerdótum et senióribus, nihil respóndit. Tunc dicit illi Pilátus: {\redx S.} Non audis, quanta advérsum te dicunt testimónia? {\redx C.} Et non respóndit ei ad ullum verbum, ita ut mirarétur præses veheménter. Per diem autem sollémnem consuéverat præses pópulo dimíttere unum vinctum, quem voluíssent. Habébat autem tunc vinctum insígnem, qui dicebátur Barábbas. Congregátis ergo illis, dixit Pilátus: {\redx S.} Quem vultis dimíttam vobis: Barábbam, an Jesum, qui dícitur Christus? {\redx C.} Sciébat enim, quod per invídiam tradidíssent eum. Sedénte autem illo pro tribunáli, misit ad eum uxor ejus, dicens: {\redx S.} Nihil tibi et justo illi: multa enim passa sum hódie per visum propter eum. {\redx C.} Príncipes autem sacerdótum et senióres persuasérunt populis, ut péterent Barábbam, Jesum vero pérderent. Respóndens autem præses, ait illis: {\redx S.} Quem vultis vobis de duóbus dimítti? {\redx C.} At illi dixérunt: {\redx S.} Barábbam. {\redx C.} Dicit illis Pilátus: {\redx S.} Quid ígitur fáciam de Jesu, qui dícitur Christus? {\redx C.} Dicunt omnes: {\redx S.} Crucifigátur. {\redx C.} Ait illis præses: {\redx S.} Quid enim mali íecit? {\redx C.} At illi magis clamábant, dicéntes: {\redx S.} Crucifigátur. {\redx C.} Videns autem Pilátus, quia nihil profíceret, sed magis tumúltus fíeret: accépta aqua, lavit manus coram pópulo, dicens: {\redx S.} Innocens ego sum a sánguine justi hujus: vos vidéritis. {\redx C.} Et respóndens univérsus pópulus, dixit: {\redx S.} Sanguis ejus super nos et super fílios nostros. {\redx C.} Tunc dimísit illis Barábbam: Jesum autem flagellátum trádidit eis, ut crucifigerétur. Tunc mílites prǽsidis suscipiéntes Jesum in prætórium, congregavérunt ad eum univérsam cohórtem: et exuéntes eum, chlámydem coccíneam circumdedérunt ei: et plecténtes corónam de spinis, posuérunt super caput ejus, et arúndinem in déxtera ejus. Et genu flexo ante eum, illudébant ei, dicéntes: {\redx S.} Ave, Rex Judæórum. {\redx C.} Et exspuéntes in eum, accepérunt arúndinem, et percutiébant caput ejus. Et postquam illusérunt ei, exuérunt eum chlámyde et induérunt eum vestiméntis ejus, et duxérunt eum, ut crucifígerent.
}\switchcolumn\portugues{
Ora Jesus compareceu perante o Governador, que O interrogou: {\redx S.} «Sois o Rei dos Judeus?». {\redx C.} Respondeu-lhe Jesus: «Tu o dizes». {\redx C.} E, sendo acusado pelos príncipes dos sacerdotes e pelos anciãos, nada respondeu. Disse-Lhe, então, Pilatos: {\redx S.} «Não ouvis as cousas de que Vos acusam?». {\redx C.} Mas Ele não respondeu, de modo que o Governador admirava-se deveras. No dia da festa, o Governador tinha o costume de soltar o preso que o povo quisesse. Havia, então, um preso notável, chamado Barrabás. Estando todos juntos, disse Pilatos: {\redx S.} «Qual quereis que solte? Barrabás ou Jesus, por apelido Cristo?». {\redx C.} Pois sabia que por inveja é que lh’O haviam entregado. Quando Pilatos estava assentado no seu tribunal, mandou-lhe dizer sua mulher: {\redx S.} «Não te ocupes desse justo, pois tive, hoje, sonhos, nos quais padeci muito por sua causa». {\redx C.}, Mas os príncipes dos sacerdotes e os anciãos persuadiram o povo que pedisse que Barrabás fosse solto e mandasse matar Jesus. Falando, pois, o Governador, disse-lhes: {\redx S.} «Qual dos dous quereis que solte?». {\redx C.} Responderam: {\redx S.} «Barrabás». {\redx C.} Pilatos observou: {\redx S.}, «Que hei-de fazer, então, de Jesus, que se chama Cristo?» {\redx C.} Responderam todos: {\redx S.}, «Seja crucificado!». {\redx C.} O Governador disse-lhes: {\redx S.} «Pois que mal fez Ele?». {\redx C.} Porém, cada vez mais alto, bradavam: {\redx S.} «Seja crucificado!». {\redx C.} Vendo Pilatos que nada conseguia, mas que o tumulto crescia, mandou vir água e lavou as mãos diante do povo, dizendo: {\redx S.} «Estou inocente do sangue deste justo; isso é lá convosco». {\redx C.} Todo o povo respondeu: {\redx S.} «Que o sangue caia sobre nós e nossos filhos!». {\redx C.} Então Pilatos soltou Barrabás, e, havendo mandado açoitar Jesus, entregou-lh’O para ser crucificado. Os soldados do governador conduziram Jesus ao Pretório, formaram em torno d’Ele toda a corte, despojaram-n’O dos vestidos e cobriram-n’O com um manto de púrpura. Depois, teceram uma coroa de espinhos, puseram-Lha na cabeça, meteram-Lhe na mão direita uma cana, como se fora um ceptro, e ajoelharam diante d’Ele, escarnecendo-O e dizendo: {\redx S.} «Salve, ó Rei dos judeus!». {\redx C.} E, cuspindo-Lhe nas faces, tiraram-Lhe a cana e bateram-Lhe com ela na cabeça. Depois, ainda O escarneceram, tiraram-Lhe o manto, vestiram-n’O, novamente, com seus vestidos e levaram-n’O para ser crucificado.
}\switchcolumn*\latim{
Exeúntes autem, invenérunt hóminem Cyrenǽum, nómine Simónem: hunc angariavérunt, ut tólleret crucem ejus. Et venérunt in locum, qui dícitur Gólgotha, quod est Calváriæ locus. Et dedérunt ei vinum bíbere cum felle mixtum. Et cum gustásset, nóluit bibere. Postquam autem crucifixérunt eum, divisérunt vestiménta ejus, sortem mitténtes: ut implerétur, quod dictum est per Prophétam dicentem: Divisérunt sibi vestiménta mea, et super vestem meam misérunt sortem. Et sedéntes, servábant eum. Et imposuérunt super caput ejus causam ipsíus scriptam: Hic est Jesus, Rex Judæórum. Tunc crucifíxi sunt cum eo duo latrónes: unus a dextris et unus a sinístris. Prætereúntes autem blasphemábant eum, movéntes cápita sua et dicéntes: {\redx S.} Vah, qui déstruis templum Dei et in tríduo illud reædíficas: salva temetípsum. Si Fílius Dei es, descénde de cruce. {\redx C.} Simíliter et príncipes sacerdótum illudéntes cum scribis et senióribus, dicébant: {\redx S.} Alios salvos fecit, seípsum non potest salvum fácere: si Rex Israël est, descéndat nunc de cruce, et crédimus ei: confídit in Deo: líberet nunc, si vult eum: dixit enim: Quia Fílius Dei sum. {\redx C.} Idípsum autem et latrónes, qui crucifíxi erant cum eo, improperábant ei. A sexta autem hora ténebræ factæ sunt super univérsam terram usque ad horam nonam. Et circa horam nonam clamávit Jesus voce magna, dicens: \cruz Eli, Eli, lamma sabactháni? {\redx C.} Hoc est: \cruz Deus meus, Deus meus, ut quid dereliquísti me? {\redx C.} Quidam autem illic stantes et audiéntes dicébant: {\redx S.} Elíam vocat iste. {\redx C.} Et contínuo currens unus ex eis, accéptam spóngiam implévit acéto et impósuit arúndini, et dabat ei bíbere. Céteri vero dicébant: {\redx S.} Sine, videámus, an véniat Elías líberans eum. {\redx C.} Jesus autem íterum clamans voce magna, emísit spíritum.
}\switchcolumn\portugues{
Ao sair da cidade encontraram um homem de Cirene, chamado Simão. Logo o obrigaram a levar a cruz de Jesus. E vieram para um lugar chamado Gólgota, que quer dizer: lugar do Calvário. Deram-Lhe vinho misturado com fel. Porém Ele, havendo-o provado, o não quis beber. Depois de O crucificarem, lançaram sortes sobre os seus vestidos (para se cumprir o que fora anunciado pelo Profeta): «Repartiram entre si os meus vestidos e lançaram sortes à minha túnica». Depois assentaram-se e assim O guardaram. Puseram, também, por cima da sua cabeça uma inscrição, indicando a causa da sua morte, assim escrita: «Este é Jesus, Rei dos Judeus». Simultaneamente, foram crucificados dous ladrões: um à direita e o outro à esquerda. E os que passavam por ali blasfemavam, movendo a cabeça e dizendo: {\redx S.} «Ah! dissestes que destruiríeis o templo de Deus e o reedificaríeis em três dias? Salvai-Vos, pois, agora! Se sois o Filho de Deus, descei da cruz». {\redx C.} Ao mesmo tempo os sacerdotes com os escribas e anciãos, escarneciam d’Ele, dizendo: {\redx S.} «Salvou os outros e não pode salvar-se a si próprio? Se Ele é o Rei de Israel, que desça da cruz, e acreditaremos n’Ele. Confiou em Deus?! Pois, se Deus O ama, que O livre, porquanto Ele disse: «Sou o Filho de Deus». {\redx C.} Os ladrões, que estavam crucificados com Ele, insultavam-n’O do mesmo modo. Desde a hora sexta até à nona, as trevas estenderam-se por toda a terra. Cerca da hora nona, exclamou Jesus em voz alta, dizendo: \cruz «Elí, Elí, lamma sabatáni?». {\redx C.} Isto é: \cruz «Meu Deus, meu Deus, porque me abandonastes?». {\redx C.} Alguns, porém, dos que ali estavam, ouvindo isto, diziam: {\redx S.} «Chama por Elias». {\redx C.} Logo, correndo um deles, foi buscar uma esponja, ensopou-a em vinagre, pô-la sobre uma cana e apresentou-Lha para beber. Os outros diziam: {\redx S.} «Deixa; vejamos se Elias vem livrá-l’O». {\redx C.} Porém Jesus, soltando de novo um grande brado, expirou!
}\switchcolumn*\latim{
\emph{(Hic genuflectitur, et pausatur aliquántulum)}
}\switchcolumn\portugues{
\emph{Todos devem ajoelhar e recolher-se em meditação durante algum tempo.}
}\switchcolumn*\latim{
Et ecce, velum templi scissum est in duas partes a summo usque deórsum: et terra mota est, et petræ scissæ sunt, et monuménta apérta sunt: et multa córpora sanctórum, qui dormíerant, surrexérunt. Et exeúntes de monuméntis post resurrectiónem ejus, venérunt in sanctam civitátem, et apparuérunt multis. Centúrio autem et qui cum eo erant, custodiéntes Jesum, viso terræmótu et his, quæ fiébant, timuérunt valde, dicéntes: {\redx S.} Vere Fílius Dei erat iste. {\redx C.} Erant autem ibi mulíeres multæ a longe, quæ secútæ erant Jesum a Galilǽa, ministrántes ei: inter quas erat María Magdaléne, et María Jacóbi, et Joseph mater, et mater filiórum Zebedǽi.
}\switchcolumn\portugues{
Imediatamente, o véu do santuário se rasgou em duas partes, de alto a baixo; a terra tremeu nos seus alicerces; as pedras partiram-se; os sepulcros abriram-se e muitos corpos dos santos, que haviam morrido, ressuscitaram; e, saindo dos seus sepulcros, depois da ressurreição de Jesus, vieram à cidade santa e apareceram a muitos. O centurião e os que com ele estavam para guardar Jesus, vendo o tremor de terra e tudo quanto se passava, tiveram medo e diziam: {\redx S.} «Realmente, Este era o Filho de Deus!». {\redx C.} Achavam-se também, ali, a distância, algumas mulheres, que haviam seguido Jesus desde a Galileia para O servirem, em cujo número se contavam Maria Madalena, Maria, mãe de Tiago e de José, e a mãe dos filhos de Zebedeu.
}\switchcolumn*\latim{
Cum autem sero factum esset, venit quidam homo dives ab Arimathǽa, nómine Joseph, qui et ipse discípulus erat Jesu. Hic accéssit ad Pilátum, et pétiit corpus Jesu. Tunc Pilátus jussit reddi corpus. Et accépto córpore, Joseph invólvit illud in síndone munda. Et pósuit illud in monuménto suo novo, quod excíderat in petra. Et advólvit saxum magnum ad óstium monuménti, et ábiit. Erat autem ibi María Magdaléne et áltera María, sedéntes contra sepúlcrum.
}\switchcolumn\portugues{
Quando já era tarde, chegou um homem rico de Arimateia, chamado José, que também era discípulo de Jesus. Este homem foi ter com Pilatos e pediu-lhe o corpo de Jesus. Pilatos mandou que lhe fosse entregue o cadáver; e, levando-O José, amortalhou-O em um lençol limpo e depositou-O em um sepulcro novo, que havia mandado abrir na rocha. Depois, colocou uma pedra pesada à entrada do sepulcro e se retirou. Estavam, ali, Maria Madalena e a outra Maria, assentadas, defronte do sepulcro.
}\end{paracol}

\emph{Interrompe-se aqui a leitura e diz-se o MUNDA COR MEUM... (como no Ordinário da Missa). Depois continua-se:}

\begin{paracol}{2}\latim{
Altera autem die, quæ est post Parascéven, convenérunt príncipes sacerdótum et pharisǽi ad Pilátum, dicéntes: Dómine, recordáti sumus, quia sedúctor ille dixit adhuc vivens: Post tres dies resúrgam. Jube ergo custodíri sepúlcrum usque in diem tértium: ne forte véniant discípuli ejus, et furéntur eum, et dicant plebi: Surréxit a mórtuis; et erit novíssimus error pejor prióre. Ait illis Pilátus: Habétis custódiam, ite, custodíte, sicut scitis. Illi autem abeúntes, muniérunt sepúlcrum, signántes lápidem, cum custódibus.
}\switchcolumn\portugues{
No dia seguinte, depois do Parasceve, os príncipes dos sacerdotes e os fariseus reuniram-se e foram ter com Pilatos, dizendo: «Senhor, lembramo-nos de que Aquele sedutor, quando era vivo, disse: «Depois de três dias, ressuscitarei». Ordenai, pois, que seu sepulcro seja guardado, até ao terceiro dia, pois não seja o caso que os discípulos roubem o cadáver e digam depois à plebe: «Ressuscitou dos mortos!». Então, seria o último embuste pior do que o primeiro». Pilatos respondeu-lhes: «Tendes aí guardas; ide e guardai-o, como entenderdes». Eles, pois, foram, cimentaram o sepulcro, selaram a pedra e puseram-lhe guardas.
}\end{paracol}

\paragraphinfo{Ofertório}{Sl. 68, 21-22}
\begin{paracol}{2}\latim{
\rlettrine{I}{mpropérium} exspectávit cor meum et misériam: et sustínui, qui simul mecum contristarétur, et non fuit: consolántem me quæsívi, et non invéni: et dedérunt in escam meam fel, et in siti mea potavérunt me acéto.
}\switchcolumn\portugues{
\rlettrine{A}{s} humilhações e os opróbrios aniquilaram-me o coração; procurei quem se compadecesse de mim e não apareceu ninguém; procurei quem me consolasse e não achei ninguém! E deram-me fel para comer e vinagre para mitigar a sede!
}\end{paracol}

\paragraph{Secreta}
\begin{paracol}{2}\latim{
\rlettrine{C}{oncéde,} quǽsumus, Dómine: ut oculis tuæ majestátis munus oblátum, et grátiam nobis devotionis obtineat, et efféctum beátæ perennitátis acquírat. Per Dóminum nostrum \emph{\&c.}
}\switchcolumn\portugues{
\rlettrine{C}{oncedei-nos,} Senhor, Vos suplicamos, que este sacrifício, que oferecemos à vossa divina majestade, nos obtenha a graça duma pia devoção e nos assegure a posse da eterna felicidade. Por nosso Senhor \emph{\&c.}
}\end{paracol}

\paragraphinfo{Comúnio}{Mt. 26, 42}
\begin{paracol}{2}\latim{
\rlettrine{P}{ater,} si non potest hic calix transíre, nisi bibam illum: fiat volúntas tua.
}\switchcolumn\portugues{
\rlettrine{M}{eu} Pai, se este cálice não pode passar sem que Eu o beba, faça-se a vossa vontade.
}\end{paracol}

\paragraph{Postcomúnio}
\begin{paracol}{2}\latim{
\rlettrine{P}{er} hujus, Dómine, operatiónem mystérii: et vitia nostra purgéntur, et justa desidéria compleántur. Per Dóminum nostrum \emph{\&c.}
}\switchcolumn\portugues{
\rlettrine{S}{enhor,} pela virtude deste mystério, fazei que sejamos purificados dos nossos vícios e cumulados de desejos santos. Por nosso Senhor \emph{\&c.}
}\end{paracol}

\paragraphinfo{Último Evangelho}{Mt. 21, 1-9.}
\begin{paracol}{2}\latim{
\cruz Sequéntia sancti Evangélii secúndum Lucam.
}\switchcolumn\portugues{
\cruz Continuação do santo Evangelho segundo S. Mateus.
}\switchcolumn*\latim{
\blettrine{I}{n} illo témpore: Cum appropinquásset Jesus Jerosólymis, et venísset Béthphage ad montem Olivéti: tunc misit duos discípulos suos, dicens eis: Ite in castéllum, quod contra vos est, et statim inveniétis ásinam alligátam et pullum cum ea: sólvite et addúcite mihi: et si quis vobis áliquid dixerit, dícite, quia Dóminus his opus habet, et conféstim dimíttet eos. Hoc autem totum factum est, ut adimplerétur, quod dictum est per Prophétam, dicéntem: Dícite fíliae Sion: Ecce, Rex tuus venit tibi mansuétus, sedens super ásinam et pullum, fílium subjugális. Eúntes autem discípuli, fecérunt, sicut præcépit illis Jesus. Et adduxérunt ásinam et pullum: et imposuérunt super eos vestiménta sua, et eum désuper sedére tecérunt. Plúrima autem turba stravérunt vestiménta sua in via: álii autem cædébant ramos de arbóribus, et sternébant in via: turbæ autem, quæ præcedébant et quæ sequebántur, clamábant, dicéntes: Hosánna fílio David: benedíctus, qui venit in nómine Dómini.
}\switchcolumn\portugues{
\blettrine{N}{aquele} tempo, como Jesus se aproximasse de Jerusalém e chegasse a Bétfage, já perto do monte das Oliveiras, mandou dous dos seus discípulos, a quem disse: «Ide à aldeia fronteira e lá encontrareis uma jumenta presa e com ela um jumentinho. Desprendei-a e trazei-os. Se alguém vos disser alguma cousa, respondei: «O Senhor precisa deles». E logo os deixarão trazer». Tudo isto aconteceu para se cumprir o que fora anunciado pelo Profeta: «Dizei à filha de Sião: «Eis o teu Rei que vem a ti com doçura, montado em uma jumenta e sobre um jumentinho, filho da que está sob o jugo». Foram os discípulos e fizeram tudo como Jesus lhes ordenara, trazendo a jumenta e o jumentinho. Então, puseram em cima deles as suas capas e fizeram-n’O montar. Ora a multidão, que era numerosa estendia as suas capas na estrada e cortava ramos das árvores com que atapetavam o caminho. E os da multidão, tanto os que O precediam, como os que O seguiam, clamavam: «Hosana ao Filho de David. Bendito seja O que vem em nome do Senhor!».
}\end{paracol}
