\subsectioninfo{Quinta-feira da l.ª Semana da Quaresma - Têmporas da Primavera}{Estação em S. Lourenço de Panisperna}

\paragraphinfo{Intróito}{Sl. 95, 6}
\begin{paracol}{2}\latim{
\rlettrine{C}{onféssio} et pulchritúdo in conspéctu ejus: sánctitas et magnificéntia in sanctificatióne eius. \emph{Ps. ibid., 1} Cantáte Dómino cánticum novum: cantáte Dómino, omnis terra.
℣. Gloria Patri \emph{\&c.}
}\switchcolumn\portugues{
\rlettrine{A}{} glória e a majestade estão na sua presença; a santidade e a magnificência brilham noseu santuário. \emph{Sl. ibid., 1} Cantai ao Senhor um cântico novo; cantai em honra do Senhor, ó povos de toda a terra.
℣. Glória ao Pai \emph{\&c.}
}\end{paracol}

\paragraph{Oração}
\begin{paracol}{2}\latim{
\rlettrine{D}{evotiónem} pópuli tui, quǽsumus, Dómine, benígnus inténde: ut, qui per abstinéntiam macerántur in córpore, per fructum boni óperis reficiántur in mente. Per Dóminum \emph{\&c.}
}\switchcolumn\portugues{
\rlettrine{O}{lhai} benigno, Senhor, Vos suplicamos, para a piedade do vosso povo, a fim de que aqueles que mortificam o corpo com a abstinência sejam enriquecidos na alma com os frutos das boas obras. Por nosso Senhor \emph{\&c.}
}\end{paracol}

\paragraphinfo{Epístola}{Ez. 18, 1-9}
\begin{paracol}{2}\latim{
Léctio Ezechiélis Prophétæ.
}\switchcolumn\portugues{
Lição do Profeta Ezequiel.
}\switchcolumn*\latim{
\rlettrine{I}{n} diébus illis: Factus est sermo Dómini ad me, dicens: Quid est, quod inter vos parábolam vértitis in provérbium istud in terra Israël, dicéntes: Patres comedérunt uvam acérbam, et dentes filiórum obstupéscunt? Vivo ego, dicit Dóminus Deus, si erit ultra vobis parábola hæc in provérbium in Israël. Ecce, omnes ánimæ meæ sunt: ut ánima patris, ita et ánima fílii mea est: ánima, quæ peccáverit, ipsa moriétur. Et vir si fúerit justus, et fécerit judícium et justítiam, in móntibus non coméderit, et óculos suos non leváverit ad idóla domus Israël: et uxórem próximi sui non violáverit, et ad mulíerem menstruátam non accésserit: et hóminem non contristáverit: pignus debitóri reddíderit, per vim nihil rapúerit: panem suum esuriénti déderit, et nudum operúerit vestiménto: ad usúram non commodáverit, et ámplius non accéperit: ab iniquitáte avértent manum suam, et judícium verum fécerit inter virum et virum: in præcéptis meis ambuláverit, et judícia mea custodíerit, ut fáciat veritátem: hic justus est, vita vivet, ait Dóminus omnípotens.
}\switchcolumn\portugues{
\rlettrine{N}{aqueles} dias, a palavra do Senhor foi-me dirigida, dizendo: «Porque é que entre vós se repete este provérbio e o aplicais às terras de Israel: «Os pais comeram as uvas verdes e os dentes dos filhos é que se embotaram» ? Eu sou vivo, diz o Senhor Deus, nunca mais vos será dada ocasião de repetir em Israel este provérbio; pois todas as almas são minhas, tanto a alma do pai, como a alma do filho. A alma que pecar morrerá. Se um homem for justo e proceder segundo o direito e a justiça; se não comer carnes imoladas nos montes; se não levantar os olhos para os ídolos da casa de Israel, nem desonrar a mulher do próximo, nem se aproximar da mulher no tempo da abstenção; se não ofender e não oprimir ninguém; se entregar o penhor ao devedor, nada guardar em rapina, der pão ao que quiser comer, vestido ao que estiver nu e não emprestar com usura, nem receber mais do que é lícito; se afastar a mão da iniquidade e for árbitro da justiça de homem para homem; se seguir os meus preceitos e observar as minhas leis para proceder segundo a verdade: então, esse é justo e viverá na vida eterna: diz o Senhor omnipotente».
}\end{paracol}

\paragraphinfo{Gradual}{Sl. 16, 8 \& 2}
\begin{paracol}{2}\latim{
\rlettrine{C}{ustódi} me, Dómine, ut pupíllam óculi: sub umbra alárum tuárum prótege me. ℣. De vultu tuo judícium meum pródeat: óculi tui vídeant æquitátem. 
}\switchcolumn\portugues{
\rlettrine{G}{uardai-me,} Senhor, como à pupila dos vossos olhos; protegei-me com a sombra das vossas asas. ℣. Que meu julgamento seja pronunciado diante de Vós; que os vossos olhos vejam a equidade da minha causa.
}\end{paracol}

\paragraphinfo{Evangelho}{Mt. 15, 21-28}
\begin{paracol}{2}\latim{
\cruz Sequéntia sancti Evangélii secúndum Matthǽum.
}\switchcolumn\portugues{
\cruz Continuação do santo Evangelho segundo S. Mateus.
}\switchcolumn*\latim{
\blettrine{I}{n} illo témpore: Egréssus Jesus secéssit in partes Tyri et Sidónis. Et ecce, múlier Chananǽa a fínibus illis egréssa clamávit, dicens ei: Miserére mei, Dómine, fili David: fília mea male a dæmónio vexátur. Qui non respóndit ei verbum. Et accedéntes discípuli ejus rogábant eum, dicéntes: Dimítte eam; quia clamat post nos. Ipse autem respóndens, ait: Non sum missus nisi ad oves, quæ periérunt domus Israël. At illa venit, et adorávit eum, dicens: Dómine, ádjuva me. Qui respóndens, ait: Non est bonum sumere panem filiórum, et míttere cánibus. At illa dixit: Etiam, Dómine: nam et catélli edunt de micis, quæ cadunt de mensa dominórum suórum. Tunc respóndens Jesus, ait illi: O múlier, magna est fides tua: fiat tibi, sicut vis. Et sanáta est fília ejus ex illa hora.
}\switchcolumn\portugues{
\blettrine{N}{aquele} tempo, partindo Jesus, retirou-se dos lados de Tiro e de Sidónia. E eis que uma mulher cananeia, vinda daquelas paragens, clamou a Jesus, dizendo: «Senhor, filho de David, tende piedade de mim, pois minha filha está fortemente atacada pelo demónio». Jesus não respondeu nem uma palavra. Então, os discípulos aproximaram-se e suplicaram-Lhe, dizendo: «Despedi-a já, porque vem a gritar atrás de nós». Porém, Ele, respondendo, disse: «Eu não fui mandado senão às ovelhas perdidas da casa de Israel». Mas ela veio e adorou-O, dizendo: «Senhor, socorrei-me!». Ao que Ele respondeu: «Não é permitido tomar o pão dos filhos e atirá-lo aos cães». E ela disse: «É verdade, Senhor; mas os cachorrinhos também comem as migalhas que caem da mesa dos seus senhores». Então Jesus disse: «Ó mulher, a tua fé é grande! Pois bem: aconteça o que tu queres!». E a sua filha foi curada naquela mesma hora.
}\end{paracol}

\paragraphinfo{Ofertório}{Sl. 33, 8-9}
\begin{paracol}{2}\latim{
\rlettrine{I}{mmíttet} Angelus Dómini in circúitu timéntium eum, et erípiet eos: gustáte, et vidéte, quóniam suávis est Dóminus.
}\switchcolumn\portugues{
\rlettrine{O}{} Anjo do Senhor rodeia aqueles que o temem e salvá-los-á. Examinai e vede como o Senhor é bom!
}\end{paracol}

\paragraph{Secreta}
\begin{paracol}{2}\latim{
\rlettrine{S}{acrifícia,} Dómine, quǽsumus, propénsius ista nos salvent, quæ medicinálibus sunt institúta jejúniis. Per Dóminum \emph{\&c.}
}\switchcolumn\portugues{
\rlettrine{S}{enhor,} Vos suplicamos, permiti que estes sacrifícios, que foram instituídos juntamente com jejuns salutares, nos salvem pela vossa misericórdia. Por nosso Senhor \emph{\&c.}
}\end{paracol}

\paragraphinfo{Comúnio}{Jo. 6, 52}
\begin{paracol}{2}\latim{
\rlettrine{P}{anis,} quem ego dédero, caro mea est pro sǽculi vita.
}\switchcolumn\portugues{
\rlettrine{O}{} pão que eu Vos der é a minha carne para a salvação do mundo.
}\end{paracol}

\paragraph{Postcomúnio}
\begin{paracol}{2}\latim{
\rlettrine{T}{uórum} nos, Dómine, largitáte donórum, et temporálibus attólle præsídiis, et rénova sempitérnis. Per Dóminum nostrum \emph{\&c.}
}\switchcolumn\portugues{
\rlettrine{C}{om} a liberalidade de vossos dons, Senhor, livrai-nos das prisões temporais e renovai-nos para a eternidade. Por nosso Senhor \emph{\&c.}
}\end{paracol}

\paragraph{Oração sobre o povo}
\begin{paracol}{2}\latim{
\begin{nscenter} Orémus. \end{nscenter}
}\switchcolumn\portugues{
\begin{nscenter} Oremos. \end{nscenter}
}\switchcolumn*\latim{
Humiliáte cápita vestra Deo.
}\switchcolumn\portugues{
Inclinai as vossas cabeças diante de Deus.
}\switchcolumn*\latim{
Da, quǽsumus, Dómine, pópulis christiánis: et, quæ profiténtur, agnóscere, et cœléste munus dilígere, quod frequéntant. Per Dóminum nostrum \emph{\&c.}
}\switchcolumn\portugues{
Senhor, concedei aos fiéis cristãos, Vos suplicamos, a graça de conhecerem a dignidade da fé, que professam, e de amarem o dom celestial, que muitas vezes recebem. Por nosso Senhor \emph{\&c.}
}\end{paracol}