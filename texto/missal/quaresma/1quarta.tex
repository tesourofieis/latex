\subsectioninfo{Quarta-feira da l.ª Semana da Quaresma - Têmporas da Primavera}{Estação em Santa Maria Maior}

\paragraphinfo{Intróito}{Sl. 24, 6, 3 \& 22}
\begin{paracol}{2}\latim{
\rlettrine{R}{eminíscere} miseratiónum tuárum, Dómine, et misericórdiæ tuæ, quæ a sǽculo sunt: ne umquam dominéntur nobis inimíci nostri: líbera nos, Deus Israël, ex ómnibus angústiis nostris. \emph{Ps. ib., 1-2} Ad te, Dómine, levávi ánimam meam: Deus meus, in te confído, non erubéscam.
℣. Gloria Patri \emph{\&c.}
}\switchcolumn\portugues{
\rlettrine{L}{embrai-Vos,} Senhor, de que a vossa bondade e misericórdia são eternas! Não permitais nunca que os nossos inimigos triunfem de nós. Ó Deus de Israel, livrai-nos de todas nossas angústias. \emph{Sl. ib., 1-2} A Vós, Senhor, elevei a minha alma: confio em Vós, ó meu Deus, pois me não deixareis ficar envergonhado.
℣. Glória ao Pai \emph{\&c.}
}\end{paracol}

\begin{paracol}{2}\latim{
\begin{nscenter} Orémus. Flectámus génua. \end{nscenter}
}\switchcolumn\portugues{
\begin{nscenter} Oremos. Ajoelhemos! \end{nscenter}
}\switchcolumn*\latim{
℟. Leváte.
}\switchcolumn\portugues{
℟. Levantai-vos!
}\end{paracol}

\paragraph{Oração}
\begin{paracol}{2}\latim{
\rlettrine{P}{reces} nostras, quǽsumus, Dómine, cleménter exáudi: et contra cuncta nobis adversántia, déxteram tuæ majestátis exténde. Per Dóminum nostrum \emph{\&c.}
}\switchcolumn\portugues{
\rlettrine{S}{enhor,} Vos rogamos, ouvi clemente nossas preces e imponde o poder da dextra da vossa majestade contra os nossos inimigos. Por nosso Senhor \emph{\&c.}
}\end{paracol}

\paragraphinfo{Lição}{Ex. 24, 12-18}
\begin{paracol}{2}\latim{
Léctio libri Exodi.
}\switchcolumn\portugues{
Lição do Livro do Êxodo.
}\switchcolumn*\latim{
\rlettrine{I}{n} diébus illis: Dixit Dóminus ad Móysen: Ascénde ad me in montem, et esto ibi: dabóque tibi tábulas lapídeas, et legem ac mandáta quæ scripsi: ut dóceas fílios Israël. Surrexérunt Moyses et Josue miníster ejus: ascendénsque Moyses in montem Dei, senióribus ait: Exspectáte hic, donec revertámur ad vos. Habétis Aaron et Hur vobíscum: si quid natum luent quæstiónis, referétis ad eos. Cumque ascendísset Moyses, opéruit nubes montem, et habitávit glória Dómini super Sínai, tegens illum nube sex diébus: séptimo autem die vocávit eum de médio calíginis. Erat autem spécies glóriæ Dómini, quasi ignis ardens super vérticem montis; in conspéctu filiórum Israël. Ingressúsque Móyses médium nébulæ, ascéndit in montem: et luit ibi quadragínta diébus et quadragínta nóctibus.
}\switchcolumn\portugues{
\rlettrine{N}{aqueles} dias, disse o Senhor a Moisés: «Sobe comigo ao monte e permanece lá: dar-te-ei as tábuas de pedra, a Lei e os Mandamentos, que escrevi para que os ensines aos filhos de Israel». Ergueu-se, então, Moisés, acompanhado por seu ministro Josué, e caminhou para o monte de Deus, dizendo antes aos anciãos: «Ficai aqui até que voltemos. Ficam convosco Aarão e Hur. Se, pois, alguma querela houver entre vós, apresentar-lha-eis, para que a resolvam». E Moisés começou a subir o monte. Logo que lá chegou, desceu do céu uma nuvem, cobrindo o monte; e a glória do Senhor permaneceu no Sinai durante seis dias, envolvendo-o sempre a nuvem. No sétimo dia o Senhor no meio da nuvem chamou Moisés. Esta manifestação da glória do Senhor aparecia aos filhos de Israel sob a forma de fogo ardente, no pico do monte. Então Moisés entrou na nuvem, subiu para o monte e aí permaneceu durante quarenta dias e quarenta noites.
}\end{paracol}

\paragraphinfo{Gradual}{Sl. 24, 17-18}
\begin{paracol}{2}\latim{
\rlettrine{T}{ribulatiónes} cordis mei dilatátæ sunt: de necessitátibus meis éripe me, Dómine. ℣. Vide humilitátem meam, et labórem meum: et dimítte ómnia peccáta mea.
}\switchcolumn\portugues{
\rlettrine{M}{ultiplicaram-se} as angústias do meu coração. Senhor, livrai-me das minhas angústias. ℣. Vede a minha humilhação e as minhas fadigas: e perdoai-me todos os pecados.
}\end{paracol}

\paragraph{Oração}
\begin{paracol}{2}\latim{
\rlettrine{D}{evotiónem} pópuli tui, quǽsumus, Dómine, benígnus inténde: ut, qui per abstinéntiam macerántur in córpore, per fructum boni óperis reficiántur in mente. Per Dóminum \emph{\&c.}
}\switchcolumn\portugues{
\rlettrine{O}{lhai} benigno, Senhor, Vos pedimos, para a piedade do vosso povo, a fim de que aqueles que mortificam o corpo com a abstinência sejam enriquecidos na alma com o fruto das boas obras. Por nosso Senhor \emph{\&c.}
}\end{paracol}

\paragraphinfo{Epístola}{3 Rs. 19, 3-8}
\begin{paracol}{2}\latim{
Léctio libri Regum.
}\switchcolumn\portugues{
Lição do Livro dos Reis.
}\switchcolumn*\latim{
\rlettrine{I}{n} diébus illis: Venit Elías in Bersabée Juda, et dimísit ibi púerum suum, et perréxit in desértum, viam uníus diéi. Cumque venísset, et sedéret subter unam juníperum, petívit ánimæ suæ, ut morerétur, et ait: Súfficit mihi, Dómine, tolle ánimam meam: neque enim mélior sum quam patres mei. Projecítque se, et obdormívit in umbra juníperi: et ecce, Angelus Dómini tétigit eum, et dixit illi: Surge et cómede. Respéxit, et ecce ad caput suum subcinerícius panis, et vas aquæ: comédit ergo et bibit, et rursum obdormívit. Reversúsque est Angelus Dómini secundo, et tétigit eum, dixítque illi: Surge, cómede: grandis enim tibi restat via. Qui cum surrexísset, comédit et bibit, et ambulávit in fortitúdine cibi illíus quadragínta diébus et quadragínta nóctibus, usque ad montem Det Horeb.
}\switchcolumn\portugues{
\rlettrine{N}{aqueles} dias, havendo Elias chegado a Bersabeia, de Judá, deixou aí o seu servo e caminhou no deserto um dia. Então, sentou-se sob um junípero e pediu ao Senhor a morte, dizendo: «Já me basta, Senhor; tirai-me a vida, pois não sou melhor do que meus pais». E deitou-se no chão, adormecendo à sombra do junípero. Eis que um Anjo do Senhor lhe tocou e disse: «Levanta-te e come». Ele olhou e viu à sua cabeceira um pão cozido na cinza e um vaso de água. Comeu, bebeu e depois adormeceu. Voltou segunda vez o Anjo do Senhor, acordou-o e disse-lhe: «Levanta-te e come; pois um longo caminho te espera ainda». Levantou-se, comeu, bebeu e caminhou com o vigor daquele pão durante quarenta dias e quarenta noites até ao Horeb, a montanha do Senhor.
}\end{paracol}

\paragraphinfo{Trato}{Sl. 24, 17, 18 \& 1-4}
\begin{paracol}{2}\latim{
\rlettrine{D}{e} necessitátibus meis éripe me, Dómine: vide humilitátem meam et labórem meum: et dimítte ómnia peccáta mea. ℣. Ad te, Dómine, levávi ánimam meam: Deus meus, in te confído, non erubéscam: neque irrídeant me inimíci mei. ℣. Etenim univérsi, qui te exspéctant, non confundéntur: confundántur omnes faciéntes vana.
}\switchcolumn\portugues{
\rlettrine{S}{enhor,} livrai-me das minhas tribulações; vede a minha miséria e as minhas penas; e perdoai todos meus pecados. ℣. A Vós, Senhor, elevei a minha alma; meu Deus, em Vós confio: não ficarei envergonhado, pois os meus inimigos não triunfarão de mim! ℣. Não serão confundidos, Senhor, os que confiam em Vós: mas serão confundidos todos os que procedem em vão.
}\end{paracol}

\paragraphinfo{Evangelho}{Mt. 12, 38-50}
\begin{paracol}{2}\latim{
\cruz Sequéntia sancti Evangélii secúndum Matthǽum.
}\switchcolumn\portugues{
\cruz Continuação do santo Evangelho segundo S. Mateus.
}\switchcolumn*\latim{
\blettrine{I}{n} illo témpore: Respondérunt Jesu quidam de scribis et pharisǽis, dicéntes: Magíster, vólumus a te signum vidére. Qui respóndens, ait illis: Generátio mala et adúltera signum quærit: et signum non dábitur ei, nisi signum Jonæ Prophétæ. Sicut enim fuit Jonas in ventre ceti tribus diébus et tribus nóctibus: sic erit Fílius hóminis in corde terræ tribus diébus et tribus nóctibus. Viri Ninivítæ surgent in judício cum generatióne ista, et condemnábunt eam: quia pæniténtiam egérunt in prædicatióne Jonæ. Et ecce plus quam Jonas hic. Regína Austri surget in judício cum generatióne ista, et condemnábit eam: quia venit a fínibus terræ audire sapiéntiam Salomónis. Et ecce plus quam Sálomon hic. Cum autem immúndus spíritus exíerit ab hómine, ámbulat per loca árida, quærens réquiem, et non invénit. Tunc dicit: Revértar in domum meam, unde exívi. Et véniens invénit eam vacántem, scopis mundátam, et ornátam. Tunc vadit, et assúmit septem álios spíritus secum nequióres se, et intrántes hábitant ibi: et fiunt novíssima hóminis illíus pe-jóra prióribus. Sic erit et generatióni huic péssimæ. Adhuc eo loquénte ad turbas, ecce, Mater ejus et fratres stabant foris, quæréntes loqui ei. Dixit autem ei quidam: Ecce, mater tua et fratres tui foris stant, quæréntes te. At ipse respóndens dicénti sibi, ait: Quæ est mater mea, et qui sunt fratres mei? Et exténdens manum in discípulos suos, dixit: Ecce mater mea et fratres mei. Quicúmque enim fécerit voluntátem Patris mei, qui in cœlis est: ipse meus frater et soror et mater est.
}\switchcolumn\portugues{
\blettrine{N}{aquele} tempo, alguns escribas e fariseus falaram a Jesus, dizendo: «Mestre, queremos ver um prodígio praticado por Vós». Ele respondeu: «Esta geração má e adúltera pede um sinal, mas lhe não será dado senão o do Profeta Jonas. Pois, assim como Jonas esteve três dias e três noites encerrado nas entranhas da baleia, assim também o Filho do homem estará três dias e três noites no seio da terra. Os Ninivitas levantar-se-ão no dia do juízo diante desta geração e a condenarão, porque eles fizeram penitência depois da pregação de Jonas. Ora, Aquele que está aqui é mais do que Jonas. A rainha do Meio-Dia erguer-se-á no dia do juízo diante desta geração e a condenará, porque veio dos confins da terra ouvir a sabedoria de Salomão. Ora, Aquele que aqui está é mais do que Salomão. Quando o espírito impuro sai dum homem, anda errante por lugares áridos, procurando repouso, sem o encontrar. Então diz ele: «Voltarei para a minha casa, donde saí»; e, entrando nela, acha-a desocupada, lavada e ornada. E vai, procura sete espíritos piores do que ele, volta outra vez para casa e aí permanecem. Na verdade, o último estado deste homem é pior do que o primeiro. Assim acontecerá também a esta geração má». Falando Ele ainda ao povo, eis que estavam lá fora sua mãe e seus irmãos, que procuravam falar-Lhe. Alguém Lhe disse: «Eis que estão lá fora vossa mãe e vossos irmãos, que Vos procuram». Jesus, porém, respondeu assim ao homem que lhe falara: «Quem é minha mãe e quem são os meus irmãos?». E, estendendo a mão para os seus discípulos, acrescentou: «Eis minha mãe e meus irmãos; pois aquele que faz a vontade de meu Pai, que está nos céus, é meu irmão, minha irmã e minha mãe».
}\end{paracol}

\paragraphinfo{Ofertório}{Sl. 118, 47 \& 48}
\begin{paracol}{2}\latim{
\rlettrine{M}{editábor} in mandátis tuis, quæ diléxi valde: et levábo manus meas ad mandáta tua, quæ diléxi.
}\switchcolumn\portugues{
\rlettrine{M}{editarei} nos vossos mandamentos, que muito amo: levantarei as minhas mãos, cumprindo os vossos mandamentos, que, repito, muito amo.
}\end{paracol}

\paragraph{Secreta}
\begin{paracol}{2}\latim{
\rlettrine{H}{óstias} tibi, Dómine, placatiónis offérimus: ut et delícta nostra miserátus absólvas, et nutántia corda tu dírigas. Per Dóminum \emph{\&c.}
}\switchcolumn\portugues{
\rlettrine{V}{os} oferecemos estas hóstias de propiciação, Senhor, a fim de que pela vossa infinita misericórdia perdoeis os nossos pecados e governeis os nossos inconstantes corações. Por nosso Senhor \emph{\&c.}
}\end{paracol}

\paragraphinfo{Comúnio}{Sl. 5, 2-4}
\begin{paracol}{2}\latim{
\rlettrine{I}{ntéllege} clamórem meum: inténde voci oratiónis meæ, Rex meus et Deus meus: quóniam ad te orábo, Dómine.
}\switchcolumn\portugues{
\rlettrine{O}{uvi} a minha súplica: atendei ao clamor da minha prece, ó meu Rei e meu Deus; pois não cessarei de Vos invocar, Senhor.
}\end{paracol}

\paragraph{Postcomúnio}
\begin{paracol}{2}\latim{
\rlettrine{T}{ui,} Dómine, perceptióne sacraménti, et a nostris mundémur occúltis, et ab hóstium liberémur insídiis. Per Dóminum \emph{\&c.}
}\switchcolumn\portugues{
\rlettrine{P}{ermiti,} Senhor, que pela recepção do vosso Sacramento sejamos lavados das nossas faltas ocultas e livres das ciladas dos nossos inimigos. Por nosso Senhor \emph{\&c.}
}\end{paracol}

\paragraph{Oração sobre o povo}
\begin{paracol}{2}\latim{
\begin{nscenter} Orémus. \end{nscenter}
}\switchcolumn\portugues{
\begin{nscenter} Oremos. \end{nscenter}
}\switchcolumn*\latim{
Humiliáte cápita vestra Deo.
}\switchcolumn\portugues{
Inclinai as vossas cabeças diante de Deus.
}\switchcolumn*\latim{
Mentes nostras, quǽsumus, Dómine, lúmine tuæ claritátis illústra: ut vidére póssimus, quæ agénda sunt; et, quæ recta sunt, agere valeámus. Per Dóminum nostrum \emph{\&c.}
}\switchcolumn\portugues{
Dignai-Vos, Senhor, iluminar as nossas almas com o brilho do vosso divino esplendor, para que possamos ver o que devemos praticar e cumprir o que seja justo. Por nosso Senhor \emph{\&c.}
}\end{paracol}
