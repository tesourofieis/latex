\subsectioninfo{Quarta-feira da 2.ª Semana da Quaresma}{Estação em Santa Cecília}

\paragraphinfo{Intróito}{Sl. 37, 22-23}
\begin{paracol}{2}\latim{
\rlettrine{N}{e} derelínquas me, Dómine, Deus meus, ne discédas a me: inténde in adjutórium meum, Dómine, virtus salútis meæ. \emph{Ps. ibid., 2} Dómine, ne in furóre tuo árguas me: neque in ira tua corrípias me.
℣. Gloria Patri \emph{\&c.}
}\switchcolumn\portugues{
\rlettrine{N}{ão} me abandoneis, Senhor, meu Deus; não Vos afasteis de mim. Apressai-Vos em socorrer-me, ó Senhor, que sois a minha força e salvação. \emph{Sl. ibid., 2} Senhor, não me repreendais com furor, nem me castigueis com ira.
℣. Glória ao Pai \emph{\&c.}
}\end{paracol}

\paragraph{Oração}
\begin{paracol}{2}\latim{
\rlettrine{P}{ópulum} tuum, quǽsumus, Dómine, propítius réspice: et, quos ab escis carnálibus prǽcipis abstinére, a noxiis quoque vítiis cessáre concéde. Per Dóminum \emph{\&c.}
}\switchcolumn\portugues{
\rlettrine{O}{lhai} propício para o vosso povo, Senhor, Vos suplicamos, e permiti que aqueles a quem ordenais a abstinência das carnes se abstenham também dos vícios, que prejudicam as suas almas. Por nosso Senhor \emph{\&c.}
}\end{paracol}

\paragraphinfo{Epístola}{Est. 13, 8-11 \& 15-17}
\begin{paracol}{2}\latim{
Léctio libri Esther.
}\switchcolumn\portugues{
Lição da Ep.ª do B. Ap.º Paulo aos Coríntios.
}\switchcolumn*\latim{
\rlettrine{I}{n} diébus illis: Orávit Mardochǽus ad Dóminum, dicens: Dómine, Dómine, Rex omnípotens, in dicióne enim tua cuncta sunt pósita, et non est, qui possit tuæ resístere voluntáti, si decréveris salváre Israël. Tu fecísti cœlum et terram, et quidquid cœli ámbitu continétur. Dóminus ómnium es, nec est, qui resístat majestáti tuæ. Et nunc, Dómine Rex, Deus Abraham, miserére pópuli tui, quia volunt nos inimíci nostri pérdere, et hereditátem tuam delére. Ne despícias partem tuam, quam redemísti tibi de Ægýpto. Exáudi deprecatiónem meam, et propítius esto sorti et funículo tuo, et convérte luctum nostrum in gáudium, ut vivéntes laudémus nomen tuum, Dómine, et ne claudas ora te canéntium, Dómine, Deus noster.
}\switchcolumn\portugues{
\rlettrine{N}{aqueles} dias, orava Mardoqueu ao Senhor, dizendo: «Senhor, Senhor, Rei omnipotente, todas as coisas estão submetidas ao vosso poder. Nada, pois, poderá resistir à vossa vontade, se determinardes salvar o povo de Israel. Criastes o céu, a terra e tudo o que encerra o âmbito do céu; sois o Senhor de todas as coisas; ninguém pode resistir à vossa majestade. Então, agora, Senhor e Rei, Deus de Abraão, tende piedade do vosso povo; pois os nossos inimigos querem perder-nos e destruir a vossa herança. Não desprezeis o vosso povo, que livrastes do Egipto para ser vosso. Ouvi a minha oração; sede propício à nação que se tornou a vossa herança; convertei o nosso luto em alegria, a fim de que, conservando a vida, louvemos o vosso nome, Senhor» Não fecheis a boca àqueles que Vos louvam, ó Senhor, nosso Deus!».
}\end{paracol}

\paragraphinfo{Gradual}{Sl. 27, 9 \& 1}
\begin{paracol}{2}\latim{
\rlettrine{S}{alvum} fac pópulum tuum, Dómine, et bénedic hereditáti tuæ. ℣. Ad te, Dómine, clamávi: Deus meus, ne síleas a me, et ero símilis descendéntibus in lacum.
}\switchcolumn\portugues{
\rlettrine{S}{alvai} o vosso povo, Senhor, e abençoai a vossa herança. A Vós, Senhor, clamei. Meu Deus, não fecheis os ouvidos à minha voz. Se me não escutais, serei semelhante àqueles que desceram ao túmulo.
}\end{paracol}

\paragraphinfo{Trato}{Página \pageref{tratoquartacinzas}}

\paragraphinfo{Evangelho}{Mt. 20, 17-28}
\begin{paracol}{2}\latim{
\cruz Sequéntia sancti Evangélii secúndum Matthǽum.
}\switchcolumn\portugues{
\cruz Continuação do santo Evangelho segundo S. Mateus.
}\switchcolumn*\latim{
\blettrine{I}{n} illo témpore: Ascéndens Jesus Jerosólymam, assúmpsit duódecim discípulos secréto, et ait illis: Ecce, ascéndimus Jerosólymam, et Fílius hóminis tradétur princípibus sacerdótum, et scribis, et condemnábunt eum morte, et tradent eum Géntibus ad illudéndum, et flagellándum, et crucifigéndum, et tértia die resúrget. Tunc accéssit ad eum mater filiórum Zebedǽi cum fíliis suis, adórans et petens áliquid ab eo. Qui dixit ei: Quid vis? Ait illi: Dic, ut sédeant hi duo fílii mei, unus ad déxteram tuam et unus ad sinístram in regno tuo. Respóndens autem Jesus, dixit: Nescítis, quid petátis. Potéstis bíbere cálicem, quem ego bibitúrus sum? Dicunt ei: Póssumus. Ait illis: Cálicem quidem meum bibétis: sedére autem ad déxteram meam vel sinístram, non est meum dare vobis, sed quibus parátum est a Patre meo. Et audiéntes decem, indignáti sunt de duóbus frátribus. Jesus autem vocávit eos ad se, et ait: Scitis, quia príncipes géntium dominántur eórum: et qui majóres sunt, potestátem exércent in eos. Non ita erit inter vos: sed quicúmque volúerit inter vos major fíeri, sit vester miníster: et qui volúerit inter vos primus esse, erit vester servus. Sicut Fílius hóminis non venit ministrári, sed ministráre, et dare ánimam suam, redemptiónem pro multis.
}\switchcolumn\portugues{
\blettrine{N}{aquele} tempo, subindo Jesus para Jerusalém, chamou de parte os doze discípulos e disse-lhes: «Vamos subindo para Jerusalém, onde o Filho do homem será entregue aos príncipes dos sacerdotes e aos escribas. E condená-lo-ão à morte e entregá-lo-ão aos gentios, para ser escarnecido, flagelado e crucificado; mas ao terceiro dia ressuscitará». Então a mãe dos filhos de Zebedeu aproximou-se d’Ele com seus filhos, adorando-O e pretendendo pedir-Lhe alguma coisa. Ele disse-lhe: «Que queres?». Respondeu ela: «Ordenai que estes meus dois filhos se sentem no vosso reino, um à direita e o outro à esquerda». E Jesus disse-lhe: «Não sabeis o que pedis. Podeis beber o cálice que eu hei-de beber». Disseram-lhe: «Podemos!». Jesus disse-lhes: «Vós bebereis, certamente, o meu cálice; porém, sentar-vos à minha direita ou esquerda, não pertence a mim conceder-vo-lo, pois esse lugar será para quem meu Pai o tiver preparado». Ouvindo isto, os outros dez indignaram-se contra os dois irmãos. Jesus, porém, chamando-os para junto de si, disse-lhes: «Sabeis que os soberanos dominam os seus povos e que os mais importantes exercem a sua força contra estes, oprimindo-os. Não será assim entre vós; aquele, pois, que quiser ser superior aos outros será seu inferior; e o que quiser ser o primeiro será o seu escravo. Assim como o Filho do homem, que não veio para ser servido, mas para servir e dar a vida em redenção de muitos».
}\end{paracol}

\paragraphinfo{Ofertório}{Sl. 24, 1-3}
\begin{paracol}{2}\latim{
\rlettrine{A}{d} te, Dómine, levávi ánimam meam: Deus meus, in te confído, non erubéscam: neque irrídeant me inimíci mei: étenim univérsi, qui te exspéctant, non confundéntur.
}\switchcolumn\portugues{
\rlettrine{A}{} Vós, Senhor, elevei a minha alma. Ó meu Deus, confio em Vós; não serei confundido, nem vencido pelos meus inimigos., porquanto todos aqueles que em Vós confiam não serão confundidos.
}\end{paracol}

\paragraph{Secreta}
\begin{paracol}{2}\latim{
\rlettrine{H}{óstias,} Dómine, quas tibi offérimus, propítius réspice: et, per hæc sancta commércia, víncula peccatórum nostrórum absólve. Per Dóminum \emph{\&c.}
}\switchcolumn\portugues{
\rlettrine{O}{lhai} propício, Senhor, para as hóstias que Vos oferecemos, e, por este sagrado Cornércio, desligai-nos das cadeias dos nossos pecados. Por nosso Senhor \emph{\&c.}
}\end{paracol}

\paragraphinfo{Comúnio}{Sl. 10, 8}
\begin{paracol}{2}\latim{
\qlettrine{J}{ustus} Dóminus, et justítiam diléxit: æquitátem vidit vultus ejus.
}\switchcolumn\portugues{
\rlettrine{O}{} Senhor é justo e amou a justiça; seu rosto volta-se benévolo para os justos.
}\end{paracol}

\paragraph{Postcomúnio}
\begin{paracol}{2}\latim{
\rlettrine{S}{umptis,} Dómine, sacraméntis: ad redemptiónis ætérnæ, quǽsumus, proficiámus augméntum. Per Dóminum \emph{\&c.}
}\switchcolumn\portugues{
\rlettrine{H}{avendo} nós recebido estes sacramentos, Senhor, concedei-nos, Vos suplicamos, que nos sirvam de aumento dos frutos da redenção eterna. Por nosso Senhor \emph{\&c.}
}\end{paracol}

\paragraph{Oração sobre o povo}
\begin{paracol}{2}\latim{
\begin{nscenter} Orémus. \end{nscenter}
}\switchcolumn\portugues{
\begin{nscenter} Oremos. \end{nscenter}
}\switchcolumn*\latim{
Humiliáte cápita vestra Deo.
}\switchcolumn\portugues{
Inclinai as vossas cabeças diante de Deus.
}\switchcolumn*\latim{
Deus, innocéntiæ restitútor et amátor, dírige ad te tuórum corda servórum: ut, spíritus tui fervóre concépto, et in fide inveniántur stábiles, et in ópere efficáces. Per Dóminum \emph{\&c.}
}\switchcolumn\portugues{
Ó Deus, reparador e amigo da inocência, encaminhai para Vós os corações dos vossos servos, a fim de que, afervorados com vosso Espírito, sejam firmes na fé e activos nas obras. Por nosso Senhor \emph{\&c.}
}\end{paracol}
