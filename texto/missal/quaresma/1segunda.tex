\subsectioninfo{Segunda-feira da l.ª Semana da Quaresma}{Estação em S. Pedro nas Cadeias}

\paragraphinfo{Intróito}{Sl. 122, 2}
\begin{paracol}{2}\latim{
\rlettrine{S}{icut} óculi servórum in mánibus dominórum suórum: ita óculi nostri ad Dóminum, Deum nostrum, donec misereátur nobis: miserére nobis, Dómine, miserére nobis. \emph{Ps. ibid., 1} Ad te levávi óculos meos: qui hábitas in cœlis.
℣. Gloria Patri \emph{\&c.}
}\switchcolumn\portugues{
\rlettrine{A}{ssim} como os olhos dos escravos estão voltados para as mãos dos seus senhores, assim os nossos olhos estão voltados para o Senhor, nosso Deus, até que Ele tenha compaixão de nós. Tende compaixão de nós, Senhor, tende compaixão de nós. \emph{Sl. ibid., 1} Levantei os olhos para Vós, que habitais nos céus.
℣. Glória ao Pai \emph{\&c.}
}\end{paracol}

\paragraph{Oração}
\begin{paracol}{2}\latim{
\rlettrine{C}{onvérte} nos, Deus, salutáris noster: et, ut nobis jejúnium quadragesimále profíciat, mentes nostras cœléstibus ínstrue disciplínis. Per Dóminum \emph{\&c.}
}\switchcolumn\portugues{
\slettrine{Ó}{} Deus, nosso Salvador, convertei-nos; e, para que o jejum quaresmal nos seja proveitoso, instruí as nossas almas com vossas celestiais lições. Por nosso Senhor \emph{\&c.}
}\end{paracol}

\paragraphinfo{Epístola}{Ez. 34, 11-16}
\begin{paracol}{2}\latim{
Léctio Ezechiélis Prophétæ.
}\switchcolumn\portugues{
Lição do Profeta Ezequiel.
}\switchcolumn*\latim{
\rlettrine{H}{æc} dicit Dóminus Deus: Ecce, ego ipse requíram oves meas, et visitábo eas. Sicut vísitat pastor gregem suum in die, quando fúerit in médio óvium suárum dissipatárum: sic visitábo oves meas, et liberábo eas de ómnibus locis, in quibus dispérsæ fúerant in die nubis et calíginis. Et edúcam eas de pópulis, et congregábo eas de terris, et indúcam eas in terram suam: et pascam eas in móntibus Israël, in rivis, et in cunctis sédibus terræ. In páscuis ubérrimis pascam eas, et in móntibus excélsis Israël erunt páscua eárum: ibi requiéscent in herbis viréntibus, et in páscuis pínguibus pascéntur super montes Israël. Ego pascam oves meas, et ego eas accubáre fáciam, dicit Dóminus Deus. Quod períerat, requíram; et quod abjéctum erat, redúcam; et quod confractum fúerat, alligábo; et quod infírmum fúerat, consolidábo; et quod pingue et forte, custódiam: et pascam illas in judício, dicit Dóminus omnípotens.
}\switchcolumn\portugues{
\rlettrine{A}{ssim} fala o Senhor Deus: «Eis que Eu próprio procurarei as minhas ovelhas e as visitarei. Assim como um pastor, durante o dia, visita as suas ovelhas, quando está no meio delas e andam dispersas, assim visitarei as minhas ovelhas e as retirarei dos lugares por onde tinham sido dispersas, no tempo das nuvens e das tempestades. Tirá-las-ei do meio dos povos; congregá-las-ei dos diversos países; conduzi-las-ei à sua terra; e apascentá-las-ei nas montanhas de Israel, à beira dos regatos, em todos os lugares do país. Levá-las-ei a pastagens abundantíssimas; sua malhada será no cimo das montanhas de Israel, sobre a erva verdejante; e pastarão em ervas viçosas nas montanhas de Israel. Eu próprio apascentarei as minhas ovelhas e prepararei o seu repouso ao meu lado, diz o Senhor Deus. Procurarei aquela que se perdera; reconduzirei a que se desgarrara; tratarei a que se ferira; fortificarei a que enfraquecera; conservarei a que estava farta e forte; e apascentá-las-ei com justiça: diz o Senhor omnipotente».
}\end{paracol}

\paragraphinfo{Gradual}{Sl. 88, 10 \& 9}
\begin{paracol}{2}\latim{
\rlettrine{P}{rotéctor} noster, áspice, Deus, et réspice super servos tuos. ℣. Dómine, Deus virtútum, exáudi preces servórum tuórum.
}\switchcolumn\portugues{
\rlettrine{O}{lhai} para nós, ó Deus, nosso protector: volvei os vossos olhos para os vossos servos. ℣. Senhor, Deus dos exércitos, ouvi as súplicas dos vossos servos.
}\end{paracol}

\paragraphinfo{Trato}{Página \pageref{tratoquartacinzas}}

\paragraphinfo{Evangelho}{Mt. 25, 31-46}
\begin{paracol}{2}\latim{
\cruz Sequéntia sancti Evangélii secundum Matthǽum.
}\switchcolumn\portugues{
\cruz Continuação do santo Evangelho segundo S. Mateus.
}\switchcolumn*\latim{
\blettrine{I}{n} illo témpore: Dixit Jesus discípulis suis: Cum vénerit Fílius hóminis in majestáte sua, et omnes Angeli cum eo, tunc sedébit super sedem majestátis suæ: et congregabúntur ante eum omnes gentes, et separábit eos ab ínvicem, sicut pastor ségregat oves ab hædis: et státuet oves quidem a dextris suis, hædos autem a sinístris. Tunc dicet Rex his, qui a dextris ejus erunt: Veníte, benedícti Patris mei, possidéte parátum vobis regnum a constitutióne mundi. Esurívi enim, et dedístis mihi manducáre; sitívi, et dedístis mihi bíbere; hospes eram, et collegístis me; nudus, et cooperuístis me; infírmus, et visitástis me; in cárcere eram, et venístis ad me. Tunc respondébunt ei justi, dicéntes: Dómine, quando te vídimus esuriéntem, et pávimus te; sitiéntem, et dedimus tibi potum? quando autem te vídimus hóspitem, et collégimus te? aut nudum, et cooperúimus te? aut quando te vídimus infírmum, aut in cárcere, et vénimus ad te? Et respóndens Rex, dicet illis: Amen, dico vobis: quámdiu fecístis uni ex his frátribus meis mínimis, mihi fecístis. Tunc dicet et his, qui a sinístris erunt: Discédite a me, maledícti, in ignem ætérnum, qui parátus est diábolo et ángelis ejus. Esurívi enim, et non dedístis mihi manducáre; sitívi, et non dedístis mihi potum; hospes eram, et non collegístis me; nudus, et non cooperuístis me; infírmus et in cárcere, et non visitástis me. Tunc respondébunt ei et ipsi, dicéntes: Dómine, quando te vídimus esuriéntem, aut sitiéntem, aut hóspitem, aut nudum, aut infírmum, aut in cárcere, et non ministrávimus tibi? Tunc respondébit illis, dicens: Amen, dico vobis: Quámdiu non fecístis uni de minóribus his, nec mihi fecístis. Et ibunt hi in supplícium ætérnum: justi autem in vitam ætérnam.
}\switchcolumn\portugues{
\blettrine{N}{aquele} tempo, disse Jesus aos seus discípulos: «Quando o Filho do homem vier na sua majestade, acompanhado com os Anjos, sentar-se-á no trono da sua glória. Então, serão chamados à sua presença todos os povos e separados uns dos outros, como o pastor separa as ovelhas dos cabritos, e colocará as ovelhas à sua direita e os Cabritos à sua esquerda. E o Rei dirá aos que estão à direita: «Vinde, benditos de meu Pai, possuir o reino, que vos está preparado desde o princípio do mundo; porque tive fome, e destes-me de comer; tive sede, e destes-me de beber; estava sem asilo, e destes-me hospedagem; estava nu, e vestistes-me; enfermo, e visitastes-me; preso, e viestes ter comigo». Então, os justos lhe darão esta resposta: «Senhor, quando foi que Vos vimos com fome, e Vos demos de comer; com sede, e Vos demos de beber? Quando foi que Vos vimos sem casa, e Vos recolhemos em nossa casa; nu, e Vos vestimos? Quando Vos encontrámos doente ou na prisão, e fomos visitar-Vos?». E o Rei lhes responderá: «Na verdade vos digo: todas as vezes que fizestes isto a algum dos meus irmãos fizeste-lo a mim mesmo». Em seguida o Rei dirá aos que estão à sua esquerda: «Afastai-vos de mim, malditos, ide para o fogo do inferno, que foi preparado para o demónio e para os seus anjos; pois tive fome, e me não destes de comer; tive sede, e me não destes de beber; era hóspede, e me não recolhestes; estava nu, e me não vestistes; estava doente e preso, e me não visitastes». Então, eles responderão: «Senhor, quando foi que Vos vimos faminto ou sequioso, hóspede, nu, doente ou preso, e não Vos servimos?». Ele lhes responderá: «Em verdade vos digo: quando não fizestes isto mesmo a um dos mais pequenos, o não fizestes a mim mesmo». E estes irão para o suplício eterno, e os justos para a vida eterna».
}\end{paracol}

\paragraphinfo{Ofertório}{Sl. 118, 18, 26 \& 73}
\begin{paracol}{2}\latim{
\rlettrine{L}{evábo} óculos meos, et considerábo mirabília tua, Dómine, ut dóceas me justítias tuas: da mihi intelléctum, et discam mandáta tua.
}\switchcolumn\portugues{
\rlettrine{E}{rguerei} a Vós os meus olhos, Senhor, e considerarei as vossas maravilhas, para que me ensineis as vossas leis. Dai-me inteligência, e aprenderei os vossos Mandamentos.
}\end{paracol}

\paragraph{Secreta}
\begin{paracol}{2}\latim{
\rlettrine{M}{únera} tibi, Dómine, obláta sanctífica: nosque a peccatórum nostrórum máculis emúndet. Per Dóminum \emph{\&c.}
}\switchcolumn\portugues{
\rlettrine{S}{antificai,} Senhor, estes dons que Vos oferecemos, e purificai-nos das manchas dos nossos pecados. Por nosso Senhor \emph{\&c.}
}\end{paracol}

\paragraphinfo{Comúnio}{Mt. 25, 40 \& 34}
\begin{paracol}{2}\latim{
\rlettrine{A}{men,} dico vobis: quod uni ex mínimis meis fecístis, mihi fecístis: veníte, benedícti Patris mei, possidéte parátum vobis regnum ab inítio sǽculi.
}\switchcolumn\portugues{
\rlettrine{E}{m} verdade vos digo: todas as vezes que fizestes isto ao mais pequeno de meus irmãos, a mim mesmo o fizestes. Vinde, benditos de meu Pai, possuir o reino preparado para vós desde o princípio do mundo.
}\end{paracol}

\paragraph{Postcomúnio}
\begin{paracol}{2}\latim{
\rlettrine{S}{alutáris} tui, Dómine, múnere satiáti, súpplices exorámus: ut, cujus lætámur gustu, renovémur efféctu. Per Dóminum \emph{\&c.}
}\switchcolumn\portugues{
\rlettrine{A}{gora,} que fomos alimentados com vosso dom salutar, Vos suplicamos, Senhor, que, assim como tivemos a alegria de o receber, assim sejamos renovados com seus efeitos. Por nosso Senhor \emph{\&c.}
}\end{paracol}

\paragraph{Oração sobre o povo}
\begin{paracol}{2}\latim{
\begin{nscenter} Orémus. \end{nscenter}
}\switchcolumn\portugues{
\begin{nscenter} Oremos. \end{nscenter}
}\switchcolumn*\latim{
Humiliáte cápita vestra Deo.
}\switchcolumn\portugues{
Inclinai as vossas cabeças diante de Deus.
}\switchcolumn*\latim{
Absólve, quǽsumus, Dómine, nostrórum víncula peccatórum: et, quidquid pro eis merémur, propitiátus avérte. Per Dóminum \emph{\&c.}
}\switchcolumn\portugues{
Absolvei-nos, Senhor, Vos suplicamos, dos laços dos nossos pecados, e afastai propício os castigos que merecemos por causa deles. Por nosso Senhor \emph{\&c.}
}\end{paracol}
