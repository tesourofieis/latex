\subsectioninfo{Segunda-feira da Semana da Paixão}{Estação em S. Crisógono}

\paragraphinfo{Intróito}{Sl. 55, 2}
\begin{paracol}{2}\latim{
\rlettrine{M}{iserére} mihi, Dómine, quóniam conculcávit me homo: tota dic bellans tribulávit me. \emph{Ps. ibid., 3} Conculcavérunt me inimíci mei tota die: quóniam multi bellántes advérsum me.
}\switchcolumn\portugues{
\rlettrine{T}{ende} misericórdia de mim, Senhor, porque o meu inimigo calca-me aos pés: todos os dias me ataca e persegue. Meus inimigos calcam-me incessantemente com seus pés; pois São muitos os que me atacam.
}\end{paracol}

\paragraph{Oração}
\begin{paracol}{2}\latim{
\rlettrine{S}{anctífica,} quǽsumus, Dómine, nostra jejúnia: et cunctárum nobis indulgéntiam propítius largíre culpárum. Per Dóminum \emph{\&c.}
}\switchcolumn\portugues{
\rlettrine{V}{os} suplicamos, Senhor, santificai os nossos jejuns e concedei-nos benignamente a indulgência de todas nossas culpas. Por nosso Senhor \emph{\&c.}
}\end{paracol}

\paragraphinfo{Epístola}{Jn. 3, 1-10}
\begin{paracol}{2}\latim{
Léctio Jonæ Prophétæ.
}\switchcolumn\portugues{
Lição do Profeta Jonas.
}\switchcolumn*\latim{
\rlettrine{I}{n} diébus illis: Factum est verbum Dómini ad Jonam Prophétam secúndo, dicens: Surge, et vade in Níniven civitátem magnam: et prǽdica in ea prædicatiónem, quam ego loquor ad te. Et surréxit Jonas, et ábiit in Níniven juxta verbum Dómini. Et Nínive erat civitas magna itínere trium diérum. Et cœpit Jonas introíre in civitátem itínere diéi uníus: et clamávit et dixit: Adhuc quadragínta dies, et Nínive subvertétur. Et credidérunt viri Ninivítæ in Deum: et prædicavérunt jejúnium, et vestíti sunt saccis a majore usque ad minórem. Et pervénit verbum ad regem Nínive: et surréxit de sólio suo, et abjécit vestiméntum suum a se, et indútus est sacco, et sedit in cínere. Et clamávit et dixit in Nínive ex ore regis et príncipum ejus, dicens: Hómines et juménta et boves et pécora non gustent quidquam: nec pascántur, et aquam non bibant. Et operiántur saccis hómines et juménta, et clament ad Dóminum in fortitúdine, et convertátur vir a via sua mala, et ab iniquitáte, quæ est in mánibus eórum. Quis scit, si convertátur et ignóscat Deus: et revertátur a furóre iræ suæ, et non períbimus? Et vidit Deus ópera eórum, quia convérsi sunt de via sua mala: et misértus est pópulo suo Dóminus, Deus noster.
}\switchcolumn\portugues{
\rlettrine{N}{aqueles} dias, falou o Senhor segunda vez ao Profeta Jonas, dizendo-lhe: «Ergue-te, vai à grande cidade de Ninive e prega lá o que Eu te inspirar». Jonas ergueu-se e foi a Ninive, segundo a palavra do Senhor. Ora Ninive era uma grande cidade, a três dias de caminho. Jonas entrou na cidade, caminhou durante um dia e começou a pregar, dizendo: «Ainda quarenta dias e Ninive será destruída». Então os ninivitas acreditaram em Deus, proclamaram um jejum público e vestiram-se com sacos, desde o maior ao mais pequeno dos seus habitantes. Chegando isto ao conhecimento do rei de Ninive, ergueu-se ele do trono, despiu a túnica real, vestiu um saco e sentou-se na cinza. Em seguida fez publicar em Ninive, pela sua boca e pelos grandes da cidade: «que nem homens, nem animais (ou bois ou ovelhas) comessem, pastassem ou bebessem água; que os homens e animais se cobrissem com sacos; que os homens clamassem ao Senhor fortemente; e que toda a criatura humana abandonasse o mau caminho e a iniquidade de que suas mãos estavam manchadas. Quem sabe se Deus se não arrependerá de nos perdoar e não voltará ao furor da sua ira, de modo que todos pereçamos? E Deus viu as suas obras; viu que se convertiam e afastavam dos maus caminhos; e teve piedade do seu povo, o Senhor, nosso Deus».
}\end{paracol}

\paragraphinfo{Gradual}{Sl. 53,4 \& 3}
\begin{paracol}{2}\latim{
\rlettrine{D}{eus,} exáudi oratiónem meam: áuribus pércipe verba oris mei. ℣. Deus, in nómine tuo salvum me fac, et in virtúte tua líbera me.
}\switchcolumn\portugues{
\slettrine{Ó}{} Deus, ouvi a minha oração: prestai atenção às palavras da minha boca. Ó Deus, pela glória do vosso nome, salvai-me; livrai-me com vosso poder.
}\end{paracol}

\paragraphinfo{Trato}{Sl. 102, 10}\label{tratosemanapaixao}
\begin{paracol}{2}\latim{
\rlettrine{D}{ómine,} non secúndum peccáta nostra, quæ fécimus nos: neque secúndum iniquitátes nostras retríbuas nobis. ℣. \emph{Ps. 78, 8-9} Dómine, ne memíneris iniquitátum nostrárum antiquárum: cito antícipent nos misericórdiæ tuæ, quia páuperes facti sumus nimis. \emph{(Hic genuflectitur)} ℣. Adjuva nos, Deus, salutáris noster: et propter glóriam nóminis tui, Dómine, líbera nos: et propítius esto peccátis nostris, propter nomen tuum.
}\switchcolumn\portugues{
\rlettrine{S}{enhor,} nos não castigueis, consoante merecemos, pelos pecados que praticámos e por causa das nossas iniquidades! ℣. \emph{Sl. 78, 8-9} Esquecei-Vos, Senhor, das nossas antigas iniquidades e apressai-Vos em revestir-nos com vossas misericórdias, pois grande é a nossa miséria! \emph{(Aqui os fiéis devem genuflectir)} ℣. Auxiliai-nos, ó Deus, nosso Salvador. Para glória do vosso Nome, perdoai-nos os nossos pecados, Senhor!
}\end{paracol}

\paragraphinfo{Evangelho}{Jo. 7, 32-39}
\begin{paracol}{2}\latim{
\cruz Sequéntia sancti Evangélii secúndum Joánnem.
}\switchcolumn\portugues{
\cruz Continuação do santo Evangelho segundo S. Lucas.
}\switchcolumn*\latim{
\blettrine{I}{n} illo témpore: Misérunt príncipes et pharisǽi minístros, ut apprehénderent Jesum. Dixit ergo eis Jesus: Adhuc módicum tempus vobíscum sum: et vado ad eum, qui me misit. Quærétis me, et non inveniétis: et ubi ego sum, vos non potéstis veníre. Dixérunt ergo Judǽi ad semetípsos: Quo hic itúrus est, quia non inveniémus eum? numquid in dispersiónem géntium itúrus est, et doctúrus gentes? Quis est hic sermo, quem dixit: Quærétis me, et non inveniétis: et ubi sum ego, vos non potéstis veníre In novíssimo autem die magno festivitátis stabat Jesus, et clamábat, dicens: Siquis sitit, véniat ad me et bibat. Qui credit in me, sicut dicit Scriptúra, flúmina de ventre ejus fluent aquæ vivæ. Hoc autem dixit de Spíritu, quem acceptúri erant credéntes in eum.
}\switchcolumn\portugues{
\blettrine{N}{aquele} tempo, os príncipes e os fariseus mandaram soldados para prender Jesus. Então, Jesus disse-lhes: «Ainda um pouco de tempo estou convosco; mas depois irei para Aquele que me mandou. Vós me procurareis e não me encontrareis; pois onde Eu estiver não podereis ir». Os judeus disseram uns aos outros: «Para onde irá Ele que o não encontraremos? Porventura irá para os que estão dispersos entre as nações e para instruir os gentios? Que significam estas palavras: «Vós me procurareis e não me encontrareis; pois onde Eu estiver não podereis ir?». No último dia da festa, que é o mais solene, Jesus estava erguido e dizia em voz alta: «Se alguém tem sede, venha a mim e beba; pois, segundo a Sagrada Escritura, «rios de água viva manarão no seio daquele que crê em mim». Isto dizia, referindo-se ao Espírito que haviam de receber aqueles que acreditassem n’Ele.
}\end{paracol}

\paragraphinfo{Ofertório}{Sl. 6, 5}
\begin{paracol}{2}\latim{
\rlettrine{D}{ómine,} convértere, et éripe ánimam meam: salvum me fac propter misericórdiam tuam.
}\switchcolumn\portugues{
\rlettrine{V}{olvei-Vos} para mim, Senhor, e livrai a minha alma! Pela vossa misericórdia, salvai-me!
}\end{paracol}

\paragraph{Secreta}
\begin{paracol}{2}\latim{
\rlettrine{C}{oncéde} nobis, Dómine, Deus noster: ut hæc hóstia salutáris et nostrórum fiat purgátio delictórum, et tuæ propitiátio majestátis. Per Dóminum \emph{\&c.}
}\switchcolumn\portugues{
\rlettrine{P}{ermiti,} Senhor, nosso Deus, que esta salutar hóstia nos purifique dos nossos pecados e nos torne propícios à vossa majestade. Por nosso Senhor \emph{\&c.}
}\end{paracol}

\paragraphinfo{Comúnio}{Sl. 23, 10}
\begin{paracol}{2}\latim{
\rlettrine{D}{óminus} virtútum ipse est Rex glóriæ.
}\switchcolumn\portugues{
\rlettrine{O}{} Senhor dos exércitos é o Rei da
glória.
}\end{paracol}

\paragraph{Postcomúnio}
\begin{paracol}{2}\latim{
\rlettrine{S}{acraménti} tui, quǽsumus, Dómine, participátio salutáris, et purificatiónem nobis tríbuat, et medélam. Per Dóminum \emph{\&c.}
}\switchcolumn\portugues{
\rlettrine{S}{enhor,} Vos suplicamos, permiti que este salutar sacramento, de que comparticipámos, nos purifique e sirva de remédio. Por nosso Senhor \emph{\&c.}
}\end{paracol}

\paragraph{Oração sobre o povo}
\begin{paracol}{2}\latim{
\begin{nscenter} Orémus. \end{nscenter}
}\switchcolumn\portugues{
\begin{nscenter} Oremos. \end{nscenter}
}\switchcolumn*\latim{
Humiliáte cápita vestra Deo.
}\switchcolumn\portugues{
Inclinai as vossas cabeças diante de Deus.
}\switchcolumn*\latim{
Da, quǽsumus, Dómine, pópulo tuo salútem mentis et córporis: ut, bonis opéribus inhæréndo, tua semper mereátur protectióne deféndi. Per Dóminum \emph{\&c.}
}\switchcolumn\portugues{
Concedei ao vosso povo, Senhor, Vos suplicamos, a saúde da alma e do corpo, a fim de que, dedicando-se às boas obras, mereça sempre ser assistido com vossa protecção. Por nosso Senhor \emph{\&c.}
}\end{paracol}
