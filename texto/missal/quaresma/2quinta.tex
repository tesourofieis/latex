\subsectioninfo{Quinta-feira da 2.ª Semana da Quaresma}{Estação em Santa Maria além Tibre}

\paragraphinfo{Intróito}{Sl. 69, 2 \& 3}
\begin{paracol}{2}\latim{
\rlettrine{D}{eus,} in adjutórium meum inténde: Dómine, ad adjuvándum me festína: confundántur et revereántur inimíci mei, qui quærunt ánimam meam. \emph{Ps. ibid., 4} Avertántur retrórsum et erubéscant: qui cógitant mihi mala.
℣. Gloria Patri \emph{\&c.}
}\switchcolumn\portugues{
\slettrine{Ó}{} Deus, vinde em meu auxílio; apressai-Vos, Senhor, em socorrer-me. Sejam confundidos e cobertos de opróbrio os meus inimigos que procuram tirar-me a vida. \emph{Sl. ibid., 4} Que se afastem de mim, cheios de vergonha, os que conjuram contra mim!
℣. Glória ao Pai \emph{\&c.}
}\end{paracol}

\paragraph{Oração}
\begin{paracol}{2}\latim{
\rlettrine{P}{ræsta} nobis, quǽsumus, Dómine, auxílium grátiæ tuæ: ut, jejúniis et oratiónibus conveniénter inténti, liberémur ab hóstibus mentis et córporis. Per Dóminum \emph{\&c.}
}\switchcolumn\portugues{
\rlettrine{C}{oncedei-nos,} Senhor, Vos suplicamos, o auxílio da vossa graça, a fim de que, estando nós convenientemente aplicados aos jejuns e às orações, sejamos livres dos inimigos da alma e do corpo. Por nosso Senhor \emph{\&c.}
}\end{paracol}

\paragraphinfo{Epístola}{Jr. 17, 5-10}
\begin{paracol}{2}\latim{
Léctio Jeremíæ Prophétæ.
}\switchcolumn\portugues{
Lição do Profeta Jeremias.
}\switchcolumn*\latim{
\rlettrine{H}{æc} dicit Dóminus Deus: Maledíctus homo, qui confídit in hómine, et ponit carnem bráchium suum, et a Dómino recédit cor ejus. Erit enim quasi myrícæ in desérto, et non vidébit, cum vénerit bonum: sed habitábit in siccitáte in desérto, in terra salsúginis et inhabitábili. Benedíctus vir, qui confídit in Dómino, et erit Dóminus fidúcia ejus. Et erit quasi lignum, quod transplantátur super aquas, quod ad humórem mittit radíces suas: et non timébit, cum vénerit æstus. Et erit fólium ejus víride, et in témpore siccitátis non erit sollícitum, nec aliquándo de sinet fácere fructum. Pravum est cor ómnium et inscrutábile: quis cognóscet illud? Ego Dóminus scrutans cor, et probans renes: qui do unicuique juxta viam suam, et juxta fructum adinventiónum suárum: dicit Dóminus omnípotens.
}\switchcolumn\portugues{
\rlettrine{I}{sto} diz o Senhor e Deus: «Maldito seja o homem que confia no homem; que faz da carne o seu arrimo; e cujo coração se afasta do Senhor; pois ele será como as tamargas do deserto. Quando vier a felicidade, ele a não verá, mas permanecerá na aridez do deserto, numa terra salgada e desabitada! Bendito seja o homem que confia no Senhor e do qual o Senhor é a esperança; pois será como a árvore, plantada à beira das águas, que estende suas raízes no ribeiro e nada receia quando vem o calor. Sua folhagem estará sempre verde, não se ressentindo com o tempo da estiagem, nem deixará de dar bom fruto! O coração humano é depravado e impenetrável. Quem poderá conhecê-lo? Eu, o Senhor, que ausculto o seu coração e observo o seu íntimo, recompensando cada qual segundo os seus intentos e o mérito das suas obras: diz o Senhor omnipotente».
}\end{paracol}

\paragraphinfo{Gradual}{Sl. 78, 9 \& 10}
\begin{paracol}{2}\latim{
\rlettrine{P}{ropítius} esto, Dómine, peccátis nostris: ne quando dicant gentes: Ubi est Deus eórum? ℣. Adjuva nos, Deus, salutáris noster: et propter honórem nóminis tui, Dómine, líbera nos.
}\switchcolumn\portugues{
\rlettrine{P}{erdoai} os nossos pecados, Senhor, para que os povos não digam: «Onde está o seu Deus?». Socorrei-nos, Senhor, nosso Salvador, e, por causa da glória do vosso nome, livrai-nos, Senhor.
}\end{paracol}

\paragraphinfo{Evangelho}{Lc. J6, 19-31}
\begin{paracol}{2}\latim{
\cruz Sequéntia sancti Evangélii secúndum Lucam.
}\switchcolumn\portugues{
\cruz Continuação do santo Evangelho segundo S. Lucas.
}\switchcolumn*\latim{
\blettrine{I}{n} illo témpore: Dixit Jesus pharisǽis: Homo quidam erat dives, qui induebátur púrpura et bysso: et epulabátur cotídie spléndide. Et erat quidam mendícus, nómine Lázarus, qui jacébat ad jánuam ejus, ulcéribus plenus, cúpiens saturári de micis, quæ cadébant de mensa dívitis, et nemo illi dabat: sed et canes veniébant et lingébant úlcera ejus. Factum est autem, ut morerétur mendícus, et portarétur ab Angelis in sinum Abrahæ. Mórtuus est autem et dives, et sepúltus est in inférno. Elevans autem óculos suos, cum esset in torméntis, vidit Abraham a longe, et Lázarum in sinu ejus: et ipse clamans, dixit: Pater Abraham, miserére mei, et mitte Lázarum, ut intíngat extrémum dígiti sui in aquam, ut refrígeret linguam meam, quia crúcior in hac flamma. Et dixit illi Abraham: Fili, recordáre, quia recepísti bona in vita tua, et Lázarus simíliter mala: nunc autem hic consolátur, tu vero cruciáris. Et in his ómnibus, inter nos et vos chaos magnum firmátum est: ut hi, qui volunt hinc transíre ad vos, non possint, neque inde huc transmeáre. Et ait: Rogo ergo te, pater, ut mittas eum in domum patris mei. Hábeo enim quinque fratres, ut testétur illis, ne et ipsi véniant in hunc locum tormentórum. Et ait illi Abraham: Habent Móysen et Prophétas: áudiant illos. At ille dixit: Non, pater Abraham: sed si quis ex mórtuis íerit ad eos, pæniténtiam agent. Ait autem illi: Si Móysen et Prophétas non áudiunt, neque si quis ex mórtuis resurréxerit, credent.
}\switchcolumn\portugues{
\blettrine{N}{aquele} tempo, disse Jesus aos fariseus: «Havia um certo homem rico, que se vestia de púrpura e de linho fino e se banqueteava todos os dias com o maior esplendor. Havia, também, um certo mendigo, chamado Lázaro, que jazia, deitado à porta do rico, coberto de úlceras, e desejava comer as migalhas que caíam da mesa do rico, mas ninguém lhas dava, vindo os cães lamber-lhe as chagas! Aconteceu, porém, que morreu o pobre, sendo levado pelos Anjos para o seio de Abraão; e morreu também o rico, que foi sepultado no inferno. Ora, erguendo o rico os olhos, enquanto jazia em tormentos, viu ao longe Abraão, e, no seu seio, viu Lázaro. Então, disse em clamor: «Pai Abraão, tende misericórdia de mim; mandai que Lázaro com a ponta do seu dedo molhada em água venha refrescar a minha língua, porque sofro cruelmente nestas chamas!». Respondeu-lhe Abraão: «Filho, recorda-te de que recebeste bens durante a vida, e de que ao mesmo tempo Lázaro sofreu males. Agora, pois, este será consolado e tu sofrerás. Além disso, um grande abysmo está cavado entre nós; de sorte que os que quiserem passar daqui para lá não poderão; nem tão-pouco os daí poderão passar para aqui». E o rico continuou: «Eu vos suplico, ó pai, que envieis Lázaro a casa de meu pai, pois tenho lá cinco irmãos, para que lhes dê testemunho destas cousas, de modo que não venham também para este lugar de tormentos». Abraão respondeu: «Eles têm lá Moisés e os Profetas; que os ouçam, pois!». Mas ele disse: «Não, pai Abraão, não os ouvirão; mas, se algum dos mortos lhes falar, farão penitência». Abraão disse-lhe: «Se não ouvem nem Moisés nem os Profetas, ainda mesmo que algum dos mortos ressuscite, tão-pouco o acreditarão».
}\end{paracol}

\paragraphinfo{Ofertório}{Ex. 32, 11, 13 \& 14}
\begin{paracol}{2}\latim{
\rlettrine{P}{recátus} est Móyses in conspéctu Dómini, Dei sui, et dixit: Quare, Dómine, irascéris in pópulo tuo? parce iræ ánimæ tuæ: meménto Abraham, Isaac et Jacob, quibus jurásti dare terram fluéntem lac et mel. Et placátus est Dóminus de malignitáte, quam dixit fácere pópulo suo.
}\switchcolumn\portugues{
\rlettrine{O}{rou} Moisés ante o Senhor, seu Deus, e disse: «Porque, Senhor, Vos irais contra o vosso povo? Saí do ardor da vossa ira: lembrai-Vos de Abraão, de Isaque e de Jacob, aos quais jurastes dar a terra onde correm leite e mel». Então o Senhor acalmou-se e não mandou ao seu povo o mal que anunciara.
}\end{paracol}

\paragraph{Secreta}
\begin{paracol}{2}\latim{
\rlettrine{P}{æsénti} sacrifício, nómini tuo nos, Dómine, jejúnia dicáta sanctíficent: ut, quod observántia nostra profitétur extérius, intérius operétur efféctu. Per Dóminum \emph{\&c.}
}\switchcolumn\portugues{
\qlettrine{Q}{ue} o presente sacrifício, Senhor, santifique os jejuns que empreendemos para glória do vosso nome, a fim de que sua observância exterior seja acompanhada de frutos interiores. Por nosso Senhor \emph{\&c.}
}\end{paracol}

\paragraphinfo{Comúnio}{Jo. 6, 57}
\begin{paracol}{2}\latim{
\qlettrine{Q}{ui} mandúcat meam carnem, et bibit meum sánguinem, in me manet, et ego in eo, dicit Dóminus.
}\switchcolumn\portugues{
\rlettrine{A}{quele} que come a minha carne e bebe o meu sangue permanece em mim e eu permaneço nele: diz o Senhor.
}\end{paracol}

\paragraph{Postcomúnio}
\begin{paracol}{2}\latim{
\rlettrine{G}{rátia} tua nos, quǽsumus, Dómine, non derelínquat: quæ et sacræ nos déditos fáciat servitúti, et tuam nobis opem semper acquírat. Per Dóminum \emph{\&c.}
}\switchcolumn\portugues{
\rlettrine{S}{enhor,} Vos suplicamos, concedei-nos que a vossa graça nos não abandone; que nos torne dedicados ao vosso santo serviço; e que nos alcance sempre o vosso auxílio. Por nosso Senhor \emph{\&c.}
}\end{paracol}

\paragraph{Oração sobre o povo}
\begin{paracol}{2}\latim{
\begin{nscenter} Orémus. \end{nscenter}
}\switchcolumn\portugues{
\begin{nscenter} Oremos. \end{nscenter}
}\switchcolumn*\latim{
Humiliáte cápita vestra Deo.
}\switchcolumn\portugues{
Inclinai as vossas cabeças diante de Deus.
}\switchcolumn*\latim{
Adésto, Dómine, fámulis tuis, et perpétuam benignitátem largíre poscéntibus: ut iis, qui te auctóre et gubernatóre gloriántur, ei congregáta restáures et restauráta consérves. Per Dóminum \emph{\&c.}
}\switchcolumn\portugues{
Amparai, Senhor, os vossos servos e concedei-lhes a misericórdia perpétua, que Vos imploram; e, como eles se gloriam de Vos ter como Autor e Rei, restabelecei-os na posse dos bens em que os unistes, e mantende o que restabelecestes. Por nosso Senhor \emph{\&c.}
}\end{paracol}
