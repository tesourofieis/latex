\subsection{Domingo da Paixão}\label{domingopaicao}

\paragraphinfo{Intróito}{Sl. 42, 1-2}
\begin{paracol}{2}\latim{
\qlettrine{J}{údica} me, Deus, et discérne causam meam de gente non sancta: ab homine iníquo et dolóso éripe me: quia tu es Deus meus et fortitúdo mea. \emph{Ps. ibid., 3} Emítte lucem tuam et veritátem tuam: ipsa me de duxérunt et adduxérunt in montem sanctum tuum et in tabernácula tua.
}\switchcolumn\portugues{
\qlettrine{J}{ulgai-me,} ó Deus, e defendei a minha causa da causa de um povo infiel; livrai-me do homem iníquo e ardiloso: pois sois, ó meu Deus, a minha fortaleza. \emph{Sl. ibid., 3} Enviai-me a vossa luz e a vossa verdade, para que me guiem até ao vosso monte sagrado, até aos vossos tabernáculos.
}\end{paracol}

\paragraph{Oração}
\begin{paracol}{2}\latim{
\qlettrine{Q}{uǽsumus,} omnípotens Deus, familiam tuam propítius réspice: ut, te largiénte, regátur in córpore; et, te servánte, custodiátur in mente. Per Dóminum \emph{\&c.}
}\switchcolumn\portugues{
\slettrine{Ó}{} Deus omnipotente, olhai propício para a vossa família, Vos rogamos, a fim de que com vossa graça sejam dominados os nossos corpos e com vossa protecção sejam preservadas as nossas almas. Por nosso Senhor \emph{\&c.}
}\end{paracol}

\paragraphinfo{Epístola}{Heb. 9, 11-15}
\begin{paracol}{2}\latim{
Léctio Epístolæ beáti Pauli Apóstoli ad Hebrǽos.
}\switchcolumn\portugues{
Lição da Ep.ª do B. Ap.º Paulo aos Hebreus.
}\switchcolumn*\latim{
\rlettrine{F}{atres:} Christus assístens Pontifex futurórum bonórum, per ámplius et perféctius tabernáculum non manufáctum, id est, non hujus creatiónis: neque per sánguinem hircórum aut vitulórum, sed per próprium sánguinem introívit semel in Sancta, ætérna redemptióne invénta. Si enim sanguis hircórum et taurórum, et cinis vítulæ aspérsus, inquinátos sanctíficat ad emundatiónem carnis: quanto magis sanguis Christi, qui per Spíritum Sanctum semetípsum óbtulit immaculátum Deo, emundábit consciéntiam nostram ab opéribus mórtuis, ad serviéndum Deo vivénti? Et ideo novi Testaménti mediátor est: ut, morte intercedénte, in redemptiónem eárum prævaricatiónum, quæ erant sub prióri Testaménto, repromissiónem accípiant, qui vocáti sunt ætérnæ hereditátis, in Christo Jesu, Dómino nostro.
}\switchcolumn\portugues{
\rlettrine{M}{eus} irmãos: Jesus Cristo, vindo como Pontífice dos bens futuros, penetrou em um tabernáculo maior e mais que não foi fabricado por mão humana, isto é, que não teve criação terrena; e, sem recorrer ao sangue dos carneiros ou dos bois, mas pelo seu próprio sangue, entrou uma vez no santuário, tendo alcançado a salvação eterna. Com efeito, se o sangue dos carneiros e dos bois e a aspersão da cinza de vitela santificam aqueles que estão manchados, dando-lhes a pureza legal e exterior, quanto mais o sangue de Cristo (que pelo Espírito Santo se ofereceu a si mesmo a Deus como vítima imaculada) purificará a nossa consciência das obras mortas, para nos tornar capazes de servir Deus vivo! Por isso Ele é o mediador do Novo Testamento, a fim de que sua morte, servindo de resgate das prevaricações cometidas no Antigo Testamento, transmita a eterna aliança prometida àqueles que foram chamados em N. S. Jesus Cristo.
}\end{paracol}

\paragraphinfo{Gradual}{Sl. 142, 9 \& 10}
\begin{paracol}{2}\latim{
\rlettrine{E}{ripe} me, Dómine, de inimícis meis: doce me fácere voluntátem tuam. ℣. \emph{Ps. 17, 48-49} Liberátor meus, Dómine, de géntibus iracúndis: ab insurgéntibus in me exaltábis me: a viro iníquo erípies me.
}\switchcolumn\portugues{
\rlettrine{L}{ivrai-me} dos meus inimigos, Senhor: ensinai-me a cumprir a vossa vontade. ℣. \emph{Sl. 17, 48-49} Livrai-me, ó Senhor, das iras dos meus inimigos: elevai-me sobre aqueles que se insurgem contra mim e defendei-me do homem iníquo.
}\end{paracol}

\paragraphinfo{Trato}{Sl. 128, 1-4}
\begin{paracol}{2}\latim{
\rlettrine{S}{æpe} expugnavérunt me a juventúte mea. ℣. Dicat nunc Israël: sæpe expugnavérunt me a juventúte mea. ℣. Etenim non potuérunt mihi: supra dorsum meum fabricavérunt peccatóres. ℣. Prolongavérunt iniquitátes suas: Dóminus justus cóncidit cervíces peccatórum.
}\switchcolumn\portugues{
\rlettrine{C}{ombateram-me} desde a minha juventude. ℣. Diga agora Israel: combateram-me desde a minha juventude. ℣. Mas nada puderam contra mim. Nas minhas costas ficaram sinais das pancadas dos pecadores. ℣. Durante bastante tempo continuaram as suas iniquidades; mas o Senhor, que é justo, cortará a cabeça dos pecadores.
}\end{paracol}

\paragraphinfo{Evangelho}{Jo. 8, 46-59}
\begin{paracol}{2}\latim{
\cruz Sequéntia sancti Evangélii secúndum Joánnem.
}\switchcolumn\portugues{
\cruz Continuação do santo Evangelho segundo S. João.
}\switchcolumn*\latim{
\blettrine{I}{n} illo témpore: Dicébat Jesus turbis Judæórum: Quis ex vobis árguet me de peccáto? Si veritátem dico vobis, quare non créditis mihi? Qui ex Deo est, verba Dei audit. Proptérea vos non audítis, quia ex Deo non estis. Respondérunt ergo Judǽi et dixérunt ei: Nonne bene dícimus nos, quia Samaritánus es tu, et dæmónium habes? Respóndit Jesus: Ego dæmónium non hábeo, sed honorífico Patrem meum, et vos inhonorástis me. Ego autem non quæro glóriam meam: est, qui quærat et júdicet. Amen, amen, dico vobis: si quis sermónem meum serváverit, mortem non vidébit in ætérnum. Dixérunt ergo Judǽi: Nunc cognóvimus, quia dæmónium habes. Abraham mórtuus est et Prophétæ; et tu dicis: Si quis sermónem meum serváverit, non gustábit mortem in ætérnum. Numquid tu major es patre nostro Abraham, qui mórtuus est? et Prophétæ mórtui sunt. Quem teípsum facis? Respóndit Jesus: Si ego glorífico meípsum, glória mea nihil est: est Pater meus, qui gloríficat me, quem vos dícitis, quia Deus vester est, et non cognovístis eum: ego autem novi eum: et si díxero, quia non scio eum, ero símilis vobis, mendax. Sed scio eum et sermónem ejus servo. Abraham pater vester exsultávit, ut vidéret diem meum: vidit, et gavísus est. Dixérunt ergo Judǽi ad eum: Quinquagínta annos nondum habes, et Abraham vidísti? Dixit eis Jesus: Amen, amen, dico vobis, antequam Abraham fíeret, ego sum. Tulérunt ergo lápides, ut jácerent in eum: Jesus autem abscóndit se, et exívit de templo.
}\switchcolumn\portugues{
\blettrine{N}{aquele} tempo, dizia Jesus à multidão dos judeus: «Qual de vós me acusará de pecado? Se vos digo a verdade, porque não me acreditais? Aquele que é de Deus, ouve a palavra de Deus. Vós não atendeis à palavra de Deus, porque não sois de Deus». Os judeus responderam-Lhe: «Não temos nós razão para dizer que sois samaritano e que estais possesso do demónio?». Jesus retorquiu: «Eu não estou possesso do demónio, mas honro a meu Pai; enquanto que vós desonrais-me. Eu não procuro a minha própria glória. Há alguém que tem esse cuidado e me fará justiça. Em verdade, em verdade vos digo: se alguém obedecer às minhas palavras, não terá a morte para sempre». Os judeus disseram-Lhe: «Agora conhecemos que possuís o demónio. Abraão morreu, assim como os Profetas, e ainda dizeis: se alguém obedecer às minhas palavras, não morrerá para sempre? Sois, porventura, maior do que nosso pai Abraão, que morreu, assim como os Profetas? Quem pretendeis ser?». Jesus respondeu: «Se me glorificasse a mim mesmo, a minha glória não era nada! Porém é meu Pai (a quem chamais vosso Deus) quem me glorifica. Contudo, vós O não conheceis; mas Eu conheço-O. Se dissesse que O não conhecia, seria mentiroso, semelhante a vós; mas Eu conheço-O e obedeço às suas palavras. Abraão, vosso pai, exultou de alegria, porque desejou ver o meu dia. Viu-o e rejubilou». Os judeus disseram-Lhe, então: «Ainda não tendes cinquenta anos e vistes Abraão? Jesus respondeu-lhes: «Em verdade, em verdade vos digo: Antes que Abraão viesse a ser feito (isto é, nascesse) já Eu existo!». Então, eles tomaram pedras, para Lhas atirarem; mas Jesus ocultou-se e saiu do templo.
}\end{paracol}

\paragraphinfo{Ofertório}{Sl. 118, 17 \& 107}
\begin{paracol}{2}\latim{
\rlettrine{C}{onfitébor} tibi, Dómine, in toto corde meo: retríbue servo tuo: vivam, et custódiam sermónes tuos: vivífica me secúndum verbum tuum, Dómine.
}\switchcolumn\portugues{
\rlettrine{S}{enhor,} louvar-Vos-ei de todo meu coração. Recompensai o vosso servo: Então viverei e observarei os vossos preceitos. Vivificai-me, segundo a vossa palavra, Senhor!
}\end{paracol}

\paragraph{Secreta}
\begin{paracol}{2}\latim{
\rlettrine{H}{æc} múnera, quǽsumus Dómine, ei víncula nostræ pravitátis absólvant, et tuæ nobis misericórdiæ dona concílient. Per Dóminum \emph{\&c.}
}\switchcolumn\portugues{
\qlettrine{Q}{ue} estas ofertas, Senhor, Vos suplicamos, esmaguem os laços da nossa malícia e nos alcancem os dons da vossa misericórdia. Por nosso Senhor \emph{\&c.}
}\end{paracol}

\paragraphinfo{Comúnio}{1 Cor. 11, 24 \& 25}
\begin{paracol}{2}\latim{
\rlettrine{H}{oc} corpus, quod pro vobis tradétur: hic calix novi Testaménti est in meo sánguine, dicit Dóminus: hoc fácite, quotiescúmque súmitis, in meam commemoratiónem.
}\switchcolumn\portugues{
\rlettrine{E}{ste} é o meu corpo, que será entregue por vós. Este é o cálice da nova aliança no meu sangue: diz o Senhor. Fazei isto em memória de mim todas as vezes que os receberdes.
}\end{paracol}

\paragraph{Postcomúnio}
\begin{paracol}{2}\latim{
\rlettrine{A}{désto} nobis, Dómine, Deus noster: et, quos tuis my
stériis recreásti, perpétuis defénde subsidiis. Per Dóminum nostrum \emph{\&c.}
}\switchcolumn\portugues{
\rlettrine{A}{ssisti-nos,} ó Senhor, nosso Deus; e concedei o vosso perpétuo auxílio aqueles a quem restaurastes com vossos mystérios. Por nosso Senhor \emph{\&c.}
}\end{paracol}
