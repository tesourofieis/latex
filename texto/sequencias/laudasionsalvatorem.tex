\subsectioninfo{Láuda Síon Salvatórem}{Procissão Corpus Christi}
\label{laudasionsalvatorem}

\begin{paracol}{2}\latim{
\rlettrine{L}{áuda} Síon Salvatórem, Láuda dúcem et pastórem, In hýmnis et cánticis.
}\switchcolumn\portugues{
\rlettrine{L}{ouva,}ó Sião, louva o teu Salvador! Louva com hinos e cânticos o teu Principe e Pastor!
}\switchcolumn*\latim{
Quantum pótes, tantum áude: Quia májor ómni láude, Nec laudáre súfficis.
}\switchcolumn\portugues{
Louva-O tanto quanto possas: nem tu podes louvá-l’O dignamente, pois está acima de todos os louvores.
}\switchcolumn*\latim{
Láudis théma speciális, Pánis vívus et vitális Hódie propónitur.
}\switchcolumn\portugues{
O assunto especial dos teus louvores, hoje, é o Pão vivo e vivificante:
}\switchcolumn*\latim{
Quem in sácræ ménsa coénæ Túrbæ frátrum duodénæ Dátum non ambígitur.
}\switchcolumn\portugues{
É aquele mesmo Pão (nós o acreditamos) que foi dado ao grupo dos Doze Irmãos na Última Ceia.
}\switchcolumn*\latim{
Sit laus pléna, sít sonóra, Sit jucúnda, sit decóra Méntis jubilátio.
}\switchcolumn\portugues{
Que o louvor seja perene, sonoro e melodioso: que seja agradável e belo, como a alegria que transporta as nossas almas!
}\switchcolumn*\latim{
Díes enim solémnis ágitur, In qua ménsæ príma recólitur Hújus institútio.
}\switchcolumn\portugues{
Eis o dia solene, que nos recorda a primitiva instituição deste Banquete divino.
}\switchcolumn*\latim{
In hac ménsa nóvi Régis, Nóvum Páscha nóvæ légis, Pháse vétus términat.
}\switchcolumn\portugues{
Nesta mesa do novo Rei a Páscoa da nova Lei acaba com a páscoa antiga.
}\switchcolumn*\latim{
Vetustátem nóvitas, Umbram fúgat véritas, Nóctem lux elíminat.
}\switchcolumn\portugues{
O rito antigo cede o lugar ao novo, como a imagem desaparece diante da realidade, e a luz apaga a noite.
}\switchcolumn*\latim{
Quod in coéna Chrístus géssit, Faciéndum hoc expréssit In súi memóriam.
}\switchcolumn\portugues{
Aquilo que Cristo praticou na Ceia, mandou que fizéssemos, também, em sua memória.
}\switchcolumn*\latim{
Dócti sácris institútis, Pánem, vínum, in salútis Consecrámus hóstiam.
}\switchcolumn\portugues{
E nós, instruídos pelo mandato divino, consagrámos o pão e o vinho em hóstia de salvação.
}\switchcolumn*\latim{
Dógma dátur christiánis, Quod in cárnem tránsit pánis, Et vínum in sánguinem.
}\switchcolumn\portugues{
É um dogma de fé para os cristãos que o pão passa para Carne de Cristo e o vinho para seu Sangue.
}\switchcolumn*\latim{
Quod non cápis, quod non vídes, Animósa fírmat fídes, Præter rérum órdinem.
}\switchcolumn\portugues{
Aquilo que não compreendeis, nem vedes, a fé viva o afirma sem alterar a ordem da natureza.
}\switchcolumn*\latim{
Sub divérsis speciébus, Sígnis tantum, et non rébus, Látent res exímiæ.
}\switchcolumn\portugues{
Debaixo de diversas espécies, distintas somente por sinais exteriores, ocultam-se sublimes realidades.
}\switchcolumn*\latim{
Cáro cíbus, sánguis pótus: Mánet tamen Chrístus tótus, Sub utráque spécie.
}\switchcolumn\portugues{
A Carne de Cristo é alimento, e o Sangue bebida; mas Ele existe inteiro em cada uma das espécies.
}\switchcolumn*\latim{
A suménte non concísus, Non confráctus, non divísus: Integer accípitur.
}\switchcolumn\portugues{
Quem O recebe, nem O parte, nem O corta, nem O divide; porém recebe-O inteiro.
}\switchcolumn*\latim{
Súmit únus, súmunt mille: Quantum ísti, tantum ílle: Nec súmptus consúmitur.
}\switchcolumn\portugues{
Quem O receba uma só pessoa, quer O recebam mil, todas recebem o mesmo: recebem-n’O sem O consumirem.
}\switchcolumn*\latim{
Súmunt bóni, súmunt máli: Sórte tamen inæquáli, Vítæ vel intéritus. Mors est mális, víta bónis:
}\switchcolumn\portugues{
Recebem-n’O bons e maus; porém com efeitos diferentes: uns encontram vida; outros a morte! Para os maus é morte e para os bons é vida.
}\switchcolumn*\latim{
Víde páris sumptiónis Quam sit díspar éxitus. Frácto demum sacraménto,
}\switchcolumn\portugues{
Vede que diferentes são os efeitos que produz o mesmo alimento!
}\switchcolumn*\latim{
Ne vacílles, sed meménto Tantum ésse sub fragménto, Quantum tóto tégitur.
}\switchcolumn\portugues{
Que a vossa fé não vacile quando a hóstia é dividida; mas lembrai-vos de que Jesus tanto está no fragmento, como na hóstia inteira.
}\switchcolumn*\latim{
Núlla réi fit scissúra: Sígni tantum fit fractúra, Qua nec státus, nec statúra Signáti minúitur.
}\switchcolumn\portugues{
A substância não é dividida: somente o sinal é que é partido, mas sem diminuição, nem no estado, nem na grandeza do que está sob as espécies.
}\switchcolumn*\latim{
Ecce Pánis Angelórum, Fáctus cíbus viatórum
}\switchcolumn\portugues{
Eis o Pão dos Anjos que se fez alimento dos homens viadores,
}\switchcolumn*\latim{
Vere pánis filiórum, Non mitténdus cánibus.
}\switchcolumn\portugues{
Verdadeiro pão dos inocentes, que não deve ser dado aos cães!
}\switchcolumn*\latim{
In figúris præsignátur, Cum Isáac immolátur, Agnus Páschæ deputátur, Dátur mánna pátribus.
}\switchcolumn\portugues{
Antigamente foi representado por figuras: imolado com Isaque e significado no cordeiro pascal e no maná do deserto.
}\switchcolumn*\latim{
Bóne pástor, pánis vére, Jésu, nóstri miserére: Tu nos pásce, nos tuére, Tu nos bóna fac vidére In térra vivéntium.
}\switchcolumn\portugues{
Ó bom Pastor, ó Pão verdadeiro, ó Jesus, tende piedade de nós: alimentai-nos, defendei-nos do mal e permiti que gozemos os verdadeiros bens da terra dos vivos.
}\switchcolumn*\latim{
Tu qui cúncta scis et váles, Qui nos páscis hic mortáles: Túos ibi commensáles, Coherédes et sodáles Fac sanctórum cívium. ℟. Amen.
}\switchcolumn\portugues{
Ó Vós, que tudo conheceis e podeis: ó Vós, que nos alimentais nesta vida mortal, tornai-nos co-herdeiros e companheiros dos habitantes da cidade celestial. ℟. Amen.
}\end{paracol}
